While Cybernetics began as prosthetic limbs and organs, modern biotechnology that allows for the growth of new limbs and organs has eclipsed the need for purely cybernetic replacements. Instead, cybernetic technology has turned towards those which \textit{enhance} the body.

\subsection{Social Effects of Cybernetics}\label{subsec:social_effects_cybernetics}
While there are certainly those who shun cybernetics, especially among the radical anti~-android movement, they are exceedingly rare. Among most people, cybernetics are as normal as eye~-glasses (in fact, eye glasses might be rarer given the state of the ease of replacement or bionic eyes).

\subsection{Statistics}\label{subsec:statistics_cybernetics}
Cybernetic modifications usually provide advantages or mitigate disadvantages; these traits are listed under Statistics along with the total point cost. The Body Modification (p. B294) rules apply, with the exception that the more detailed Operations rules below supersede the Surgical Modifications rules on p. B295.

Cybernetic advantages often have the limitation Temporary Disadvantage (Electrical, -20\%), which means the advantage is vulnerable to electrical surges, power draining, etc. See Electrical, p. B134.

Cybernetic replacement parts for specific body locations are bought as a crippling disadvantage with the Mitigator (-70\%) limitation. This limitation is assumed to include the effects of the Electrical, Maintenance (1 person, monthly) (p. B143), and Unhealing (p. B160) disadvantages for that body part.

Cybernetic implants generally supercede (or mitigate) existing natural or biological traits. Thus, if someone with Night Vision 1 gains a bionic eye that provides Night Vision 2, the levels don't stack together. Modify the character's point total accordingly; if paying character points for the advantage, base the cost on the net change (if positive).

\subsection{Availability}\label{subsec:availability_cybernetics}
Each modification specifies the type of procedure, the cost of the cybernetics, and the LC. Procedures are classified as simple, minor, complex, or radical -- see below.

\subsection{Procedure}\label{subsec:procedure_cybernetics}
Installing cybernetic modifications requires, except for simple procedures, involved neuro~-surgery. For players or NPCs performing this surgery, use the Surgery (Cybernetics) skill (Defaults: Surgery-2). When receiving this surgery at a reputable hospital, no skill rolls are generally required.

The Surgical Procedures Table shows the difficulty modifiers (parenthetical modifier is for brain or eye surgery), time per attempt, and the injury caused by a failed roll (which is applied to the body part being operated on). In addition to these modifiers, apply \textit{General Surgery Modifiers} (\textbf{GURPS Bio-Tech}, p. 138). Success installs the modification, but it won't work until after the specified recovery period. The fee is the surgical fee charged at a clinic or hospital -- ignore it when doing your own work.

These times assume the surgeon is using precise, robotic instruments; without them, double the recovery time and damage. (Increase the damage from a failed Simple procedure to 1d/2 HP.)

A modification is not functional until the recovery time has passed. Additionally, follow the rules for \textit{Recovery} (\textbf{Bio-Tech}, p. 139). If a disadvantage is mitigated by the modification -- e.g., One Hand for Bionic Hand -- the patient will suffer the disadvantage until the recovery time is completed.

On a critical success, halve the recovery time. A critical failure means that the cybernetic is defective in some way. This may mean it will break down at some point or it might cause an inconvenient disadvantage.

\multicolinterrupt{
    \begin{table}[H]
        \hrule height 1pt\medskip
        \subsection{Surgical Procedure Tables}
        \centering
        \rowcolors{1}{}{\colormedical}
        \begin{tabularx}{\linewidth}{lXcXcXcXcXr}
             \textbf{Procedure} && \textbf{Modifiers} && \textbf{Time} && \textbf{Injury} && \textbf{Recovery Period} && \textbf{Fee} \\
             Simple && +4 (+2) && 15 min. && 1 HP && 1 hour && \$100 \\
             Minor && +2 (+0) && 1 hour && 1d HP && 1 day && \$1,000 \\
             Major && +0 (-2) && 2 hours && 2d HP && 1 week && \$10,000 \\
             Radical && -3 (-5) && 4 hours && 3d HP && 4 weeks && \$100,000 \\
        \end{tabularx}
        % \caption{Caption}
        \label{tab:surgical_procedures}
    \end{table}
}
\subsection{Biofabrication}\label{subsec:biofabrication_cybernetics}
Some surface implants -- notably skin coatings and dermal armor -- can be grown by immersing the patient in a biofab, which assembles the implant similarly to a 3D printer.

This process requires a Physician roll (modified by the quality of the tank) and takes the specified time on the Surgical Procedures Table. During this time, the patient is unconscious. On a failed Physician roll, the process must be repeated. On a critical failure, something goes gruesomely wrong, resulting in 1d corrosion damage for every 2 hours the process took.

\subsection{Detecting and Removing Cybernetics}\label{subsec:detecting_removing_cybernetics}
A diagnostic bed (p. \pageref{subsec:diagnostic_bed}), medscanner (p. \pageref{subsec:medscanner}), or X~-ray scanner (p. \pageref{itm:x-ray_scanner}) can detect concealed implants on a successful Electronics Operation (Medical) or Diagnosis roll.

Cybernetics can be safely removed in the same fashion they are installed, but the operation is easier: add +1 to Surgery rolls. If the parts don't need to be removed intact, add +2 and halve the time required.

Cybernetics may be rigged to cause unpleasant effects (e.g. see Bomb Implant, p. \pageref{subsec:bomb_implant}) if removal is attempted. A successful Traps-4 roll is required to notice a cyber~-trap before it goes off; roll at no penalty if specifically looking for it. Disarming a booby trap requires an appropriate Traps roll prior to the surgery.

\subsection{Second~-Hand Cybernetics}\label{subsec:second-hand_cybernetics}
Second~-hand parts may be available, usually at 20-70\% the normal cost. This may or may not be a bargain, and there may be damage that is not immediately evident.

Looting cybernetics from bodies results in salvaged cybernetics worth 10-35\% of the original value. The process of salvaging them follows the removal rules above if the victim is living. Salvaging from a corpse is much faster. It takes only one~-third the procedure time and, if paying someone, costs 1/10th as much. A Mechanic (Robotics) skill roll can be substituted for surgery. Failing the roll means the parts require major repairs; critical failure destroys them.

\subsection{Repairing Cybernetics}\label{subsec:repairing_cybernetics}
Use Mechanic (Robotics) skill to repair physical damage or malfunctioning cybernetics, or to diagnose second~-hand parts to see if they have any hidden flaws. Minor damage to bionic body parts can be repaired from the outside, without surgery. For implants and major damage, the part must be completely removed before repairs can take place.

\subsection{Powering Cybernetics}\label{subsec:powering_cybernetics}
Cybernetic devices are generally powered by body heat and motion, though there are some exceptions. Bionic limbs require cell replacement or recharging on a monthly basis (this is part of the maintenance requirement subsumed in their Mitigator limitation)

\subsection{Armoring Cybernetics}\label{subsec:armoring_cybernetics}
As machines of metal, plastic, and ceramic, bionic replacements have inherent DR that can be improved with purpose-built armor. Most implants do not have inherent DR, because of fragility, small size, diffuse nature, or flexibility requirements. Bionic hearts, gill implants, hive implants, and ripsnakes are rigid and compact enough to be armored. Count them as extremities for armoring purposes.

\subsubsection{Rigid Armor}
All rigid cybernetics can be armored beyond the free DR 2 they get as machinery. The normal maximum DR that can be built-in is equal to the crippling threshold (pp. B420-421) of the replaced part in HP ~x 2. For each additional level of Unnatural Features (up to 5), add +1 to the multiplier. Increased rigid armor supersedes the DR that comes with basic bionic limbs.

Rigid armor costs \$50 and weighs 1 lbs. per point of DR for whole-body coverage; reduce cost and weight to the fraction covered for that bionic part.

\subsubsection{Flexible Flesh-Like Armor}
Realistic flesh biomorphics can be made with tougher material -- up to half the maximum DR of equivalent rigid parts (above). Realistic flesh biomorphics have Semi-Ablative and Flexible. Living flesh and synthetic organs can have up to 1/4 DR of equivalent rigid parts; these versions have Ablative and Tough Skin.

Flexible flesh-like armor is a weak form of dermal armor; don't combine them with real dermal armor (see TODO, for the results of more than one layer). This type of armor should be purchased as lightweight partial suits of flexible armor (p. \pageref{subsec:ballistic_armor}).

Bionic limbs \textit{can} have both rigid internal armor and flesh-like external armor. Bionic eyes can only have rigid armor.

\subsection{Acquiring Cybernetics}\label{subsec:acquiring_cybernetics}
At character creation, cybernetics are paid for with Character Points but during play, cybernetics are bought using money and you do not pay the Character Point cost.

You cannot acquire Cybernetics without paying for them, even if you have the Character Points.

\section{Implanted Gadgets}
Many external devices can be implanted into living bodies. Tiny implants are powered by the user's motion, body-heat gradient, or nanomachines consuming blood sugar. Any implanted gadget that uses a B cell or larger probably needs real power cells and an access port to change them or the bio-power tap (p. TODO). Cybernetics located in the torso can have their power-cell port or charging socket located between two ribs and covered by synthetic flesh.

Tiny cybernetics implants (less than 0.05 lb.) can be placed anywhere in the body. Small ones (0.05-0.25 lb.) should be anchored to bones (ribs, a spinal disk, skull, pelvis) to prevent internal lacerations from hard acceleration like falls, slams, or high-G maneuvers. Medium-sized implants (0.25-1 lb.) should be suspended in the body on shock mounting or distributed into multiple pieces around the body. Larger implants (larger than 1 lb.) almost always should be distributed into smaller pieces. Double all devices''' weights to account for the bio-compatible shell and added internal support structures and any access ports for data cables, ammunition, or power cells. Many sensors can be built as distributed arrays in the same manner as ladar smartskin (Ultra-Tech p. 64), taking up negligible internal space, but with added weight.

Most simple implants count as an Accessory perk. However, some may be versatile enough or the user is so familiar with them that they may be better represented as full advantages.
%See Perks vs. Powers (below) for more discussion on these options.

Basic implantation has +2 CF and is a minor procedure that gives the device push-button activation (e.g. pushing on a biomonitor to activate the screen, or tapping on a subdermal printed computer). A device implanted just below or in the skin that doesn't require precise placement is typically -1 CF, and the surgery is one step easier than if it were placed elsewhere. If the device has a reflexive muscle trigger to turn it on and off (like cyber-claws) add +1 CF, or +2 CF if its function can be dialed up and down like a dimmer. If it requires biometric, positional, or physical feedback from the body (like an autoinjector, or the radar skin on p. 00), include +1 CF.

To send to \textit{or} receive from an existing simple sense (like hearing or basic skin pressure) is +4 CF; +8 CF for complex information (like vision or the full sense of touch); or +12 CF for any number of senses. To both send and receive via an existing sense, add half again the CF (an extra +2, +4, or +6). To send non-human sensory information via another human sense (e.g., sensing magnetic field density via simulated tactile pressure) is also half again the CF of the target sense. Interfacing with complex senses is a major procedure and probably requires eye or brain surgery.

To transmit information from the device to the user via surface thoughts is +8 CF, or +18 CF to include full two-way com-
munication. Both versions need major brain surgery procedures.

A device that must use an implant computer or implant radio instead of linking directly to a brain is only +2 CF. However, the implant computer or radio must be acquired separately. 

If the cybernetic system requires modifying most of the body (for example, the skin, all nerves, or all blood vessels) rather than a simple implant, increase the surgery difficulty by one step.

\section{Body Modifications}\label{sec:body_modifications}
These include modifications to the body's limbs and organs, as well as implanted devices. Also see \textit{Pyramid Issue 3/51}, pp. 15-21 for more options and rules for bionics.

\subsection{Biomonitor Implant}\label{subsec:biomonitor_implant}
This implant monitors vital signs: pulse, heartbeat, blood pressure, respiration, brainwaves, blood sugar, and alcohol levels, as well as the overall condition of the user's other cybernetics. It includes a small wrist display, and can connect to a neural interface (p. \pageref{sec:neural_interfaces}) or computer implant (p. \pageref{subsec:computer_implant}). It gives a +2 bonus to any First Aid, Diagnosis or Physician rolls on the cyborg, as long as the medic can see the display. Halve the bonus if the user can see it but has to describe it to the medic. If the medic has a neural interface or a computer, he can jack it into a port beside the visual readout and monitor the cyborg directly.

\textit{Statistics:} Accessory (Biomonitor) [1]. \textit{1 point}.

\textit{Availability:} Simple procedure. \$100. LC4.

\subsubsection{Autoinjector Implant}\label{subsubsec:autoinjector_implant}
A biomonitor autoinjector (above) implanted in the torso. This is not under the user's control. Instead, it triggers based on physiological conditions. The refill/interface device is \$100, 0.5 lbs. LC4.

\textit{Statistics:} Accessory (Autoinjector) [1]; Accessory (Biomonitor) [1]. \textit{2 points.}

\textit{Availability:} Minor procedure. \$400. LC4.

\subsection{Bionic Arm or Hand}\label{subsec:bionic_arm_hand}
This is slightly stronger than the original, but constrained by the limits of the flesh~-and~-bone shoulder it is attached to.

\subsubsection{One Bionic Arm}\label{subsubsec:one_bionic_arm}
\textit{Statistics:} Arm ST+2 (One arm; Temporary Disadvantages, Electrical, -20\%, and Maintenance, 1 person, weekly, -5\%) [5]; DR 2 (One arm, -40\%) [6]; One Arm (Mitigator, -70\%) [-6]. \textit{5 points}.

\textit{Availability:} Major procedure. \$12,000. LC4.

\subsubsection{Two Bionic Arms}\label{subsubsec:two_bionic_arms}
\textit{Statistics:} Arm ST+2 (Both arms; Temporary Disadvantages, Electrical, -20\%, and Maintenance, 1 person, weekly, -5\%) [8]; DR 2 (Arms, -20\%) [8]; No Fine Manipulators (Mitigator, -70\%) [-9]. \textit{7 points}.

\textit{Availability:} Two major procedures. \$24,000. LC4.

If the recipient already has one existing bionic arm, use the Availability entry for one bionic arm.

\subsubsection{Bionic Hand}\label{subsubsec:bionic_hand}
A cybernetic hand and wrist.

\textit{Statistics:} Arm ST+1 (One arm, Temporary Disadvantage, Electrical, -20\%) [3]; DR 2 (One hand, -80\%) [2]; One Hand (Mitigator, -70\%) [-4]. \textit{1 point}.

\textit{Availability:} Major procedure. \$8,000. LC4.

\subsection{Bionic Ears}\label{subsec:bionic_ears}
Crude cybernetic implants to repair damaged or lost hearing were available even in ancient history; these are much more advanced, providing some benefits over natural ears.

\subsubsection{Bionic Ear}\label{subsubsec:bionic_ear}
\textit{Statistics:} Hard of Hearing (Mitigator, -70\%) [-3]. \textit{-3 points}.

\textit{Availability:} Minor procedure. \$500. LC4.

\subsubsection{Bionic Ears}\label{subsubsec:bionic_ears}
\textit{Statistics:} Protected Hearing [5]; Deafness (Mitigator, -70\%) [-6]. \textit{-1 point}.

\textit{Availability:} Two minor procedures. \$1,000. LC4.

\subsubsection{Advanced Bionic Ears}\label{subsubsec:advanced_bionic_ears}
These ears are connected to a computerized sound~-profiling database.

\textit{Statistics:} Discriminatory Hearing (Temporary Disadvantage, Electrical, -20\%) [12]; Protected Hearing [5]; Deafness (Mitigator, -70\%) [-6]. \textit{11 points}.

\textit{Availability:} Two minor procedures. \$5,000. LC4.

\subsection{Bionic Eyes}\label{subsec:bionic_eyes}
The eye is a complex organ, but modern cybernetics are able to replace it with something that works just as well or better than the original. Standard features are roughly equivalent to night vision contact lenses, including a video display option and low~-light and telescopic (2×) optics.

\subsubsection{One Bionic Eye}\label{subsubsec:one_bionic_eye}
\textit{Statistics:} Accessory (Video Display) [1]; Nictitating Membrane 2 (One eye, -50\%) [1]; Night Vision 2 (Temporary Disadvantages, Electrical and No Depth Perception, -35\%) [2]; Telescopic Vision 1 (Temporary Disadvantages, Electrical and No Depth Perception, -35\%) [4]; One Eye (Mitigator, -70\%) [-4]. \textit{4 points}.

\textit{Availability:} Major eye procedure. \$5,000. LC4.

\subsubsection{Two Bionic Eyes}\label{subsubsec:two_bionic_eyes}
\textit{Statistics:} Accessory (Video Display) [1]; Nictitating Membrane 2 [2]; Night Vision 2 (Temporary Disadvantage, Electrical, -20\%) [2]; Protected Vision [5]; Telescopic Vision 1 (Temporary Disadvantage, Electrical, -20\%) [4]; Blindness (Mitigator, -70\%) [-15]. \textit{-1 points}.

\textit{Availability:} Two major eye procedures. \$10,000. LC4.

\subsection{Bionic Leg}\label{subsec:bionic_leg}
A single cybernetic leg is limited by the capabilities of the remaining original leg. A pair of legs are more useful.

\subsubsection{One Bionic Leg}\label{subsubsec:one_bionic_leg}
\textit{Statistics:} DR 3 (One leg, -40\%) [9]; Missing Legs (Mitigator, -70\%) [-6]. \textit{3 points}.

\textit{Availability:} Major procedure. \$8,000. LC4.

\subsubsection{Two Bionic Legs}\label{subsubsec:two_bionics_legs}
\textit{Statistics:} Basic Move +1 (Temporary Disadvantages, Electrical, -20\%, and Maintenance, 1 person, weekly, -5\%) [4]; Super Jump 1 (Temporary Disadvantages, Electrical, -20\%, and Maintenance, 1 person, weekly, -5\%) [8]; DR 3 (Legs, -20\%) [12]; Legless (Mitigator, -70\%) [-9]. \textit{15 points}.

\textit{Availability:} Two major procedures. \$16,000. LC4.

\subsection{Bionic Vital Organs}\label{subsec:bionic_vital_organs}
Complete cybernetic replacement of the heart, lungs, or other vital organs is usually performed only to save a life. This may be combined with additional implants that improve on the original organ.

\subsubsection{Bionic Organ Transplants}\label{subsubsec:bionic_organ_transplants}
One of the most common operations, this extends older medicine with better pacemakers, artificial lungs, etc. The procedure is common, and therefore cheaper than most cybernetics.

\textit{Statistics:} Hard to Kill (Temporary Disadvantage, Electrical, -20\%) +2 [4]; Terminally Ill (Up to one month; Mitigator, -70\%) [-30]. \textit{-26 points}.

\textit{Availability:} Major procedure. \$7,000. LC4.

\subsubsection{Boosted Heart}\label{subsubsec:boosted_heart}
This combination of a cybernetic heart upgrade and arterial reinforcement allows the recipient to temporarily boost his metabolism beyond human norms. It can be added to either a healthy or a bionic heart.

Statistics: Basic Speed +1 (Costs Fatigue 1, -5\%; Temporary Disadvantage, Electrical, -20\%) [15]; Immunity to Heart Attack (Temporary Disadvantage, Electrical, -20\%) [4]. \textit{19 points}.

Availability: Major procedure. \$10,000. LC4.

\subsection{Bionic Voicebox}\label{subsec:bionic_voicebox}
This implant replaces the recipient's voicebox, and may include an artificial tongue if the original was damaged. Someone with a damaged or recovering voicebox can croak or gurgle, but cannot actually speak. These implants can also be used to give animals the power of speech, although at IQ 5 or less, they can only parrot words.

\subsubsection{Cybervoder}\label{subsubsec:cybervoder}
\textit{Statistics:} Cannot Speak (Mitigator, -70\%) [-4]. \textit{-4 points}.

\textit{Availability:} Minor procedure. \$1,000. LC4.

\subsubsection{Silvertongue Implant}\label{subsubsec:silvertongue_implant}
\textit{Statistics:} Cannot Speak (Mitigator, -70\%) [-4]; Voice (Temporary Disadvantage, Electrical, -20\%) [8]. \textit{4 points}.

\textit{Availability:} Minor procedure. \$5,000. LC4.

\subsection{Bio-Power Tap}\label{subsec:bio-power_tap}
An implanted flexible power cell and electric trickle-charger powered by the user's blood sugar using the same technology as gastrobot power supplies (p. \pageref{subsec:power_supply_swarmbot}). Unlike the ones for tiny implants, this produces enough energy to charge larger power cells. However, it requires the user to consume significant quantities of food. A bio-power tap can charge a C cell in one day.

This device powers bionics or implants, including those that would otherwise require removable power cells, such as the gill implant (p. \pageref{subsec:gill_implant})

\textit{Statistics:} Accessory (C Cell) [1]; Internal Create 1 (Electrical Energy; Temporary Disadvantage, Increased Consumption 1, -10\%) [4]. \textit{5 points}.

\textit{Availability:} Minor procedure. \$4,000. LC4.

\subsection{Bomb Implant}\label{subsec:bomb_implant}
This explosive charge is attached to a timed or radio~-triggered detonator and placed in the subject's head or torso. Implanted bombs could be suicide devices under the control of the implantee, or used to insure the loyalty of untrustworthy subordinates. Implanted bombs are often wired into other implants to prevent tampering -- see \textit{Detecting and Removing Cybernetics} (p. \pageref{subsec:detecting_removing_cybernetics}).

An ounce of explosive will inflict 6d+1 crushing damage with the explosive modifier; wounding to the victim is tripled (as per a vital hit) for a torso charge in the vitals, or quadrupled (as per a skull injury) for one buried in the head. An exploding skull inflicts 1d-3 cutting fragmentation damage to anyone nearby.

A nasty variation on the implanted bomb is to place it under the control of a computer implant (p. \pageref{subsec:computer_implant}) which shares the victim's consciousness. Such a system is much harder to fool than a guard with a radio trigger! 

\textit{Statistics:} An implanted bomb may qualify as an Involuntary Duty.

\textit{Availability:} Simple procedure. Use the cost and LC of a smart grenade (p. \pageref{subsec:smart_grenades}); a 25mm or 40mm can fit in the torso, a 15mm in a limb, 10mm elsewhere in the body.

\subsection{Boosted Reflexes}\label{subsec:boosted_reflexes}
These implanted glands release chemicals on mental command, triggering a controlled adrenaline~-like response. 

\textit{Statistics:} Basic Speed +1 (Costs Fatigue, 2 FP, -10\%) [18]. \textit{18 points.}

\textit{Availability:} Minor procedure. \$9,000. LC3.

\subsection{Cyber Claws}\label{subsec:cyber_claws}
The recipient's hands or feet are equipped with ceramic or metal claws. The claws are retractable, triggered by muscle contractions.

\textit{Statistics:} Sharp Claws (Switchable, +10\%) [6]. 

\textit{Availability:} Minor procedure. \$6,000. LC3.

Halve the cost and treat as a simple procedure if adding these to a bionic hand or arm.

\subsection{Filter Implant}\label{subsec:filter_implant}
A self~-regenerating particle~-filtration system integrated into the recipient's lungs.

\textit{Statistics:} Filter Lungs [5].

\textit{Availability:} Minor procedure. \$2,500. LC4.

\subsection{Flesh Pocket}\label{subsec:flesh_pocket}
This is a surgically implanted pocket or pouch, sealed by a flap of skin. It can be used to smuggle small objects. A flesh pocket is normally installed in the torso; up to five levels are possible. Each level allows the pocket to hold up to Basic Lift/10 lbs. If placed elsewhere, a maximum of one level can be installed, and the amount of weight that can be carried is divided by 4 (leg), 8 (arm), or 16 (head or neck).

\textit{Statistics:} Payload 1-5 [1/Level]. \textit{1-5 points}.

\textit{Availability:} Simple procedure. \$200 per level. LC3.

\subsection{Gyrobalance}\label{subsec:gyrobalance}
This is a miniature electronic gyroscope implanted in the inner ears (both ears -- but treat as a single operation), and interfaced to improve the recipient's sense of balance.

\textit{Statistics:} Klutz (Mitigator, -70\%) [-2]; Perfect Balance (Temporary Disadvantage, Electrical, -20\%) [12]. \textit{10 points}.

\textit{Availability:} Minor procedure. \$7,000. LC4.

\subsection{Hidden Compartments}\label{subsec:hidden_compartments}
A cybernetic arm or leg may have a compartment large enough for any small object of up to Basic Lift/10 lbs. weight.

\textit{Statistics:} Payload 1 [1].

\textit{Availability:} Simple procedure. \$500. No operation required if purchased with the limb. LC4.

\subsection{Implant Radio}\label{subsec:implant_radio}
This ``implant communicator'' is a radio (p. \pageref{subsec:radio_comms}) with a range of one mile. It is spliced into the recipient's auditory nerve; the user may speak normally or subvocalize. A character with an implant radio can use it to subscribe to a cell phone or net service provider.

\textit{Statistics:} Radio (Reduced Range, 1/10, -30\%; Secure, +20\%; Temporary Disadvantage, Electrical, -20\%) [7]. \textit{7 points}.

\textit{Availability:} Simple procedure. \$100. LC4.

\subsection{Implant Video Comm}\label{subsec:implant_video_comm}
This implanted radio communicator (p. \pageref{subsec:radio_comms}) is spliced into the recipient's optic nerves to provide a video display. It has a range of one mile, and can be used to subscribe to a cell phone or net service provider.

\textit{Statistics:} Radio (Reduced Range, ×1/10, -30\%; Temporary Disadvantage, Electrical, -20\%; Video, +40\%) [9]. \textit{9 points}.

\textit{Availability:} Simple procedure. \$200. LC4.

\subsection{Memory Flesh}\label{subsec:memory_flesh}
These synthetic flesh implants allow the recipient to shift between two different sets of facial and bodily features: his own and another set specified when it is installed.

\textit{Statistics:} Alternate Form (Cosmetic, -50\%, Temporary Disadvantage, Electrical, -20\%) [5]. \textit{5 points}.

\textit{Availability:} Major procedure. \$20,000. LC3.

If the subject already has bioplastic skin (p. \pageref{subsec:bioplastic_skin}), this is a minor procedure.

\subsection{Radar Skin}\label{subsec:radar_skin}
An array of subcutaneous radar emitters and receivers distributed throughout the body. Built as three small multi-mode radar arrays (for 360\degree coverage; p. \pageref{subsec:multi-mode_radar}) implanted (+2 CF) throughout the body with positional feedback (+1 CF and difficult surgery). It also has two-way mental communication (+18 CF) and a one-way full sensory output (+12 CF) which overlays its map in the visual field with hearing and touch as secondary sub-channels. Bearing and range to target comes as subconscious knowledge thanks to the two-way mind interface.

\textit{Statistics:} Radar (Extended Arc, 360\degree, +125\%; Low-Probability Intercept, +10\%; Maximum Duration, 8 hours, -5\%; Multi-Mode, +50\%; Targeting, +20\%; Temporary Disadvantage, Electrical, -20\%) [56]. \textit{56 points}.

\textit{Availability:} Major brain procedure. \$102,000, 3B/8 hrs. LC3.

\subsection{Subdermal Armor}\label{subsec:subdermal_armor}
This flexible armor is implanted under the skin. The operation uses nanotechnology to grow the armor under the skin. A careful tactile examination, a diagnostic bed, or a medscanner can detect the armor, but it is invisible to the naked eye and does not appear on metal detectors. It provides DR 12 vs. piercing and cutting damage and DR 4 vs. other damage.

\textit{Statistics:} DR 8 (Limited, Piercing and Cutting, -20\%; Tough Skin, -40\%) [16]; DR 4 (Tough Skin, -40\%) [12]. \textit{28 points}.

\textit{Availability:} Major procedure. \$2,000. LC2.

\subsection{Smart Tattoos}\label{subsec:smart_tattoos}
These tattoos are made with video ink. They can follow preprogrammed scripts, or even act in response to changes their sensors detect in the skin (sweat, temperature, etc.). A tiger tattoo might roar when it detects anger, or purr when the wearer is aroused.

\textit{Statistics:} Distinctive Features 1 (Switchable, -10\%) [0]. \textit{0 points}.

\textit{Availability:} Simple procedure. \$200. LC4.

\subsection{Stinger}\label{subsec:stinger}
This concealed implant houses a single disposable hypo (p. \pageref{subsec:disposable_hypo}) sheathed under a fingernail or in a body cavity. The recipient has no ability to manufacture drugs or toxins; he must buy hypos loaded with injectable drugs, poisons, or metabolic weapons. It takes 10 seconds to remove and replace a hypo in the mount.

A fingernail~-mounted stinger attacks just like a jab with a disposable hypo. It has reach C, does 1 HP damage for penetration purposes, but with no wounding, and delivers a follow~-up attack based on whatever agent was loaded into it. If the user has claws rather than normal fingernails, the injection can be a follow~-up attack to the claw's damage. A body~-cavity stinger is mostly useful to deliver a surprise attack during an intimate moment; a Touch-8 sense roll can notice the tiny mount before it can be used. A stinger in the mouth can also be a follow~-up attack to a bite.

\textit{Statistics:} Extra Arm (Switchable, +10\%; Takes Recharge, -10\%; Weapon Mount, -80\%) [2]. \textit{2 points}.

\textit{Availability:} Minor procedure. \$500 (hypos not included). LC3.

\subsection{Weapons Mounts}\label{subsec:weapon_mounts_cyber}
These are modular weapons installations attached to a cyborg's body. Each can mount a single weapon that weighs no more than the recipient's Basic Lift. Mounted weapons cost cash -- not points -- and their weight counts as encumbrance.

A mounted weapon is plugged~-in, not built~-in. The user can swap it for another weapon with a suitable interface. It takes five seconds to mount or remove a weapon.

Concealing a weapon mount is similar to hiding a firearm of similar bulk. A Bulk -5 rifle built into one's arm will have a protruding muzzle, while a Bulk -1 holdout pistol may have only a tiny gun~-port built into the user's palm.

All mounted weapons can be detected by searches. Enemies or the authorities can unplug and confiscate them, just like carried weapons.

\subsubsection{Bionic Arm or Hand Mount}\label{subsec:bionic_arm_hand_mount}
This is a weapon mount built into an existing bionic arm or hand (p. \pageref{subsec:bionic_arm_hand}). It may be mounted above or below the arm, or fire out through the palm. The weapon's weight may not exceed Basic Lift if in an arm, or half of Basic Lift if in a hand.

\textit{Statistics:} Extra Arm (Weapon Mount, -80\%) [2].

\textit{Availability:} Minor procedure. \$100/lb. of weapon weight. LC of weapon.

\subsubsection{Heavy Weapon Arm}\label{subsec:heavy_weapon_arm}
This weapon mount replaces the user's arm with a socket joint and a hardpoint for attaching a weapon, usually rifle~-sized. The mounted weapon must weigh equal to or less than the user's Basic Lift.

\textit{Statistics:} To replace one arm with a weapon mount, take Extra Arm (Weapon Mount, -80\%) [2] and One Arm [-20]. \textit{-18 points}.

\textit{Availability:} Minor procedure. \$100/lb. of weapon weight. LC of weapon.

\subsection{Accelerated Reflexes}\label{subsec:accelerated_reflexes}
A system of electronic nerves and computer hardware that replaces large sections of the nervous system.

\textit{Statistics:} Extra Attack 1 (Temporary Disadvantage, Electrical, -20\%) [20]. \textit{20 points}.

\textit{Availability:} Radical procedure. \$50,000. LC2.

\subsection{Bioplastic Skin}\label{subsec:bioplastic_skin}
This modification covers the recipient's body with a sheath of living smart bioplastic (p. \pageref{subsec:smart_bioplastic}). It is thin and sensitive enough that it looks and behaves like normal skin, and even heals itself. It is invisible armor covering the entire body. It has DR 15 vs. burning or piercing damage, DR 5 vs. other types of damage.

\textit{Statistics:} DR 10 (Limited, Burning and Piercing damage, -20\%; Tough Skin, -40\%) [20]; DR 5 (Tough Skin, -40\%) [15]. \textit{35 points}.

\textit{Availability:} Major procedure. \$20,000. LC3.

If installed first, bioplastic skin reduces the cost and difficulty of certain other cybernetic skin modifications.

\subsection{Cyberhair}\label{subsec:cyberhair}
This implant replaces sections of ordinary hair with thin cybernetic tendrils attached to a reinforced scalp. Cyberhair does not grow and cannot be cut by ordinary razors or scissors, but it can coil close to the scalp when the recipient needs a ``haircut.''

Cyberhair can be used as a simple manipulator, which may be useful if the user is grappled or tied up. It must be at least shoulder~-length to be effective.

\subsubsection{Shoulder~-Length Cyberhair}\label{subsubsec:shoulder-length_cyberhair}
\textit{Statistics:} Extra Arm 1 (Extra~-Flexible, +50\%; Short, -50\%; Temporary Disadvantages, Electrical, -20\%, and Maintenance, 1 person, weekly, -5\%; Weak, 1/4 body ST, -50\%) [3]. \textit{3 points}.

\textit{Availability:} Major procedure. \$3,000. LC4.

\subsubsection{Waist~-Length Cyberhair}\label{subsubsec:waist-length_cyberhair}
\textit{Statistics:} Extra Arm 1 (Extra~-Flexible, +50\%; Temporary Disadvantages, Electrical, -20\%, and Maintenance, 1 person, weekly, -5\%; Weak, 1/4 body ST, -50\%) [8]. \textit{8 points}.

\textit{Availability:} Major procedure. \$4,000. LC3.

\subsubsection{Knee~-Length Cyberhair}\label{subsubsec:knee-length_cyberhair}
Longer and tougher, with a reinforced scalp to support the hair's capabilities.

\textit{Statistics:} Extra Arm 1 (Extra~-Flexible, +50\%; Long, +100\%; Temporary Disadvantages, Electrical, -20\%, and Maintenance, 1 person, weekly, -5\%; Weak, 1/2 body ST, -25\%) [20]. \textit{20 points}.

\textit{Availability:} Major procedure. \$10,000. LC3.

\subsection{Variskin}\label{subsec:variskin}
The recipient's skin is replaced or coated with smart film. He can change its color and texture to blend in with the surroundings. If nude, he gets +2 to Stealth skill when perfectly still, or +1 if moving. Clothing reduces this to +1 when perfectly still. It takes one second to alter skin pigment, and unnatural colors such as green or chrome are possible. The skin can also function as a video display terminal for data run through neural interface or computer implant.

\textit{Statistics:} Accessory (Video terminal) [1]; Chameleon 1 (Controllable, +20\%) [6]. \textit{7 points}.

\textit{Availability:} Minor procedure. \$1,000. LC2.

If the patient has bioplastic skin (p. \pageref{subsec:bioplastic_skin}) the procedure is simple.

\subsection{Gill Implant}\label{subsec:gill_implant}
This implant allows the recipient to breathe underwater, using a device that extracts oxygen from water (it's not a true set of fish gills). It uses two C cells per day of operation. Maintenance involves opening an access panel in his chest or back, cleaning filters, and installing new power cells.

\textit{Statistics:} Doesn't Breathe (Gills, -50\%; Temporary Disadvantages, Maintenance, 1 hour, daily, -10\%, and Electrical, -20\%) [4]. \textit{4 points}.

\textit{Availability:} Major procedure. \$8,000. LC4.

\subsection{Hive Implant}\label{subsec:hive_implant}
This implanted swarmbot hive (p. \pageref{subsec:swarmbot_hive}) can carry a single swarm measuring one square yard. It includes recharging ports hidden by a skin flap, allowing the swarm to recharge from a power system or the included C cell. The swarm and a control system (such as an implant computer and advanced com implant) must be acquired separately.

\textit{Statistics:} Accessory (Swarmbot Hive) [1]; Payload 1 [1]. \textit{2 points}.

\textit{Availability:} Minor procedure. \$1,000. LC4.

\subsection{Intestinal Recycler}\label{subsec:intestinal_recycler}
The human digestive system is imperfect, so waste matter always contains useful chemicals that could have been metabolized and used by the body. This implant collects waste matter and reprocesses it.

\textit{Statistics:} Reduced Consumption 2 [4]. \textit{4 points}.

\textit{Availability:} Major procedure. \$4,000. LC4.

\subsection{Nanoweave Subdermal Armor}\label{subsec:nanoweave_subdermal_armor}
Advanced flexible armor implanted under the patient's skin. It has DR 18 vs. piercing and cutting damage and DR 6 vs. other damage.

\textit{Statistics:} DR 12 (Limited, Piercing and Cutting, -20\%; Tough Skin, -40\%) [24]; DR 6 (Tough Skin, -40\%) [18]. \textit{42 points}.

\textit{Availability:} Major procedure. \$5,000. LC2.

\subsection{Polyskin}\label{subsec:polyskin}
A combination of micromachines, smart bioplastic implants, and artificial glands that allow the recipient to alter his appearance. He can adjust apparent weight, skin color and tone, and facial structure.

The system can be purchased for the face or for the entire body. It can also be combined with sexmorph (p. \pageref{subsec:sexmorph}).

If the recipient already has bioplastic skin (p. \pageref{subsec:bioplastic_skin}) divide the dollar cost of the implant and the recovery time by 2.

\subsubsection{Polyskin Body}\label{subsubsec:polyskin_body}
\textit{Statistics:} Elastic Skin (Temporary Disadvantage, Electrical, -20\%) [16]. \textit{16 points}.

\textit{Availability:} Radical procedure. \$36,000. LC2.

This is only a major procedure if the recipient already has bioplastic or living metal skin.

\subsubsection{Polyskin Face}\label{subsubsec:polyskin_face}
\textit{Statistics:} Elastic Skin (Face only, -25\%; Temporary Disadvantage, Electrical, -20\%) [11]. \textit{11 points}.

\textit{Availability:} Major procedure. \$15,000. LC2.

\subsection{Reinforced Skeleton}\label{subsec:reinforced_skeleton}
Micro~- or nanomachines can reinforce a patient's skeleton with carbon fibers, transforming his bones into structures with the strength of metal. Implants take over the function of bone marrow and produce blood cells. Weight does not increase. While the reinforced bones do not show up on metal detectors, they can be identified with X~-rays, diagnostic beds, and other advanced sensors.

\textit{Statistics:} HP+5 [10]; DR 20 (Skull only, -70\%) [30]; DR 10 (Limited, Crushing, -40\%; Tough Skin, -40\%) [10]. \textit{50 points}.

\textit{Availability:} Radical procedure. \$50,000. LC3.

\subsection{Ripsnake}\label{subsec:ripsnake}
This cybernetic assassin's weapon is a concealed bionic limb linked to the user's nervous system. It uncoils from a natural body opening (usually the mouth) when deployed, and can attack semi~-autonomously.

In certain situations, a ripsnake can deliver an automatically successful attack to the vital organs. For example, if a would~-be assassin with a ripsnake concealed in his mouth kisses someone, it can coil out and down his victim's throat.

\textit{Statistics:} Extra Attack 1 (Ripsnake Only, -20\%) [20]; Impaling Striker (Cannot Parry, -40\%; Long, +1 SM, +100\%; Temporary Disadvantage, Electrical, -20\%) [12]. \textit{32 points}.

\textit{Availability:} Major procedure. \$26,000. LC2.

\subsection{Sexmorph}\label{subsec:sexmorph}
This suite of sphincter valves, synthetic hormone glands and memory or bioplastic implants allows the recipient to switch gender in 10 seconds. If desired, a user can also adopt a neuter phase (no obvious genitalia or breasts) or mixed phase (male genitalia, female breasts, or vice versa) with voice and features as desired.

\textit{Statistics:} Hermaphromorph (Temporary Disadvantage,
Electrical, -20\%; transsexual form also, +20\%) [5]. \textit{5 points}. 

\textit{Availability:} Major procedure. \$10,000, LC3.

\subsection{Slickskin}\label{subsec:slickskin}
The recipient's skin is covered with a switchable smart matter nanofilm. When activated, most of his skin becomes virtually frictionless. The palms of the hands and the soles of the feet are not affected.

\textit{Statistics:} Slippery 3 [6]. \textit{6 points}.

\textit{Availability:} Major procedure. \$12,000. LC3.

\subsection{Stickskin}\label{subsec:stickskin}
The recipient's palms and soles are covered with switchable gecko adhesive skin (p. \pageref{subsubsec:gecko_adhesive}). When activated, the user can adhere to any solid surface. They can be combined with slickskin at no penalty, since they cover nonoverlapping areas.

\textit{Statistics:} Clinging (Temporary Disadvantage, Electrical, -20\%) [16]. \textit{16 points}.

\textit{Availability:} Major procedure. \$16,000. LC3.

This is only a minor procedure and half cost if the recipient
already has bioplastic skin.

\subsection{Swimskin}\label{subsec:swimskin}
The recipient's skin is covered with a biomemetic swim surface and a permanent water-repellant coating (see \textit{Waterproof Clothing}, p. \pageref{subsubsec:waterproof_clothing}). The user will have to use sonic showers (p. \pageref{subsec:sonic_shower}) or wash with non-water-based solvents. The microscopic layer of trapped air reduces water drag, increasing Water Move when unclothed. Provides at most Slippery 3 when combined with Slickskin.

\textit{Statistics:} Water Move +2 [10]; Resistant to Contact Agents (+3) [3]; Slippery 1 [2]. \textit{15 points}.

\textit{Availability:} Major procedure. \$25,000. LC3.

This is only a minor procedure and half cost if the recipient already has bioplastic skin.

\subsection{Thermal Imaging Eyes}\label{subsec:thermal_imaging_eyes}
This is a pair of bionic eyes with tiny infrared imaging cameras, day/night telescopic optics, and a HUD (p. \pageref{itm:terminal_hud}) chipped into the optic nerves.

\textit{Statistics:} Accessory (HUD) [1]; Infravision (Temporary Disadvantage, Electrical, -20\%) [8]; Nictitating Membrane 2 [2]; Telescopic Vision 1 (Temporary Disadvantage, Electrical, -20\%) [4]; Blindness (Mitigator, -70\%) [-15]. \textit{0 points}.

\textit{Availability:} Two major procedures (minor if replacing existing bionic eyes). \$8,000. LC4.

\subsection{Wonder Glands}\label{subsec:wonder_glands}
This upgraded version of the encapsulated cell implant (\textbf{GURPS Bio-Tech}, p. 120) includes a specialized, programmable biomonitor that releases a dose when certain conditions are met. Any drug (see \textit{Drugs and Nano}, \pageref{sec:drugs_nano} and \textit{Psi-Drugs and Nanodrugs}, \pageref{ch:psi-drugs_and_nanodrugs}) may be used; the usual choice is Ascepaline or Hyperstim, triggered if the subject is badly injured (HP 0 or worse) or unconscious.

Wonder glands typically dispense drugs that must be administered immediately, such as combat enhancers or emergency healing drugs, although some people use them for lifestyle chemicals. This implant can also be used to refill the drug and poison reservoirs of stingers (p. \pageref{subsec:stinger}) and similar weapons.

\textit{Statistics:} Any advantages or disadvantages granted, modified to fit the dosage requirements and triggering criteria. Aftermath, Limited Use, Takes Recharge, and Trigger are all appropriate modifiers.

\textit{Availability:} Simple procedure. \$500 plus a microbe culture worth the cost of a single dose at LC4, multiplied by 10 for every LC lower. LC is that of the drug released.

\section{Brain Implants}\label{sec:brain_implants}
Brain implants are inserted into the recipient's skull and linked to his central nervous system. Some societies may see brain implants as sinister. Others may consider altering the mind to be more socially acceptable than modifying the body.

Brain implants are riskier than other implants. Critical failure on major or radical procedures may cause brain injury, resulting in a loss of one point of IQ, or a disadvantage like Epilepsy or Phantom Voices.

\subsection{Braintap}\label{subsec:braintap}
A braintap is an advanced form of implant communicator that lets the recipient transmit his experiences as sensies (p. \pageref{sec:sensies}). Others can use the receiver to experience or record the sensory information experienced by the braintapped character.

Braintaps are used by sensie stars, journalists, or anyone else who wants to record his personal experiences. A normal braintap can be turned on or off by the user, but remote~-controlled braintaps are also possible. It is possible to implant a braintap in an animal. If well~-trained, such animals make very useful scouts or familiars. A braintap can also be implanted without someone's knowledge, during other surgery -- a favorite trick of intelligence agencies.

\subsubsection{Braintap Jack}\label{subsubsec:braintap_jack}
This incorporates a plug~-in cable jack (p. \pageref{itm:cable_jack}) plus a sensie transmission module. If the user has a computer implant (below), he can store data in it.

\textit{Statistics:} Cable Jack (Send Only, -50\%; Sensie, +80\%) [7].

\textit{Availability:} Minor procedure. \$3,000. LC4.

\subsubsection{Wireless Braintap}\label{subsubsec:wireless_braintap}
This incorporates a radio transmitter with a one~-mile range and a sensie~-transmission module. If the user has a computer implant, he can store data in it.

\textit{Statistics:} Radio (Secure, +20\%; Reduced Range, ×1/10, -30\%; Send Only, -50\%; Sensie, +80\%) [12].

\textit{Availability:} Minor procedure. \$12,000. LC3.

\subsection{Computer Implant}\label{subsec:computer_implant}
This is a computer implanted in the recipient's head and controlled through its own direct neural interface. It takes only a thought to call up a file or access a database. Data scrolls across the periphery of the user's vision, and he hears the computer as a voice in his head.

The implant includes an optical~-recognition feature using the user's eyes and ears as sensors. It can speed~-read documents, for example, and store them in its database. (It takes normal time to read them later, though one could ask the computer to provide a synopsis).

The interface can also run ``virtual tutor'' (p. \pageref{sec:virtual_tutor}) augmented reality programs that not only talk but show how to do things by overlaying instructions on the user's visual field. Although several programs could theoretically be run at once, the user can only focus on one at a time. A computer implant is most useful with an implanted communicator.

The recovery time represents the amount of time needed to master the computer's functions.

\subsubsection{Computer Implant}\label{subsubsec:computer_implant}
The user should also have a neural jack (p. \pageref{subsubsec:neural_jack}) or implant radio (p. \pageref{subsec:implant_radio}).

\textit{Statistics:} Accessory (Tiny computer) [1]; Photographic Memory (Temporary Disadvantage, Electrical, -20\%; Recorded data only, -20\%) [6]. \textit{7 points}.

\textit{Availability:} Minor procedure. \$4,000 + the cost of a tiny computer (p. \pageref{itm:tiny_com}) with the compact option. LC4.

\subsection{Chip Slots}\label{subsec:chip_slots}
A chip slot is a sterile dime~-sized skull socket, covered with a cap, into which modular brain implants (``chips'') can be inserted. Each chip interfaces directly with the recipient's brain and nervous system, providing knowledge and ability. Chips containing individual skills, techniques, or mental advantages can be manufactured and purchased.

\textit{Chips:}\label{itm:chip_slots_chips} Chips themselves are tiny plugs that weigh 0.05 lbs. (just under an ounce). Skill chips cost \$500 per point. Most are LC3.

\textit{Statistics:} Variable. Buy the Cip Slots advantage (p. B71) with the limitation (Temporary Disadvantage, Electrical, -20\%). The maximum number of slots is 3, and each chip can be worth no more than 15 points.

Availability: Minor procedure. Cost is \$5,000 per slot plus \$3,000 per point of abilities that a chip can hold. LC3.

\subsubsection{Skip Slot}\label{subsec:skip_slot}
This is a general~-purpose chip slot optimized for skills. Running a chip with 4-points in skills, it can give someone an easy skill at attribute+2, an average skill at attribute+1, a hard skill at attribute+0, or a very hard skill at attribute-1. It can also add +1 to any existing skill level.

\textit{Statistics:} Chip Slots 1 (4) (Temporary Disadvantage, Electrical, -20\%) [14].

\textit{Availability:} Minor procedure. \$17,000. LC3.

\subsection{Computer Implant Template}\label{subsec:computer_implant_template}
\hfill \large{\textbf{-17 Points}}\normalsize

An implanted computer can be a PC or associated NPC! This is a built~-in computer implant with a mind of its own. The template's Mindlink is with the person it is implanted into. If it's a PC, the person it is implanted into is usually an Ally or Dependent.

Combine this with any AI template (p. \pageref{subsec:ai}) or the Mind Emulation template (p. \pageref{subsec:mind_emulation}).

\begin{hangparas}{1em}{1}
    \textbf{Attribute Modifiers:} ST 0 [-100]; HT+4 [40].
    
    \textbf{Secondary Characteristic Modifiers:} HP +2 [4]; Basic Move -6 [-30].
    
    \textbf{Advantages:} Absolute Direction (Requires signal, -20\%) [4]; AI [32]; Doesn't Breathe [20]; Doesn't Eat or Drink [10]; DR 5 (Can't Wear Armor, -40\%) [15]; Injury Tolerance (No Eyes, No Head, No Neck) [17]; Machine [25]; Mindlink [5]; Mind Reading (Mindlink Required, -40\%*; Sensory Only, -20\%; Touch~-Based, -20\%) [6]; Sealed [15]; Radio (Burst, +30\%; Secure, +20\%; Video, +40\%) [19].
    
    \textbf{Perks:} Accessories (Tiny computer) [1].
    
    \textbf{Disadvantages:} Electrical [-20]; Quadriplegic [-80].
    
    \textbf{Features:} Taboo Trait (Fixed ST, DX, HT, HP).
    
    \textbf{Availability:} Minor procedure. \$4,000 plus the cost of a tiny computer (p. \pageref{itm:tiny_com}) and AI software (p. \pageref{sec:machine_intelligence_lenses}) or a mind emulation (p. \pageref{subsec:mind_emulation}) program. LC4.
    
    * Works the same way as the identical limitation for Possession (p. B76).
\end{hangparas}
\subsubsection{Lenses}\label{subsubsec:lenses_computer_implant_template}

\begin{hangparas}{1em}{1}
    
\textit{Puppeteer} (+100 points). The computer can use biopresence software to possess the body of its host. This can also be combined with any of the above sub~-race options. Add Possession (Mindlink Required, -40\%; No Memory Access, -10\%; Telecontrol, +50\%) [100]. If the host has a biopresence implant so that possession is automatic, add Puppet [5], increasing the cost to +105 points. Add the cost of biopresence software (p. \pageref{subsec:biopresence_software}).

\textit{Mind Interface} (+6 points). The computer implant can sense the surface thoughts of its host. Remove the Sensory Only limitation on Mind Reading. This can be combined with Puppeteer. +\$10,000. LC3.
\end{hangparas}\medskip

\subsection{Neural Interface Implant}\label{subsec:neural_interface_implant}
A neural interface (p. \pageref{sec:neural_interfaces}) permits the user to control electronic devices using his mind. It picks up electronic impulses and translates them into electrochemical signals in his brain. There are two models in common use, and some people may implant both:

\subsubsection{Neural Jack}\label{subsubsec:neural_jack}
This is a socket implanted in the body (usually the back of the neck, base of the spine, or skull) with a communications interface. The user can plug an optical cable (p. \pageref{itm:optical_cable}) into it and connect to a phone line, modem, etc.

\textit{Statistics:} Cable Jack (Sensie, +80\%) [9]. \textit{9 points}.

\textit{Accessibility:} Minor procedure. \$4,000. LC3.

\subsubsection{Wireless Neural Interface}\label{subsubsec:wireless_neural_interface}
This is a wireless neural interface radio with a one~-mile range. It can also function as a radio communicator. 

\textit{Statistics:} Radio (Reduced Range, ×1/10, -30\%; Secure, +20\%; Sensie, +80\%) [17]. \textit{17 points}.

\textit{Accessibility:} Minor procedure. \$5,000. LC3.

\subsection{Neurotherapy Implant}\label{subsec:neurotherapy_implant}
Computer chips may be surgically implanted into the brain to restore misbehaving or damaged functions, or to act as a bridge between injured and healthy areas.

A neurotherapy implant can be implanted to neutralize mental or physical disadvantages that impair brain or neurological function, such as Dyslexia, Epilepsy, Killjoy, Non~-Iconographic, Neurological Disorder, and Short Attention Span.

If brain damage such as a stroke or bungled brain surgery causes DX or IQ loss or other disadvantages (e.g., Blindness or Mute, or partial paralysis resulting in a disadvantage such as One Arm), the GM may also allow the implant to fix it.

\textit{Statistics:} Add the Mitigator (-70\%) limitation for the disadvantage.

\textit{Availability:} A persona map of the patient is required before a neurotherapy implant can be installed; see \textit{Brainscanner} (p. \pageref{subsec:brainscanner}). This data is used to program the implant. Minor procedure. \$500 per -1 point of disadvantage. LC3.

\subsection{Psych Implant}\label{subsec:psych_implant}
This implant stimulates areas of the brain to produce psychological reactions. Moderate regimes use them as an alternative to prison or psychiatric treatment -- repressive ones rely on them for mind control.

A psych implant gives the subject an additional mental disadvantage. Common implants induce Combat Paralysis, Gullibility, Pacifism, or Slave Mentality, and are used to restrain violent individuals or render the subject easily controllable. Illegal implants are available that compel Berserk, Dyslexia, Paranoia, or a Phobia. An implant can not create self~-imposed mental disadvantages such as Code of Honor.

The disadvantage is not active until after the recovery period, although the subject will feel a growing urge to act in the fashion indicated. Any implant~-induced disadvantage ends when the implant is removed. However, anyone who has worn a psych implant for three or more months may acquire the disadvantage permanently. After the implant is removed, the implantee should make a Will roll at +4 to avoid the disadvantage continuing, with a penalty of -1 for each doubling of time, e.g., Will+3 at six months, Will+2 after a year, Will+1 after two years, etc.

Therapeutic implants also exist which negate mental disadvantages, such as Bad Temper or Phobias; use the rules for Neurotherapy Implants. After several months the effect may become permanent. Roll vs. Will as above when the implant is removed -- if the roll fails, the disadvantage is gone. The GM may require it to be bought off with character points.

\textit{Statistics:} A disadvantage granted by a psych implant will have the (Temporary Disadvantage, Electrical, -20\%) limitation. A disadvantage that is negated by a psych implant will have the (Mitigator, -70\%) limitation.

\textit{Availability:} A persona map of the patient is required; see \textit{Brainscanner} (p. \pageref{subsec:brainscanner}). This data is used to program the implant. Minor procedure. \$1,000 per -1 point of disadvantage added or mitigated. LC3.

\subsection{Biological Operating System (BOS) Implant}\label{subsec:biological_operating_system_implant}\label{subsec:BOS_implant}
This implant controls biofeedback systems and diagnostic monitors, as well as nanomachine drug factories that that help the user manage his body's physiological state.

\textit{Statistics:} Alcohol Tolerance [1], Deep Sleeper [1], Metabolism Control 1 [5], No Hangover [1]. \textit{8 points}.

\textit{Availability:} Major procedure. \$10,000. LC3.

\subsection{Sensie Transceiver Implant}\label{subsec:sensie_transceiver_implant}
These brain implants enable a person to transmit or receive live or recorded sensory impressions from another person. They are essentially two~-way braintaps.

\subsubsection{Sensie Transceiver Jack}\label{subsubsec:sensie_transceiver_jack}
This incorporates a plug~-in cable jack (p. \pageref{itm:cable_jack}) plus a sensie transceiver module. If the user has a computer implant (p. \pageref{subsec:computer_implant}), he can store data in it.

\textit{Statistics:} Cable Jack (Sensie Only, +0\%) [5].

\textit{Availability:} Minor procedure. \$2,500. LC4.

\subsubsection{Wireless Sensie Transceiver}\label{subsubsec:wireless_sensie_transceiver}
This incorporates a radio transmitter with a one~-mile range and a sensie~-transmission module. If the user has a computer implant, he can store data in it.

\textit{Statistics:} Radio (Secure, +20\%; Reduced Range, ×1/10, -30\%; Sensie Only, +0\%) [9].

\textit{Availability:} Minor procedure. \$4,500. A/1 year. LC3.

\subsection{Cognitive Enhancement}\label{subsec:cognitive_enhancement}
This implant establishes new connections between itself and different parts of the brain. Normal neurons are replaced with cybernetic duplicates. The benefits provided are capabilities at which electronic computers exceed the capabilities of human brainpower, such as spatial awareness, memory and processing speed.

\textit{Statistics:} Choose from IQ+1 to IQ+3 [20/level], 3D Spatial Sense [10], Eidetic Memory [5], Enhanced Time Sense [45], Intuition [15], Language Talent [10], Lightning Calculator [2 or 5], Oracle [15], Mathematical Ability [10/level], Musical Ability [10/level], Single~-Minded [5], Visualization [10]. The maximum points available per operation are 15. Then add Temporary Disadvantage (Electrical, -20\%).

\textit{Availability:} Major procedure. \$5,000 × point cost (before applying Temporary Disadvantage). LC3.

\subsection{Puppet Implant}\label{subsec:puppet_implant}
A puppet implant allows someone else to remotely control a cyborg's body. The teleoperator must have appropriate software, hardware, and access -- see \textit{Biopresence Software}, p. \pageref{subsec:biopresence_software}. A puppet implant requires a communications implant or sensie implant.

\textit{Statistics:} This may count as the Involuntary form of Duty (p. B133) if someone else holds the access codes.

\textit{Availability:} Major procedure. \$45,000. LC2.

\subsubsection{Personality Implant}\label{subsubsec:personality_implant}
The combination of a puppet implant and a computer implant (p. \pageref{subsec:computer_implant}) housing a digital mind is also called a personality implant. It can take possession of the cyborg. Personality implants might be used for coercive purposes -- for example, a cult leader might implant them in his followers. They could also be voluntary, with people owning and accessing personalities that are programmed to obey them, or storing the personas of deceased friends, lovers, or ancestors in their heads.

\section{Cybernetic Uplift}\label{sec:cybernetic_uplift}
These modifications are normally added to pets or working animals in order to give them additional capabilities.

\subsection{Enhanced Voicebox}\label{subsec:enhanced_voicebox}
This implant gives an animal that can't speak the ability to form human words, much like a parrot can. It can be added to any mouse~-sized or larger animal.

The animal's Cannot Speak disadvantage is nullified as long as the implant is functional. Giving a non~-sapient (IQ 5 or less) animal a voicebox does not mean that it can actually learn a language, but it can be taught to speak a few words.

\subsection{Finger Paws}\label{subsec:finger_paws}
Finger paws can be added to a bionic or organic leg on an animal with walking paws, such as a rat, dog, cat, or tiger. The paws are replaced with crude hands that can be used both for walking and grasping objects. Neural implants help the animal become comfortable with its new digits.

\textit{Statistics:} Basic Move -1 [-5]; Bad Grip 1 [-5], Foot Manipulators (2 arms) [-6]; No Fine Manipulators (Mitigator, -70\%) [-9]. -25 points. This replaces No Fine Manipulators [-30].

\textit{Availability:} Two major operations. \$10,000. LC3.

\subsection{Neural Uplift}\label{subsec:neural_uplift}
This procedure improves the intelligence of a non~-sapient animal by implanting computer components that emulate higher neural and brain functions. It may not be added to an animal with IQ 6+.

\textit{Statistics:} IQ+1 (Temporary Disadvantage, Electrical, -20\%) [16]; Wild Animal or Domestic Animal (Mitigator, -70\%) [-9]; Stress Atavism (Mild, 12) [-10]. -3 points. This replaces Wild Animal or Domestic Animal [-30].

\textit{Availability:} Major procedure. Reduce difficulty of the procedure by one step (e.g., major to minor) if performed on an animal with a racial IQ 1-3. \$5,000 × average racial IQ before the cybernetic uplift operation. LC3.

\section{Total Cyborg Brain Transplants}\label{sec:total_cyborg_brain_transplants}
A total cyborg is someone whose entire body has been replaced with artificial parts. Only his brain, parts of the spinal cord, and a few other nerves remain human.

Robot bodies large enough to house human brain cases have a total cyborg mentality lens. The most common are androids, but other types are possible. Robot templates with No Brain, Diffuse, or Homogenous are precluded.

Generally, due to advances in modern technology, a robot body need only have been built for at least a personal computer (p. \pageref{itm:person_com}) to have room for a human~-sized brain. Nonhuman brains may require a larger or smaller volume.

Total cyborgs may not have other cybernetics, with the exception of brain implants.

\textit{Statistics:} The character takes on the robot body's racial template with the Cyborg lens; his brain is the same. See Mind Transfer (p. B296).

\textit{Availability:} Major operation. \$40,000 for the brain case, plus the cost of robot body. The cyborg's body is functional after the operation.

\section{Uploading}\label{sec:uploading}
Memories are encoded within the physical structure of the brain. Uploading is the process of copying this into a digital form. Uploads can create a mind emulation -- a computer program that emulates the workings of the original person's mind. A mind emulation is not just a recording, but a working model of the way a particular brain functions.

\subsection{Destructive Uploading}\label{subsec:destructive_uploading}
This technology involves the preservation and destructive analysis of the subject's brain so that chemically stored memories can be recorded as digital media. For example, the subject's brain may be placed into biostasis, then sliced by robotic surgeons into tiny segments, each of which is scanned at very high resolution.

This procedure is fatal and may be controversial: Is it suicide or transcendence? Individuals may choose destructive uploading to obtain a form of immortality, often out of a desire to live as a posthuman entity in a superhuman robot body. They may also have no choice; with destructive uploading, the dead can be revived and interrogated. (See \textit{Uploading the Dead}, below).

Destructive uploading requires the patient (or his preserved brain). The surgery is performed using a modern surgical facility, chrysalis machine, or automed. Make a Physician roll at -5, and an Electronics Operation (Medical) roll at -5.

Success means the data was gained. If either roll fails by 1, it means only enough data for a low-res copy (p. \pageref{subsec:low-very-low-res_copies}) was gained; if either roll fails by 2, it means only enough data for a very~-low~-res copy is gained. If either roll fails by 3+, or is a critical failure, the upload fails. Only one try is possible; success or failure destroys the brain.

\subsection{Non~-Destructing Uploading}\label{subsec:non~-destructive_uploading}
This process uses advanced scanning systems to record the mind without destroying the brain. This makes it far easier to make copies which exist at the same time as the original person. It might use an advanced form of magnetic resonance imaging (HyMRI) or other non-invasive scanner. It may also require physical probes. These might be a development of the scanning, tunneling microscope (STM), or use sensors placed in the brain to assist in mapping it.

Since the original is not destroyed, it is practical for people to store ``backup'' copies of themselves. This is very useful if mind downloading (p. \pageref{sec:downloading_minds}) is possible, allowing the original mind to be replicated in a new body.

A typical upload-resolution scan takes an hour per attempt and a successful Electronics Operation (Medical) roll. Multiple tries are possible. This roll is made secretly by the GM.

\textit{Modern Imaging:} This can only make low-res scans (p. 220). On a critical failure, the user has made a very~-low~-res scan without realizing it.

\subsection{Uploading the Dead}\label{subsec:uploading_the_dead}
As long as the brain is intact, it is possible to upload a dead person's mind and retrieve a mind emulation. Either invasive or noninvasive methods can be used. Uploading is impossible when the brain suffers total destruction (-10 × HP), exposure to 5,000+ rads, or death from a failed HT roll that resulted from damage to the skull or eye.

Deep structures containing long-term memories may survive for hours after death, but not indefinitely. Use the usual success rolls, but apply a -2 penalty for working with a corpse, as well as an additional -1 penalty per hour past death unless the brain is preserved or in nanostasis. Uploading a corpse preserved via freezing is at an extra -3 due to cell damage from freezing. Memories from the last 1d × 20 minutes before the person died will usually be lost in the uploading process. This means someone revived via uploading might have no memory of how he died.

\subsection{Mind Emulation}\label{subsec:mind_emulation}
A successful upload provides the necessary ``braintape'' of the subject. Bringing it to life requires coding that into a mind emulation -- a model of the living brain.

\textit{Ghost Compiler:}\label{itm:ghost_compiler} Required to allow someone with Computer Programming skill to create a ghost mind emulation (below). Complexity 10, normal cost. LC2.

\textit{Ghost-Editor Program:}\label{itm:ghost-editor_program} This allows someone to use Brainwashing skill on a mind emulation. Complexity 11, normal cost. LC1.

Where does a mind emulation go after it's been uploaded? There are several possibilities.

\subsubsection{Backup Storage}\label{subsubsec:backup_storage}
A mind emulation can simply be stored, unconscious, as data; if non-destructive uploading is possible, old backups may be regularly deleted and replaced by newer updates. A mind emulation requires about 100 TB. The location and security of one's backups may be a paramount concern, with specialized facilities that are devoted to protecting them. Insurance agencies, organizations, or governments might maintain ``memory vaults'' that store backup copies of members or citizens. Backups could be awakened, interrogated, enslaved, or worse if they fell into the wrong hands, and entire adventures could revolve around recovering one.

Treat an accessible backup as the Extra Life (Copy) advantage.

\begin{calloutbox}
    \subsection{Low- and Very-Low-Res Copies}\label{subsec:low-very-low-res_copies}
    Sometimes an uploading or downloading proce-
dure doesn't work out right -- or shortcuts are taken.
The result is a low- or very-low-res copy.

\textit{Low-Res:}\label{itm:low-res_mind_copy} This is an imperfect copy. It can be deliberately made at +2 to skill and 50\% of the normal time and cost, or it may be the result of an accident. A low-res mind-emulation (above) or mind download (below) will have only half as many points in skills. The subject may also suffer from Partial Amnesia [-10]. If so, the subject has a 50\% chance of having the Flashbacks (Mild) [-5] disadvantage, representing partial memories.

\textit{Very-Low-Res:}\label{itm:very-low-res_mind_copy} This is a badly degraded copy. If a mind emulation or download is made from it, the copy has -1 IQ [-20] and Total Amnesia [-25]. It has a personality, but only one-quarter its normal points in skills (round down). Advantages based on emotional sensitivity -- e.g., Charisma, Empathy, Fashion Sense, Rapier Wit -- are lost. The subject has a 50\% chance of gaining Flashbacks (Mild) [-5] as above.
\end{calloutbox}

\subsubsection{Robot Bodies}\label{subsubsec:robot_bodies}
A mind emulation can be run in a computer brain in a robot body. A person might want to return to a body that resembled his original form, a younger and healthier version, or a completely new shape.

\subsubsection{Ghost Comps and Communities}\label{subsubsec:ghost_comps_communities}
A mind emulation might reside in a computer rather than a robot. Mind emulations may travel by copying themselves, or move through expansive virtual realities. Entire communities of emulations and AIs may exist on a computer network, or a single gigantic computer system.

\subsubsection{Computers in Biological Shells}\label{subsubsec:computers_in_biological_shalls}
A computer brain running a mind emulation may be implanted in a biological body -- perhaps even the original body, or a clone of it, depending on its condition after the emulation was created. Use the rules for living flesh androids (p. \pageref{sec:biomorphic_lens}). The body may wear out, but the computer can always be removed or the data copied to a brain in a new body.

\subsubsection{Multiple Bodies, Multiple Personalities}\label{subsubsec:multiple_bodies_multiple_personalities}
It's possible to copy a mind emulation many times over. Mind emulations may take advantage of this by existing in many different forms at the same time!

\subsubsection{Downloading Mind}\label{subsubsec:downloading_mind}
Superscience may allow an uploaded mind to be imprinted on a living brain. See \textit{Downloading Minds} (below).

\subsection{Mind Emulation Templates}\label{subsec:mind_emulation_templates}
Any computer with sufficient Complexity (see below) can run a sapient mind emulation. A mind emulation differs from an AI in lacking the Automaton meta~-trait and possessing other metatraits. This creates a sub~-race (p. B454) version of the template.

\subsection{Mind Emulation (``Ghost'') Programs}\label{subsec:mind_emulation_ghost_programs}
A mind emulation's Complexity depends on the average racial IQ of the being that was uploaded: Complexity 4 + (IQ/2), rounded up. Thus, a race with IQ 10 requires a Complexity 9 program. Lower Complexity by 1 for beings with the ``Fixed IQ'' taboo trait (as in the Domestic and Wild animal meta~-traits); e.g., a dog with IQ 5 requires only a Complexity 6 emulation.

A mind emulation has the advantage Digital Mind [5] and the taboo trait ``Complexity-Limited IQ,'' and optionally one or more of the lenses under \textit{Optional Intelligence Lenses} (p. \pageref{sec:machine_intelligence_lenses}). For a nonhuman, also apply the race's IQ, Perception, and Will modifiers, and all racial mental traits.

\section{Downloading Minds}\label{sec:downloading_minds}
This is the transfer of an uploaded mind into a living brain. Modern methods primarily use nanomachines to replicate neural connections, rebuilding a new brain into a copy of the desired mind.

For ethical (and possibly technical) reasons, downloading is normally performed on a ``blank mind'' -- for example, a clone that was developed in a coma, with no memories or personality of its own. However, downloading into another person's brain may also be possible.

Downloading requires an Electronics Operation (Medical) roll. Success means the mind emulation replaces the original's memories and personality (if it had any). Failure means that the transfer process fails and destroys the brain of the body that was going to receive the download. Critical failure, or any failure by 5 or more, means the transfer seems to work, but there's a hidden flaw. The subject may suffer Partial Amnesia or a Split Personality, or the wrong emulation may have been transferred!

The difficulty of downloading depends on how different the new body's brain structure is from the mind emulation's original body. This allows someone to become a person of a different sex, age, or species. Such downloads are useful for spies or students of alien cultures, or as punishment or torture (``work off your bad karma as a dog''). However, the GM is free to rule that two species are too dissimilar for a transformation to be possible.

If downloading to the brain of another person of the same species, roll at -1 to skill. For transfer to a different species, apply physiology modifiers (p. B181).

The effects of successful downloading are covered under Mind Transfer (p. B296). That is, the old racial template is replaced by that of the new body. The Mind vs. Brain (p. B296) rule should apply except in science~-fantasy settings; downloading an emulated human mind into a cat's brain would result in a drop in IQ, for example.

A failed download will result in a low-res copy (p. \pageref{subsec:low-very-low-res_copies}). A critical failure results in a very-low-res copy. It may take some time to realize this, however; the GM rolls this secretly.

\textit{Clinical Mind Transference Equipment:} The host body must be placed inside this coffin-sized unit. \$500,000, 250 lbs., E/200 hr. LC3.

\subsection{Personality Overlays}\label{subsec:personality_overlays}
``Overdubbing'' a conscious mind may either overwrite that mind (destroying it and creating the new person), or result in an unstable blend of both minds. If the latter, the effect is a Split Personality (p. B156) with -10 to -30 additional points of different mental disadvantages for each personality. Flashbacks, Manic-Depressive, On the Edge, and Paranoia are all appropriate.

\subsection{Incarnation}\label{subsec:incarnation}
If a mind emulation of a formerly biological entity can be downloaded into flesh, why not a digital mind? A former AI struggling to live as a biological entity could be a very interesting character!