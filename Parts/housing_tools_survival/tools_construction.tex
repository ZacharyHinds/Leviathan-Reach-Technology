%\addcontentsline{toc}{section}{Tools and Tool Kits}


\section{Tools and Tool Kits}

\subsection{Laser and Plasma Torches}\label{subsec:laser_plasma_torches}
These are close-focus energy beam projectors that excel at heavy cutting and welding. A torch projects a continuous jet: in combat, treat this as a melee attack that uses Beam Weapons (Projector) skill, and can't be parried except by a force blade or equivalent. The jet inflicts tight-beam burning damage.

\subsection{Power Tools}\label{subsec:power_tools}
A box of power tools for shaping wood and other construction materials. A box of power tools includes a nail gun (p. \pageref{subsec:nail_gun}), and either an industrial water knife (below), a vibroblade (p. \pageref{subsec:vibroblade}), a chainsaw, or a laser torch (p. \pageref{subsec:laser_plasma_torches}). The tools provide a +3 (quality) bonus to Carpentry skill. \$700, 7 lb., 2C/7 hr. LC4.

\subsection{Rope}\label{subsec:rope}
These are synthetic lines and ropes made of carbon nanotubes or biphase composites. 

\textit{1/8" diameter:}\label{itm:rope_1/8} Supports 800 lb. 10 yards of line: \$2, 0.1 lb.

\textit{3/16" diameter:}\label{itm:rope_3/16} Supports 2,000 lb. 10 yards of rope: \$5, 0.25 lb.

\textit{3/8" diameter:}\label{itm:rope_3/8} Supports 8,000 lb. 10 yards of rope: \$20, 1 lb.

\textit{3/4" diameter:}\label{itm:rope_3/4} Supports 32,000 lb. 10 yards of rope: \$80, 4 lb.

This is the safe working load; the theoretical breaking strain is five times as much. If exceeding the safe load, roll vs. the rope's HT 12 at -1 per multiple of working load whenever it is stressed to see if it snaps. 

\subsection{Instructor Kits}\label{subsec:instructor_kits}
Instructor kits have radio frequency tags and dedicated computer chips on all the components, from circuits to screws or bricks. If the user has a HUD or neural interface, the device will show exactly where and how a component fits into other components. It will signal when it has been properly put together, indicate what tool is needed, and so on. These will appear on handy pop~-up diagrams overlaid on the user's visual field.

Instructor kits are avilable for most devices, as well as homes, model kits, and more. They cost 50\% of the cost of the device, and take one man~-hour per \$1,000 of cost to assemble. They require an appropriate skill roll -- usually Mechanic or Electronics Repair -- but this is made at +5 to skill if the user can read the virtual tags as they build it. This makes assembly easy, even if the user has only a default level of skill. A failed skill roll means more time is required; a critical failure means something breaks or malfunctions later.

\subsection{Industrial Water Knife}\label{subsec:industrial_water_knife}
This device resembles a thick hacksaw with a five~-inch gap where its blade should be, plus a switch and power cell built into the hand and an attached hose. When connected to a water source and switched on, a jet of hyper~-velocity water crosses the gap, forming a ``blade'' capable of slicing through flesh, wood, and even thin metal.

\multicolinterrupt{
    \begin{table}[H]
        \centering
        \hrule height 1pt\medskip
        {\noindent BEAM WEAPONS (PROJECTOR)(DX-4, other Beam Weapons-4)}
        \rowcolors{2}{}{\colorhousing}
        \begin{tabularx}{\linewidth}{lXlXXXXrlX}
            \textbf{Weapon} && \textbf{Damage} && \textbf{Reach} & \textbf{Parry} & \textbf{Cost} & \textbf{Weight} & \textbf{ST} & \textbf{Notes}  \\
            Heavy Laser Torch && 4d(2) burn && C,1 & no & \$800 & 12/Dp & 7† & 15 min. \\
            Heavy Plasma Torch && 4d+1(5) burn && C,1 & no & \$2,000 & 40/Dp & 8† & 15 min. \\
            Laser Torch && 2d(2) burn && C,1 & no & \$100 & 3/C & 5 & 15 min. \\
            Mini Laser Torch && 1d(2) burn && C & no & \$50 & 0.25/B & 1 & 3 min. \\
            Mini Plasma Torch && 1d+2(5) burn && C & no & \$100 & 1/B & 3 & 3 min. \\
            Plasma Torch && 2d(5) burn && C,1 & no & \$250 & 5/C & 6 & 15 min. \\
            Fusion Torch && 8d+2(5) burn && C & no & \$2,000 & 40/Dp & 8† & 15 min. 
        \end{tabularx}
        \label{tab:laser_torches}
    \end{table}
}

Since the water recirculates through the system, little splash off -- the knife only uses one gallon per hour. Water knives are also safer than a chainsaw. If the blade can't cut through something, the only ``danger'' is a spray of water.

Water knives do not have to be sharpened or cleaned, though for use in sterile environments, there is a filtered, self~-sterilizing version. This sterile version sees significant use in surgery and medicine, but is also favored for slicing meat in restaurants and slaughterhouses.

The blade can be used as an improvised weapon. It does 3d+2 cutting damage with a Reach 1 and cannot be parried with. Roll DX-5 to attack with it.

\$160, the saw itself weighs 8 lb. Most come standard with a backpack which holds two gallons, connected to the knife with a two~-yard hose. The backpack tank weighs 2 lb. empty and 18 lb. full. C/12 hr. LC4.

\subsection{Portable Antimatter Trap}\label{subsec:portable_antimatter_trap}
This is a portable magnetic bottle for storage and transfer of antimatter. It can safely store up to 100 micrograms of antimatter and is designed to interlock with antimatter reactors for safe transfer. If the power supply is turned off or the power cells removed while containing antimatter, the result is an explosion: antimatter has an REF (p. B415) of 10,000,000, and a single microgram can do 6d×9 cr ex damage!

The power will not turn off instantly: a built~-in capacitor stores enough for 30 seconds of operation. Unless sabotaged or deactivated, a warning system will sound a buzzer and display a countdown to detonation. A biometric lock (p. \pageref{itm:briometric_scanlock}) prevents unauthorized tampering or release, and the trap itself is ruggedized (p. \pageref{subsec:rugged}). It also has a redundant power cell socket so that cells can be changed without turning it off.

A standard unit is \$20,000, 20 lb., 2D/10,000 hr. LC3 (without antimatter).

\subsection{Monowire Spool}\label{subsec:monowire_spool}
A spool of 100 yards of superstrong monowire (p. \pageref{subsec:monowire_blade}), with a handle on each end. Monowire ha DR 10 and HP 1. Ued a rope, it will upport a working load of 1,000 lb. A standard spool is \$1,000 and 0.1 lb. See Monowire Fences (p. \pageref{subsubsec:monowire_fence}) for some additional uses. LC3.

\subsection{Tool Kits}\label{subsec:tool_kits}
%\addcontentsline{toc}{subsection}{Tool Kits}
Tool kits are used for repair skills (p. B190): Armoury, Electronics Repair, Electrician, Machinist, and Mechanic. They determine the equipment modifiers that apply when using these skills.

Each kit contains a variety of powered and unpowered tools and an array of spare parts. All kits have power cells for the tools in the kit. Even without power, the tools may still be usable for minor jobs at -2 to skill.

Different tool kits are required for each skill and each specialty. There is no single Armoury kit -- if you want to repair a pistol, use an Armoury (Small Arms) kit.

Armoury (Vehicle Armor) or Mechanic (Vehicle Type) tool kits and workshops can perform major repairs on vehicles up to 10 tons. For larger facilities, multiply the cost and weight of the kit by vehicle weight/10. For example, a Mechanic (Submarine) kit (2,000 ton~-capacity) is 200 times the normal cost and weight.

\subsubsection{Portable Tool Kit}\label{subsubsec:portable_tool_kit}
This is the standard tool kit. Most versions fit in a heavy tool box or backpack. It provides basic equipment for the specific skill and specialization it is designed for, and gives a -2 (quality) modifier for other specializations within that skill. Most kits for Armoury, Electrician, Mechanic, or Machinist specializations are \$600, 20 lb., 10B/10 hr. Those for Electronics Repair and Mechanic (Micromachines or Nanomachines) are \$1,200, 10 lb., 10A/10 hr. LC4.

\subsubsection{Mini~-Toolkit}\label{subsubsec:mini-tool}
This is a belt~-sized tool kit. It gives a -2 (quality) equipment modifier for the specific skill and specialization it is designed for. Mini~-Tool Kits for most Armoury, Electrician, Mechanic, or Machinist specializations are \$200, 4 lb., 5B/10 hr. Those for Electronics Repair and Mechanic (Micromachines or Nanomachines) are \$400, 2 lb., 5A/10 hr. LC4.

\subsubsection{Portable Workshop}\label{subsubsec:portable_workshop}
An elaborate version of the portable tool kit. It has everything necessary for emergency repairs, plus a wide range of spare parts that can be tooled to specific requirements. It is a modular system that can be set up in any large vehicle or building; it takes an hour to pack or unpack. It gives a +2 (quality) bonus to skill, or +1 if not unpacked.

Most workshops for Armoury, Electrician, Mechanic, or Machinist skill are \$15,000, 200 lbs., 10C/100 hr. Workshops for Electronics Repair and Mechanic (Micromachines or Nanomachines) are \$30,000, 100 lb., 10B/100 hr. LC4.

\subsubsection{Robotic Workshop}\label{subsubsec:robotic_workshop}
This automated workhop can attempt to fix any piece of broken or damaged equipment. It uses its sensors and programmed repair menuals to diagnose the problem, then repairs it with its tool~-equipped manipulator arms. It has skill 13 in whatever skill and specialty the workshop is designed for. However, it can only maintain and repair devices that are in its database (or closely related). If the workshop encounters a problem it cannot fix, it can be programmed to call for help using a buil~-in tiny radio.

If a human technician is directing a robotic workshop, it is as good as a portable workshop, with an additional +1 bonus due to its extensive technical database and usefulness as an automated assistant. It is double the cost and weight of an equivalent workshop, and requires twice as many power cells; it has the same LC.

\subsection{Micro~-Manipulator Tool Bench}\label{subsec:micro-manipulator_tool_bench}
This is a robot arm with micro~-scale manipulators and sensors, designed to be controlled by VR gloves (p. \pageref{itm:vr_gloves}) or a neural interface (p. \pageref{sec:neural_interfaces}). It gives the user super~-fine motor skills, adding +5 to DX for tasks such as Jeweler, and DX~-based rolls to do \textit{fine} work with Artist, Machinist, or Mechanic skills. \$2,000, 4 lb., C/100 hr. LC4.

\subsection{Nail Gun}\label{subsec:nail_gun}
A tool for rapidly and accurately driving nails. It has a targeting system incorporating a computerized laser, passive infrared, and imaging radar rangefinder that can see through up to six inches of wood or similar low~-density material (including flesh). It uses this system to automatically determine the force needed to drive a nail to the
desired depth.

To attack with a nail gun, use DX-4 or Guns (Pistol) skill. It inflicts 1d+1(2) piercing damage, with Acc 0 (+3), Range 5/25, RoF 10, Bulk -3, Rcl 1, Shots 50(5). Nail velocity is variable (see Liquid-Propellant Slugthrowers, p. \pageref{subsec:liquid-propellant_slugthrowers}) and its blueprint display system is equivalent to a smartgun feature.

All of this makes the nail gun a highly accurate weapon system in a pinch. However, its sensors are programmed \textit{not} to fire if its targeting system detects something that matches the warmth and consistency of living flesh. Disabling this safety feature requires an Electronics Operation (Security) roll, one minute per attempt. The gun cannot detect flesh underneath armor with DR3 or better.

\subsubsection{Smart Nail Gun}\label{subsubsec:smart_nail_gun}
Uses binary liquid propellant. \$250, 4 lb. Its A cell powers the targeting system for a day. Its magazine holds 50 nails (0.5 lb.). A separate propellant bottle (1 lb.) holds enough propellant to fire 1,000 nails. LC4.

\subsubsection{Gauss Nail Gun}\label{subsubsec:gauss_nail_gun}
An electromagnetic nail gun. The magazine holds 50 nails (0.5 lb.). \$300, 3.5 lb. B/300 shots. LC4.

\subsection{Slipspray}\label{subsec:slipspray}
This aerosol lubricant can turn smooth ground into a nearly frictionless surface. Anyone crossing it at faster than Move 1 must make a DX roll (at +3 if crawling, -3 if sprinting) every second to avoid falling. Vehicles must make a control roll (p. B469) at -5 to avoid losing control. Slipspray breaks down in about an hour in air. A can covers 100 square feet, spraying 10 square feet per second from up to 2 yards away. \$30, 0. lb. LC3.

\subsection{Super Adhesives}\label{subsec:super_adhesives}
Pulling two objects apart that have been glued together requires a Regular Contest of ST vs. ST 23. The bond is limited by the strength of the weaker of the two objects (so flesh bonded to something else could be torn away, inflicting 1d-4 damage).

\subsubsection{Gecko Adhesive}\label{subsubsec:gecko_adhesive}
Sticky adhesive based on gecko setae (feet hairs). The pads have millions of tiny artificial hairs, covered by a protective coating. A small electrical pulse from an included wand causes the hairs to extend or release. A one~-square~-inch patch can hold 800 lb. indefinitely in any environment, including in the vacuum of space and underwater. \$0.10 per square inch for double~-sided pads, and \$1 per foot length of 2" wide single~-sided gecko tape.

\subsubsection{Molecular Glue}\label{subsubsec:molecular_glue}
This glue bonds nearly any substance and comes in non~-conductive and conductive (metal~-impregnated) varieties. It sets in 10 seconds. The glue can only be removed by a special solvent, which takes one minute to weaken each dose of the glue. A does of solvent can weaken up to \$10 applications of molecular glue. Each application of molecular glue is \$0.50, but a dose of solvent is \$1. LC4.

\subsection{Construction Foam}\label{subsec:construction_foam}
Construction foam is a liquid polymer with suspended nanoparticles that “foams~-up” with nitrogen and cures with oxygen. As a result, it expands in air, hardening to form a durable substance. A barrier has DR 2 per inch of thickness and HP based on the weight of foam used (see Object HP Table, p. B558; the foam is homogenous). Construction foam is usually combined with additional chemicals so that it cures much more rapidly. Most applications are mundane, such as creating temporary structures, sealing electronics components, and quick casts for injured limbs. Riot police and soldiers also use construction foam for temporary walls and bunkers, usually by forming barriers and filling them with water or earth.

Construction foam does not burn easily (30 points of burning damage will set it aflame), but does decompose when exposed to flame, turning into a foul black sludge and releasing toxic fumes. These cause 1 HP toxic injury per minute of exposure if breathed. The foam will not cure if it stays wet, but is waterproofed once it has hardened. Construction foam floats.

Three gallons can form five cubic feet of hardened foam; a barricade three yards long, a yard high, and a foot thick takes about 16 gallons of foam and has DR 24, HP 36. It requires one minute to completely set and harden. It comes in a variety of applicator types, from spray cans to large storage tanks for use in construction. Construction foam costs \$10 and weighs 5 lb. per gallon.

\subsection{Industrial Nanocleanser}\label{subsec:industrial_nanocleanser}
This industrial-strength version of domestic nanocleanser is designed to eliminate bacteriological spills, rotten food, dead bodies, and other biological or medical waste. Any organic target covered by industrial nanocleanser takes 1d-1 corrosion damage each minute for five minutes. Inorganic sealed DR takes no damage. If sprayed on plants, industrial nanocleanser will strip all foliage within a minute.

Industrial nanocleanser removes all forensic evidence such as blood stains, skin flakes and other organic residue. As with domestic nanocleanser, Forensics will be able to identify the exact brand used.

An application of industrial nanocleanser can cover up to 30 square feet. \$100, 1 lb. LC3.

\subsection{Morph Axe}\label{subsec:morph_axe}
A standard morph axe is a climbing tool made of memory metal (p. 90). It can go from straight~-handled to bent~-handled, from pick to hammer to adze to hook to crowbar to walking stick, on command. It can be used to cut steps, climb vertical frozen walls, or stop a climber’s potentially disastrous slide on ice, and is sharp enough cut rock if the wielder has ST 12 or better. In combat, it requires Axe/Mace skill and does swing+1 damage (cutting, impaling, or crushing, depending on configuration). Otherwise, treat as a hatchet. \$500, 2 lbs. LC4.

Any morph tool with more than 10 forms, or with illegal forms, costs double.

A morphing tool that includes any explicit weapon forms is a concealable weapon, and LC3. It could become any weapon form appropriate for its weight. For example, a quarterstaff could turn into an axe, a spear, or an oversized broadsword. \$500 per pound of weight, minimum \$1,000.

\subsection{Sonic Probe}\label{subsec:sonic_probe}
This is a multipurpose sensor the size of a cigarette package. It can be used as a short~-ranged ultrasonic scanner that can give the user a rough image of the interior of objects or containers; it has a small screen on the device, but the data is usually uplinked to a HUD. Roll against Electronic Operation (Sonar) to use; the probe has a maximum range of six inches, and the skill penalty is -1 for each 10 DR it must penetrate.

The probe’s imaging ability lets it serve as basic equipment for simple medical Diagnosis rolls for physical injuries, Mechanic rolls to find out what is wrong with a small device, Explosives rolls to disarm bombs (unless set to be triggered by vibrations!), and similar tasks. Its scanning abilities make it a useful tool for picking mechanical combination locks (+2 to Lockpicking skill) in conjunction with a lockpick. The sonic field can also be intensified and tuned to assist in cleaning delicate objects... or brushing teeth. \$500, 0.25 lbs., B/12 hr. LC4.

\section{Worker Robot}\label{sec:worker_robots}
%\addcontentsline{toc}{section}{Worker Robot}
\subsection{Techbot}\label{subsec:techbot}
\hfill \large{\textbf{156 points}}
\normalsize

This is a general~-purpose technical robot. It has a cylindrical body and pair of arms, and moves on legs. It can operate in a wide variety of environments, and may be used for everything from working in a garage to hazardous wasted disposal. Its payload space usually holds tools.

Its a weapon mount is built into its rotating head; it can carry a weapon up to 6 lb. weight, which is more likely to be a laser or plasma torch than an actual weapon.

\begin{hangparas}{1em}{1}
    % \normalsize
    \textbf{\textit{Attribute Modifiers:}} ST-2 [-20]; HT+1 [10.

    \textbf{\textit{Secondary Characteristic Modifiers:}} SM-1; HP+6 [12].

    \textbf{\textit{Advantages:}} Absolute Direction [5]; Doesn't Breathe [20]; DR 15 (Can’t Wear Armor, -40\%) [45]; Extra Arm (Weapon Mount, -80\%) [2]; High Manual Dexterity 2 [10]; Machine [25]; Microscopic Vision 3 [15]; Radio (Burst,+30\%; Secure, +20\%; Video, +40\%) [19]; Payload 2 [2]; Protected Senses (Hearing, Vision) [10]; Radiation Tolerance 5 [10]; Reduced Consumption 2 [4]; Sealed [15]; Vacuum Support [5].

    Perks: Accessory (Personal computer, fast option) [1].

    \textbf{\textit{Disadvantages:}} Cannot Float [-1]; Electrical [-20]; Maintenance (one person, bi~-weekly) [-3]; Restricted Diet (Very Common, power cells) [-10].

    \textbf{\textit{Availability:}} \$10,000, 50 lb., D/24 hr. 
\end{hangparas}

\subsubsection{Optional Enhancements}
The following optional enhancements are standard for Techbots:

\begin{hangparas}{1em}{1}
    \normalsize
    \textit{Gwhel~-Contragrav:} By making use of small~-scale gwhel-generators, these techbots hover. Flight (Planetary, -5\%) [38]; Aerial [0]. +\$12,000.
\end{hangparas}

\subsection{Worker Swarms}\label{subsec:worker_swarms}
%\addcontentsline{toc}{subsection}{Worker Swarms}
Equipment packages for microbot swarms.

\subsubsection{Construction Swarm}\label{subsubsec:construction_swarm}
This swarm is designed to tunnel, dig ditches, etc. Each bot is equipped with small arms and digging jaws. A square yard of swarm can dig as if it had ST 3 and a pick and shovel (see p. B350). They are often employed for mining, or civil or military engineering. They can also pile up loose earth and rock into ramparts, dikes, or walls. \$1,000/square yard. LC3.

\subsubsection{Decontamination Swarm}\label{subsubsec:decontam_swarm}
Decontamination swarms remove traces of most biotoxins, persistent chemicals, nanotechnology, or radioactive fallout. The type of hazard removed depends on the individual swarm's specialization. A square yard of swarm can decontaminate a one~-square~-yard area to a soil depth of 2" every minute, and can do so 20 times before requiring replacement. It has Hazardous Materials (Biological, Chemical, Nanotech, or Radioactive) skill at 12 and serves as basic equipment for hazmat disposal. \$1,000/square yard LC3.

\subsubsection{Defoliator Swarm}\label{subsubsec:defoliat_swarm}
This swarm kills plants but has no effect on other living creatures. Even so, a large swarm might serve as an ecological sabotage weapon. It takes the swarm 10 seconds to strip a square yard clean of bushes or foliage. It can be programmed to carefully trim plants; this takes one minute per square yard. It may also be programmed to affect specific plants (for example, weeds) or to mow lawns. \$1,500/square yard. LC3.

\subsubsection{Harvester Swarm}\label{subsubsec:harvest_swarm}
These can harvest crops with an effective Farming skill of 12. \$2,000/square yard. LC4.

\subsubsection{Pesticide Swarm}\label{subsubsec:pesticide_swarm}
The swarm is equipped to hunt down fleas, spiders, and other pests. Flier swarms can also destroy flies and mosquitoes. They will inflict 1d corrosion damage per turn to swarms composed of real insects! The swarm’s actions are harmless to humans, although they can be entertaining or distracting. Like defoliator swarms, they can be an ecological threat. \$1,000/square yard. LC3.

\subsubsection{Pollinator Swarm}\label{subsubsec:pollinat_swarm}
The swarm functions as artificial bees, spreading pollen or seeds. This is useful if normal insects are not available, or cannot adapt to the local climate or ecology. Base cost is \$1,000/square yard. LC4.

\subsubsection{Repair Swarm}\label{subsubsec:repair_swarm}
The swarm has the tools and programming to repair a single, specific model of equipment, plus appropriate Armoury, Electronics Repair, or Mechanic skills at 12 (specify which). A single swarm fixes things at about 1/10 the speed of a human, but up to 10 swarms can combine to make repairs. Base cost is \$500 per square yard, plus \$250/square yard per additional model of equipment the bots are programmed to fix, to a maximum of four types of equipment per swarm. LC4.

\section{Heavy Equipment, Salvage, and Rescue Gear}\label{sec:heavy_equip_salvage_rescue}
%\addcontentsline{toc}{section}{Heavy Equipment, Salvage, and Rescue}
\subsection{Blast Foam}\label{subsec:blast_foam}
Ballistic foam forms a non~-conductive polymer~-ceramic blanket. Designed to be sprayed over a bomb, it hardens in 3 seconds and forms a thick layer than can absorb explosions, contain fragments, and sterilize chemical, biological, and radiological agents. Each second of spray can coat a square yard, providing ablative DR 40 against crushing and burning damage, and ablative DR 20 against other types of damage. If the foam contains the blast, it also treated as sealed with radiation PF 5. Each square yard of foam weighs 10 lb.

Anyone completely coated with the foam may suffocate (p. B436); he can inflict his normal thrusting damage on the foam to try to escape. A canister of foam that can cover three square yards is \$100, 30 lb. LC3.

\subsection{Fire Extinguisher}\label{subsec:fire_extinguisher}
This multi~-purpose dry chemical extinguisher can put out ordinary blazing combustibles, flammable liquids, or electircal fires. Three sizes are available:

\textit{Fire Extinguisher Tube:}\label{itm:fire_extinguish_tube} A pocket device with a six~-second discharge and two~-yard range. \$10, 1 lb. LC4.

\textit{Small Fire Extinguisher:}\label{itm:small_fire_extinguish} A standard extinguisher bottle with a 15~-second discharge and a three~-yard range. \$50, 3 lb. LC4.

\textit{Large Fire Extinguisher:}\label{itm:large_fire_extinguish} A heavy backpack model with a handheld projector connected to the pack. Three~-yard range, 45~-second discharge. \$200, 10 lb. LC4.

Any fire extinguisher can also be used as a weapon. Use Liquid Projector (Sprayer) skill, using the jet rules: treat them as a melee weapon. A hit to the face is treated as an Affliction with the Contact Agent modifier. On a failed HT-3 resistance roll, the victim is stunned, and suffers the Blindness disadvantage for seconds equal to the margin of failure.

\section{Demolitions}\label{sec:demolitions}
%\addcontentsline{toc}{section}{Demolitions}

\subsection{Explosives}\label{subsec:explosives}
Explosives are rated for theri relative explosive force (REF) compated to TNT; see p. B415. Some common types:

\textit{Antimatter:}\label{itm:exp_antimatter} A microgram of antimatter (see \textit{Antimatter Trap}, p. \pageref{subsec:portable_antimatter_trap}) is \$2,500. LC0.

\textit{Plastex B:}\label{itm:exp_plastex_b} This is a powerful moldable high explosive. It is very stable and can only be detonated with an explosive detonator. REF 4. \$20 per pound. LC2.

\textit{High~-Energy Explosive:}\label{itm_exp_high-energy} An exotic explosive that stores energy in metallic hydrogen. REF 6. \$40 per pound. LC2.

For more exotic types, see also \textit{Warheads and Ammunition} (p. \pageref{ch:warheads_ammunition}).

Detonators for explosives can use communicators or timers. \$20, neg. weight, LC3.

\subsection{Taggants}\label{subsec:taggants}
Most commercial and military explosives have embedded taggants: inert materials that are not destroyed in the explosion, which can be analyzed later to determine the type of explosive, manufacturer, and lot number. Taggants add +5 to Research or Forensics rolls to find the origin of the explosive.

Taggants are nearly universally required by law, but it is possible to find explosives \textit{without} them, if you know where to look. A chemistry lab can be used to test a sample; decontaminate swarms (p. \pageref{subsubsec:decontam_swarm}) can be used to remove taggants. Any explosives \textit{without} taggants are always one lower LC.