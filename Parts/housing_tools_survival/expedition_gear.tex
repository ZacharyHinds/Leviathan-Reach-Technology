\section{Lights}\label{sec:lights}
%\addcontentsline{toc}{section}{Lights}
\subsection{Flashlights and Searchlights}\label{subsec:flashlights_searchlights}
These can project an infrared, ultraviolet, or visible light beam, which is also tunable from a wide flashlight cone to a pencil~-thin red or blue~-green laser pointer (range is multiplied by 10). It can function as a blinding weapon in a pinch -- see Dazzle Laser (p. \pageref{subsec:laser_dazzler}). The light eliminates darkenss penalties out to its listed range. Use 75 times this distance for signaling range.

\textit{Penlight:}\label{itm:penlight} This emits a 10~-yard beam. May be helmet or belt~-mounted or attached to a firearm accessory rail. \$3, 0.1 lb., 2A/24 hr.

\textit{Mini Flashlight:}\label{itm:mini_flashlight} This projects a 30~-yard beam. May be helmet~-mounted or attached to a firearm accessory rail. \$10, 0.25 lb., B/24 hr.

\textit{Heavy Flashlight:}\label{itm:heavy_flashlight} This projects a 100~-yard beam, and can be used as a baton. \$20, 1 lb., 2B/24 hr.

\textit{Searchlight:}\label{itm:searchlight} Heavy~-duty searchlights are often mounted on vehicles or buildings. A searchlight projects a 4,000~-yard beam. \$500, 10 lb., C/12 hr. LC4.

\subsection{Glow Sticks}\label{subsec:glow_sticks}
Chemical lights that glow when snapped and shaken; they don't require power cells. Each provides 2 days of light illuminating a two~-yard radius. They're available in different colors, white light, and infrared light. \$2, 0.1 lb.

\subsection{Firefly Swarm}\label{subsec:firefly_swarm}
This is a swarm of glow~-in~-the~-dark microbots. They can be ordered into small spaces for illumination, serve as mobile lamps, or provide a diffuse candle~-like glow for romantic occasions. They can turn on or off, change colors or dim their lights on command, and glow in the infrared, ultraviolet, or visible spectrums. They can't provide the equivalent of full daylight (unless multiple swarms are stacked) but they are bright enough to read by. A firefly swarm is \$100/square yard. LC4.

\section{Navigation Instruments}\label{sec:nav_instruments}
%\addcontentsline{toc}{section}{Navigation Instruments}

\subsection{Global Positioning System (GPS) Receiver}\label{subsec:gps_receiver}
Most gadgets have this as a built~-in feature. If a planet has an orbital navigation satellite network, the GPS system links the user to it, enabling him to always know his exact position if he consults a properly~-scaled map. It is accurate to about 5 yards. The system can also store the coordinates of a location it has visited as a waypoint. It can then direct the user to that waypoint or transmit the data via communicator to other GPS systems. With a computer, it can show the user's position on a moving~-map display.

\subsection{Inertial Navigation}\label{subsec:inertial_nav}
These devices indicate the direction and distance traveled from any preset point on a planetary surface. It can be set for the location at which the user is physically present, or for any other coordinates (requires a Navigation roll if the precise coordinates are uncertain). An inertial navigation system lets the user always know which way is north. She can retrace any path she has following within the past month, no matter how faint or confusion. It does not work in environments such as interstellar space, but it \textit{does} work underground, underwater, and on other planets.

\textit{Inertial Compass:}\label{itm:inertial_compass} A palm-sized inertial navigation system. Includes a tiny computer (p. \pageref{itm:tiny_com}), a GPS (above), and a video screen. If it has access to a digital map, the compas can superimpose the user's position and path on the map and display it on its screen. The compass can also connect to an HUD. It gives a +3 bonus to Navigation (Air, Land, and Sea). \$60, 0.1 lb., A/200 hr.

\subsubsection{Inertial Navigation System}\label{subsubsec:inertial_nav_system}
An extremely precise system. It has the capabilities of an inertial compass, but adds a +5 (quality) bonus to Navigation (Air, Land, and Sea). \$5,000, 20 lb., B/100 hr. LC4.

\section{Containers and Load~-Bearing Equipment}\label{sec:containers_load-bearing}
%\addcontentsline{toc}{section}{Containers and Load~-Bearing Equipment}

\subsection{Hovercart}\label{subsec:hovercart}
A flat round cart, two feet in diameter, that floats quietly on an air cushion generated by a ducted fan. It can be towed or pushed at the pusher's Move. They can carry 500 pounds over smooth ground or water, and make a humming sound audible to a normal Hearing roll at 30 yards. Voice~-controlled robot versions with Move 4 are available at double cost. \$300, 4 lb., C/12 hr. LC4.

\subsection{Pressure Box}\label{subsec:pressure_box}
A pressurized container often used for carrying fragile items or pets through vacuum or hostile environments. Its internal dimensions are 2 × 1 × 1 feet. It includes a 12~-hour air tank and its own life~-support pack that regulates the environment. It provides a room~-temperature environment from -459\degree{}F to +200\degree{}F, as well as DR 10 and radiation PF 5. The modular walls can link together to form a larger container from several boxes. Sealing or unsealing the pressure box takes six seconds; linking boxes together takes 10 seconds per box. \$600, 7 lb., 2C/1 wk. LC4.

\section{Survival and Camping Gear}\label{sec:survival_camping}
%\addcontentsline{toc}{section}{Survial and Camping Gear}

\subsection{Envirobag}\label{subsec:envirobag}
An insulated, heated sleeping bag designed for extremes of temperature. It has the same performance as a Heat Suit (p. \pageref{subsubsec:heatsuit}). It can also be sealed and hooked up to air tanks. \$80, 2 lb., C/72 hr.

\subsection{Filtration Canteen}\label{subsec:filtration_canteen}
This canteen holds a quart of water. It removes impurities, salts, microbes, and poisons. It can filter salt water, but not raw sewage or toxic waste. On its own, it adds a +1 (quality) bonus to Survival skill when living off the land; it's also included in survival kits.

One quart can be purified in 10 minutes. The filter must be replaced every 1,000 quarts; a color change signals when it's time to change. An ``exhausted'' filter still has a few quarts of capacity, but there's no way of knowing how many. 

The canteen is \$180, 1 lb. (empty) or 3 lb. (full). LC4. Replacement filters are \$18, neg. weight. LC4.

\subsection{Gripboots}\label{subsec:gripboots}
This smart climbing footwear is tough, but still provides tactile feedback to the wearer. Additionally, the boot can change shape for a better grip, and can grow crampons or a forward~-place spike on command. Add +1 per die to the damage the wearer inflicts with a kick. Gripboots give a +1 equipment modifier to Climbing, or +2 on ice. Combat statistics are identical to assault boots (p. \pageref{subsec:assault_boots}). \$500, 2 lb. LC4.

\subsection{Modular Cage}\label{subsec:modular_cage}
This kit allows the user to assemble any size or shape of cage, with a maximum volume of 10 cubic yards. Assembling a cage takes 3 minutes per cubic yard of volume; several cages may be combined to build a larger one. Traps skill is required to build anything but a simple cubical cage, or to assemble a cage quickly in half the time.

Cage bars are 1/2" in diameter. DR 100. \$1,000, 200 lb. LC4.

\subsection{Modular Environmental Cage}\label{subsec:modular_env_cage}
As the modular cage, but takes three times as long to put together. Once assembled the cage is sealed and, if connected to an external air and power supply, can duplicate and maintain nearly any planetary environment (except for gravity). A 6" airlock allows access fo rfeeding. \$10,000, 400 lb. LC4.

\subsection{Pressure Tent (Personal)}\label{subsec:pressure_tent}
This airtight tent is strong enough to be inflated to one atmosphere in a vacuum. The metallized fabric incorporates minor (PF 2) radiation protection, but users planning a long stay in a vacuum or trace atmosphere are advised to place the tent in a sheltered location to provide protection from solar and cosmic radiation. The tent has DR 15. The tent's air tanks hold air based on the tent's capacity:

\textit{One~-Man Tent:}\label{itm:one-man_tent} Two~-days of air. \$1,500, 60 lb. LC4.

\textit{Three~-Man Tent:}\label{itm:three-man_tent} Six~-days of air. \$3,000, 100 lb. LC4.

\textit{Eight~-Man Tent:}\label{itm:eight-man_tent} Sixteen~-days of air. \$15,000, 200 lb. LC4.

\subsection{Rocket Piton}\label{subsec:rocket_piton}
A pistol~-grip launcher which fires a rocket~-propelled, explosive~-set piton. It can ashoot an attached line up to 200 yards. A successful Climbing roll (made by the GM) means the piton is securely lodged and will support weight; a critical failure means the firer only thinks it is! Roll vs. DX-4 to hit if used as a weapon. It inflicts 1d+2 impaling damage, with Acc 2, range 70/200, RoF 1, Bulk -3, Rcl 2, Shots 1(5). \$40, 2 lb. Reloads are \$1, 0.5 lb. per shot. LC4.

\subsection{Shelterpack}\label{subsec:shelterpack}
Originally designed by a retired Survey Service scout, and popular with soldiers, refugees, and recreational backpackers, the shelterpack uses memory bioplas, buzz fabric, and solar paint to compress many wilderness survival tools into the lightest possible package. It can pack itself into a box the size of an attaché case for easy storage, or unfold to form the following items.

\textit{Pack:} A standard frame backpack (p. B288) or hard suitcase (p. B288) with five square feet of solar paint exposed.

\textit{Tent:} A one-man unpressurized tent. Survival modifier +2. In Earth-normal daylight, the solar paint provides enough power to run a vapor canteen (p. \pageref{subsec:vapor_canteen}) and recharge power cells. Shelterpacks can be joined together to make larger tents for more people: If 16 or more shelterpack tents are hooked up together, they constitute a solar power array providing external power (p. \pageref{itm:solar_array}).

\textit{Waterproof Poncho:} A hooded black cloak that protects against wind and rain and provides some protection against both heat and cold (-20\degree to 120\degree F if worn over ordinary clothing and suitable footwear). This is often worn over an expedition suit (p. \pageref{subsubsec:expedition_suit}); its solar paint helps keep the power plant charged. It's also useable as a heavy cloak in combat (p. B287) and has DR 1 and +4 to Holdout.

\textit{Boat:} The shelterpack can be changed to two possible configurations: a two-person enclosed kayak, or a flat-bottomed open coracle.

\textit{Sled:} It can be used as a toboggan, a pulka, or a stretcher.

\textit{Saddle and Saddlebags:} The standard setting is for an equine or robot equivalent, but it can be reprogrammed for other mounts.

When completely empty, the shelterpack can reconfigure itself between forms in 1d+9 seconds. Anything left inside it will be ejected downward; this causes the reconfiguring process to take (1d+4) times as long, and can be awkward for the occupant. A shelterpack incorporates a printed tiny computer (p. \pageref{itm:tiny_com}) and datapad, and can respond to verbal commands; smart owners secure these with a voiceprint or other ID, to prevent pranksters turning their boats into suitcases while midstream.

Standard military-issue shelterpacks are rugged (p. \pageref{subsec:rugged}) and available only in black: \$1,500, 12 lbs, LC4. Halve the cost for a non-rugged (but possibly more colorful) civilian version.

Military shelterpacks may also incorporate infrared cloaking (+\$1,500; p. \pageref{subsec:infrared_cloaking}) and/or reversible chameleon cloak lining (+\$1,000 to + \$8,000; p. \pageref{subsubsec:chameleon_cloak}) for use in tent or poncho mode. Optional extras for both military and civilian models include upgraded computers (p. \pageref{sec:computer_hardware}) and communicators (p. \pageref{sec:comms}).

\subsection{Balloon Piton}\label{subsec:balloon_piton}
A partially inflated two-inch balloon covered with gecko adhesive and attached to a swivel-ring, it's used by pressing it to a smooth surface or into a crack and electrically triggering the gecko surface. It can hold up to 5,000 pounds. A set of 10 gives +1 to Climbing skill. Each: \$20, 0.05 lb., AA/40 uses. LC4.

\subsubsection{Molecular Suction Cup}\label{subsubsec:molecular_suction_cup}
A tiny balloon piton capable of supporting 800 lbs. \$5, 0.02 lbs., AA/1,000 uses. LC4.

\subsection{Smart Pitons}\label{subsec:smart_pitons}
These high~-tech pitons adjust to the shape of the crack they're in, and repot on their status via an integral microcommunicator. They come free on command. While a piton will report if it is obviously loose, it cannot check its own stability under load, so the climber must also do a manual check. Used properly, they give +1 to Climbing skill. 10 pitons: \$100, 1 lb. LC4.

\subsection{Genius Piton}\label{subsec:genius_piton}
This is a combination of a smart piton (above) with additional gecko adhesive material (p. \pageref{subsubsec:gecko_adhesive}) along its length, a balloon piton, and a small tube of splat piton material (supports an extra two tons after one minute) and catalyst. The functions can be triggered independently, so a loose piton can be reset or removed. Nominal working load is 5,000 pounds; increase cost and weight linearly for higher loads. A set of 10 gives +1 to Climbing skill. Each: \$20, 0.05 lb., AA/40 uses. LC4. Each: \$15, 0.1 lb. LC4.

\subsubsection{Digging Piton}\label{subsubsec:digging_piton}
Advanced genius pitons that use burrow dart technology (p. \pageref{subsec:burrow_darts}) to dig themselves further into the surface. Used properly, a set of 10 gives +2 to Climbing skill (with no loss of speed). Each: \$25, 0.1 lb., AA/10 uses. LC4.

\subsection{Spider Cage}\label{subsec:spider_cage}
A captured device with a starfish~-like shape consisting of a floor base surrounded by a few dozen jointed arms. Stepping on the base triggers a pressure sensor, causing the hinged bars to spring up and bend forward at high speed to form a roofed cage. If the victim is not surprised, a successful Dodge roll allows jumping away in time. The padded bars cause minimal injury, but the closing cage will do 1d-2 crushing damage if the victim is larger than the area of the trap.

A spider cage uses bars of padded memory metal with DR 20. The separation between the bars is 2" wide. It also features a door on the side with a conventional electronic and mechanical lock. It adds a +2 (quality) bonus to Survival skill rolls made for trapping creatures. \$2,000 and 10 lb. for a cage capable of trapping a creature with SM 0. Double cose and weight per +1 SM; halve it for each -1.

\subsection{Splat Piton}\label{subsec:splat_piton}
This two~-inch sphere has a ring attahed for a rope. When broken against rock or another hard surface, a fast~-drying glue is released. In one minute, the ring can safely support 16 tons.

The sphere may be fired out of a mortar as far as 50 yards, unreeling a light line. The line unreeled must be used to pull a climbing rope through the piton ring. If a climbing rope is launched, range drops to 10 yards. A catalyst can unstick the glue, allowing the piton to be removed. It is not reusable. \$10, 0.05 lb. LC4.

\subsection{Vapor Canteen}\label{subsec:vapor_canteen}
This canteen draws moisture from the atmosphere. How quickly it works varies with the amount of water vapor in the air -- with an Earth~-standard humidity of 50\%, the time required to extract a quart of water is three hours. It has a capacity of one quart, and adds a +2 (quality) bonus to Survival skill for an individual living off the land. \$450; 2 lb. (empty) or 4 lb. (full), B/100 quarts. LC4.

\subsection{Vapor Collector}\label{subsec:vapor_collector}
A larger version of a vapor canteen for base camps, settlements, etc. It is 60 times faster, producing one quart every three minutes. It adds a +2 (quality) bonus to Survival skill for a group living off the land. Usually connected to a water tank, but has an internal capacity of 20 quarts. \$10,000, 120 lb., E/30 days. LC4.

\subsection{Smart Rope}\label{subsec:smart_rope}
A cable constructed of memory metal and plastic fibers, or non~-metallic bioplastic; it also includes a radio microcommunicator (p. \pageref{itm:radio_comm_micro}). A smart rope has half the support strength of rope (p. \pageref{subsec:rope}). It gives a +3 (quality) bonus to Knot~-Tying skill, and can be ordered via radio signal to ``flex'' or go ``rigid.''

In flex mode, the rope behaves exactly like ordinary rope. In rigid mode, the rope locks into its current position as if it were a stiff metal wire. In this position, it cannot be untied. Removing a rigid rope without ordering it into flex mode requires cutting through it. If a smart rope is severed, the pieces lose their ``smart'' properties, but retain the flexible or rigid quality the rope had when cut. Smart rope may be purchased in a variety of standard lengths, starting at 1-yard increments. Smart rope is twice as expensive as ordinary rope; other statistics are identical. LC4.

\subsection{Survival Watch}\label{subsec:survival_watch}
A heavy~-duty wristwatch built to survive in extreme environments. It includes a biomonitor (p. \pageref{subsec:biomonitor}), a chronometer, a GPS (p. \pageref{subsec:gps_receiver}) receiver, an inertial compass (p. \pageref{itm:inertial_compass}), a magnetic compass, a homing beacon (p. \pageref{subsec:homing_beacon}), and a tiny computer (p. \pageref{itm:tiny_com}) with a small 2D display (about one square inch).

The watch is voice controlled. It is waterproof, and can survive 10 atmospheres of pressure or a vacuum. It is powered by a small flywheel battery that can be recharged by body motion. If not worn, it goes to sleep for up to five years, turning off all functions except timekeeping. A vigorous shake will power the watch up to full operation. \$300, 0.5 lb., B/3 months. LC4.

\subsection{Survival Module}\label{subsec:survival_module}
A programmable bioplastic box the size of a hardcover book. When activated, it draws air out of the surrounding environment and inflates itself, becoming a comfortable two~-person cabin that can hold four in a pinch. It has transparent plastic windows, pull~-out inflatable tables, chairs and beds, and an airlock door that takes four seconds to cycle. It is pressurized, with a complete life~-support system including an air filter and reducer/respirator. If oxygen is unavailable, air tanks will be required.

The module is light and can be blown away with a strong wind if unoccupied and not weighed or tied down. \$600, 4 lb., C/2 wk. LC4

\section{Environmental}\label{sec:environmental}
%\addcontentsline{toc}{section}{Environmental}
This device is a man~-sized plastic bag with a self~-inflation system and self~-sealing flap. To use it, pull the bag on, and activate the seal. Once the seal is closed, the bag inflates automatically, forming an airtight bubble. Those found in vehicles are often connected by an air hose to an external life~-support system, but they can be disconnected. If this happens, each bubble provides 15 minutes of air.

It takes four seconds to don and inflate the bubble (make a Survival or Vacc Suit skill roll to halve the time). If disconnected, it is flexible enough to move in, at a Move of 1. It floats in water. Bubbles are generally built into vehicle seats or worn on belt packs. Opening it while inflated spills its air. The tough plastic has DR 1. \$600, 3 lb.

The compressed air cartridge can be recharged from a life support system in about an hour, allowing the bubble to be repacked an reused.

\subsection{Worldscaping}\label{subsec:worldscaping}
Technological advancements have made it possible to ecologically engineer an entire planet, terraforming them to transform an uninhabitable planet into one more Earthlike.

This process is still incredibly lengthy and involved and not all planets are suitable candidates. Terraforming a planet like Mars takes about 10,000 years (more if it is notably inhospitable). On the other hand, planets like Venus have yet to be terraformed, modern terraforming industrialists like H.F. Shigura have shown promising results.

\subsection{Gravity Control}
Gravity control technologies create localized zones in which gravity is higher (artificial gravity) or lower (contragravity). 

There are two proven and implemented forms of controlling gravity:

\subsubsection{Gwhel-Generators:}\label{subsubsec:gwhel_generator} 
The exact technology is proprietary to NovaCorp subsidiary Grav~-Tech, these gravity generators manipulate gravity by making use of the Ravens known as Gwhel's unique ability to control gravity. They are also used to generate artificial gravity (or contragravity) in other vehicles or buildings. Area~-affecting gwhel-generators produce a gravity-field based on their rated strength and size. To create up to 1G (or counteract 1G) of gravity costs \$4,000 per yard radius covered. For greater G requirements, multiply this by the difference in gravity needed (i.e. a gwhel~-generator rated to counteract 6G down to 1G, -5G, for up to 10 yards would cost \$200,000.)

Gwhel-generators are also used to create Gwhel~-drives (p. TODO). Gwhel~-drives can also provide artificial gravity to the ship when not using their full acceleration to propel the ship. This is easier for the Gwhel~-drives then their normal task so each 0.1G of acceleration not being used to propel the ship can provide 1 G of artificial or contragravity to the entire ship. Note that when the ship is using all possible acceleration, the gwhel~-drives cannot provide artificial gravity.

Additionally, Grav~-Tech recently unveiled their high-efficiency Gravity Plates, no longer requiring several massive gwhel~-generators to provide a large ship with artificial gravity. They are still rolling out universe wide, but soon they will become a universal fixture. These plates are rated for the cubic volume and strength of the gravity field. \$400, 0.4 lb. per cubic foot of area per +1G. External power. LC3.

\subsubsection{Psi-Gravity}\label{subsubsec:psi-grav}
By making use of some espers' ability to manipulate gravity, psionic researchers proposed and demonstrated the efficacy of manipulating gravity through espers. The problem with this, though, was the rarity of the required espers and the ethical concerns around the need for the espers to provide gravity at all times. 

\subsection{Industrial Megaprojects}\label{subsec:industrial_megaproj}
In addition to terraforming worlds, other massive engineering projects are possible or currently being developed by companies universe wide.

\subsubsection{Moving Black Holes}\label{subsubsec:moving_black_holes}
While rare, the occasional black hole the mass of asteroid have become the focus of industrial research. These million billion ton mini~-black holes are moved by placing a magnetic sail in orbit and feeding matter into it into a controlled fashion to generate thrust.

Additionally, the scientist~-baffling ``blue holes'' \label{note:blue_holes} act like tiny black holes and are moved in a similar way but on a much smaller scale.

\subsection{Weather Control Satellites}\label{subsec:weather_control_sat}
A weather control satellite can shift the weather in a 1,000~-square~-mile region. It cna only make changes that fit within the region's normal climate, such as diverting (or creating) a storm during hurricane season. Roll vs. Meterology skill to control the system; roll weekly for long~-term effects or daily for violent weather. Failure can produce unpredictable results (-3 on rolls to fix them), while critical failure may cause a disaster (-6 on rolls to fix). A weather control satellite must be controlled by a Complexity 8 computer. Long-term support from weather control satellites provides a +3 bonus to Farming skill for raising crops. \$300 million, 4 × 60,000 lb. LC1.

\section{Exploration, Safari, and Salvage Robots}\label{sec:exploration_safari_salvage_robots}
%\addcontentsline{toc}{section}{Exploration, Safari, and Salvage Robots}

\subsection{Robot Mule}\label{subsec:robot_mule}
This is a rugged robot cargo cart that moves on big tires and does what it is told. It has no limbs. It can also be ridden, although the passenger must give it commands. LC4.\\
\textbf{ST} 20; \textbf{DX} 10\*; \textbf{IQ} 6\*; \textbf{HT} 12. \\
\textbf{Will} 10; \textbf{Per} 10; \textbf{Speed} 6; \textbf{Dodge} 9; \textbf{Move} 6.\\
SM -2; \$3,000, 150 lb., D/8 hr. LC4.

\begin{hangparas}{1em}{1}
    \textbf{\textit{Traits:}} Absolute Direction; Accessories (Small computer); A.I.; Automaton; DR 10; Electrical; Ground Vehicle; Machine; No Legs (Wheeled); Payload 2 (16 lb.).

    \textbf{\textit{Skills:}} Area Knowledge-10.
\end{hangparas}
*May be teleoperated.

\subsection{Scout Robot}\label{subsec:scout_robot}
\hfill {\large{\textbf{37 points}}}

This is a compact machine about 10 inches long, usually with a fish~- or bee~-shaped body. Variants are available for different environments. I has a sensor head, two manipulator arms, and a single hardpoint that can be equipped with anything from a camera or searchlight to a carbine or submachine gun.

\begin{hangparas}{1em}{1}
    \textbf{\textit{Attribute Modifiers:}} ST~-7 [-70]; HT+2 [20].
    
    \textbf{\textit{Secondary Characteristic Modifiers:}} SM-4.
    
    \textbf{\textit{Advantages:}} Absolute Direction [5]; Accessory (Tiny Computer) [1]; Doesn't Breathe [20]; DR 15 (Can't Wear Armor, -40\%) [45]; Extra Arm (Weapon Mount, -80\%) [2]; Machine [25]; Protected Hearing [5]; Protected Vision [5]; Radio (Burst, +30\%; Secure, +20\%; Video, +40\%) [19]; Radiation Tolerance 5 [10]; Sealed [15].
    
    \textbf{\textit{Disadvantages:}} Electrical [-20]; Maintenance (one person, weekly) [-5]; Restricted Diet (Very Common, power cells) [-10]; Restricted Vision (Tunnel Vision) [-30].
    
    \textbf{\textit{Availability:}} \$5,000, 4 lb. 2B/8 hrs. LC4
\end{hangparas}
\subsubsection{Configuration Lenses}
\begin{hangparas}{1em}{1}
    Also select one of these lenses:
    
    \textit{Aerial Scout} (+107 points): This uses ducted fans for quiet flight, and is equipped with surveillance sensors. Add Aerial [0]; Enhanced Move 1 (Air) [20]; Flight [40]; Hyperspectral Vision [25]; Parabolic Hearing 3 [12]; Telescopic Vision 2 [10].

    \textit{Submarine} (+62 points): This uses water jets, and is equipped to operate in the ocean depths. Add Enhanced Move 1 (Water) [20], Pressure Support 2 [10], Sonar (LPI, +10\%; Multi-Mode, +50\%) [32]; Aquatic [0].
\end{hangparas}

\subsection{Explorer Swarm}\label{subsec:explorer_swarm}
The most efficient way to perform exploration tasks over a large area may be to saturate it with a swarm of tiny mobile robots. The swarm uses its contact sensors to take minute chemical samples of materials encountered. Explorers may be programmed to look for mineral or chemical traces, explosives, water, organic molecules, etc.

After a predetermined search pattern, the swarms return to a portable lab (p. \pageref{subsec:portable_laboratories}), which may be equippped to collect aand analyze these samples, or beam data out. By analyzing where and when the swarm found items or encountered impassable bariers (such as water, if the swarm cannot swim or fly), a computer can build up a map of the area epxlored. \$500/square yard. LC4.