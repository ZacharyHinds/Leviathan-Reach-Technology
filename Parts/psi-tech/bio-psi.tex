% TODO: Write the limitations of psi-drugs
\section{Psi-Drugs}

\subsection{Blocker}\label{subsec:blocker_psi-drug}
This drug temporarily increases the recipient's natural telepathic resistance, making him less vulnerable to mental intrusion. The user gains Resistant to Telepathy (+3) for six hours, but must roll vs. HT to avoid a headache (treat as moderate pain, p. B428) for the duration. A telepath resists the headache at -3, and gains the defensive benefit at the cost of -3 to \textit{his} rolls against Telepathy skills. Hypo form only. \$100/dose. LC4.

\subsection{Blue Fire}\label{subsec:blue_fire_psi-drug}
This is the nickname of a dangerous neurotransmitter that enhances the brain's ability to handle psionic energy. It's also a deadly poison! Fortunately, its effects are relatively predictable and an antidote exists; thus, many psis are willing to risk taking it when a boost is absolutely necessary.

As soon as a dose of Blue Fire takes effect, the user suffers 2 HP of injury and gains +2 to every psi Talent he possesses. He loses an additional 1d-2 HP (minimum 1 HP) per minute until the drug is counteracted by an antitoxin. Blue Fire \textit{will} eventually kill him if it isn't neutralized. Multiple doses have no extra effect.

A psi under the influence of Blue Fire must make a Will roll at -1 immediately after his first active use of psi (that is, after rolling against a psi skill) and again each minute after that until the drug is neutralized. Failure means he may stop rolling but gains the disadvantage Overconfidence (9), also until the Blue Fire is counteracted. Someone who becomes overconfident will enjoy the sensational high of using his psi abilities and tend to ignore the fact that the drug is eating away at his body.

Neutralizing Blue Fire requires a simple injection of Neurovine, an antidote for nerve toxins (\$30/dose, hypo form only). However, a user who has become overconfident due to Blue Fire's effects must make a Will roll at -1 to convince himself to administer the Neurovine. If he fails, he may roll again every minute.

Blue Fire is available in hypo, inhaler, and pill form. A hypo takes effect instantly, an inhaled dose requires five seconds, and a pill takes 30 seconds. Effects last until neutralized. \$20/dose. LC2.

\subsection{Brainstorm}\label{subsec:brainstorm}
This powerful drug makes it much easier for the user to enhance psi abilities through extra effort (\textbf{GURPS Psionic Powers}, p. 7): Reduce the penalty to his extra-effort rolls from -1 per +10\% increase to -1 per +20\%. Make a HT roll determine duration, which is 10 minutes times margin of success but always at least 10 minutes. Multiple doses have no additional effect.

Brainstorm's benefits come at a potentially severe price, however! First, any critical failure with a boosted ability automatically cripples it rather than requiring a Will roll to avoid crippling. The GM will roll on the table under \textit{Optional Crippling} Rules (\textbf{Psionic Powers}, p. 7) instead.

Moreover, the rush of power is \textit{addictive}. Anyone who exploits Brainstorm to boost an ability has to roll vs. HT to avoid addiction. Failure means that he must have at least one dose every day. Otherwise, he suffers the effects of withdrawal from a physiological dependency (p. B440) and one of his psi abilities acquires the Uncontrollable limitation (unless this was already the case) for the duration of his addiction.

Hypo form only. Takes effect in two seconds. \$150/dose. LC3.

\subsection{Catalyst Drug}\label{subsec:catalyst_drug}
This is a massive dose of psi-boosting drug combined with a specialized psychotropic agent. Anyone who takes it must roll vs. HT-3. Success means no effect. Failure results in 2d hours of unconsciousness, during which the victim experiences bizarre, often terrifying hallucinations; treat this as a Fright Check on awakening, at -1 per hour spent unconscious. Unless the user is a latent esper -- that is, someone with Talent but no abilities -- there's no further effect.

If the recipient \textit{is} a latent psi, however, he may manifest psi abilities paid for with unspent points, bypassing any normal limitations on such purchases in the campaign. The user may take new mental disadvantages -- in addition to anything the Fright Check inflicts -- and spend the associated points on more psi capabilities. In all cases, abilities gained this way often match the drug-induced visions; e.g., a nightmare of burning could catalyze Pyrokinesis.

These drugs are strictly controlled. Available as a hypo (takes effect in 30 seconds), inhaled form (takes a minute), and pill form (takes five minutes). \$3,000/dose. LC1.

\subsection{Mind Hype}\label{subsec:mind_hype}
A reliable psi-boosting drug, Mind Hype focuses the user on his inner self, dissolving the boundaries between body and mind. He gets +1 to IQ, raising Will, Per, and \textit{all} skills based on IQ, Wil, or Per. \textit{Psi} skills are at +3 instead of merely +1. To determine the duration of  these bonuses, make an HT roll; benefits last for 10 minutes per point of success, but always at least an hour. More than one dose has no noticeable effect.

However, each dose taken requires an IQ roll, at -1 per dose past the first in a 24-hour period. Failure means the user has difficulty concentrating on less-abstract things -- e.g., walking across a room. He has -2 to DX and all DX-based skills. Moreover, for anything not having to do with psionics or with abstract reasoning, he suffers Absent-Mindedness (p. B122). If he already has this disadvantage, he makes all rolls to remember trivial things or sustain interest in boring activities at an additional -3! These side effects last for 1d hours.

Mind Hype isn't addictive. Only available in a hypo form. Takes effect in 10 minutes. \$100/dose. LC3.

\subsection{Muffler}\label{subsec:muffler}
This drug works to shut down all of the subject's psionic neural pathways. He must roll vs. HT-4 on being injected. Failure means that he cannot exercise \textit{any} psi powers for (25 - HT)/4 hours, minimum one hour. Hypo form only. \$800/dose. LC2.

\subsection{Psi-Boosters}\label{subsec:psi-booster}
Psi-boosters are a \textit{series} of designer pharmaceuticals which are engineered to induce a specific psi ability; e.g., psi-booster (Telereceive). A psi-booster has no effect on a user lacks the underlying power (such as Telepathy for Telereceive). If he has that power, he gains the ability in question -- at level 1, for a gift with more than one level. If he already possess the specific ability, he gets a 50\% increase in level (per the effect of extra effort), rounded down, if it comes in levels, or a free +50\% enhancement if it doesn't. In all cases, duration is HT/2 minutes.

Psi-boosters are reasonably safe when taken in moderation. The only side effect of a single dose is that the strain reduces the user's Talent for the power associated with the ability gained or boosted, lowering it by a level for a day afterward. Minimum Talent is 0.

However, taking two or more doses of the same psi-booster inside of eight hours is dangerous! An additional dose within that period provides no extra enhancement beyond extending duration if the drug is still active, and requires a HT-4 roll (HT-8, if the drug's effects are still active). Failure inflicts 1d HP of injury times the margin, in the form of a brain hemorrhage. This high risk of potentially fatal complications means that a psi-booster can be used as a deadly -- if costly -- poison by delivering two doses or transmitting a second dose to a dosed victim.

Psi-boosters are available in hypo and inhaler form; either takes effect immediately. Cost per dose is \$100 times the point cost of the ability provided. For an ability that comes in multiple levels, use the cost of the first level. LC2

\subsection{Shatter}\label{subsec:shatter}
Anyone with any psionic power -- even a latent -- who uses this drug becomes tipsy (p. B428). This affliction lasts for minutes equal to the total point value of his psionic Talents and abilities. After that, he may roll vs. HT every minute. Success means the drug wears off.

Using psionic powers while under the influence of Shatter risks worsening its effects to severe vertigo and nausea. After attempting any psi skill roll, the user must make an immediately HT-3 roll to avoid becoming nauseated \textit{and} drunk (p. B428) for minutes equal to the margin of failure. Multiple doses of Shatter increase the effects' duration but not their severity. Too much of it is toxic, however. Each dose past the first in a 24-hour period inflicts 1d-3 points of toxic damage (minimum 1) on the psi.

Individuals without psi powers suffer \textit{none} of these effects. Because of this, Shatter can be used to ``test'' for psionics. A psi who's affected by it must make a Will-2 roll to avoid reacting visibly (showing signs of dizziness, sweating, etc.), even if he doesn't use his powers. Law-enforcement agencies more typically use Shatter to restrain \textit{known} psis.

Available forms are hypo (can be delivered by darts shot from any weapon capable of firing drugged rounds), gas (in spray cans or any of the biochemical aerosol warheads on p. \pageref{subsec:biochemical_aerosol}), and pill. Hypo form works instantly; gas or pill form takes effect in 10 seconds. \$20/dose. LC2.

\subsection{Trance}\label{subsec:trance}
The opposite of drugs like Blocker (p. \pageref{subsec:blocker_psi-drug}), this drug \textit{lowers} the user's natural psi resistance. This makes Trance useful to telepathic psychologists -- and interrogators. As it can render subjects more susceptible to mental control, it may gain a sinister reputation. Paranoid people might fear that groups employing psis are putting it in the water supply!

Each dose of Trance gives the subject -2 to Will to resist Telepathy abilities. Up to three doses are cumulative, but this can be dangerous. After taking each dose, the user must make a HT+3 roll, at -1 per dose past the first. Failure means he lapses into a coma (p. B429) \textit{and} suffers HP of injury equal to his margin of failure.

Trance is available in pill and hypo form. Either takes effect in 10 seconds and lasts six hours, regardless of dosage. \$15/dose. LC3.

\subsection{Window}\label{subsec:window_psi-drug}
Given to someone with the Telepathy power, this drug enhances his gifts, granting +2 to Telepathy Talent. However, it also dissolves his normal psychic barriers, giving him the Supersensitive disadvantage (p. B158) \textit{and} completely suppressing any psionic Mind Shield advantage (p. B70) he may possess. Psis who take it unwillingly may roll vs. HT-3 to resist. The effects last for (25 - HT)/4 hours, minimum one hour.

Window is available in hypo, inhaler, and pill form. \$250/dose. LC2.

% \section{Psibernetics}
\section{Psibernetics and Surgery}

\subsection{Brain-Tissue Graft}
This is a risky and controversial method of giving someone ``artificial'' psi powers. Brain tissue from the parts of the brain responsible for psionic powers is selectively removed from a donor who is a known esper and is transplanted into a recipient.

The donor psi can be dead (if he isn't, this procedure kills him), but \textit{not} from damage to the brain. If deceased, he must have been put on life support within four minutes of death. Brain tissue from a single donor is good for 1-3 grafts (roll 1d6/2 round up), +1 per full 50 points of psionic abilities, Talent, and skills he had. It \textit{is} possible to use cloned neural tissue, allowing one donor can theoretically provide infinite grafts.

Transplantation takes 24 hours and and gives -5 to Surgery. See \textbf{GURPS Bio-Tech} pp. 135-140 for the complete surgery rules. If the surgery roll fails, the patient loses a point of IQ permanently (1d+1 points on critical failure) and gains nothing.  Regular success means he acquires one level of Talent in up to 1d/2 powers that the donor had; as a latent psi, he can go on to learn actual abilities. Critical success also gives him one of the donor's psi abilities (GM chooses) -- at level 1, if it comes in levels.

Brain tissue doesn't suffer from rejection, but the subject must make a HT+4 roll to assimilate it. On any success, there's no special effect. Failure results in the permanent loss of points of IQ equal to the margin. Critical failure means death!

\subsection{Drones}
A ``drone'' is a human whose brain has been altered -- through an ultra-tech surgical or chemical ``brainwipe'' -- to have as little ego or creativity as possible while retaining a minimum level of useful intellect. The result is someone with an adult IQ of 8, but Will 3 and Slave Mentality. Further tinkering adjusts the brain's biochemistry to put the subject in a permanent hypnagogic state (in effect, he lives in a waking dream), making him unusually receptive to Telepathy: +3 on all rolls against Telepathy skills to influence or contact him.

For tasks that don't require individuality or initiative, drones make ideal servants for telepathic masters. They tend to be hard workers with little or nothing in the way of personality, unless encouraged to develop surface mannerisms through rote. Drones are usually motivated through conditioning or direct telepathic control.

Drones can be produced surgically. Using modern surgical techniques, a living person may be reduced to drone status. Adjust skills accordingly. The victim may have brief moments of lucidity, but he's effectively a zombie most of the time. The operation costs \$10,000. This process is LC1, and usually legal only if performed on individuals who have no civil rights.

It is also possible to use cloning and genetic modification to \textit{grow} a drone. %Use the standard rules for customized clone bodies in the setting.

\subsection{Neuro-Psi Implants}
A neuro-psi implant is tiny device packed with psychotronic microcircuitry. It goes in the brain, where it works to produce psionic powers artificially by exciting the growth of deeply buried embryonic psi abilities too weak for psionic testing or training to discover. In some settings, implantation is as trivial as ear-piercing; in others, it involves genuine brain surgery, requiring two hours and a Surgery roll at an extra -2, with failure causing 2d HP of injury.

The implant requires a day or so to take effect, during which time the recipient feels mildly dizzy (-1 to DX and DX-based skills). After that, he must roll against Will at +2. Success means that the implant has stimulated the development of a psi power -- or \textit{two} powers, on a critical success. Regular failure indicates that the patient gains nothing.

On a critical failure, though, the subject's brain proves incompatible with neuro-psi implants, and he'll suffer problems. Make a HT-2 roll to determine the severity of these effects. Any success (even a critical success) on this roll means the patient experiences a seizure (p. B429) that lasts for 3d seconds and costs him 1 FP per second. Failure further inflicts HP of injury equal to the FP loss -- and worse, brain damage that reduces his IQ by one. On a second critical failure, he dies!

Regardless of the outcome, only one attempt is allowed. Further neuro-psi implants have no effect on those who've already received them. If the subject does gain a power, roll 2d once (twice, on a critical success) to determine what he gets:\\

\begin{hangparas}{1em}{1}
    \textbf{2 --} Astral Projection
        
    \textbf{3 --} Psychokinesis

    \textbf{4 --} Psychic Vampirism

    \textbf{5 --} Psychic Healing

    \textbf{6 --} Telepathy

    \textbf{7 --} ESP

    \textbf{8 --} Telepathy

    \textbf{9 --} Anti-Psi

    \textbf{10 --} Psychokinesis

    \textbf{11 --} Ergokinesis

    \textbf{12 --} Teleportation or other power (e.g., Biokinesis)\\
\end{hangparas}

The patient gains 1d character points of abilities or perks in that power; if necessary, add appropriate limitations to reduce the cost. The subject can learn psi skills and improve his abilities normally with experience, but all abilities gained -- both initially and from later experience -- only function while the implant is worn. Buy them with gadget limitations (pp. B116-117): Breakable, DR 2, SM -9, -20\% and Can Be Stolen, Must be forcefully removed, Won't work immediately for thief, -5\%, worth a net -25\%. The implant has HP 1. If someone with implant-generated psi later gains psi powers by any other method, he loses the implant's benefits.

Neuro-psi implants only work on people without natural psi powers. If someone who has any psionic potential -- latent or fully realized -- receives such an implant, he suffers the effects of a critical failure on the Will+2 roll described above. He must immediately make the HT-2 roll to determine the severity of the effects!

A neuro-psi implant is \$15,000, plus \$2,000 for the operation. LC3.

\subsection{Psibernetic Implants}
Psychotronic devices may be implanted in a subject to give him artificial psi abilities. These implants \textit{do not work} for those with psi abilities or latent psi abilties. A psibernetic implant has the same stats and point cost as a ``natural'' psi ability, but adds three modifiers (which add up to 0\%, so there's no change to cost):

\begin{itemize}
    \item \textit{Reliable +5, +25\%.} This gives +5 to skill rolls to use the ability, representing the implant's innate psi-amplifying abilities. See \textbf{GURPS Powers}, p. 109.
    \item \textit{Temporary Disadvantage, Electrical, -20\%.} The ability is vulnerable to electrical surges, power drains, etc. See p. B134.
    \item \textit{Temporary Disadvantage, Maintenance, 1 person, Weekly, -5\%.} The implant calls for a weekly checkup and maintenance procedure using Electronic Operations (Psychotronics). This does not require removing the device. The maintenance generally involves replacing power cells and performing diagnostic scans. See p. B143.
\end{itemize}

Physically, psibernetic implants go in the \textit{brain}. Assume that they have some form of tiny skull port -- hidden under the scalp, in the neck, etc. -- to allow maintenance. While they're made of non-ferrous materials, a careful physical examination or diagnostic scan might find them; this requires a Diagnosis roll, which is at -5 if not deliberating looking for implants.

Installation or removal of a psibernetic implant is major brain surgery! The procedure requires two hours and a roll against Surgery (p. B223) at an extra -2. Failure inflicts 2d HP of injury; critical failure doubles this or adds other, more-exotic side effects.

Recovery time is two weeks, halved if the Surgery roll was a critical success. Also halve recovery time \textit{and} any injury if using TL9+ robotic surgical instruments. And due to the synergy between the subject's mind and body, reduce the final recovery time of a \textit{successful} operation by 10\% per level of appropriate psionic Talent he possesses (e.g., Telepathy Talent, for a Telesend implant).

The recovery time must pass before a repeated attempt is possible (if the operation failed) or before the implant becomes functional (if it succeeded). Optionally, in the last half of the recovery period, treat the implant as functional but possessing a temporary limitation such as Emergencies Only -- or Unconscious Only and Uncontrollable, if that would be more interesting. This wears off at the end of the recovery time.

\textit{Psibernetic Implant:} Start with any psionic ability and add Reliable +5, +25\%; Temporary Disadvantage, Electrical, -20\%; and Temporary Disadvantage, Maintenance, 1 person, Weekly, -5\%. Cost is \$10,000 ~x ability's point cost. Most psibernetic implants are LC2, but those with obvious espionage or offensive applications are LC1.

\section{Psiborgs}
``Psiborgs'' are disembodied animal brains -- or human ones -- encased in life-support systems and cybernetically linked to powerful psi-amplifiers. They're encouraged by direct neural stimulation to produce a single psionic power and to obey their masters. Unlike standard cyborgs (p. \pageref{sec:cyborgs}), they're enslaved constructs with little or no volition. Still, while a psiborg's only thoughts might be of its duties, torment, or drug-induced bliss, it's a living creature that can be detected psionically and affected by Telepathy, Psychic Vampirism, and so on.

A psiborg requires a brain with at least some psionic potential. At a minimum, this may involve genetic manipulation or drug experiments to grow or create an animal brain with a latent psi power. More powerful psiborgs might call for transplanted or vat-grown human or Raven brains, possibly with fully realized psi abilities.

A basic psiborg consists of a brain housed within a ``psiborg box'' -- a self-contained life-support unit with monitoring equipment. Psiborgs are given orders and communicate telepathically, or via chemical or electrical psychotronic systems designed to simulate the brain directly in order to enforce compliance and amplify the subject's psi abilities. Some also have built-in terminals that allow communication with or information display for use by non-telepaths.

All psiborgs have the following meta-trait to reflect this status quo:\\

\textit{Psiborg:} A brain-in-a-box, with a reserve of energy for powering psionic abilities. Blindness [-50]; Deafness [-20]; DR 5 [25]; Electrical [-20]; Energy Reserve 16 (Psi) [48]; Fearlessness 5 [10]; Injury Tolerance (No Neck) [5]; Less Sleep 4 [8]; Machine [25]; Maintenance (Electronics Repair; 1 person; Weekly) [-5]; Mute [-25]; No Legs (Portable)* [-30]; No Manipulators [-50]; No Sense of Smell/Taste [-5]; Numb [-20]; Reduced Consumption 2 [4]; Reprogrammable [-10]; Restricted Diet (Nutrient pastes and fluids; Occasional) [-30]; Sealed [15]; Slave Mentality [-40]; and Unattractive [-4]. \textit{-169 points}.\\

* No Legs (Portable) is a less-extreme form of No Legs (Sessile). The character can be carried, worn, or implanted for easy transportation, but has -6 to DX where legs are needed, as for Lame (Legless). %See Transhuman Space: Changing Times, p. 42.

Most psiborgs also have ST 0 and HP appropriate to their mass. Four typical models appear below; in all cases, maintenance costs are 1\% of purchase price per week (for the special serums and nutrients), and an unmaintained psiborg will die in 1d days. These psiborgs can serve as worked examples for the GM who wants to create custom models -- including ones where the disembodied brain has unique psi abilities or a larger Energy Reserve, is built into a mobile cybernetic body (alter No Legs as needed), or retains its personality (remove Slave Mentality, and probably Reprogrammable).

\subsection{Watchdog}
This is a simple psiborg that detects the presence of telepathic activity. It uses a rat brain sensitized to telepathic impressions, encased in a life-support unit. The braincase and life-support machinery are about the size of a large briefcase. Attached to them is a monitor that shows the watchdog's brainwave fluctuations. The brain is super-sensitive to psionic impressions, and becomes painfully agitated if anyone nearby is the target of or uses any Telepathy ability. Through telepathic conditioning, it can be ``introduced'' to friendly psi signatures and trained to exclude them. A skilled observer can interpret the watchdog's brainwave fluctuations to deduce the type of intrusion; make an Electronics Operation (Psychotronics) roll to determine the specific Telepathic ability at work.\\

\begin{hangparas}{1em}{1}
    \textbf{ST} 0; \textbf{DX} 8; \textbf{IQ} 4; \textbf{HT} 8.
    \textbf{HP} 11; \textbf{Will} 10; \textbf{Per} 10; \textbf{Speed} 4.00; \textbf{Dodge} 0; \textbf{Move} 0.
    SM -4 (1 hex); 20 lbs.

    \textbf{Traits:} Accessory (Monitor); Anti-Psi Talent 3; Domestic Animal; Psi Sense 4; Psiborg.
    
    \textbf{Skills:} Psi Sense-16.
    
    \textbf{Techniques:} Exclusion (Psi Sense)-16.
    
    \textbf{Cost:} \$100,000. LC2.
\end{hangparas}

\subsection{Guardian}
This more-sophisticated development of the watchdog (above) uses a higher animal brain -- typically that of a predator of some sort, with IQ 4-5 but Will 12. The psiborg is designed to react painfully to uses of psi power within 100 yards, using Anti-Psi to strike back with Screaming or to foil attackers by creating a Reflective Shield. As Screaming also blinds the guardian to the target's location, it will usually stop after a few seconds and only resume if it detects further psi use within range.\\

\begin{hangparas}{1em}{1}    
    \textbf{ST} 0; \textbf{DX} 10; \textbf{IQ} 5; \textbf{HT} 8.
    
    \textbf{HP} 16; \textbf{Will} 12; \textbf{Per} 10; \textbf{Speed} 4.50; \textbf{Dodge} 0; Move 0.

    SM -2 (1 hex); 60 lbs.

    \textbf{Traits:} Anti-Psi Talent 4; Domestic Animal; Psi Sense 3; Psiborg; Psionic Shield 3; Screaming 5.

    \textbf{Skills:} Psi Sense-14; Psionic Shield-14; Screaming-12.

    \textbf{Techniques:} Exclusion (Psi Sense)-14; Reflective Shield (Psionic Shield)-14; Tiring Scream-12.

    \textbf{Cost:} \$300,000. LC2.
\end{hangparas}

\subsection{Inquisitor}
The inquisitor is conditioned to sift through another being's mind. A bulbous central module holds a disembodied human brain, psi-amplifying circuitry, and a complex life-support system. Half a dozen tubes radiate out from this, attaching it to cooling systems and nutrient pumps. Two long sensor cables end in electrodes, ready to be placed on the subject's head. The psiborg is crowned by a monitor screen to display the subject's thoughts. It also has a voice synthesizer and a microphone that enable it to communicate verbally with its masters.

Attached to a subject and activated, the inquisitor will relentlessly probe his mind. Any surface thoughts the psiborg picks up are translated into text and visual form on the monitor screen as they occur to the subject. Unless given orders to hunt for something specific, the psiborg merely displays whatever the subject was thinking about when he was probed. If an inquisitor is attacked, it's trained to generate a Psionic Shield to defend itself, trap its attacker in a cage of thought, and then repeatedly stab his mind to death. It will not otherwise use Mental Stab (as it has trouble locating victims).\\


\begin{hangparas}{1em}{1}
    \textbf{ST} 0; \textbf{DX} 10; \textbf{IQ} 10; \textbf{HT} 9.
    
    \textbf{HP} 17; \textbf{Will} 13; \textbf{Per} 10; \textbf{Speed} 4.75; \textbf{Dodge} 0; \textbf{Move} 0.
    
    SM -1 (1 hex); 70 lbs.
    
    \textbf{Traits:} Accessory (Monitor); Mental Stab 3; Psiborg; Psionic Shield 2; Telereceive 2.
    
    \textbf{Skills:} Mental Stab-14; Psionic Shield-16; Telereceive-16.
    
    \textbf{Techniques:} Deep Probe (Telereceive)-16; Mind Trap (Psionic Shield)-16.
    
    \textbf{Cost:} \$350,000. LC1.
\end{hangparas}

\subsection{Sponge}
A sponge is a telepathic receiver and storage system the size of a bulky equipment case. It consists of a living brain (sometimes taken from a psionically gifted child) with boosted Telereceive capabilities, conditioned to act exclusively as a telepathic recorder. Sponges are most often used for surveillance, and occasionally as secretaries.

Once activated, the psiborg reads all unshielded surface thoughts within range and records them in its brain. The radius is global -- a sponge hidden in a basement could read minds on the upper floors. It can store up to 100 man-hours of thoughts before it shuts down. These can be played back on the screen of an inquisitor (above) or scanned by anyone with Telereceive. Individual minds come across as distinct mental ``voices''; once a mind reader concentrates on a specific person, there's no difficulty ``hearing'' that mind over the crowd.

\begin{hangparas}{1em}{1}
    \textbf{ST} 0; \textbf{DX} 10; \textbf{IQ} 8; \textbf{HT} 9.
    
    \textbf{HP} 14; \textbf{Will} 7; \textbf{Per} 10; \textbf{Speed} 4.75; \textbf{Dodge} 0; \textbf{Move} 0.
    
    SM -4 (1 hex); 40 lbs.
    
    \textbf{Traits:} Photographic Memory (limited to 100 hours); Psiborg; Telereceive 5 (Shallow).
    
    \textbf{Skills:} Telereceive-20.
    
    \textbf{Techniques:} Multiplicity-20.
    
    \textbf{Cost:} \$200,000. LC2.
\end{hangparas}