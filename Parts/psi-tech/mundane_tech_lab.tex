In this case, \textit{mundane technology} refers to items that aren't psychotronic superscience but that are useful to psis or those researching psionics. 

\section{Psi Testing Equipment}
The portable labs outlined on p. \pageref{subsec:portable_laboratories} can also be specialized for the study of psionics, specifically the Expert Skill (Psionics). Possible equipment included in these labs include tools such as Zener cards (flash cards with symbols that those with Telepathy can guess), electronic number generators (for testing ESP and Probability Alteration), biological tests for Psyhic Healing, and so on. 

The costs the same and provides the same bonuses.

One way this equipment can be used is to test a subject's psionic abilities. For each test session, roll against the researcher's Expert Skill (Psionics) at +4 or unmodified Psychology. Failure means the subject comes up negative. Otherwise, \textit{if the subject is cooperating}, he may then roll against IQ, plus any Talent for the power question. (\textit{Exception:} the ability being tested for normally requires a Per or Will roll, use that score instead.) On a success by 10+ or a critical success, he demonstrates his latent gift in a measurable way. If he isn't cooperating, then the experiment fails -- although it may be possible to trick him into revealing powers.

\section{Sensory Deprivation}\label{sec:sensory_deprivation}
Sensory deprivation techniques isolate the subject from external stimuli, serving to focus his mind inward. Intended for psychiatric testing or therapy, they can also be used as a meditation aid and a focusing tool by psis prior to operations where a high level of skill is necessary -- or to make a subject (sometimes a prisoner) especially receptive to mental intrusion. 

After an hour of sensory deprivation, the user must roll against the higher of Will or Meditation. Success indicates that the isolation has successfully focused his mind; he gets +1 to use psi skills while the sensory deprivation endures, plus an additional bonus equal to 1/3 of his margin of success (rounded down), to a maximum of +5. Failure means that he gains no benefit. If he remains in isolation, he may roll again once per hour, but further rolls are at a cumulative -1 (although never worse than -5).

There's also some danger. If the subject fails this roll by 5+, or critically fails, then the prolonged sensory deprivation hasn't merely failed to increase his concentration -- the isolation and loss of sensation have worked to unhinge his mind! Treat this as a failed Fright Check. Roll 3d on the Fright Check Table (p. B360) and add his margin of failure. Quirks or Phobias (p. B149) acquired this way are liable to relate to the sensory deprivation; Autophobia, Claustrophobia, and Scotophobia are especially appropriate.

Regardless of his roll, the subject is also more open to any form of telepathic contact. He has no physical sensations to distract him! Every hour spent in sensory deprivation gives +1 to all skill rolls for Telepathy abilities (friendly or hostile) used on him, to a maximum of +5 after five or more hours.

Various types of sensory deprivation equipment, most notably a sensory deprivation tank (p. \pageref{itm:sensory_deprivation_tank}). When using a tank, additional psi~-tech may be used, but it must be designed to be waterproof (1.2× cost, or included in rugged, p. \pageref{subsec:rugged}). Total VR (p. \pageref{subsec:total_vr}) can also be used with a properly designed virtual environment, though the virtual nature makes it difficult to use any additional psi~-boosting tech, any bonuses are halved without software to incorporate them into the VR (generally at least Complexity 6, depending on specific needs).

\newcommand{\eeg}{Electroencephalography}
\section{\eeg (EEG)}\label{sec:eeg}
\textit{\eeg} is the recording of spontaneous electrical activity (“brainwaves”) produced by the firing of neurons within the brain. Its primary clinical use is to diagnose neurological abnormalities such as epilepsy. However, EEG is also used for cognitive research and brain-computer interfaces.

An EEG machine has several components: sensing electrodes, amplifiers, a computer control system, and a display device. T

Old EEG machines had several components: sensing electrodes, amplifiers, a computer control system, and a display device. Modern EEGs are much more compact, even built into diagnostic beds (\pageref{subsec:diagnostic_bed}), medscanners (\pageref{subsec:medscanner}), and HyMRI scanners (p. \pageref{subsec:hymri_scanner}).

Modern scientific understanding of psionics has allowed for the identification os esper abilities simply by analyzing EEG readings. Databases of such information are standard with most diagnostic beds, but will need to be acquired separately for medscanners. If the scanner has access to such a database, a successful Electronics Operation (Medical or Scientific) roll determines what power is involved (e.g., Telepathy). Success by 5+ identifies the specific ability (e.g., Telesend). Furthermore, most databases also are able to detect when a subject is being \textit{affected} by a psi power that induces mental effects: Psychic Vamprisom, Telepathy, Dream Control, etc.

% \section{EM Field Sensor}
\section{Psis and HyMRIs}
HyMRI scanners (p. \pageref{subsec:hymri_scanner}) can also function as EEGs (above) and can also map the exact areas of the brain that are active during psi use. They can also be useful for locating brain~-tissue grafts (p. TODO), other psi implants, or where to target an electro~-psionic neutralizer (p. TODO).