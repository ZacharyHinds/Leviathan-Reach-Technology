\section{Materials}\label{sec:def_materials}

\subsection{Smart Bioplastic}\label{subsec:smart_bioplastic}
This is a tough, flexible, pseudo~-alive smart material. Every square inch of it contains electrically~-active muscles, fibers, and nerve endings. A coded electrical impulse can command these muscles to move, allowing an item constructed of bioplas to change its shape.

\subsection{Threat Protection}\label{subsec:threat_protection}
High DR doesn't provide much defense against chemical weapons, great heat, microbes, and so on. These dangers demand specialized protection that corresponds to particular advantages. Below are several common classes of ``threat protection'' used in descriptions of protective gear:

\textit{Climate Control:} The equipment provides protection against climatic extremes equivalent to the Temperature Tolerance advantage. Climate control system remove waste heat as well as providing insulation and air conditioning. They extend the wearer's comfort zone to the range noted. If the suit is not sealed, treat as if it were merely air conditioned and insulated. If the wearer's own comfort zone is greater, the equipment may fail before its user does!

\textit{Air Supply:} The equipment provides air for the wearer. The air supply times listed are an approximation and assume an external pressure of one atmosphere or less. For game purposes, assume that the standard applies to most adults, while children under 12 consume half as much. It takes 10 seconds to hook up a tank and two seconds to jettison it. Air refills are \$5 per hour, but most vehicles with life support systems incorporate air compressors that can top them up for free.

\textit{Glare~-Resistant:} The equipment screens out bright light. It is equivalent to Protected Vision (p. B78), and works against deleterious effects of ``dazzle,'' ``flash,'' and ``strobe'' weapons.

\textit{Hearing Protection:} The equipment screens out noise, and is equivalent to Protected Hearing (p. B78).

\textit{Radiation PF:} The equipment has a radiation Protection Factor. Divide radiation (when listed in rads) by the PF before applying its effects, as if the user had Radiation Tolerance (p. B79).

\textit{Pressurized:} The equipment is resistant to pressures greater than one atmosphere. Pressurized comes in three levels, each equivalent to a level of Pressure Support (p. B77). This protects against crushing ocean depths and superdense atmospheres like thos of Venus and Jupiter.

\textit{Sealed:} Impervious to penetration by liquids and gases. This corresponds to the Sealed advantage (p. B82). It prevents all harm from noncorrosive bioweapons, chemicals, and nano, as well as ordinary rust and waterlogging.

\textit{Vacuum Support:} Protects the wearer or occupants from the deleterious effects associated with vacuum and decompression (other than lack of air). This corresponds with the Vacuum Support advantage (p. B96)

\section{Body Armor}\label{sec:body_armor}
These unpowered suits and armored garments require no special skill to use.

All desired body armors are sold in any number of colors and patterns (including camouflage). It takes three seconds per piece to don or remove most body armor.

\subsubsection{Body Armor Styles}\label{subsubsec:body_armor_styles}
Body armor comes in a variety of styles:

\textit{Bodysuit:}\label{itm:bodysuit_body_armor} This outfit covers the torso, groin, arms, and legs. The neck, head, or extremities are uncovered, making it easy to add customized boots, gloves, and helmets.

\textit{Gloves:}\label{itm:gloves_body_armor} A pair of armored gloves. They're made of thinner material than other armor types to avoid compromising the wearer's manual dexterity. 

\textit{Jacket:}\label{itm:jacket_body_armor} This is a heavier outfit that covers the torso and arms. It zips up and has plenty of pockets.

\textit{Suit:}\label{itm:suit_body_armor} This head~-to~-toe outfit includes a hood and face mask with eye slit. It is used as the basis for tailored armor (p. TODO).

\textit{Trousers:}\label{itm:trousers_body_armor} A pair of long pants, protecting the groin and legs (but not the feet). It is not obviously armor, and can pass for a normal pair of work pants or jeans.

\textit{Vest:}\label{itm:vest_body_armor} A sleeveless t~-shirt covering the torso.

All concealable armor styles can pass as normal clothing, although bodysuits and complete suits are likely to be conspicuous. 

Additional styles can be created using the tailored armor (p. TODO) rules.

\subsection{Ballistic Armor}\label{subsec:ballistic_armor}
This armor uses flexible materials to resist high velocity projectile attacks as well as cutting blows. It is the modern successor to ancient bulletproof vests. 

Ballistic armor is flexible with a split DR: it provides full protection against piercing and cutting attacks, and uses its reduced DR against all other types of damage.

Advances in material technology has led to improvements in ballistic armor. Modern ballistic armor is generally \textit{nanoweave}. This armor is a fabric woven from para~-aramid fibers, polyethylene, or sythetics inspired by the molecular structure of spider silk. This fabric is then reinforced by woven carbon nanotubes. It can be fitted with various accessories, using ``smart'' properties that can be engineered into it.

\multicolinterrupt{
    \hrule height 1pt
    \subsection{\textit{Ballistic Armor Table}}
    \begin{table}[H]
        \centering
        % \caption*{Nanoweave Ballistic Armor Table}
        \rowcolors{1}{}{\colordefenses}
        \begin{tabularx}{\linewidth}{lXcXcXrXcXcXr}
            % \hline\\
             \textbf{Armor} && \textbf{Location} && \textbf{DR} && \textbf{Cost} && \textbf{Weight} && \textbf{LC} && \textbf{Notes} \\
             % \rowcolor{blue!10}
             Nanoweave Bodysuit && body, limbs && 18/6* && \$900 && 6 && 3 && \\ 
             Nanoweave Gloves && hands && 9/3* && \$30 && neg. && 4 && \\
             Nanoweave Jacket && arms, torso && 18/6* && \$450 && 3 && 3 && \\
             Nanoweave Suit && all && 18/6* && \$1,200 && 8 && 3 && \\
             Nanoweave Trousers && groin, legs && 18/6* && \$280 && 2.8 && 3 && \\
             Nanoweave Vest && torso && 18/6* && \$300 && 2 && 3 &&
        \end{tabularx}
        \label{tab:concealable_ballistic_armor}
    \end{table}
    * Flexible.
}

\subsection{Tactical Vest}\label{subsec:tactical_vest}
This thick, sleeveless, jacket~-like vest covers the torso and groin, with front and back pockets for inserting rigid ceramic or alloy plates. 

A tactical vest is made of similar materials to concealable body armor, but is heavier, and is obviously body armor. It provides full protection against cutting and piercing damage, and reduced protection against all other attacks. Its trauma plates provide full protection against all damage types. It takes three seconds to insert or remove the plates.

\multicolinterrupt{
    \hrule height 1pt
    \subsection{\textit{Tactical Vest Table}}
    \begin{table}[H]
        \centering
        % \caption*{Nanoweave Ballistic Armor Table}
        % \rowcolors{1}{}{\colordefenses}
        \begin{tabularx}{\linewidth}{lXcXcXrXcXcXr}
            % \hline\\
             \textbf{Armor} && \textbf{Location} && \textbf{DR} && \textbf{Cost} && \textbf{Weight} && \textbf{LC} && \textbf{Notes} \\
             \rowcolor{\colordefenses}
             Nanoweave Tactical Vest && torso, groin && 24/10* && \$900 && 9 && 2 && \\
             \rowcolor{\colordefenses}
             \textit{trauma plates} && torso && +46 && +\$600 && +9 && 2 && 
        \end{tabularx}
        \label{tab:tactical_vest}
    \end{table}
    * Flexible.
}

\subsection{Assault Boots}\label{subsec:assault_boots}
These armored boots add metal or ceramic inserts to the sole of a ballistic fiber. They provide their full DR against attacks to the underside of the foot (e.g., stepping on a stake, a contact~-detonation mine, etc.) but half DR against attacks from other angles.

\textit{Hiking:} Modern combat boots incorporate smart~-matter responsive fabrics and biomaterials that treat or prevent blistering from long marches. They count as the best quality equipment and add +5 to Hiking skill.

\multicolinterrupt{
    \hrule height 1pt
    \subsection{\textit{Assault Boots Table}}
    \begin{table}[H]
        \centering
        % \caption*{Nanoweave Ballistic Armor Table}
        \rowcolors{1}{}{\colordefenses}
        \begin{tabularx}{\linewidth}{lXcXcXrXcXcXr}
            % \hline\\
             \textbf{Armor} && \textbf{Location} && \textbf{DR} && \textbf{Cost} && \textbf{Weight} && \textbf{LC} && \textbf{Notes} \\
             % \rowcolor{blue!10}
             Assault Boots && feet && 18/9 && \$150 && 3 && 4 && \\ 
        \end{tabularx}
        \label{tab:assault_boots}
    \end{table}
}

\subsection{Laser~-Resistant Body Armor}\label{subsec:laser-resistant_body_armor}
These forms of armor are optimized to counter laser weaponry.

\subsubsection{Ablative Nanoplas}\label{subsubsec:ablative_nanoplas}
This is similar to nanoweave armor, but made of plastic fabric (strengthen with tailored carbon nanotubes) designed to vaporize when struck by a laser beam. Since the armor is damaged by the attack, ablative armor is more useful against single assassin than it is a lengthy combat mission!

Ablative armor has a split DR. Its full DR is used against the burning or crushing explosive damage inflicted by any type of laser. This DR is also semi~ablative: For every 10 points of basic laser damage rolled, remove one point of DR from the locations struck, regardless of whether the attack penetrated or not.

Its lower DR is used against all other attacks, and is not ablative.

\subsubsection{Reflec}\label{subsubsec:reflec}
Reflec is a light, highly~-reflective armor of polished metallic fibers that reflects laser fire. It is useless against other attacks, but can be worn over other armor. Reflec has a split DR: It gets its full DR against microwaves and lasers (but not X~-ray or gamma~ray lasers), but provides no protection against other weapons.

Reflec is an excellent radar reflector: any stealth benefits against radar are negated, and add +1 (+2 if wearing a full suit) to rolls to detect its wearer.

Any rigid helmet can be made reflective for \$50. It gains +20 DR vs. lasers and microwaves.

\multicolinterrupt{
    \hrule height 1pt
    \subsection{\textit{Laser~-Resistant Armor Table}}
    \begin{table}[H]
        \centering
        % \caption*{Nanoweave Ballistic Armor Table}
        \rowcolors{1}{}{\colordefenses}
        \begin{tabularx}{\linewidth}{lXcXcXrXcXcXr}
            % \hline\\
             \textbf{Armor} && \textbf{Location} && \textbf{DR} && \textbf{Cost} && \textbf{Weight} && \textbf{LC} && \textbf{Notes} \\
             % \rowcolor{blue!10}
             Reflec Helmet && head && 20/0* && \$25 && 0.5 && 4 && \\ 
             Reflec Jacket && torso, arms && 20/0* && \$150 && 1 && 4 && \\
             Reflec Suit && all && 20/0* && \$300 && 2 && 4 && \\
             Ablative Nanoplas Jacket && arms, torso && 36/6* && \$450 && 3 && 3 && \\
             Ablative Nanoplas Suit && all && 36/6* && \$1,200 && 8 && 3 && \\
             Ablative Nanoplas Trousers && groin, legs && 36/6* && \$280 && 2.8 && 3 && \\
        \end{tabularx}
        \label{tab:laser-resistant_armor}
    \end{table}
    * Flexible
}

\subsection{Bioplas Armor}\label{subsec:bioplas_armor}
Bioplas is a strong, pseudo~-alive smart matter material that is light and comfortable to wear -- see \textit{Smart Bioplastic} (p. \pageref{subsec:smart_bioplastic}). Flexible armored suits and clothing are made out of this material. Like other bioplastic equipment, it can heal rips and tears if it has access to moisture and heat, such as sweat and body heat. Bioplas is also a popular material for swimwwear and other sports clothing.

Bioplas is flexible armor with a split DR. Unlike ballistic body armor, bioplas provides its full DR against burning and piercing damage, but gets only one~-third DR vs. other damage types. Thus, it's very effective against a bullet or most energy beams, but not that much use against a powerful blow or vat of acid.

See \textit{Space Biosuit} (p. \pageref{subsubsec:space_biosuit}) for a sealed environmental suit version.

\multicolinterrupt{
    \hrule height 1pt
    \subsection{\textit{Bioplas Armor Table}}
    \begin{table}[H]
        \centering
        % \caption*{Nanoweave Ballistic Armor Table}
        \rowcolors{1}{}{\colordefenses}
        \begin{tabularx}{\linewidth}{lXcXcXrXcXcXr}
            % \hline\\
             \textbf{Armor} && \textbf{Location} && \textbf{DR} && \textbf{Cost} && \textbf{Weight} && \textbf{LC} && \textbf{Notes} \\
             % \rowcolor{blue!10}
             Bioplas Bodysuit && body, limbs && 15/5* && \$1,800 && 3 && 3 && \\ 
             Bioplas Gloves && hands && 15/5* && \$60 && neg. && 4 && \\
             Bioplas Suit && all && 15/5* && \$2,400 && 4 && 3 && \\
        \end{tabularx}
        \label{tab:bioplas_armor}
    \end{table}
    * Flexible
}

\subsubsection{Transparent Bioplas}\label{subsubsec:transparent_bioplas}
This is an option for any bioplas outfit and the space biosuit (p. \pageref{subsubsec:space_biosuit}). It does not protect against laser fire, but is otherwise the same as any other bioplas vest or suit. THe suit adjusts around the user's body, and is almost invisible when worn. (A Vision roll from a yard or less will spot it, and anyone touching the wearer will notice it.) Transparent bioplas comes in translucent colors. Transparent bioplas costs twice as much as ordinary bioplas, but is otherwise identical.

\section{Tailoring Armor}\label{sec:tailoring_armor}
For a combination of fashion and safety, individuals may wear flexible armor in styles other than those described on the armor tables. Specialty shops, such as those provided by Urban Ephemera, design tailored armor to order using computerized manufacturing systems.

\subsection{Armor Type}\label{subsec:armor_type}
Select ballistic (p. \pageref{subsec:ballistic_armor}), reflec (p. \pageref{subsubsec:reflec}), ablative (p. \pageref{subsubsec:ablative_nanoplas}), or bioplas body armor. Recrod the statistics of the suit version of that armor type. Its DR, cost, weight, and LC will be used as the basis of the rest of the outfit.

\subsection{Coverage}\label{subsec:coverage}
Choose the locations that are covered by that outfit. Each location has its own multiplier; add the multipliers for all locations covered. This will give the ``coverage multiplier'' of the entire outfit. The numbers add up to 1 (all locations covered). This multiplier is applied to the armor's weight and cost.

\subsection{\textit{Coverage Table}}
\begin{table}[H]
    \centering
    % \caption*{Coverage Table}
    \rowcolors{1}{}{\colordefenses}
    \begin{tabularx}{\columnwidth}{XlXlX}
        % \hline
         & \textbf{Multiplier} && \textbf{Location} &\\
         & 0.05 && skull &\\
         & 0.05 && face and eyes &\\
         & 0.025 && neck &\\
         & 0.125 && both arms &\\
         & 0.05 && both hands &\\
         & 0.25 && torso &\\
         & 0.10 && groin &\\
         & 0.25 && both legs &\\
         & 0.10 && both feet &
    \end{tabularx}
    \label{tab:coverage}
\end{table}
Outfits may be designed that only protect a location (other than eyes or face) from the front (such as a low~-cut dress) or the back (such as a cape). Halve the multiplier.

Outfits can be designed to protect only part of a location. For example, a miniskirt protects just part of the legs; a bikini bottom provides skimpy coverage to the groin. Halve the coverage multiplier for \textit{halve~-coverage}; multiply by 0.25 for \textit{skimpy} coverage (about 25\% of the area). If a partly~-covered location is struck, make an activation roll (see p. B116) using 3d to see if the protected area was struck. This is an 11 or less for armor with half~-coverage, or 8 or less for skimpy coverage. Any armor on the upper torso \textit{always} protects the vitals, and any armor on the face always protects the eyes.

\subsection{Style}\label{subsec:style}
Now that the coverage has been selected, decide whether the clothing is heavy, normal, light, diaphanous. This will multiply DR, cost, and weight, and may affect LC.

\textit{Heavy:} Trench coats, winter clothing, etc. If it is supposed to be anything else, it's easily recognized as a protective outfit. Multiply weight, cost, and DR by 1.5. Reduce LC by 1.

\textit{Normal:} The outfit can pass as typical civilian attire, such as shirts, jackets, skirts, and trousers. Use the base values.

\textit{Light:} This is typical of T~-shits, evening wear, summer wear, and many undergarments. It can be easily worn \textit{under} clothing. Multiply weight, cost, and DR by 2/3. Increase LC by 1.

\textit{Diaphanous:} This is typical of wispy lingerie or swimsuits. It doesn't look like armor at all, and can be owrn under other outfits. Multiply weight, cost, and DR by 1/2. Increase LC by 1.

\subsection{Cut}\label{subsec:cut}
Finally, decide whether the outfit is of average cut (no extra cost), stylish (four times cost), or a fashion original (20 times cost). These multipliers are cumulative with all others, including accessories that were added to the outfit, except power supply costs.

\subsection{Accessories}\label{subsec:outfit_accesories}
Any appropriate accessories or clothing options (e.g. \hyperref[subsec:buzz_fabric]{buzz fabric}) may be added at the usual cost.

\section{Rigid Body Armor}\label{sec:rigid_body_armro}
These are non~flexible pieces of non~sealed armor used to protect particular body parts.

\subsection{Headgear}\label{subsec:headgear_rigid}
This armor protects the head or eyes. It is made of rigid armor plastic or composites.

\subsubsection{Armored Shades}\label{subsubsec:armored_shades}
Sunglasses with armored lenses. They are glare~-resistant and can be built into any video glasses (p. \pageref{itm:conf_video_glass}).

\subsubsection{Light Infantry Helmet}\label{subsubsec:light_infantry_helmet}
These helmets resemble those used by 20th~-century soldiers. They hav eno built~-in electronics, and are often worn with armoed shades (above) or an optional visor attachement. The visor is glare~-resistant, and is often fitted with a HUD (p. \pageref{itm:terminal_hud}), although this is not standard.

\multicolinterrupt{
    \hrule height 1pt
    \subsection{\textit{Rigid Body Armor Table}}
    \begin{table}[H]
        \centering
        % \caption*{Nanoweave Ballistic Armor Table}
        \rowcolors{1}{}{\colordefenses}
        \begin{tabularx}{\linewidth}{lXcXcXrXcXcXr}
            % \hline\\
             \textbf{Armor} && \textbf{Location} && \textbf{DR} && \textbf{Cost} && \textbf{Weight} && \textbf{LC} && \textbf{Notes} \\
             % \rowcolor{blue!10}
             Armored Shades && eyes && 15 && \$100 && +0.1 && 4 && \\ 
             Light Infantry Helmet && skull && 24 && \$250 && 3 && 3 && \\
             + \textit{Visor} && eyes, face && 20 && +\$100 && +3 && 3 && \\
        \end{tabularx}
        \label{tab:rigid_body_armor}
    \end{table}
}

\subsection{Clamshell Armor}\label{subsec:clamshell_armor}
This hinged cuirass consists of sloped, molded composite laminate reinforced by an inner layer of flexible armor. It is favored by soldiers who don't want to carry around the weight of a full suit of armor, but do want plenty of protection where it counts.

As a laminate armor, clamshell armor is especially effective in protecting against shaped explosions and plasma guns. Reduce the armor divisor of such weapons by 1; (10) becomes (5), (5) becomes (2), and (2) becomes none.

\multicolinterrupt{
    \hrule height 1pt
    \subsection{\textit{Clamshell Armor Table}}
    \begin{table}[H]
        \centering
        % \caption*{Nanoweave Ballistic Armor Table}
        \rowcolors{1}{}{\colordefenses}
        \begin{tabularx}{\linewidth}{lXcXcXrXcXcXr}
            % \hline\\
             \textbf{Armor} && \textbf{Location} && \textbf{DR} && \textbf{Cost} && \textbf{Weight} && \textbf{LC} && \textbf{Notes} \\
             % \rowcolor{blue!10}
             Heavy Clamshell && torso && 60 && \$900 && 18 && 2 && \\ 
             Light Clamshell && torso && 45 && \$600 && 12 && 2 && \\
        \end{tabularx}
        \label{tab:clamshell_armor}
    \end{table}
}

\section{Environmental Gear and Suits}\label{sec:environmental_gear_suits}
These masks and suits are designed to protect the user from the environment as well as from injury. Environmental suit styles vary widely; civilian workers' styles are loud, garish colors for easy recognition, Urban Ephemera offers stylish streetwear cuts, and combat outfits generally come camouflaged.

\subsection{Air Masks and Breathing Gear}\label{subsec:air_mask_breathing_gear}
These are used when full~-equipped suit is unavailable or inappropriate. Each mask covers the entire face, providing the Protect Vision and Protect Smell advantages. 

All masks take three seconds to put on, one second to remove. In all instances, a warning light blinks when power (or air, or filtration) capacity is 90\% gone. All systems contain microcommunicators (p. \pageref{sec:comms}) for presenting remaining capacity on a HUD.

\subsubsection{Air Mask}\label{subsubsec:air_mask}
This mask is used in environments with an unbreathable but otherwise harmless atmosphere. It requires air tanks (below) or a filter (below).% It takes two seconds to put on and one to take off.

\subsubsection{Air Tanks}\label{subsubsec:air_tanks}
Lightweight tanks that store pressurized air mixtures for breathing. All durations assume the use of rebreather systems that recycle and ruse air.

\textit{Large Tank:}\label{itm:large_tank_air} Holds 36 hours. \$200, 10 lb.

\textit{Medium Tank:}\label{itm:medium_tank_air} Holds 18 hour. \$80, 4 lb. LC4.

\textit{Mini Tank:}\label{itm:mini_tank_air} Holds 15 minutes. \$50, 0.5 lb. LC4.

\textit{Small Tank:}\label{itm:small_tank_air} Holds six hours. \$60, 2 lb. LC4.

\subsubsection{Artificial Gill}\label{subsubsec:artificial_gill}
An artificial gill extracts oxygen from water and mixes it with buffer gases, allowing the user to breathe normally while submerged in any body of water that contains dissolved air. This includes most terrestrial sease, but not polar waters and some freshwater bodies. The gill is backpack~-mounted, and includes a mask, an intake system, and a device for separating dissolved air from the water. It runs on a D cell; endurance is 24 hours.

\subsubsection{Filter Mask}\label{subsubsec:filter_mask}
This mask can filter out ordinary contaminants such as dust, pollen, smoke, and even tear gas. It is only effective against nerve gas or other contact agents if combined with a Sealed outfit. The filter medium must be replaced periodically; cost varies from a \$10 cartridge (to filter heavy dust or pollen) to replacing the whole mask (in a chemical~-warfare environment). 

\subsubsection{Respirator}\label{subsubsec:respirator}
This makes thin or low~-oxygen atmospheres breathable by concentrating the oxygen. It includes goggles to protect the eyes from the effects of thin air. It runs on a B cell for three days.

\subsubsection{Reducing Respirator}\label{subsubsec:reducing_respirator}
This mask makes dense or very dense oxygen atmospheres breathable by chemically reducing the partial pressure of oxygen. It includes glare~-resistant goggles to protect the eyes from the burning effects of too much oxygen. It requires power and a monthly chemical recharge (\$50, 1 lb.). It runs on a C cell for three days.

\multicolinterrupt{
    \begin{table}[H]
        \hrule height 1 pt\medskip
        \subsection{Environmental Gear Table}
        \centering
        \rowcolors{1}{}{\colordefenses}
        \begin{tabularx}{\columnwidth}{lXcXcXrXcXcX}
             \textbf{Type} && \textbf{Location} && \textbf{DR} && \textbf{Cost} && \textbf{Weight} && \textbf{LC} & \\
             % name && location && dr && \$ && weight && LC & \\
             Air Mask && eyes, face && 10 && \$100 && 1 && 4 & \\
             Artificial Gill && eyes, face && 10 && \$2,000 && 25 && 4 & \\
             Filter Mask && eyes, face && 10 && \$100 && 3 && 4 & \\
             Respirator && eyes, face && 10 && \$300 && 3 && 4 & \\
             Reducing Respirator && eyes, face && 10 && \$500 && 5 && 4 & \\
        \end{tabularx}
        % \caption{Caption}
        \label{tab:environmental_gear}
    \end{table}
}

\subsection{Civilian Survival Suits}\label{subsec:civilian_survival_suits}
These are flexible, multi~-environmental, and fully~-insulated survival suits, including gloves and a hood with clear visor. The suits are light and comfortable. With the hood sealed and the addition of an air mask or respirator, they protect against atmospheric pollutants or chemical or biological contamination; use NBC Suit skill, but there is no DX penalty.

These suits are popular with natives of hostile regions, survey teams, and rangers; while not armor, their compound~-fiber fabric is resistant damage. The suits are generally legal, but people may frown upon individuals wandering about with the mask sealed. The suits don't protect the face when the mask is rolled up.

Survival suits are often equipped with programmable camouflage (p. \pageref{subsec:program_camouflage}) for safari or tactical purposes.

\subsubsection{Desert Environmental Suit}\label{subsubsec:desert_env_suit}
This full~-body survival suit insulates the wearer from the extremes of desert heat and cold. It provides climate control (-20\degree{}F and 120\degree{}F). It also recycles 90\% of the wearer's body fluids, collecting pure water in a reservoir from which the wearer may drink; the user can survive on one~-tenth as much water as normal. The water recycling system also acts as part of the suit's cooling system. If the suit is out of power, it can't recycle.

\subsubsection{Drysuit}\label{subsubsec:drysuit}
A one~-piece, light underwater survival suit that is sealed and insulated. It is useful for diving in cold or toxic water. It covers the user's entire body except the face. With a gill mask, the suit is sealed and provides climate control (-50\degree{}F to 90\degree{}F).

\subsubsection{Heatsuit}\label{subsubsec:heatsuit}
A heated suit for survival in freezing conditions, including a mask to protect the face. It provides climate control (-250\degree{}F to 100\degree{}F). With a respirator (above), it's useful at very high altitudes or on some alien worlds. If the heatsuit runs out of power, it still provides some benefits due to its insulation: climate control is -50\degree{}F to 90\degree{}F.

\subsubsection{Protective Suit}\label{subsubsec:protective_suit}
A simple sealed suit, with a fireproof and chemical~-retardant coating but no other features. Cargo handlers, hazmat teams, hangar~-bay crews, and some industrial workers often wear them, usually in white or a bright color such as orange or yellow. A rip in the suit causes the smart fabric to change color at the rip. It is sealed with the addition of an air mask (p. \pageref{subsubsec:air_mask}).

\subsubsection{Expedition Suit}\label{subsubsec:expedition_suit}
This suit uses nanocatalytic filtration systems and transistor thermocouples woven into the fabric for heating, cooling, and recycling liquid waste. It recycles 95\% of the user's body fluids and provides climate control (-120\degree{}F to 120\degree{}F). It prevents heat exhaustion with micropores which enable it to ``breathe.'' These pores can also seal shut in hostile environments. Worn with an air mask (p. \pageref{subsubsec:air_mask}), it is sealed. If the suit runs out of power, it provides climate control (-50\degree{}F to 90\degree{}F) and cannot recycle.

\subsubsection{Gill Suit}\label{subsubsec:gill_suit}
This full~-body suit is identical to the drysuit (p. \pageref{subsubsec:drysuit}) in all respects, except that its surface absorbs oxygen from water. This allows the user to breathe underwater as long as the power supply lasts. It includes a belt~-mounted power pack.

\multicolinterrupt{
    \begin{table}[H]
        \hrule height 1pt\medskip
        \subsection{Civilian Survival Suits Table}
        \centering
        \rowcolors{1}{}{\colordefenses}
        \begin{tabularx}{\columnwidth}{lXcXcXrXcXrXcX}
             \textbf{Type} && \textbf{Location} && \textbf{DR} && \textbf{Cost} && \textbf{Weight} && \textbf{Power} && \textbf{LC} & \\
             % name && location && dr && cost && weight && power && lc & \\
             Desert Environment Suit && all && 2* && \$1,000 && 10 && C/1 wk. && 4 & \\
             Drysuit && all && 2* && \$200 && 5 && -- && 4 & \\
             Heatsuit && all && 2* && \$500 && 10 && C/24 hr. && 4 & \\
             Protective Suit && all && 2* && \$50 && 3 && -- && 4 & \\
             Expedition Suit && all && 5* && \$1,500 && 6 && 2C/1 wk. && 4 & \\
             Gill Suit && all && 5* && \$2,000 && 10 && D/24 hr. && 4 & \\
        \end{tabularx}
        % \caption{Caption}
        \label{tab:civilian_survival_suits}
        \begin{flushleft}
            * Flexible
        \end{flushleft}
    \end{table}
}

\subsection{Flexible Combat Suits}\label{subsec:flexible_combat_suits}
These are sealed suits made of flexible armored fabric. All come with pockets, attachment points, and harnesses for weapons of gadgets.

\subsubsection{Nanoweave Tacsuits}\label{subsubsec:nanoweave_tacsuits} %Reflex, Nanoweave, and Monocrys Tacsuits
These tactical suits are chemically~-coated, contamination~-proof coveralls made of flexible nanoweave ballistic fabric. The suit has a split DR: it provides full DR against cutting and piercing damage, and half DR against other damage types. NBC Suit skill is used to get in or out of the suit quickly or gauge its state of repair, but a tactical suit does not limit DX. In fact, the suit is very comfortable to wear, thanks to its internal microclimate control system.

Tacsuits incorporate biomedical sensors (p. TODO). With an air mask (p. \pageref{subsubsec:air_mask}) or combat infantry helmet (p. TODO), the suit is sealed and provdes climate control (-40\degree{}F to 120\degree{}F).

\multicolinterrupt{
    \begin{table}[H]
        \hrule height 1pt\medskip
        \subsection{Tacsuit Table}
        \centering
        \rowcolors{1}{}{\colordefenses}
        \begin{tabularx}{\columnwidth}{lXcXcXrXcXrXcX}
             \textbf{Type} && \textbf{Location} && \textbf{DR} && \textbf{Cost} && \textbf{Weight} && \textbf{Power} && \textbf{LC} & \\
             % name && location && dr && cost && weight && power && lc & \\
             Nanoweave && all && 30/15* && \$3,000 && 15 && C/18 hr. && 2 & \\
        \end{tabularx}
        % \caption{Caption}
        \label{tab:tacsuit}
        \begin{flushleft}
            * Flexible. See above for the split DR explanation.
        \end{flushleft}
    \end{table}
}

\subsection{Counterpressure Vacc Suits}\label{subsec:counterpressure_vacc_suits}
These vacc suits do not inflate. They incorporate a mechanical counter~-pressure (MCP) system which uses elastic layers in direct contact with the skin to prevent the expansion of gases and water vapor in blood vessels and tissues. 

Several types are available. All require Vacc Suit skill to use.

\subsubsection{Skinsuit}\label{subsubsec:skinsuit}
A form~-fitting elastic garment resembling a body stocking, with a rigid collar ring for attaching a helmet. A skinsuit is much thinner than a conventional vacc suit (see below), omitting radiation shielding and heavy~-duty climate control. It is often worn as normal day~-to~-day clothing by space crews who don a full suit only for extravehicular excursions. It is also worn on worlds with poisonous atmospheres but moderate climates. The suit does not include air tanks (p. \pageref{subsubsec:air_tanks}), which must be provided separately. With the addition of a vacc suit helmet (p. TODO), it is sealed, providing climate control (-50\degree{}F to 150\degree{}F) and vacuum support.

\subsubsection{Vacc Suit}\label{subsubsec:vacc_suit}
A vacc suit covers the whole body, including a rigid, removable helmet and life support pack. It's usually festooned with exterior pockets, sticky patches, straps, and hooks for access to equipment, plus at least two lifeline hooks for safety when outside a vessel. The suit has a back~-mounted life~-support pack (LSP), which provides heat regulation, cooling, and energy for the suit's systems. It also includes an air tank with a 12~-hour air supply.

The suit has built~-in biomedical sensors (p. TODO). It is sealed with the addition of a vacc suit helmet (p. TODO), providing climate control (-459\degree{}F to 250\degree{}F) (p. \pageref{subsec:threat_protection}), pressure support (p. \pageref{subsec:threat_protection}) up to 10 atmospheres, radiation protection (PF 2) (p. \pageref{subsec:threat_protection}), and vacuum support (p. \pageref{subsec:threat_protection}). A vacc suit takes 30 seconds to put on or take off, though this time can be halved with a successful Vacc Suit skill roll.

Different vacc suit models are available:

\textit{Civilian Vacc Suit:}\label{itm:civilian_vacc_suit} An ordinary vacc suit worn by most spacers.

\textit{Nanoweave Vacc Suit:}\label{itm:nanoweave_vacc_suit} A heavy~-duty tactical vacc suit reinforced with impact~-resistant balistic armor. It has a split DR: Use the higher DR against piercing and cutting damage, and the lower DR against all other damage types.

\textit{Smart Vacc Suit:}\label{itm:smart_vacc_suit} An improved civilian vacc suit design using advanced nano~catalytic system to reduce the life support system's bulk.

\subsubsection{Space Biosuit}\label{subsubsec:space_biosuit}
This flexible ``living'' counterpressure vacc suit resembles a form~-fitting jumpsuit. Made of smart bioplastic, it absorbs sunlight and recycles waste, giving it an extended air supply (some wastage occurs, but the suit provides full life support for six weeks as long as its power supply can be charged). A small belt pack contains the air needed for recycling and a power pack to supplement the solar system.

The space biosuit is self~-sealing for punctures up to an inch in diameter, and more extensive damage is slowly repaired. It is powered by the user's body heat and lives off his waste products. The suit also includes flexible bioplas gloves and transparent hood~-helmet, which are stored in the belt pack when not in use. These meld seamlessly with the suit worn. No clothing or armor can be worn under a space biosuit.

The suit is sealed with the hood on, providing climate control (-459\degree{}F to 250\degree{}F), pressure suppot up to 10 atmosphere, and vacuum support. LIke bioplas, the biosuit has a split DR: use its higher DR vs. most attacks, but its lower R against corrosion, crushing, and toxic damage. The suit is also a small computer (p. \pageref{itm:small_com}) with the ``printed'' option for flexibility.

\multicolinterrupt{
    \begin{table}[H]
        \hrule height 1pt\medskip
        \subsection{Counterpressure Vacc Suit Table}
        \centering
        \rowcolors{1}{}{\colordefenses}
        \begin{tabularx}{\columnwidth}{lXcXcXrXcXrXcX}
             \textbf{Type} && \textbf{Location} && \textbf{DR} && \textbf{Cost} && \textbf{Weight} && \textbf{Power} && \textbf{LC} & \\
             % name && location && dr && cost && weight && power && lc & \\
             Civilian Vacc Suit && all && 6* && \$10,000 && 25 && 2C/24 hr. && 4 & \\
             Skinsuit && all && 2* && \$1,500 && 4 && -- && 3 & \\
             Smart Vacc Suit && all && 6* && \$5,000 && 15 && 2C/36 hr. && 4 & \\
             Space Biosuit && all && 15/3* && \$10,000 && 5 && 2C/6 wk. && 3 & \\
             Nanoweave Vacc Suit && all && 30/15* && \$12,000 && 30 && 2C/36 hr. && 2 & \\
        \end{tabularx}
        % \caption{Caption}
        \label{tab:counterpressure_vacc_suit}
        \begin{flushleft}
            * Flexible.
        \end{flushleft}
    \end{table}
}

\subsection{Sealed Combat Armor}\label{subsec:sealed_combat_armor}
These enclosed suits of rigid combat armor are designed to resist modern rifle fire as well as explosive and biochemical munitions. THanks to advances in micro~-climate control systems and power supplies, they are comfortable to wear, but more expensive than flexible armor.

\subsubsection{Combat Hardsuit}\label{subsubsec:combat_hardsuit}
This is a sealed suit of combat armor designed for operations in a terrestrial environment. It is heaviest over the torso, but articulated plates and molded pieces also protect the rest of the body. An anti~-radiation layer provides radiation PF 2.

It incorporates an inner garment including biomedical sensors (p. TODO), a waste relief system (p. TODO), and a microclimate control system (p. \pageref{subsec:threat_protection}). The back of the torso clamshells open so the user can step into the armor (it takes three seconds to step in or out). The helmet is \textit{not} included. When worn with either a combat infantry helmet (p. TODO) or space helmet (p. TODO) the suit is sealed, with climate control (-140\degree{}F to 140\degree{}F) and radiation protection (PF 5).

A hardsuit isn’t pressurized and can’t operate in vacuum, but with air tanks and a mask or appropriate helmet, it can operate in areas with unbreathable or contaminated air.

\subsubsection{Space Armor}\label{subsubsec:space_armor}
This complete suit of articulated and pressurized plate armor enables its wearer to operate in almost any environment. 

The suit includes biomedical sensors (p. TODO) and a climate~-control system. It is sealed if worn with space helmet (below), privind climate control (-459\degree{}F to 250\degree{}F), pressure system (10 atm.), radiation protection (PF 10), and vacuum support. Each suit has a split DR; use its higher DR for attacks to the torso, and its lower DR for attacks to other areas.

\multicolinterrupt{
    \begin{table}[H]
        \hrule height 1pt\medskip
        \subsection{Sealed Combat Armor Table}
        \centering
        \rowcolors{1}{}{\colordefenses}
        \begin{tabularx}{\columnwidth}{lXcXcXrXcXcX}
             \textbf{Type} && \textbf{Location} && \textbf{DR} && \textbf{Cost} && \textbf{Weight} && \textbf{LC} & \\
             % name && location && dr && cost && weight && lc & \\
             Combat Hardsuit && all && 75/45 && \$10,000 && 30 && 2 & \\
             Space Armor && all && 75/45 && \$20,000 && 45 && 2 & \\
        \end{tabularx}
        % \caption{Caption}
        \label{tab:sealed_combat_armor}
    \end{table}
}

\subsection{Sealed Helmets}\label{subsec:sealed_helmets}
These helmets protect the entire head. They take three seconds to attach or remove. Each helmet has a split DR: use its higher DR for attacks to the skill, and its lower DR for attacks to the face and the eyes.

\subsubsection{Combat Infantry Helmet}\label{subsubsec:combat_infantry_helmet}
This rigid full~-face visored helmet is usually worn with either the combat hardsuit (p. \pageref{subsubsec:combat_hardsuit}) or a tacsuit (p. \pageref{subsubsec:nanoweave_tacsuits}). It has built~-in GPS (p. \pageref{subsec:gps_receiver}), hearing protection (p. \pageref{subsec:threat_protection}), a small radio (p. \pageref{itm:radio_comm_small}), and an infrared visor (p. \pageref{itm:infrared_goggles_visor}). Filter masks (p. \pageref{subsubsec:filter_mask}) are built into the cheek pieces. With the visor locked into place, the helmet provides an airtight seal to hardsuit and tacsuits.

\subsubsection{Space Helmet}\label{subsubsec:space_helmet}
These enclosed helmets are designed to be worn with suits that are sealed or provide vacuum support. There are four styles:

\textit{Bubble Helmet:}\label{itm:bubble_helmet} A fishbowl helmet made of rigid transparent plastic. The user should wear her own vision and communication gear.

\textit{Space Combat Helmet:}\label{itm:space_combat_helmet} A heavily~-armored combat helmet often worn in conjunction with space armor (p. \pageref{subsubsec:space_armor}). It has hearing protection (p. \pageref{subsec:threat_protection}), a small radio (\pageref{itm:radio_comm_small}), and an infrared visor (p. \pageref{itm:infrared_goggles_visor}).

\textit{Visored Space Helmet:}\label{itm:visored_space_helmet} An enclosed helmet with a transparent faceplate. This incorporates a small radio (p. \pageref{itm:radio_comm_small}), an infrared visor (p. \pageref{itm:infrared_goggles_visor}), and hearing protection (p. \pageref{subsec:threat_protection}).

\textit{Flexible Space Helmet:}\label{itm:flexible_space_helmet} Essentially a pressurized bag, this is made of light, flexible plastic, inflated by a puff of air from the suit. It can be rolled up and stored in a pocket; the user must wear their own communications and vision gear.

\multicolinterrupt{
    \begin{table}[H]
        \hrule height 1pt\medskip
        \subsection{Sealed Helmet Table}
        \centering
        \rowcolors{1}{}{\colordefenses}
        \begin{tabularx}{\columnwidth}{lXcXcXrXcXrXcX}
             \textbf{Type} && \textbf{Location} && \textbf{DR} && \textbf{Cost} && \textbf{Weight} && \textbf{Power} && \textbf{LC} & \\
             % name && location && dr && cost && weight && power && lc & \\
             Bubble Helmet && head && 9 && \$2,000 && 5 && B/36 hr. && 4 & \\
             Combat Infantry Helmet && head && 27/18 && \$2,000 && 5 && B/18 hr. && 2 & \\
             Flexible Space Helmet && head && 5* && \$500 && 0.5 && -- && 4 & \\
             Space Combat Helmet && head && 60/45 && \$3,000 && 7 && B/36 hr. && 2 & \\
        \end{tabularx}
        \label{tab:sealed_helmets}
        \begin{flushleft}
            * Flexible.
        \end{flushleft}
    \end{table}
}