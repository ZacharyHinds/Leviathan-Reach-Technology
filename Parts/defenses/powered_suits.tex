Powered suits enhance the wearer's strength and mobility. They come in two styles: open \textit{exoskeletons} and enclosed \textit{battlesuits}. Most powered suits provide a bonus to Lifting ST (p. B65) and Striking ST (p. B88)

\section{Exoskeletons}\label{sec:exoskeletons}
A powered exoskeleton (or ``exo'') is an open framework of artificial ``muscles.'' When the user moves, the sensors in the suit react to and match his movements. The wearer uses the physical attributes of the exoskeleton rather than his own.

Exoskeletons provide little protection, but unless noted, they may be worn over clothing or any flexible armor.

\subsection{Full~-Body Exoskeletons}\label{subsec:full-body_exoskeletons}
These are attached to the body and limbs. They provide a bonus to Lifting ST and Striking ST. Battlesuit skill limits DX and DX~-based skills; see p. B192. With th power on, a full~-body exoskeleton's weight is \textit{not} counted toward encumbrance.

\subsubsection{Heavy Exoskeleton}\label{subsubsec:heavy_exoskeleton}
A rugged, heavy~-duty exoskeleton designed for cargo loading and construction work. It's often used as a substitute for a fork~-lift truck or construction robot. It is very strong, but the oversized arms are not suited for fine work.

It stands eight feet tall (SM+1). The wearer gains Lifting ST+12 and Striking ST+8. In addition to any penalties for low skill, the wearer is Ham~-Fisted (-3 DX).

The exoskeleton has a built~-in laser torch (p. \pageref{subsec:laser_plasma_torches}), a mini tool kit (p. \pageref{subsubsec:mini-tool}) for Mechanic skill, and a fire extinguisher tube (p. \pageref{itm:fire_extinguish_tube}).

\subsubsection{Light Exoskeleton}\label{subsubsec:light_exoskeleton}
This is a lower~-powered but less bulky exoskeleton. It grants the wearer Lifting ST+10 and Striking ST+6.

\subsubsection{Ranger Exoskeleton}\label{subsubsec:ranger_exoskeleton}
Basically a battlesuit without the armor, this light but powerful exoskeleton is used for military or paramilitary operations. Its leg braces and motors boost the wearer's agility as well as his strength. It grants Lifting ST+12, Striking ST+12, and Super Jump 2.

\subsubsection{Stealth Exoskeleton}\label{subsubsec:stealth_exoskeleton}
This lightweight exoskeleton can be conealed under heavy clothing, such as a jacket or trousers. It can only be worn over skimpy clothing. The stealth exoskeleton gives Lifting ST+4 and Striking ST+4.

\subsection{Lower~-Body Exoskeleton}\label{subsec:lower-body_exoskeleton}
Lower~-body exoskeletons are worn by porters, solders, and anyone else who needs to carry heavy loads without straining. They include an exo~-supported backpack capable of carrying up to a 70~-pound Payload; when the power is on, this load is \textit{not} counted toward the encumbrance. Battlesuit skill onnly limits DX and skills for tasks that require lower~-body agility, such as melee attacks or jumping.

\subsection{Power Sleeve}\label{subsec:power_sleeve}
A bulky ``power glove'' and arm brace that enhances gripping power. The glove can also be set on ``auto~-grip,'' which makes it ``freeze'' in any desired position; the user can then slip her hand out of the glove and leave it clamped onto something. It gives Arm ST+6 for crushing, gripping, and holding to the arm it is worn on (the user can wear one glove on each arm, if desired). It requires Battlesuit skill, but this only limits DX and skills for the power~-sleeved arm.

\multicolinterrupt{
    \begin{table}[H]
        \hrule height 1pt\medskip
        \subsection{Powered Exoskeleton Table}
        \centering
        \small
        \rowcolors{1}{}{\colordefenses}
        \begin{tabularx}{\columnwidth}{lXcXcXrXcXcXcX}
            \textbf{Armor} && \textbf{Location} && \textbf{DR} && \textbf{Cost} && \textbf{Weight} && \textbf{Power} && \textbf{LC} \\
            % name && location && dr && \$ && weight && power && LC \\
            Heavy Exoskeleton && all && 20/0 && \$50,000 && 200 && E/24 hr. && 3 \\
            Light Exoskeleton && body, limbs && 10/0 && \$25,000 && 50 && D/12 hr. && 3 \\
            Lower Body Exoskeleton && groin, legs && 8/0 && \$12,000 && 30 && 2D/24 hr. && 4 \\
            Power Sleeve && one arm and hand && 8/0 && \$2,000 && 2 && C/12 hr. && 4 \\
            Ranger Exoskeleton && body, limbs && 20/0 && \$50,000 && 50 && D/12 hr. && 3 \\
            Stealth Exoskeleton && body, limbs && 12/0 && \$10,000 && 10 && 2C/8 hr. && 4 \\
        \end{tabularx}
        \label{tab:powered_exoskeleton}
        \begin{flushleft}
            If an exoskeleton has a split DR, use the higher DR against any \textit{swinging} melee attacks, falls, or collisions. Use the lower DR against \textit{all other damage types.}
        \end{flushleft}
    \end{table}
}

\section{Battlesuits}\label{subsec:battlesuits}
A battles is an armored exoskeleton. Its strength~-amplifying feature lets a battlesuit trooper carry squad~-support weapons like heavy machine guns or semi~-portable blasters. Many battlesuits have built~in tactical systems such as sensors or weapon mounts, and are designed for hostile environments.

Battlesuits are much more expensive than ordinary combat armor, and require more training to use, but they greatly increase effectiveness. A single battlesuit trooper with heavy weapons can be as effective as an entire squad, and nearly as mobile as an armored vehicle.

Battlesuits do not run any faster, since the user's speed is limited by the length of their legs, but suits with strong leg muscles can move quickly by using a series of jumps, which may provide both the Super Jump advantage and an increase in Basic Move. Wearing a suit is not fatiguing; except for the helmet, the armor's weight does not count as encumbrance while powered up. If the suit loses power, the wearer can still move (unless they're in a combat walker), but they must use their own ST to carry the weight!

Unless otherwise noted, a battlesuit opens at the waist so that the user can easily step in or out. This takes three seconds, plus another three to screw on the separate helmet, if there is one. However, it takes 30 seconds to do this \textit{and} perform all the subsystem checks, power everything up, and connect all features (such as the waste relief and biomedical telemetry). This time is halved on a successful Battlesuit skill roll. It's also possible to omit the check~-out procedure and just start moving, but if so, the user risks internal systems failing at the worst times. 

Battlesuits need to be fitted to the wearer. Refitting takes two hours and requires an Armoury (Body Armor)+2 roll. Failure means another attempt (and another 2 hours) is required. Critical failure damages the suit in some way, possibly requiring repairs before it is usable or it suffers a fault that will not be apparent until it is used in action. It is possible to use an unfitted suit, provided the user is the same size, shape, and body~-type as the last wearer (height/weight should be no more than 2\% off). However, the wearer will suffer a -1 penalty to DX and all DX~-based skills.

\subsubsection{Flying Battlesuits}\label{subsubsec:flying_battlesuits}
Many battlesuits are used with flight systems, since the suit's strength amplification makes it easy to carry extra gear. For maximum flexibility, these systems are not included in the suit designs, but are usually worn as extra gear. For maximum flexibility, these systems are not included in the suit designs, but are usually worn as external packs or belts. See \textit{Flight Pack} (p. \pageref{sec:flight_packs}) for various options. Flight systems interface with battlesuits, allowing the suit's own navigation displays to be used for flight control.

\subsubsection{Underwater Battlesuits}\label{subsubsec:underwater_battlesuits}
Units operating underwater will need to use an aquatic propulsion pack (p. \pageref{sec:diver_propulsion_systems}) for submerged mobility. A few suits have integral aquatic propulsion.

\subsection{Combat Walkers}\label{subsec:combat_walkers}
These early designs have a barrel~-shaped torso that blends into the head. There is no neck or waist articulation; the user must rely on sensors to see behind him and cannot twist his torso around. The sut's hands are also crude (but very strong) grippers. Combat walkers buit for humans stand eight feet tall (SM+2).

A combat walker is more mobile than a tank, but its agility remains limited. The suit can sit or kneel, but the user cannot crawl, get up from a prone position, jump, or swim. On the other hand, the walker is covered with depleted uranium composite laminate over high hardness steel alloy, and can shrug off fire from light anti~-tank weapons.

The suit's exoskeleton provides Lifting ST+20 and Striking ST+20. Due to its longer legs, it also adds +1 to Basic Move. While wearing the suit, the wearer suffers Bad Grip 2 (p. B123). The entire suit's weight is ignored for encumbrance. However, if the combat walker loses power, the wearer is effectively paralyzed until he leaves the suit.

The suit has several standard accessories: a GPS (p. \pageref{subsec:gps_receiver}), hearing protection (p. \pageref{subsec:threat_protection}), biomedical sensors (p. \pageref{subsec:biomedical_sensors}), and a waste relief system (p. \pageref{itm:waste-relief_system}). Its helmet electronics include a hyperspectral visor (p. \pageref{itm:hyperspec_goggles}), a medium radio (p. \pageref{itm:radio_comm_medium}), and a small laser comm (p. \pageref{itm:laser_comm_small}). The helmet has audio sensors so the user can hear outside the suit, but it lacks olfactory sensors; unless the hatch is opened, the user suffers from No Sense of Smell/Taste (p. B146) when dealing with the outside world. The suit's surface has a tactical ESM (p. \pageref{itm:tactical_esm_detector}), and incorporates infrared cloaking (p. \pageref{subsec:infrared_cloaking}) and radar stealth (p. \pageref{subsec:radar_stealth}).

A combat walker is slightly more roomy than most other battlesuits. This means that it is a ``one size fits all'' suit that does not require special fitting to each user. It has a hatch at the back that the user must climb into; due to its hight, the suit should be in a kneeling posture to enter, or it takes an extra second to clamber into it. Entry and exit are otherwise similar to most battlesuits.

Different variations of combat walkers are described below. These battlesuits have mostly become obsolete, having been replaced by heavy battlesuits (p. \pageref{subsec:heavy_battlesuit}), but vintage and refurbished walkers can still be found.

\subsubsection{Infantry Combat Walker}\label{subsubsec:infantry_combat_walker}
This is the standard model, designed for operations in terrestrial conditions. It is sealed (p. \pageref{subsec:threat_protection}), with a filter mask (p. \pageref{subsubsec:filter_mask}), climate control (-20\degree{}F to 140\degree{}F) (p. \pageref{subsec:threat_protection}), and radiation protection (PF 10) (p. \pageref{subsec:threat_protection}). It can be equipped with air tanks (p. \pageref{subsubsec:air_tanks}), but these add to its weight.

\subsubsection{Marine Combat Walker}\label{subsubsec:marine_combat_walker}
This model can swim underwater using ballast tanks and a waterjet propulsion system. It has Water Move 4, and a built~-in small sonar (p. \pageref{itm:small_sonar}). It is sealed, and provides climate control (-20\degree{}F to 150\degree{}F), pressure support (10 atm.), and radiation protection (PF 10). It has a large air tank, giving it a 36~-hour supply, though some notable old models only have a 24~-hour air supply.

\subsubsection{Space Combat Walker}\label{subsubsec:space_combat_walker}
This battlesuit is designed for operations on hostile planets. It can walk underwater, but cannot float or swim. It has vacuum support, and can operate underwater or in superdense atmospheres. It is sealed, and provides climate control (-459\degree{}F to 300\degree{}F), pressure support (30 atm.), radiation protection (PF 10), and vacuum support. It has two large air tanks, givig it a 72 hour air supply, though particularly old models only have a 48 hour air supply.

\subsection{Powered Combat Armor}\label{subsec:powered_combat_armor}
This is a standard medium~-weight combat battlesuit. It is seven feet tall, made of articulated plates of metal~-matrix composites with an inner layer of reflex armor. Powered combat armor is intended to resist rifles or light machine guns, but can't stand up to anti~-tank weapons. It is small enough to fit though ordinary doors, making it a superb tool for house~-to~-house fighting, urban warfare, and boarding actions. 

Powered combat armor gives +10 to Lifting and Striking ST and Super Jump 1. Biomedical sensors (p. \pageref{subsec:biomedical_sensors}) and a waste relief system (p. \pageref{itm:waste-relief_system}) are standard features. The suit's surface has a tactical ESM (p. \pageref{itm:tactical_esm_detector}).

The helmet comes with a filter mask (p. \pageref{subsubsec:filter_mask}), a GPS (p. \pageref{subsec:gps_receiver}), hearing protection (p. \pageref{subsec:threat_protection}), a small radio (p. \pageref{itm:radio_comm_small}), a small laser comms (p. \pageref{itm:laser_comm_small}). The helmet has olfactory and audio sensors so the user can hear and smell outside the suit.

With the helmet on, the suit is sealed. It provides climate control (-459\degree{}F to 250\degree{}F), pressure support (10 atm.), radiation protection (PF 10), and vacuum support. It has a large air tank with 36 hours of air (very old models only have 24 hours). In a contaiminated but breathable atmosphere, it can operate using the standard filter mask.

Powered combat armor incorporates infrared cloaking (p. \pageref{subsec:infrared_cloaking}). Chameleon surfaces (p. \pageref{subsec:chameleon_surface}) are common but not standard.

\subsection{Zero~-G Worksuit}\label{subsec:zero-g_worksuit}
This suit is more like a miniature spaceship than a vacc suit. It is a rigid pressurized cylinder with a transparent helmet dome; the whole thing is slightly larger than a humman. It has no legs, but is propelled by an integral thruster pack mounted in the base. The suit's thrusters accelerate or decelerate it at up to three yards/second$^2$, with enough fuel for 300 seconds of acceleration. A Piloting (Spacecraft) roll is required to quickly change direction.

In addition to its normal suit sleeves, it has three ST 15 waldoes -- remote~-controlled arms -- for heavy duty work; they can be used as normal arms but at a -3 DX penalty. Any two waldoes may be used at once. One waldo also mounts as an integral laser torch (p. \pageref{subsec:laser_plasma_torches}). A waldo's grip can be powered~-locked onto a structure (e.g., a ship's hull) to hold the suit steady while the other limbs are used for work.

It is sealed, providing climate control (-459\degree{}F to 300\degree{}F), radiation protection (PF 10), and vacuum support. It has two weeks of air. A small (eight~-inch diameter) airlock in its side is used to transfer small items (such as tools, food, or air) without breaking suit integrity. It has a built~-in medium radio (p. \pageref{itm:radio_comm_medium}). The suit is powered by an E cell, and has sockets for a second cell. 

Like a combat walker (p. \pageref{subsec:combat_walkers}), a zero~-G worksuit does not need to be specifically fitted to each user.

\subsection{Commando Battlesuit}\label{subsec:commando_battlesuit}
This is a lightweight, agile, form~-fitting powered armor suit.

Its exoskeleton gives +15 to Lifting and Striking ST and Super Jump 2. Biomedical sensors (p. \pageref{subsec:biomedical_sensors}) and a waste relief system (p. \pageref{itm:waste-relief_system}) are standard features. The suit's surface has infrared cloaking (p. \pageref{subsec:infrared_cloaking}), radar stealth (p. \pageref{subsec:radar_stealth}), and a tactical ESM (p. \pageref{itm:tactical_esm_detector}).

The helmet comes with a filter mask (p. \pageref{subsubsec:filter_mask}), an inertial compass (p. \pageref{itm:inertial_compass}), hearing protection (p. \pageref{subsec:threat_protection}), a small radio (p. \pageref{itm:radio_comm_small}), a small laser comm (p. \pageref{itm:laser_comm_small}), and a hyperspectral visor (p. \pageref{itm:hyperspec_goggles}). The helmet has olfactory and audio sensors so the user can hear and smell outside the suit.

With the helmet locked down the suit is sealed, providing climate control (absolute zero to 500\degree{}F), pressure support (20 atm.), radiation protection (PF 10), and vacuum support. It has a large air tank (p. \pageref{itm:large_tank_air}) with 36 hours of air. In a contaminated but breathable atmosphere, it can operate using the filter mask.

\subsection{Heavy Battlesuit}\label{subsec:heavy_battlesuit}
These highly~-mobile suits can fight in almost any environment. They have enough life support to keep the user alive for days in a contaminated war zone. They are smaller than combat walkers, but still stand seven feet tall and are bulky (SM+1).

The armor is a thick shell of laminated nanocomposites and ceramic armor over an inner layer of shock~-absorbing liquid armor. This gives torso protection equal to at least three inches of steel plate. It also has integral superconductor~-based electromagnetic armor, which doubles the suit's DR against shaped~-charge warheads and plasma bolts. The electromagnetic armor operates off a separate D cell and is good for 10 uses.

A powered exoskeleton amplifies the user's muscles and grounds speed (see below). Except for the helmet, the armor's weight does not count as encumbrance while powered up. It is powered by an integral radiothermal generator which operates it for up to 10 years.

The suit's helmet includes a filter mask (p. \pageref{subsubsec:filter_mask}), an inertial compass (p. \pageref{itm:inertial_compass}), hearing protection (p. \pageref{subsec:threat_protection}), a hyperspectral visor (p. \pageref{itm:hyperspec_goggles}), a small laser comm (p. \pageref{itm:laser_comm_small}), and a medium radio (p. \pageref{itm:radio_comm_medium}). The suit's body incorporates biomedical sensors (p. \pageref{subsec:biomedical_sensors}), trauma maintenance (p. \pageref{subsec:trauma_maintenance}), a provisions dispenser (p. \pageref{itm:provisions_dispenser}) with a week's provisions, tactical ESM (p. \pageref{itm:tactical_esm_detector}), and a waste relief system (p. \pageref{itm:waste-relief_system}). The suit also has infrared cloaking (p. \pageref{subsec:infrared_cloaking}) and radar stealth (p. \pageref{subsec:radar_stealth}).

With its helmet on, it is sealed, and has climate control (-459\degree{}F to 500\degree{}F), pressure support (10 atm.), radiation protection (PF 5), and vacuum support. It has two large air tanks with a 72~-hour air supply.

The suit's exoskeleton grants Lifting and Striking ST+20. It has Basic Move +2 and Super Jump 1.

\subsubsection{Command Battlesuit and Scout Battlesuit}\label{subsubsecLcommand_scout_battlesuit}
These variations have almost identical statistics.

\textit{Command Battlesuit:}\label{itm:command_battlesuit} A suit designed for officers, not quite as strong as a heavy battlesuit, but with equivalent armor and greater mobility. It includes a medium laser comm (p. \pageref{itm:laser_comm_medium}) and a large radio (p. \pageref{itm:radio_comm_large}). Its exoskeleton grants Lifting and Striking ST+18, Basic Move +3, and Super Jump 2.

\textit{Scout Battelsuit:}\label{itm:scout_battlesuit} A variation on the command battlesuit for reconnaissance and special ops units. It has a chemsniffer (p. \pageref{subsec:chemsniffer}), and a deceptive radar jammer (p. \pageref{itm:deceptive_radar_jammer}). The helmet has a small genius computer (p. \pageref{itm:small_com}). Its exoskeleton is less strong but faster: it grants Lifting and Striking ST+16, Basic Move +4, and Super Jump 3.

\subsection{HEX Suit}\label{subsec:hex_suit}
The Hostile Environment eXosuit is a suit of powered space armor reinforced for operations in extremely dangerous environments.

The suit is eight feet tall (SM+1). Its bulbous, heavily~-armored body is reinforced and shielded to resist extremes of pressure, temperature and radiation, and it has a heavy~-duty life support. It is sealed, providing climate control (-459\degree{}F to 800\degree{}F), pressure support (50 atm.), radiation protection (PF 100), and vacuum support. It has 120 hours of air and water.

Its exoskeleton provides Lifting ST+8 and Striking ST+4, and cancels the weight of the suit for encumbrance. Unlike most other battlesuits, it does not otherwise increase the wearer's mobility. The suit requires 60 seconds to put on or take off.

Standard accessories include a waste relief system (p. \pageref{itm:waste-relief_system}), a provisions dispenser (p. \pageref{itm:provisions_dispenser}) with a week's water and rations, and an automatic backscratcher. The helmet has hyperspectral goggles (p. \pageref{itm:hyperspec_goggles}), a small multi~-mode radar (p. \pageref{subsec:multi-mode_radar}), and a medium radio (p. \pageref{itm:radio_comm_medium}).

Battlesuit skill is used to operate the suit, and \textit{does} limit DX and skill use (see p. B192). 

\multicolinterrupt{
    \begin{table}[H]
        \hrule height 1pt\medskip
        \subsection{Battlesuit Table}
        \small
        \centering
        \begin{tabularx}{\columnwidth}{lXcXcXrXcXcXcX}
            \textbf{Armor} && \textbf{Location} && \textbf{DR} && \textbf{Cost} && \textbf{Weight} && \textbf{Power} && \textbf{LC} & \\
            % name && all && dr && \$ && weight && power && LC & \\
            \rowcolor{\colordefenses}
            Combat Walker && all && 200/120 && \$300,000 && 800 && E/24 hr. && 1 & \\
            Marine Combat Walker && all && 200/120 && \$320,000 && 900 && E/24 hr. && 1 & \\
            \rowcolor{\colordefenses}
            Powered Combat Armor && body && 70/50 && \$80,000 && 150 && E/18 hr. && 1 & \\
            \rowcolor{\colordefenses}
            + \textit{Helmet} && head && 70/50 && +\$10,000 && 15 && C/18 hr. && 1 & \\
            Space Combat Walker && all && 200/120 && \$330,000 && 950 && 2E/48 hr. && 1 & \\
            \rowcolor{\colordefenses}
            Zero~-G Worksuit && all && 40 && \$60,000 && 150 && E/48 hr. && 3 & \\
            Commando Battlesuit && body && 105/75 && \$80,000 && 150 && E/24 hr. && 1 & \\
            + \textit{Helmet} && head && 10/75 && +\$10,000 && 15 && C/24 hr. && 1 & \\
            \rowcolor{\colordefenses}
            Heavy Battlesuit && body && 150/100 && \$200,000 && 480 && 10 yr. && 1 & \\
            \rowcolor{\colordefenses}
            + \textit{Helmet} && head && 150/100 && +\$10,000 && 20 && 10 yr. && 1 & \\
            HEX Suit && all && 140 && \$200,000 && 2,000 && 2E/1 wk. && 3 & \\
        \end{tabularx}
        \label{tab:battlesuit}
        \begin{flushleft}
            Battlesuits with split DR use the higher DR against attacks to the torso (and skull, for helmets or suits that cover all locations); the lower DR protects other locations.\bigskip

            * Flexible.
        \end{flushleft}
    \end{table}
}