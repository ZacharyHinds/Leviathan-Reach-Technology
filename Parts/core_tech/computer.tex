\section{Hardware}
%\addcontentsline{toc}{section}{Hardware}
\label{sec:computer_hardware}
Every computer has a "Complexity``rating. This is an abstract measure of processing power. Each Complexity level represents a tenfold increase in overall capability over the previous level. 

A computer's Complexity determines what programs it can run, and may be a prerequisite for certain options, such as Sentient. Software also has a Complexity rating, and can only run on a computer of that Complexity level or higher; e.g., a Complexity 2 program requires a Complexity 2 computer or better.

Complexity determines how many programs a computer can run simultaneously. It can run two programs of its own Complexity, 20 programs of one Complexity level less, 200 programs of two Complexity less, and so on.

Computers are rated for their data storage in petabytes.

% \subsection{Computer Models}
% %\addcontentsline{toc}{section}{}
\subsection{Computer Models}
\label{subsec:computer_models}
%\addcontentsline{toc}{subsection}{Computer Models}
These are standard sizes of ordinary computers that lack any sort of self-awareness.

These systems include the processor, the power supply, the casing, and a storage system, plus an operating system. Computers may also have a cable jack (p. \pageref{itm:cable_jack}) and microcommunicator (p. \pageref{subsubsec:standard_comm_sizes}) at no extra cost, although these may also be omitted in order to isolate the computer for security purposes.

Displays and controls are not included. Even so, the computer can be used “as is” via a neural interface (p. \pageref{sec:neural_interfaces}), or installed into a robot body or vehicle. Also, if the computer is equipped with AI software, users can interact with it just by talking to it. Otherwise, they should be equipped with a terminal (p. \pageref{subsec:terminals}) or a communicator.

\subsubsection{Tiny Computer}
The smallest multi-purpose computer in regular use. It is Complexity 5 and stores 1 PB. \$50, 0.05 lb., 2A/20 hr. LC4. \label{itm:tiny_com}

\subsubsection{Small Computer}
It has Complexity 6 and stores 10 PB. \$100, 0.5 lb., 2B/20 hr. LC4. \label{itm:small_com}

\subsubsection{Personal Computer}
Average computer found in almost every middle-class household. It has Complexity 7 and stores 100 PB. \$100, 5 lb., 2C/20 hr. or external power. LC4. \label{itm:person_com}

\subsubsection{Microframe}
High-end cabinet-sized machine. It has Complexity 8 and stores 1,000 PB. \$10,000. 40 lb., external power. LC3. \label{itm:microframe}

\subsubsection{Mainframe}
It has Complexity 9 and stores 10,000 PB. \$100,000, 400 lb., external power. LC3. \label{itm:mainframe}

\subsubsection{Macroframe}
Massive computers. They have Complexity 10 and store 100,000 PB. \$1,000,000, 4,000 lb., external power. LC3. \label{itm:macroframe}

\subsubsection{Megacomputer}
A computer the size of a building. It has Complexity 11 and stores 1,000,000 PB. \$10,000,000. 40,000 lb., external power. LC2. \label{itm:mega_com}

\subsubsection{Customizing Hardware}
\label{subsubsec:customize_hardware}
Various options are available to customize computer hardware. Multiple options can be chosen, but each option can only be taken once. Modifiers to Complexity, cost, etc. apply to the hardware statistics. Cost and weight multipliers are multiplied together. Complexity and LC modifiers are additive, but LC cannot go below LC.

\textit{Compact:} Double the cost, halve the weight. Halve the number of power cells and the operating duration \label{itm:com_upg_compact}

\textit{Fast:} This option cannot be combined with Slow or Genius. +1 Complexity. Multiply the cost by 20. \label{itm:com_upg_fast}

\textit{Genius:} This option cannot be combined with Fast or Slow. +2 Complexity. Multiply the cost by 500 and reduce LC by 1. \label{itm:com_upg_genius}

\textit{Hardened:} +3 to HT to resist electromagnetic pulses, microwaves, and other attacks that target electrical gadgets. Double the cost, double the weight. \label{itm:com_upg_hardened}

\textit{High-Capacity:} Run 50\% more programs simultaneously. Cost is 1.5 times normal. \label{itm:com_upg_high-cap}

\textit{Printed:} Printed on a flexible surface, such as fabric or even skin. Requires 4 sqft per pound of weight. Must use solar cells or flexible cells for power. Breaking the surface destroys the computer. This option is not compatible with quantum computers. -1 Complexity, and storage is reduced to TB instead of PB. \label{itm:com_upg_printed}

\textit{Quantum:} Drastically reduces the time required to perform certain processes. Multiply the cost by 10, and double the weight. -1 LC. \label{itm:com_upg_quantum}

\textit{Slow:} This option cannot be combined with Fast or Genius. -1 Complexity and 1/10th storage. Divide cost by 20. \label{itm:com_upg_slow}

\textit{Data Storage:} Additional built-in data storage can be purchased for \$1 and 0.0001 lb. per additional PB. \label{itm:com_upg_data}

\subsection{Terminals}
%\addcontentsline{toc}{subsection}{Terminals}
\label{subsec:terminals}
The standard types of terminal are:

\textit{Datapad:} Tiny color video screen with built-in touch screen, resembling a cellphone. Can be built into a computer or be separate (or even worn similarly to a wristwatch). Includes a microcommunicator, a cable jack, a speaker/microphone, and a mini-camera. Any tasks requiring use of the keyboard and screen for lengthy or complex periods are at -2 to skill. It has a data-chip removable drive. \$10, 0.05 lb. 2A/20 hr. LC4. \label{itm:terminal_datapad}

\textit{Head-Up Display (HUD):} A 3D video display integrated into glasses or a helmet visor, or designed to be projected onto a windscreen. A HUD can also be printed onto a flat surface. Many vehicles, suits, sensor goggles, and the like incorporate a HUD at no extra cost. If bought separately: \$50, neg., uses external power. LC4. \label{itm:terminal_hud}

\textit{Sleeve Display:} A square of touch-sensitive digital cloth woven into the fabric of clothing, uniforms, and body armor. It has a built-in speaker. \$50, neg, Af/10 hr. LC4. \label{itm:terminal_sleeve}

\textit{Portable Terminal:} A small but functional color video display and multi-system interface (keyboard, mouse, speakers, mic, video camera), typical of laptop computers. Also used as a remote control for many types of devices, such as sensors, communicators, and drones. It's adequate for most tasks, though time-consuming or graphics-intensive tasks require a desktop workstation (see below) to avoid a -1 penalty. It has both data-chip and removable drives. \$50, 0.5 lb., 2B/20 hr. LC4. \label{itm:terminal_portable}

\textit{Workstation Terminal:}\ A complete desktop, vehicular console, or office system with the same capabilities as a portable terminal. It has a larger keyboard, a full-size 3D monitor, a document scanner/printer, and a hook-up for VR. \$500, 5 lb. C/10 hr. or external power. LC4. \label{itm:terminal_wrkst}

\textit{Computerized Crew Station:} This is a high~-end workstation with controls that can be reconfigured, multi~-function programmable displays, and a padded, adjustable seat. Generally, these stations are required for controlling complex systems in vehicles like spaceships and in power stations. \$2,000, 50 lb., uses external power. LC4. \label{itm:terminal_com_crew}

\textit{Holographic Crew Station:} As the computerized crew station, but that uses holographic projection to immerse the user in 3D imagery. \$10,000, 50 lb., uses external power. LC4. \label{itm:terminal_holo_crew}

\textit{Multisensory Holographic Crew Station:} As the computerized crew station, but the controls and displays can be configured for nonhuman senses--for example, ultrasonic, infrared, or even olfactory outputs. \$50,000, 100 lb.; uses external power; LC4.\label{itm:terminal_multisense_holo_crew}

\textit{Holoprojection:} Any terminal can also be made using a holoprojector instead of a screen. This is generally a mini holoprojector (see p. ). \label{itm:terminal_holoproj}

\section{Software}
%\addcontentsline{toc}{section}{Software}
\label{sec:computer_software}
A system can be programmed to do just about anything, but good programming is expensive.

\subsection{Programs}
%\addcontentsline{toc}{subsection}{Programs}
\label{subsec:programs}
Programs are rated for their cost, LC, and Complexity, which determines what systems they can run on. Descriptions of programs are found in relevant sections. See Encryption, Sensies, Tactical Programs, and Virtual Reality.

Software cost may vary depending on the nature of the program and its provenance (shareware, pirated, demo copy, open-source, etc.). Many programs have free versions, not all of which are legal. Free programs often lack novice~-friendly interfaces and manuals, so a Computer Operation roll is generally required to find, install, or use them.

\subsection{Software Cost Table}
\begin{table}[H]
    \centering
    \rowcolors{1}{}{\colorcoretech}
    \begin{tabularx}{\columnwidth}{lXr}
        \textbf{Complexity} && \textbf{Cost} \\
        Complexity 1 && \$1 \\
        Complexity 2 && \$3 \\
        Complexity 3 && \$10 \\
        Complexity 4 && \$30 \\
        Complexity 5 && \$100 \\
        Complexity 6 && \$300 \\
        Complexity 7 && \$1,000 \\
        Complexity 8 && \$3,000 \\
        Complexity 9 && \$10,000 \\
        Complexity 10 && \$30,000 \\
        Complexity 11 && \$100,000 \\
        Complexity 12 && \$300,000 \\
        Complexity 13 && \$1,000,000 \\
    \end{tabularx}
    \label{tbl:software_cost}
\end{table}

\subsection{Software Tools}
%\addcontentsline{toc}{subsection}{Software Tools}
\label{subsec:software_tools}
IQ-based technological skills generally require software to function at full effectiveness when performing any task involving research, analysis, or invention. Software tools are also generally needed or useful for skills such as Accounting, Artillery, Market Analysis, Strategy, Tactics, and Writing.

\textit{Basic} programs are incorporated into dedicated systems integrated into the devices used to perform the skill and provide no bonus. \label{itm:software_basic}

\textit{Good}~-quality programs give a +1 bonus. These are Complexity 4 for Easy skills, Complexity 5 for Average, Hard or Very Hard skills. \label{itm:software_good}

\textit{Fine}~-quality programs give a +2 bonus. These are Complexity 6 for Easy skills, Complexity 7 for Average, Hard, or Very Hard skills. \label{itm:software_fine}

\subsection{Artificial Intelligence}
%\addcontentsline{toc}{subsection}{Artificial Intelligence}
\label{subsec:ai}
An artificial intelligence (AI) is a sentient or sapient computer system. AIs range from barely-sentient insect-level intelligence to godlike minds, though most (especially those used in robots) are sapient. Sapient AIs are also classed as dedicated, non-volitional, or volitional.

\textit{Dedicated AI:} This is a simple AI program that lacks initiative or personality. It is incapable of learning and is at most a "smart tool." Complexity is (IQ/2)+1. LC4. \label{itm:ai_dedicated}

\textit{Non~-Volitional AI:} An AI capable of understanding natural speech, learning technological skills, and learning by itself. It, however, lacks initiative and is essentially an automaton. Most would not consider them to be persons. Complexity is (IQ/2)+2. LC4 unless their IQ 15+, then LC3. \label{itm:ai_nonvolition}

\textit{Volitional AI:} ``Strong AI" with just as much initiative and creativity as a living creature of equivalent intelligence. Complexity is (IQ/2)+3. LC4 if IQ 6-8, LC3 if IQ 9-14, LC2 if IQ 15-19, or LC1 if IQ 20+. \label{itm:ai_volition}

\subsection{Databases}
\label{subsec:databases}