A \textit{robot} is a computer~-controlled machine capable of perceiving and manipulating its environment. While many robots exist simply to serve their designed purpose, Androids are robots with volitional AI who are considered people in their own right.

Various robots are described in this book. They can be found in the chapters relevant to their function, e.g., combat robots in the Weaponry part. Racial templates are provided for machines that are suitable as player characters, or which may be associated NPCs (such as allies). Other robots are described as animals or equipment.

Robots are also characterized by the type of intelligence inhabiting them. Any given robot body can have different types of intelligence depending on its software, or the replacement of its directing computer with a cyborg brain.

\section{Digital Intelligences}\label{sec:digital_intelligences}
%\addcontentsline{toc}{section}{Digital Intelligences}
The most typical robot is a machine controlled by a digital intelligence: a sapient self~-aware computer program. 

The complexity of the computer hardware and the software will set a limit on the robot's IQ. 

Most digital intelligences are Artificial Intelligences (p. \pageref{subsec:ai}), or AIs. For robots that do only what you tell them to do, install a non~-volitional AI. For robots that have free will, also known as androids, install a volitional AI. Note, though, that in most civilized societies Android are considered living beings and have the associated rights.

Digital intelligences can also be mind emulations created from uploading human (or other) brains as detailed in Chapter 8. See \textit{Uploading} (p. \pageref{sec:uploading}) and \textit{Mind Emulation (``Ghost'')} Programs (p. \pageref{subsec:mind_emulation_ghost_programs}).

For traits associated with different digital intelligences, see \textit{Machine Intelligence Lenses} (p. \pageref{sec:machine_intelligence_lenses}).

\section{Drones}\label{sec:drones}
A robotic drone is a remotely~-controlled machine that is not sentient: it has IQ 0. It usually has a computer onboard that handles some autonomic functions, such as helping to stabilize a walking or flying drone, but a drone isn't self~-aware. Drones are also known as remotely~-piloted vehicles (RPVs) or teleoperated robots. 

Drones are popular as a physical form of telepresence, an alternative to the digital telepresence of VR.

With the correct command codes, any robot body -- even one housing an AI or cyborg -- can be teleoperated as a drone. 

A drone's computer runs a simple software program (Complexity 3) that controls its body and communication systems. A robot body that is \textit{only} being used as a drone has the drone lens -- see \textit{Machine Intelligence Lenses} (p. \pageref{sec:machine_intelligence_lenses}).

\section{Cyborgs}\label{sec:cyborgs}
A cyborg is a fusion of biological and machine parts. There are two classes of cyborg:

\textit{Partial Cyborgs} are living creatures whose bodies contain mechanical or electronic parts. They do not qualify for the Machine meta-trait. Someone with an artificial heart, bionic leg, or a neural interface implant is a partial cyborg. These cybernetic modifications are covered in \hyperref[ch:cybernetics_uploading]{\textit{Cybernetics and Uploading}}.

\textit{Total Cyborgs} are robot bodies that house an living brain and (sometimes) parts of the spinal cord. Aside from this, they are machines. A total cyborg has a computer that controls many of its functions, but the guiding intelligence is the biological brain. In the case of a total cyborg, the robot's computer is reduced one size (e.g., a personal computer becomes a small computer) and a cyborg brain case inserted.

No special lens is required for a total cyborg: use the unmodified racial template, except that the computer is one size smaller than indicated. Some robot bodies aren't big enough to contain a human-sized brain case; see the individual descriptions. The cyborg brain rules (p. \pageref{sec:total_cyborg_brain_transplants}) specify the space required.

\section{More Information and Options}
For a break-down of relvent meta-traits, attribute scores, advantages, disadvantages, etc. see \textbf{GURPS Ultra-Tech} pp. 29-35.

\section{Swarmbots}\label{sec:swarmbots}
Swarmbots are an alternative to conventional robots. They are insect- to microbe-sized machines, controlled by computers the size of pinhead. These run simple programs modeled on insect behavior patterns. (Microbots might also be cyborgs, containing tiny insect brains!)

A swarm consists of hundreds or thousands of microbots (or countless nanobots) programmed to act in concert. By following a specified pattern of cooperative behavior, the swarm can perform its tasks and then (if so programmed) return to base. Its collective intelligence is much greater than that of any component part.

Swarmbots may supplement or replace conventional robots in industrial, agricultural, medical, espionage, and military applications. They may live within a vehicle's machinery or the structure of a building, performing routine maintenance and repair tasks. Swarmbot toy sets may exist, such as model farms, zoos, communities, or battlefields, all populated by microbot people, vehicles, or animals.

Individual swarmbots are rarely larger than fleas, so it is most convenient to measure swarms in square yards. A typical swarm is one-square-yard in size, but swarms can be larger. Up to 10 swarms can effectively ``stack,'' and a dense swarm can be more effective.

A swarm is defined by picking its area in square yards, its size (microbot or nanobot), and its type. In addition, it may have various chassis or power system options.

\subsection{Microbot and Nanobot Swarms}\label{subsec:microbot_nanobot_swarms}
There are two sizes of swarmbot: microbot and nanobot.

\subsubsection{Microbot}\label{subsubsec:microbot}
Individual microbots are insect-sized, from the size of a fly to a barely visible speck. They may have any chassis (see below) except Dust. A swarm of microbots is sometimes called a ``cyberswarm.'' 

\subsubsection{Nanobot}\label{subsubsec:nanobot}
Nanobots are a brand new technology, less than a decade old. Developed by the now megacorporation Mikra, this technology is primed to revolutionize many aspects of modern technology.

Individual nanobots range from the size of a dust mote to that of a cell. The ``nanoswarm'' is dense enough to be visible, but not easily identified -- those with ground or water movement chassis resemble a slick ``goo,'' while an airborn nanoswarm resembles a cloud of mist or fog. They are sometimes called ``nanomist.''

A nanoswarm can flow through the tiniest holes, and ooze through porous barriers and narrow cracks.

Given how new the technology is, all nanoswarms cost ~x10 normal cost and are at most LC2.

\subsection{Swarm Chassis}\label{subsec:swarm_chassis}
The chassis provides the basic body, motive system, sensors, and brain. A standard swarmbot sensor suite is roughly equivalent to that of a typical insect, such as an ant or bee. A swarmbot brain is collectively equivalent to a non-volitional AI.

Select the chassis for the swarm and calculate its cost. All costs are per square yard of swarm; for swarms larger than a square yard, multiply by the number of square yards.

\subsubsection{Aerostat}\label{subsubsec:aerostat}
This is a tiny lighter-than-air balloon with an air turbine. Nanoswarms with aerostat chassis often resemble clouds of drifting mist or fog. Microbots are Air Move 2; nanobots are Air Move 1. Aerostat swarms are normal cost. 

\subsubsection{Crawler}\label{subsubsec:crawler}
Each swarmbot usually resembles a tiny metallic ant or beetle, or a miniature tracked vehicle. It can move on the ground or swim. Move 3; Water Move 1. Normal cost.

\subsubsection{Crawler, Armored}\label{subsubsec:crawler_armored}
Similar to the crawler (above), but with a tougher shell. Armored crawlers can survive corrosive atmospheres or high pressures, such as the surface of Venus. Armored crawlers are harder to injure: a swarm has twice as many HP. Move 2. +100\% cost.

\subsubsection{Dust (Nanobot only)}\label{subsubsec:dust_nanobot}
The swarm resembles a cloud of dust motes unless examined using Microscopic Vision, bughunters (p. \pageref{subsec:bughunter_swarm}), or a chemscanner (p. \pageref{subsubsec:laser_chemscanner}). Dust drifts until settling to the ground or sticking to solid objects; the individual motes are capable of anchoring themselves. Only Surveillance swarms (p. \pageref{subsec:surveillance_swarm}) may have this option. Dust swarms are -80\% cost.

\subsubsection{Flier}\label{subsubsec:flier}
This looks like a tiny helicopter, or a mechanical wasp or bee. Microbots are Move 1; Air Move 6. Nanobots are Move 1; Air Move 3. Flier swarms are +100\% cost.

\subsubsection{Hopper}\label{subsubsec:hopper}
Each swarmbot slightly resembles a tiny metallic flea or cricket, with long rear legs. Each swarm has Move 4 (including a level of Super Jump). Not available for nanoswarms. Hopper swarms are +50\% cost.

\subsubsection{Space}\label{subsubsec:space_chassis}
The swarm can link together to function as a solar or magnetic sail, accelerating at up to 0.0001 G within the inner solar system (or faster if accelerated by an external laser cannon or particle beam). It can also crawl on the ground at Move 1. Normal cost.

\subsubsection{Swimmer}\label{subsubsec:swimmer_chassis}
The swarm's components resemble tiny robot submarines, tadpoles, or water insects, with teeth and arms. Water Move 4 for microbot swarms, or Water Move 1 for nanobot swarms. Swimmer swarms are normal cost.

\subsection{Disguise}\label{subsec:disguise_swarmbot}
Most swarms can be disguised as a swarm of insects, or built to resemble something else of appropriate size (such as miniature toy soldiers). This costs an extra \$1,000/square yard. A space swarm's disguise is only effective when crawling or drifting. Aerostat swarms cannot be disguised in this way.

Swarms can be given chameleon systems (p. \pageref{subsec:chameleon_surface}) for the same cost as a suit of armor (the swarms have much less weight, but similar surface area).

A disguised swarm's true identity can be determined if it takes damage. 

An RTG-powered swarm (see below) also shows up on radiation detectors at very close range (a few yards). 

\subsection{Power Supply}\label{subsec:power_supply_swarmbot}
Various types of power supplies are available.

\subsubsection{Power Cell}
Swarms use tiny or nanocatalytic fuel cells that are similar to but far smaller than AA cells. These power each bot for 12 hours. Each square yard of a swarm's power cells are equivalent to a single C cell.

A swarm that isn't doing anything consumes minimal power, as does a space swarm that is flying using its solar or magnetic sail. It can remain operational indefinitely.

For Flyer swarms, each hour of flight consumes as much power as two hours of crawling.

The swarm can be charged by entering a swarm hive (p. \pageref{subsec:swarmbot_hive}) and hooking up to an attached power supply; this is just like recharging a C cell. 

\subsubsection{Beamed Power}
The swarm is powered by beamed microwaves (and designed to avoid being fried by them). This works based on the \textit{Beamed Power} (p. \pageref{subsec:beamed_power}); each square yard requires as much power as a C cell.+50\% to cost.

\subsubsection{Gastrobot}
These ``live of the land'' while performing their duties. They eat more than a similar-sized swarm of insects: each square yard of swarm consumes 0.1 lb./hr. of biomass. They breathe air, and cannot survive in vacuum or very low pressures. Combat-capable gastrobots can hunt and kill animals to survive. +100\% cost. 

\subsubsection{Radio-Thermal Generator}
Each swarmbot has a miniscule radio-thermal generator. These use tiny amounts of radioactive material, the decay of which releases energy enough to power the microbot for a year. The swarm can be detected by Geiger counters or other radiation detectors at close range. Due to the radioactive material in their power supply, RTGs are usually limited to space or other hostile environments. +100\% cost. LC1.

\subsubsection{Solar Cell}
The robots in this swarm have built-in solar panels as well as batteries. They recharge energy sufficient for 3 hours of operating time for each hour they remain dormant in full sunlight. +50\% cost.

\subsection{Swarm Types}
A swarm's function depends on the specialized tools, manipulators, programming, and sensors of its robots. (A swarm might actually represent several different types of microbots working together.) Individual swarm types are described in appropriate sections, e.g., terminator swarms in the Weapons chapter. The table below provides a quick reference to the types and their cost per square yard.

\begin{table}[H]
    \hrule height 1pt\medskip
    \subsection{Swarm Type Table}
    \rowcolors{1}{}{\colorcoretech}
    \begin{tabularx}{\columnwidth}{lXrXcXr}
        \textbf{Swarm Type} && \textbf{Cost} && \textbf{LC} && \textbf{Page} \\
        Bughunter && \$4,000 && 3 && \pageref{subsec:bughunter_swarm} \\
        Cleaning && \$1,000 && 4 && \pageref{subsec:cleaning_swarm} \\
        Construction && \$1,000 && 4 && \pageref{subsubsec:construction_swarm} \\
        Decontamination && \$1,000 && 3 && \pageref{subsubsec:decontam_swarm} \\
        Defoliater && \$1,000 && 3 && \pageref{subsubsec:defoliat_swarm} \\
        Devourer && \$8,000 && 1 && \pageref{subsec:devourer_swarm} \\
        Explorer && \$500 && 4 && \pageref{subsec:explorer_swarm} \\
        Firefly && \$100 && 4 &&\pageref{subsec:firefly_swarm} \\
        Forensic && \$4,000 && 3 && \pageref{subsec:forensic_swarm} \\
        Gremlin && \$2,000 && 2 && \pageref{subsec:gremlin_swarm} \\
        Harvester && \$2,000 && 4 && \pageref{subsubsec:harvest_swarm} \\
        Massage && \$200 && 4 && \pageref{subsubsec:massage_swarm} \\
        Paramedical && \$6,000 && 3 && \pageref{subsec:paramedical_swarm} \\
        Pesticide && \$1,000 && 3 && \pageref{subsubsec:pesticide_swarm} \\
        Play && \$200 && 4 && \pageref{subsubsec:play_swarm} \\
        Pollinator && \$1,000 && 4 && \pageref{subsubsec:pollinat_swarm} \\
        Repair && \$500* && 4 && \pageref{subsubsec:repair_swarm} \\
        Security && \$1,000 && 3 && \pageref{subsec:security_swarm} \\
        Sentry && \$5,000 && 3 && \pageref{subsec:sentry} \\
        Stinger && \$1,500 && 2 && \pageref{subsec:stinger_swarm} \\
        Surveillance && \$50 && 3 && \pageref{subsec:surveillance_swarm} \\
        Terminator && \$1,500 && 1 && \pageref{usbsec:terminator_swarm} \\
    \end{tabularx}
    \label{tbl:swarm_type}
\end{table}

* +\$250 per equipment type it can repair.

\subsection{Swarmbot Operation}
A swarm can take orders from any computer running an appropriate program (see below). Swarms can send and receive radio, laser, or infrared signals, with a range of about 0.01 for infrared or laser and 0.1 miles for radio. The operator must know the command codes for that swarm. Orders are limited to actions related to the swarm's equipment package, movement, or recharging.

\textit{Swarm Controller Software:} Lets a user command and control a microbot swarms using a radio, laser, or infrared communicator. The GM can make a secret Electronics Operation (Robots) roll to see if the swarm understands the orders (apply penalties for confusing instructions). Failure means the swarm does not do exactly what was intended. A separate program is needed for each swarm type. Complexity 4, normal cost. LC is that of the swarm.

\subsubsection{Combat}
Swarms capable of combat usually attack any entity they find while following a preprogrammed path -- e.g., to ``sterilize'' an area or to sweep a security perimeter. Swarms may be programmed to differentiate by species or even by sex, using chemical sensors equivalent to Discriminatory Smell (this will not work on targets in airtight armor).

\subsubsection{Multiple Swarms}
Friendly swarms can work together, but swarms generally avoid ``stacking'' unless commanded to do so.

\subsubsection{Fighting Swarms}
Cannibal, Disassembler, Devourer, Gremlin, Sentry, Stinger, or Terminator types may make attacks. Use the rules for Swarm Attacks (p. B461), except that only Stinger swarms are slowed by clothing. The Attacking a Swarm rules also apply -- the swarm is treated as Diffuse, but it can be stomped or swatted.

All swarms are assumed to have the Sealed advantage. Any swarm with the Gastrobot power plant has the equivalent of Doesn't Breathe (Oxygen Combustion, -50\%); others have Doesn't Breathe and Vacuum Support.

\subsection{Swarmbot Hive}\label{subsec:swarmbot_hive}
This container can house a square yard of swarmbots, allowing them to recharge from external pose. \$200, 10 lb. LC4.

\multicolinterrupt{
    \begin{table}[H]
        \hrule height 1pt\medskip
        \subsection{Swarm Statistics Table}
        \rowcolors{1}{}{\colorcoretech}
        \centering
        \begin{tabularx}{\linewidth}{lXccccccccXcXc}
            \textbf{Type} && \textbf{ST} & \textbf{DX} & \textbf{IQ} & \textbf{HT} & \textbf{BL} & \textbf{HP} & \textbf{Will} & \textbf{Per} && \textbf{Basic Speed} && \textbf{Basic Move} \\
            Microbot && 2 & 10 & 3 & 10 & 0.8 lb. & 10 & 10 & 10 && 5 && varies \\
            Nanobot && 1 & 10 & 2 & 10 & 0.2 lb. & 20 & 10 & 9 && 5 && varies \\
        \end{tabularx}
        \label{tbl:swarm_statistics}
    \end{table}
}