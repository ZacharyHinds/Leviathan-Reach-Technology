A \textit{robot} is a computer~-controlled machine capable of perceiving and manipulating its environment. While many robots exist simply to serve their designed purpose, Androids are robots with volitional AI who are considered people in their own right.

Various robots are described in this book. They can be found in the chapters relevant to their function, e.g., combat robots in the Weaponry part. Racial templates are provided for machines that are suitable as player characters, or which may be associated NPCs (such as allies). Other robots are described as animals or equipment.

Robots are also characterized by the type of intelligence inhabiting them. Any given robot body can have different types of intelligence depending on its software, or the replacement of its directing computer with a cyborg brain.

\section{Digital Intelligences}\label{sec:digital_intelligences}
%\addcontentsline{toc}{section}{Digital Intelligences}
The most typical robot is a machine controlled by a digital intelligence: a sapient self~-aware computer program. 

The complexity of the computer hardware and the software will set a limit on the robot's IQ. 

Most digital intelligences are Artificial Intelligences (p. \pageref{subsec:ai}), or AIs. For robots that do only what you tell them to do, install a non~-volitional AI. For robots that have free will, also known as androids, install a volitional AI. Note, though, that in most civilized societies Android are considered living beings and have the associated rights.

Digital intelligences can also be mind emulations created from uploading human (or other) brains as detailed in Chapter 8. See \textit{Uploading} (p. TODO) and \textit{Mind Emulation (``Ghost'')} Programs (p. TODO).

For traits associated with different digital intelligences, see \textit{Machine Intelligence Lenses} (p. TODO).

\section{Drones}\label{sec:drones}
A robotic drone is a remotely~-controlled machine that is not sentient: it has IQ 0. It usually has a computer onboard that handles some autonomic functions, such as helping to stabilize a walking or flying drone, but a drone isn't self~-aware. Drones are also known as remotely~-piloted vehicles (RPVs) or teleoperated robots. 

Drones are popular as a physical form of telepresence, an alternative to the digital telepresence of VR.

With the correct command codes, any robot body -- even one housing an AI or cyborg -- can be teleoperated as a drone. 

A drone's computer runs a simple software program (Complexity 3) that controls its body and communication systems. A robot body that is \textit{only} being used as a drone has the drone lens -- see \textit{Machine Intelligence Lenses} (p. TODO).

\section{Cyborgs}\label{sec:cyborgs}
A cyborg is a fusion of biological and machine parts. There are two classes of cyborg:

\textit{Partial Cyborgs} are living creatures whose bodies contain mechanical or electronic parts. They do not qualify for the Machine meta-trait. Someone with an artificial heart, bionic leg, or a neural interface implant is a partial cyborg. These cybernetic modifications are covered in \hyperref[ch:cybernetics_uploading]{\textit{Cybernetics and Uploading}}.

\textit{Total Cyborgs} are robot bodies that house an living brain and (sometimes) parts of the spinal cord. Aside from this, they are machines. A total cyborg has a computer that controls many of its functions, but the guiding intelligence is the biological brain. In the case of a total cyborg, the robot's computer is reduced one size (e.g., a personal computer becomes a small computer) and a cyborg brain case inserted.

No special lens is required for a total cyborg: use the unmodified racial template, except that the computer is one size smaller than indicated. Some robot bodies aren't big enough to contain a human-sized brain case; see the individual descriptions. The cyborg brain rules (p. \pageref{sec:total_cyborg_brain_transplants}) specify the space required.
