\section{Power Cells}\label{sec:power_cells}
%\addcontentsline{toc}{section}{Power Cells}
The most popular and common power cells in the universe are those produced by Andromeda Electric using Susanoo's storm in the Andromeda Galaxy. While there are non-AE power cells, they are uncommon.

There are several sizes of power cell, designated by letter from AA (the smallest) to F (the largest). Power cells increase in power exponentially. An A cell is 10 times as powerful as an AA cell, a B cell has 10 tomes the power of an A cell, and so on.

\textit{AA cell:} These tiny cells operate devices which need minimal power. \$1, 0.0005 lb. \label{itm:aa_cell}

\textit{A cell:} The size of a watch battery, or postage stamp-sized for flexible cells (see below). \$2, 0.005 lb. \label{itm:a_cell}

\textit{B cell:} The size of a pistol cartridge or 21st century AA battery. \$3, 0.05 lb. \label{itm:b_cell}

\textit{C cell:} The most common type of power cell. Most equipment designed for larger or smaller cells often include an adapter for C-cell operation. Each cell is about the same size as a pistol magazine. \$10, 0.5 lb. \label{itm:c_cell}

\textit{D cell:} Often worn as a power pack, they're about the size of a thick paperback book. \$100, 5 lb. LC4. \label{itm:d_cell}

\textit{E cell:} About the size of a backpack. \$2,000, 20 lb. LC4. \label{itm:e_cell}

\textit{F cell:} About the size of a compact car engine. \$20,000, 200 lb. LC4. \label{itm:f_cell}

%\addcontentsline{toc}{subsubsection}{Flexible Cells}
% \section{Flexible Cells}
\subsubsection{Flexible Cells}
Andromeda Electric also sells flexible variants of their power cells. They are attached like stamps and peeled off when exhausted. Gadgets noted as using flexible cells use them \textit{instead} of normal power cells; they're also embedded into smart labels, smart paper, and similar disposable items. AA and A cost the same but larger sizes are 4 times the normal cost.

%\addcontentsline{toc}{subsubsection}{Non-Rechareable Power Cells}
\subsubsection{Non-Rechargeable Power Cells}

Normal cells and flexible cells are rechargeable, but there are also non-rechargeable cells. They last twice as as long (or provide twice as many shots) but may not be recharged. They are otherwise identical.

%\addcontentsline{toc}{subsection}{Replacing Power Cells}
\subsection{Replacing Power Cells}
It takes three seconds to replace an A, B, or C cells with a new one, or 5 seconds to replace a tiny AA or heft D or E cell, or 20 seconds to replace an F cell. Cells can only be replaced if the user is strong enough to light them out.

Fast-Draw (Ammo) skill can be used to reduce the time for cells loaded into weapons. A successful skill roll reduces the replacement time by one second.

% \begin{table}[b]
%     \centering
%     \caption*{Replacing Power Cells}
%     \begin{tabular}{|r|l|}
%         Power Cell & Time to Replace \\
%         \hline
%         A,B,C & 3 seconds \\
%         AA,D,E & 5 seconds \\
%         F & 20 seconds
%     \end{tabular}
%     \label{tab:replace_power}
% \end{table}

%\addcontentsline{toc}{subsection}{Exploding Power Cells}
\subsection{Exploding Power Cells}
AE Power Cells contain the volatile energy of Susanoo's storm. As a result, they can explode if destroyed. When they explode, they act as an explosive of the same weight as the cell with REF of 2. The explosion does crushing explosion surge damage.

%\addcontentsline{toc}{subsection}{Jury-Rigging Power cells}
\subsection{Jury-Rigging Power cells}
Even though devices are designed for a specific size of cell, different cells can be jury-rigged to work. Ten cells that are one size category smaller can substitute for a single larger cell. This requires a roll against Electrician-2 and 10 minutes of work per attemt; critical failure damages the gadget.

Given the prevalence of AE's cells, their design has become the standard and even other company's cells will work in products designed for them. The exception to this is when dealing with older cells. Rolls to use them are at an additional penalty which varies based off the age of the cell.

\subsection{Capacitor Fungus}
Capacitor fungus, also known as ``Zappers,'' are a common growth on the conductive parts of spaceships, sapping their power. If these mushrooms sap enough power, they can act as single-use power cells. They grow based on the amount of power they have consumed, roughly the same size as their equivalent power cell. 

After being harvested, they lose their power quickly unless plugged into a power system, an unused, harvested Zapper is drained of all its power after 24 hours. This can be staved off 

Using Zappers as power cells requires jury-rigging (above), though a proper Raven-based skill (Biology (Raven), Mechanic (Raven-Tech), etc.) can be used in place of Electrician at an additional -2. Furthermore, in addition to the damage to the gadget, the person jury-rigging takes burn surge damage as the zapper shocks them: 1d-3 for equivalent of AA or A, 1d-1 for A or B, 1d for C or D, 1d+1 for E, and 2d for F. 

\begin{ravenbox}
    \subsection{Zappers}\label{subsec:zappers_raven}
    Akin to barnacles on boats, Zappers are a Raven-fungus which cling to the conductive parts of spaceships and sapping energy from their power systems. They get their name from the small shocks they can deliver on touch (generally 1d-2 to 2d burn sur damage). They are edible, requiring their electricity being discharged with a Cooking roll. Additionally, they are used as power cells (above) and can be used as ``self-constructing'' wires or power connections.
    
    Large enough growths will occasionally eject clouds of electrified spores. Ships which fly through these clouds may have their electronics negatively temporarily disabled. 
    
    If they grow large enough, they begin reducing the number of Power Points available for the ship. Removing them is a part of the regular maintenance done to spaceships.
\end{ravenbox}

%\addcontentsline{toc}{section}{Generators}
\section{Generators}
%\addcontentsline{toc}{subsection}{Fission Generators}
\subsection{Fission Generators}
\textit{Semi~-Portable Fission Reactor:} A small fission reactor that fits in a truck bed. \$100,000, 1,000 lb. Typically provides external power for 5 years before refueling and maintenance; Refueling and maintenance is \$50,000. LC2. \label{itm:sp_fission_react}

%\addcontentsline{toc}{subsection}{Fusion Generators}
\subsection{Fusion Generators}
\textit{Semi~-Portable Fusion Reactor:} A small nuclear fusion reactor. Fuses hydrogen into helium. \$200,000, 100 lb. Its internal fuel supply lasts for up to 20 years; refueling and maintenance is \$20,000. LC3. \label{itm:sp_fusion_react}

%\addcontentsline{toc}{section}{Energy Collection}
\section{Energy Collection}
%\addcontentsline{toc}{subsection}{Solar Power}
\subsection{Solar Panels}
%\addcontentsline{toc}{subsubsection}{Solar Power Array}
\subsubsection{Solar Power Array}
Semi-portable array of solar panels that provides external power. It takes a minute to deploy and covers about 400 square feet. \$10,000, 500 lb. These sizes are based on earth-like levels of sunlight; multiply cost and weight by relative light levels for other environments. LC4 \label{itm:solar_array}

%\addcontentsline{toc}{subsubsection}{Solar Paint}
\subsubsection{Solar Paint}
Cheap plastic solar cells that can be painted onto any surface, including clothing or rooftops. A coating of solar paint is only 20\% of the cost and weight of regular solar panels, but requires twice the surface area and has no DR. LC4 \label{itm:solar_paint}

\subsubsection{Solar Backups}
Adding small solar panels to gadgets lets them trickle-charge in daylight. It costs 20\% of the cost of the power cells. Recharging could take a few days to weeks, depending on the device's surface area relative to power capacity.

\subsection{Backpack Power Unit}
Andromeda electric offers a new, powerful \textit{and} portable power supply for all your recharging needs. A solid-state portable nuclear battery unit, this device can recharge an E cell in one hour and has connectors for any size. It is guaranteed to last upwards of one year, afterwhich it should be replaced. It has DR 40, HP 15, and HT 15. Warning, if damaged and disabled (fails an HT check), it leaks radiation (1 rad/hour), but cannot explode. \$50,000, 50 lb. LC2. 

%\addcontentsline{toc}{section}{Beamed and Broadcast Power}
\section{Beamed and Broadcast Power}\label{sec:beamed_broadcast_power}
Devices may operate on beamed or broadcast power coming from a central station. Generally, they must remain within line of sight. This is commonplace in the Andromeda Galaxy as AE has set up widespread infrastructure. Other planets may have something similar set up as well.

If needed, the cost of this power per month is generally equal to 1\% the cost of a customer's power receivers

%\addcontentsline{toc}{subsection}{Beamed Power}
\subsection{Beamed Power}\label{subsec:beamed_power}
Constructed like power cells, these weight the same as the normal power cell they are replacing. They operate indefinitely as long as it is in the line of sight of the transmitter. If sight is interrupted, the receiver's backup system provides power for 1/10th as long as the normal power cell. Cost is the same, but only D cells or larger are available. Beamed power transmitters are usually 10 times the cost and double the weight of an equivalent power cell per mile or fraction of a mile of range; they power one system at a time.

%\addcontentsline{toc}{subsection}{Broadcast Power}
\subsection{Broadcast Power}\label{subsec:broadcast_power}
This relatively recent innovation from Andromeda Electric works similarly to beamed power, but does not require line of sight for transmission. Broadcast receivers are 10 times as expensive as normal power cells and can come in any size, not just D and up. Broadcast power transmitters are generally double the cost and weight of an equivalent power cell per yard of radius.