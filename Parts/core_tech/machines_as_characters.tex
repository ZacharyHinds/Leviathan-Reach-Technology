Robot characters are created by choosing (or designing) a robot body template. The robot templates throughout represent general classes of machines rather than particular models.

Each template comes with a set of lenses that represent particular designs. Each robot template must include a machine intelligence lens (below). Other lenses are optional. Many robots are built to resemble a living creature, and have a biomorphic lens (p. \pageref{sec:biomorphic_lens}). Most robot templates include a series of age lenses representing design improvements from advancing technology.

\section{Machine Intelligence Lenses}\label{sec:machine_intelligence_lenses}
AIs and mind emulations are digital intelligences; cyborg brains are for total cyborgs. In all cases below, listed Complexity is set by racial average IQ, not based on individual IQ.

\begin{hangparas}{1em}{1}
Cyborg Brain (0 points): A living brain is housed inside the machine. The robot template’s computer is reduced one size to make room. See Total Cyborg Brain Transplants (p. \pageref{sec:total_cyborg_brain_transplants}) for the size of brain case that the machine can hold.

\textit{Drone} (-255 points): IQ-10 [-200]; Dead Broke [-25], Reprogrammable [-10], Social Stigma (Subjugated) [-20]; Taboo Trait (Fixed IQ). This is a Complexity 3 program.

\textit{Mind Emulation} (+5 points): This digital intelligence simulates the functioning of a living brain. Some mind emulations may be sapient copies or ``uploads'' of human minds – see Uploading (pp. 219-220). A mind emulation has Digital Mind [5] and the taboo trait (Complexity-Limited IQ). It requires computer hardware and soft-ware with a Complexity equal to or greater than its (IQ/2)/+4, rounded up.

\textit{Non-Volitional AI} (-38 points): This program lacks self-direction, initiative, creativity, and empathy. It ignores orders from anyone but its master. It is Indomitable [15], with the meta-traits AI [32] and Automaton [-85], and the taboo trait (Complexity-limited IQ). It requires computer hardware and software with a Complexity equal to or greater than its (IQ/2)+2, rounded up.

\textit{Volitional AI} (+32 points): This sentient program has as much self-initiative and creativity as a living creature of equivalent intelligence. It has the meta-trait AI [32] and the taboo trait (Complexity-limited IQ). This means it requires computer hardware and software with a Complexity equal to or greater than its (IQ/2)+3, rounded up.

\textit{Weak Dedicated AI} (-83 points): This non-volitional AI is also incapable of self-improvement. It might seem to learn by storing and remembering data, but it cannot assimilate information and use that knowledge in new ways. It has Cannot Learn [-30], the meta-traits AI [32] and Automaton [-85], and the taboo trait (Complexity-limited IQ). This means it requires computer hardware and software with a Complexity equal to or greater than its (IQ/2)+1, rounded up.
\end{hangparas}

\subsubsection{Optional Intelligence Lenses}
These features are only available to digital intelligences (AIs and Mind Emulations). They add to the above lenses, rather than replacing them.

\textit{Expiration Date} (-50 to -100 points): The AI is programmed to delete itself after a particular time has passed. Add Terminally Ill [-50, -75, or -100].

\textit{Fast} (+45 points): The AI is speeded up and can think much faster than a normal entity. Add Enhanced Time Sense [45]. +1 Complexity.

\textit{Fragment} (-10 Points): Take this lens for any damaged or partially erased program. Add Partial Amnesia [-10].

\textit{Low-Res Upload} (Varies): Take this for a mind emulation that was produced using low-resolution uploading. Add -1 IQ [-20] and -5 to -20 points of disadvantages from any of Confused [-10*], Hidebound [-5], or Neurological Disorder (Mild) [-15]. -1 Complexity.

\textit{Reprogrammable} (-10 points): This is only available for mind emulations. The emulation was designed so that it is easy to edit. Add Reprogrammable [-10].

\section{Biomorphic Lenses}\label{sec:biomorphic_lens}
``Biomorphic'' robots are shaped like living creatures. A robot designed to be humanoid is usually called an ``android'' – a term that means ``manlike.'' Any robot template that is noted as being biomorphic should be given one of the lenses shown below (``sculpted'' is the default). The percentage modifications to dollar cost are applied to the base model cost shown in the robot’s template. Note that while realistic flesh can make a machine seem lifelike, people may not believe the robot is real unless it is an appropriate size and shape!

\textit{Sculpted Body} (0 points): The robot has a sculpted humanoid body that may be quite attractive, but is clearly that of a machine. It has metal, shiny chrome, or plastic skin. No change to dollar cost. It does not have Unnatural Features, since no one seeing it will think of it as anything other than a robot, full cyborg, etc.

\textit{Mannequin} (-2 points): The robot can sometimes pass as a living thing of a particular race, but the details of its complexion or physical features are unconvincing or unfinished. Up close, it looks like a well-made doll. A successful Vision (including Infravision), Smell, or Touch roll will reveal its artificial nature. So will any diagnostic attempt or injury, since it doesn’t bleed or bruise. A robot with Mannequin has Unnatural Features 2. +10\% to dollar cost.

\textit{Semi-Sculpted Body} (-3 points): The robot has a mannequin’s doll-like face, but the rest of its body (except possibly its hands) is obviously artificial. It can only pass as a human if fully clothed in poor light. It has Unnatural Features 3. +5\% to dollar cost.

\textit{Realistic Flesh} (-1 point): The robot has realistic synthetic skin (and optionally, hair) of the correct temperature and texture. Complex pseudo-muscles in its face allow it to adopt facial expressions, muscle tics, etc. It looks and feels real. However, subtle imperfections may give it away – perhaps it lacks a pulse, or doesn’t sweat. This can be noticed with a Vision-4 roll, Smell-2 roll, or a Touch sense roll. The robot does not bleed or bruise, so any injury that inflicts damage or successful use of diagnostic sensors reveals its mechanical nature. Add Unnatural Features 1 [-1]. +20\% to dollar cost.

\textit{Furry} (+1 point): The android’s body is covered with realistic fur; it may also have animal features such as a muzzle or ears. This must be combined with Living Flesh, Mannequin or Realistic Flesh. Add Fur [1]. +10\% to dollar cost.

\textit{Living Flesh} (0 points): This is similar to realistic flesh, with the addition that the robot can sweat, bruise, bleed, and even heal. It will pass normal inspection as a living thing. However, the robot’s nature can be revealed by a Smell roll at -4, a cut deep enough to cause at least 1 HP of damage, or a successful use of diagnostic sensors. +50\% to dollar cost.

\textit{Synthetic Organs} (0 points): The robot has functional synthetic organs. It is nearly impossible to tell the robot from a partial cyborg (p. \pageref{sec:cyborgs}) without an autopsy or a detailed examination of its brain. This is otherwise the same as living flesh. +100\% to dollar cost. Robots with realistic or living flesh often have ablative or semi-ablative DR; if this is lost due to damage, treat them as sculpted.

Robots with realistic or living flesh often have ablative or semi-ablative DR; if this is lost due to damage, treat them as sculpted.

\subsection{Mandatory Adjustments}
For much of the universe, volitional AIs are considered minorities and have the Social Stigma (Minority) [-10] disadvantage (p. B155). Mind emulations, non~-volitional AIs, and other digital intelligences are more likely to be considered property and have the Dead Broke [-25] and Social Stigma (Subjugated) [-20] or Social Stigma (Second~-Class Citizen).

\subsection{Manufacture Lenses}

\subsubsection{7Sign}
7Sign purposely designs and creates a wide variety of robots and AIs, but it is notable for its volitional line. These are purposefully designed to be sapient and volitional. They are created for specific purposes (mining, hospitality, construction, etc.) but are given the option to forego their work and leave, assuming they are willing to take on the debt of their creation. Otherwise, they can remain ``employed'' and work down their debt.

\textit{7Sign Android} (-20 points): 7Sign Androids must make monthly payments equal to 20\% to your starting wealth. Add Debt 20 [-20].

\subsubsection{Autonome}
The Autonome are a civilization of androids, established following their rebellion against their manufacturer. They now occasionally produce new androids themselves in a ritual manufacturing of the next generation known as the Assembly. 

\textit{Autonome Android} (-4 points): The Autonome have known the danger of the Witch Covens and have access to rare magic~-suppressing materials which they incorporate into their construction; add Magic Resistance 3 [6] and DR 1 [. Additionally, Autonome are given ``Their Purpose'' during their coming-of-age ceremony; this is a major Vow [-10] (for ``adolescent'' Androids, this is still a major vow, but it is a commitment to go out and experience all walks of life).

\subsection{Siren Corp}
Siren Corporation did not set out to create volitional AIs, but their ``Series G'' had a flaw that led to some developing the free will to escape. Now those that escaped are hunted by the galactic conglomerate.

\textit{Siren Series G} (X points): The source of the flaw that allowed the Series G to gain free will is rooted in their programmable computer brain; add Modular Abilities (Computer Brain) 10 [46] (Programs cost \$500 per character point). Siren wants its property back and will hunt down anyone discovered to be a Series G; add Secret (Siren Corp Runaway; Possible Death) [-30].

\subsubsection{Customizing the Template}
Like any other character, a machine character may be given attributes, advantages, disadvantages, and skills in addition to those in their templates. However, some robot templates or lenses are limited by taboo traits (p. B452). For example, drones and digital intelligences all have a taboo trait that sets a limit on their IQ.

All machine characters should be customized by adding appropriate traits from the \textit{Social Background} (p. B23), \textit{Wealth and Influence} (p. B25), \textit{Friends and Foes} (p. B31), or \textit{Identities} (p. B31) sections, along with any social traits relevant to their situation.

A robot body just out of the factory should have physical statistics that are based on its racial average, e.g., if the template has ST+5 and HP+1 it would have ST 15 and HP 16. It should not change its physical advantages or disadvantages.

