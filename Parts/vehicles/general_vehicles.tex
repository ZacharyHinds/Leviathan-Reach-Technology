\section{Planetary Travel}\label{sec:planetary_travel}

% \subsection{Slidewalks}\label{subsec:slidewalks}
% In large space habitats or the downtown core of crowded cities, roads and sidewalks might be replaced by the slidewalk, a passenger conveyor belt similar to a high-speed horizontal escalator. The slidewalk is made up two sets of belts running in opposite directions with a platform in between. Each set operates at five-mph intervals from 5 mph to 30 mph. The slowest belts are nearest the platform.

% While walking is easy, running on a slidewalk is difficult. Add or subtract two mph for each point of Move to the slidewalk's speed to give the speed of travel. Make a DX+3 roll each turn to avoid falling. This roll is -1 per 10 mph if running in the direction that the slidewalk is moving, or at -3 per 10 mph if running in the opposite direction. Running on a slidewalk is ill~-mannered, and may be illegal. GMs can impose reaction penalties or legal penalties depending on the culture.

% An uncommon sight on most planets, but still found here and there such as 

\subsection{Self~-Driving Vehicles}\label{subsec:self-driving_vehicles}
On the more high~-tech planets, such as Eurydice or Armitage, automated vehicles vehicles have become the norm, making use of intertial guidance systems, GPS, anti~-collision radar, and computer autopilts. Generally speaking these services vary from planet to planet, but fare is generally in the range of \$2 to \$5 a mile.

In some particularly dense and/or tightly controleld regions, such as Riveria, Armitage, it is actually illegal for a non~-AI or AI~-assisted human to drive.

\subsection{Mag~-Lev Trains}\label{subsec:mag-lev_trains}
Subway and commuter trains may use magnetic levitation (mag-lev) for propulsion, eliminating rail friction and allowing speeds of up to 300 mph. Mag-lev lines are most efficient when constructed in evacuated tunnels, removing all air resistance and enabling the trains to reach supersonic speeds. If intercontinental tunnels are built, 1,000-mph mag-lev trains replace aircraft and surface shipping. The capital investment for an evacuated mag-lev system is enormous; only very wealthy societies can afford one. Operating costs are comparatively low, however, so a railway with the capital investment paid off can have cheap fares. Government subsidies may help pay for the infrastructure.

Some planets and cities use magnetic leviation (mag~-lev) trains as a primary form of transit both for people and goods. Before its annihilation, the subterranean settlements on Awei in the Milky Way were known for their immense planetary train system. 

Mag-lev is also cheap on worlds with no atmosphere. Under these circumstances, fares average \$20 per person (or \$100 per ton of cargo) per 1,000 miles.

\subsection{Super Airships}\label{subsec:super_airships}
In atmosphere, cargo and passenger airships may be the most economical long-distance transport. Airships are cheap and reliable, though limited to worlds with dense or standard atmospheres and reasonably placid weather. Airships average 50 mph over long hauls, and streamlined vessels using airfoils and aerostats may fly faster than 100 mph. Fares are usually \$10 per 100 miles, per person or per half ton of cargo. Airships need only minimal facilities, and can get by with no more than a mast to tie up to, so they are an attractive option for undeveloped planets.

The floating cities of Criaphus are a prime example of this infrastructure. The Large airships suffle people from one floating city to the next atop the planet's dense atmosphere.

\subsection{Ballistic Liners}\label{subsec:ballistic_liners}
Hypersonic suborbital space planes can carry 100 or more passengers to anywhere on an Earth-sized planet in less than three hours. The cost is \$500 to \$1,000 per person or ton of cargo. Suborbital vehicles require extensive takeoff and landing facilities, and are unlikely to make stops at small towns, frontier outposts, or lonely archaeological sites.

\subsection{Transcontinental Tunnels}\label{subsec:transcontinental_tunnels}
Robots and advanced boring machines can dig tunnels between continents for about \$10 million per mile. (In contrast, the Channel Tunnel between France and the United Kingdom is 31 miles long and cost about \$20 billion.) These are generally used for supersonic mag~-lev rail lines.

\subsection{Fusion~-Powered Watercraft}\label{subsec:fusion-powered_watercraft}
Surface ships are one of the most economical ways to move massive cargoes over intercontinental distances. Vehicle-sized fusion reactors are likely to be used in both spacecraft and ocean-going vessels.

\subsection{Contragravity}\label{subsec:contragravity}
Gwhel generators have had a significant impact on transportation, across sectors. While they are primarily employed in gwhel~-drive engines to propel spaceships, recent advances have seen smaller gwhel~-generators entering the market, allowing for much smaller vehicles and structures to make use of their contragravity technology.

\section{Space Travel}\label{sec:space_travel}
Space travel is covered in detail in the \textit{Spaceships} chapter, p. TODO. That said, for space travel between the largest, most populated of worlds, spaceships land and depart from space ports, many of which reside out in the planets' orbits, rather than on their surfaces. The primary method of getting between these space ports and the surface is through space elevators.

\subsection{Space Elevator}\label{subsec:space_elevator}
A space elevator, or beanstalk, is a super-strong cable running from the equator on a planetary surface into geostationary orbit (about 21,700 miles up for Earth). The beanstalk is built using carbon nanotube cables, thickest at the base, narrower at the top. A counterweight -- either an extension of the cable (useful for snagging and hurling spacecraft) or an asteroid or space station -- is attached to the other end. Elevator cars run up and down the cable, taking anywhere from a day to a week to reach the top. After the construction cost has been paid, it might cost as little as \$3/lb. to reach orbit.

Beanstalks may range from satellite-sized systems with hair-thin cables to giant mega-structures with bus-sized elevator cars for passengers and cargo. The cost of construction is \$40 billion per ton/day of cargo capacity (assume that passengers require about a ton each due to safety and life support requirements). Multiply cost by the square of local gravity, e.g., a lunar beanstalk (1/6 G) is 1/36 as expensive. The cost of construction may increase if space junk, satellites, or low-orbiting moons must be cleared away first! LC2.

Riding a beanstalk elevator to or from orbit costs \$500 and takes from six hours to a week, depending on elevator speed. The best elevator cabins resemble those of trains, with food service, entertainment, and a spectacular view.

\subsection{Vehicle Systems}\label{subsec:vehicle_systems}
\textit{Crashweb:} An ``smart'' airbag that provides 10 ablative DR for seated vehicle occupants involved in a crash or collision. An activated crashweb will prevent the user from doing anything until he gets free (DX-2 roll to do so each turn).

If a collision is expected and the occupant is not worried about surviving it, he can turn off the crashweb. This feature is common in military designs or libertarian societies, but civilian passenger vehicles sold in CR3+ societies may require the crashweb to be operational. To disable a crashweb in such a civilian vehicle, make an Electronics Repair (Security) roll; each attempt takes one minute.

\textit{Full Life Support:} A vehicle with full life support is completely sealed (p. \pageref{subsec:threat_protection}). It recycles air and water supply for its occupants as long as it has power. It can function normally in vacuum or other hostile environments. Its climate control system provides a comfort zone extending from absolute zero to 500\degree{}F. Some vehicles can moderate even higher temperatures.

\textit{Limited Life Support:} This functions like full life support as long as it has power, but does not recycle air or water; it only has enough for a limited duration, specified in man~-days. Six man~-days of support can provide air and water for one adult for six days, two for three days, and so on. The vehicle can replenish this supply if it has a source of breathable air or drinkable water.

\textit{NBC (Nuclear-Biological-Chemical) Kit:} This is an environmental control system equipped with sensors to detect contaminants, filters, and an overpressure system (the interior is kept at a higher pressure than outside) to keep impure air out. Much like a filter mask (p. \pageref{subsubsec:filter_mask}), it defends against nuclear fallout, germs, and chemicals such as pollution or poison gas. Only people entirely inside the vehicle may benefit from an NBC kit.

\section{ATVs}\label{sec:atvs}
These ground vehicles are designed for off~-road travel in trackless wilderness. Modern ATVs aren't just designed for Earth -- they're built for hostile environments on alien worlds as well.

\subsection{Wheeled ATV}\label{subsec:wheeled_atv}
This is an eight~-wheeled all-terrain vehicle used by survey teams and prospectors. The wheels have oversized self~-inflating tires and independent electric motors. The vehicle is powered by a pair of F cells, giving it a range of 500 miles. Its tough composite hull can survive up to 30 atmospheres of pressure. It has full life support (p. \pageref{subsec:vehicle_systems}) and radiation PF 2.

Standard equipment includes headlights, a one~-man airlock, an inertial navigation system (p. \pageref{subsec:inertial_nav}), a large radio (p. \pageref{itm:radio_comm_large}), a personal computer (p. \pageref{itm:person_com}), and three workstation terminals (p. \pageref{itm:terminal_wrkst}). An auxiliary solar panel can operate all onboard systems except the motors indefinitely, allowing the vehicle to be used as a base camp.

Use Driving (Heavy Wheeled) skill to operate it. It has a watertight hull and auxiliary hydrojet propulsion system, and can swim at Move 1/4.

\subsection{Exo~-Spider}\label{subsec:exo-spider}
This is a car~-sized eight~-legged vehicle designed for the roughest terrain, such as icy mountains or craters. It is powered by a closed~-cycle turbine engine with a range of 360 miles. Its hull can withstand 30 atmospheres of pressure, and it has limited life support (10 man~-days) and radiation PF 10.

It has the same equipment as the wheeled ATV (above), except that it is equipped with a holographic crew station (p. \pageref{itm:terminal_holo_crew}). Use Driving (Mecha) skill to operate it.

\section{Personal Vehicles}\label{sec:personal_vehicles}
These are standard commuter vehicles.

\subsection{Smart Car}\label{subsec:smart_car}
This is an electric car with a motor in each wheel, a light alloy and composite body, and a fuel cell power plant. It runs for six hours with a cruising range of 400 miles.

It is equipped with a computerized crew station (p. \pageref{itm:terminal_com_crew}), an inertial compass (p. \pageref{itm:inertial_compass}), an infrared surveillance camera (p. \pageref{itm:infrared_surveillance_cam}), a rugged personal computer (p. \pageref{itm:person_com}), small ladar (p. \pageref{itm:small_ladar}), a small cellular radio (p. \pageref{itm:radio_comm_small}), and an entertainment console (p. \pageref{itm:entertainment_console}). Each occupant has a crashweb (p. \pageref{subsec:vehicle_systems}). The car has a biometric lock (p. \pageref{itm:briometric_scanlock}) on its doors, and the vehicle is also equipped with headlights and tail lights.

The driver uses Driving (Automobile) skill to operate it, although the vehicle is often driven under computer control.

\subsection{Dynamic Car}\label{subsec:dynamic_car}
This four-wheeled electric vehicle is powered by superconductor cells. It can operate for 12 hours with a cruising range of 1,000 miles. The vehicle's reconfigurable smart skin subtly adjusts the shapes of the body and the wheels for optimum aerodynamics and ground traction.

The car has the same systems as the smartcar described above, but its body has programmable camouflage (p. \pageref{subsec:program_camouflage}) allowing it to change color and pattern on command. The interior features self-adjusting memswear (p. \pageref{itm:memswear}) seats. If empty, it can fold into a SM+2 box for ease of parking. A \$100, LC3 hack lets it do this on command, doing thrusting crushing damage to occupants based on its ST.

\subsection{Flying Cars}\label{subsec:flying_cars}
Modern advances in composites, flight-control software, and miniaturized power plants make the old dream of flying automobiles possible.

All flying cars have these standard systems: computerized controls, headlights, an inertial compass (p. \pageref{itm:inertial_compass}), a personal computer (p. \pageref{itm:person_com}), and a small multi~-mode radar (p. \pageref{subsec:multi-mode_radar}). Other details depend on the model, as described below.

\subsubsection{Air Car}\label{subsubsec:air_car}
This is a streamlined automobile with a bubble canopy. It flies using thrust from four pod~-mounted ducted fans, but it also has an ordinary wheeled suspension and electric drivetrain that lets it operate like an ordinary car. It can hover in mid-air, or take off and land vertically, or fly as fast as a light airplane with a range of 900 miles. The quoted performance statistics are for ducted~-fan flight; as a ground vehicle, the air car has Handling/SR +1/3 and Move 3/45* on the ground, with an 1,800 mile range.

The vehicle has two doors and four seats. It is operated with Piloting (Vertol) and Electronics Operation (Sensors); Navigation (Air) is also useful!

It can fly with only two engines, but if a total systems failure occurs or it runs out of fuel while airborne, it deploys a landing parachute that will usually bring it down safely (assume Parachuting-11). All occupants are also provided with crashwebs (p. \pageref{subsec:vehicle_systems}).

\subsection{Grav Jeep}\label{subsec:grav_jeep}
Part of Grav~-Tech's latest line of products, the Grav Jeep, also known as a Gwhel Jeep, uses mini~-gwhel~-drives (p. TODO). It is an open~-topped and resembles a streamlined automobile with no wheels. It has a personal computer (p. \pageref{itm:person_com}) and a small intertial compass (p. \pageref{itm:inertial_compass}). A pair of F cells give it a 2,000-mile range. The pilot uses Piloting (Contragravity) skill.

\multicolinterrupt{
    \begin{table}[H]
        \hrule height 1pt\medskip
        \subsection{Ground Vehicle Table}
        \centering
        \rowcolors{1}{}{\colorvehicles}
        \begin{tabularx}{\linewidth}{lXXcccccrcrrc}
            \textbf{Vehicle} && \textbf{ST/HP} & \textbf{Hnd/SR} & \textbf{HT} & \textbf{Move} & \textbf{LWt.} & \textbf{Load} & \textbf{SM} & \textbf{Occ.} & \textbf{DR} & \textbf{Cost} & \textbf{Locations} \\
            % name && sthp & hndsr & ht & move & lwt & load & sm & occ & dr & \$ & locations \\
            Smart Car && 46 & +1/5 & 12 & 3/60* & 1.4 & 0.6 & +3 & 1+4 & 5 & \$20K & G4W \\
            Wheeled ATV && 100 & -1/4 & 12 & 2/40 & 10 & 2 & +5 & 1+9PVS & 40 & \$200K & g8W \\
            Dynamic Car && 40 & +2/5 & 12 & 6/75* & 1.1 & 0.6 & +3 & 1+4 & 10 & \$30K & G4W \\
            Exo~-Spider && 80 & +2/2 & 13 & 8/16 & 5 & 1 & +4 & 1+4PVS & 70 & \$200K & g6L \\
        \end{tabularx}
        % \caption{Caption}
        \label{tab:ground_vehicle}
        \hrule height 1pt\medskip
    \end{table}
}
\section{Tanks}\label{sec:tanks}

\subsection{Light Battle Tank}\label{subsec:light_battle_tank}
This tank has a tough composite-laminate hull that is reinforced by electromagnetic armor (p. \pageref{subsec:electromagnetic_armor}), but it relies on stealth and sensors to get the first shot. It runs quietly on rubber-band tracks, powered by a hybrid diesel-electric engine. The crew (a driver and commander-gunner) are stationed in the hull, protected by a NBC kit (p. \pageref{subsec:vehicle_systems}).

Its unmanned turret is armed with a 100mm tank cannon (p. \pageref{itm:tank_cannon_100mm}) with the electrothermal upgrade (p. \pageref{subsec:electrothermal-chemical}) and a coaxial 15mm chaingun (p. \pageref{itm:heavy_chaingun_15mmcl}). Atop the turret is a smaller turret with a strike laser (p. \pageref{itm:strike_laser}) for air defense and missile interception. The rear hull houses 10 tactical missile launchers (p. \pageref{subsubsec:tml_100mm}) in fixed upward-facing mounts.

Electronics include two holographic crew stations (p. \pageref{itm:terminal_holo_crew}), a hyperspectral imaging sensor (p. \pageref{subsec:hyperspec_imaging_sensor}), a medium laser comm (p. \pageref{itm:laser_comm_medium}), a medium radio (p. \pageref{itm:radio_comm_medium}), a tactical ESM (p. \pageref{itm:tactical_esm_detector}), and a tactical sound detector (p. \pageref{itm:tactical_sound_detector}).

It is operated by a driver who uses Driving (Tracked) and Electronics Operation (Sensors) and a commander/gunner with Artillery (Guided Missiles), Gunner (Beams, Cannon, Machine Gun), and Electronics Operation (Comm, ECM, Sensors).

\subsection{Hovertank}\label{subsec:hovertank}
This tank rides on a cushion of air, using auxiliary jump jets to cross rough terrain. The hull is sealed with full life support (p. \pageref{subsec:vehicle_systems}). The main turret has a 40mm railgun (p. \pageref{itm:railgun_40mm}) or a plasma cannon (p. \pageref{itm:plasma_cannon}) in a stabilized mount. The vehicle also has a small turret with a strike laser (p. \pageref{itm:strike_laser}) in a stabilized mount for point defense. Electronics include two holographic crew stations (p. \pageref{itm:terminal_holo_crew}), hyperspectral imaging sensors (p. \pageref{subsec:hyperspec_imaging_sensor}), a medium laser comm (p. \pageref{itm:laser_comm_medium}), a medium radio (p. \pageref{itm:radio_comm_medium}), and a tactical sound detector (p. \pageref{itm:tactical_sound_detector}). The vehicle is protected by infrared cloaking (p. \pageref{subsec:infrared_cloaking}), a multispectral chameleon surface (p. \pageref{subsubsec:multispec_chameleon_surf}) that aids any roll to avoid visual detection, radar stealth (p. \pageref{subsec:radar_stealth}), and a tactical ESM (p. \pageref{itm:tactical_esm_detector}).

Crew and skill requirements are the same as the light tank with the exception that Driving (Hovercraft) is used. LC1.

\subsection{Grav Tanks}\label{subsec:grav_tanks}
While Grav~-Tech does not publicly sell these, they are contracted with many military organizations to produce gwhel~-drive~-based military equipment, such as these contragravity tanks.

Grav tanks combine the agility of an attack helicopter with the armor and firepower of the main battle tank. The grav tank has the same capabilities as the hovertank, but is capable of transonic flight using its gwhel~-drives. Use Piloting (Contragravity) instead of Driving skill.

\multicolinterrupt{
    \begin{table}[H]
        \hrule height 1pt\medskip
        \subsection{Tanks Table}
        \centering
        \small
        \rowcolors{1}{}{\colorvehicles}
        \begin{tabularx}{\linewidth}{lccXcXcrcrrc}
            \textbf{Vehicle} & \textbf{ST/HP} & \textbf{Hnd/SR} & \textbf{HT} & \textbf{Move} & \textbf{LWt.} & \textbf{Load} & \textbf{SM} & \textbf{Occ.} & \textbf{DR} & \textbf{Cost} & \textbf{Locations} \\
            % name && sthp & hndsr & ht & move & lwt & load & sm & occ & dr & \$ & locations \\
            Light Battle Tank & 150 & -2/5 & 11 & 2/25 & 30 & 1 & +5 & 2S & 500/200 & \$2M & 2CTt \\
            Hovertank & 150 & -3/4 & 11 & 2/50 & 30 & 1 & +5 & 2S & 700/300 & \$3M & Tt \\
            Grav Tank & 150 & +1/5 & 11 & 10/500 & 30 & 1S & +5 & 2SV & 700/300 & \$3.5M & Tt \\
        \end{tabularx}
        % \caption{Caption}
        \label{tab:tanks}
        \hrule height 1pt\medskip
    \end{table}
}

\section{Hovercraft}\label{sec:hovercraft}
These ground~-effect vehicles ride on a cushion of air. Hovercraft are less maneuverable than conventional ground vehicles, but can travel on land and water.

\subsection{Armored Hovercraft}\label{subsec:armored_hovercraft}
This armored vehicle is designed to transport a squad of soldiers and provide them with fire support. It is also very effective as a coastal patrol (or smuggling!) craft. It uses a gas turbine or fuel cell power plant, and has a range of 800 miles.

Its crew compartment holds two, and is accessed through a top hatch. A powered rear ramp leads into a compartment that can carry up to eight passengers and cargo. Electronics include two holographic crew stations (p. \pageref{itm:terminal_holo_crew}), hyperspectral sensors (p. \pageref{subsec:hyperspec_imaging_sensor}), an inertial navigation system (p. \pageref{subsec:inertial_nav}), a medium laser comm (p. \pageref{itm:laser_comm_medium}), a medium radio (p. \pageref{itm:radio_comm_medium}), a personal computer (p. \pageref{itm:person_com}) with the hardened option, tactical AESA (10 mile range) (p. \pageref{subsubsec:aesa}), and a tactical ESM (p. \pageref{itm:tactical_esm_detector}). Defenses include infrared cloaking (p. \pageref{subsec:infrared_cloaking}) and radar stealth (p. \pageref{subsec:radar_stealth}).

It has a small stabilized turret atop the hull. Its primary sensors are in a telescoping mast~-mounted periscope (the X location) that extends up to 15 feet for over~-the~-horizon reconnaissance. It is protected by composite armor.

It has a small independent turret that is a stabilized mount; it can be fitted with up to 400 lbs. of weapon systems appropriate to its TL. At TL9, a typical weapon mix would be a 25mm autocannon (p. \pageref{itm:assault_cannon_25mmcl}), an MLAWS (p. \pageref{subsubsec:mlaws_64mm}), and two 40mm mortar boxes (p. \pageref{itm:mortar_box_40mmplb}). Appropriate skills are Artillery (Guided Missile), Driving (Hovercraft), Electronics Operation (Sensors), and Gunner (Machine Gun).

\subsection{Hover Jeep}\label{subsec:hover_jeep}
This open-topped light hovercraft is suitable for both civilian and military uses. It is powered by a single F cell for six hours, giving it a range of 240 miles. It has two front and four rear seats. The pilot uses Driving (Hovercraft) skill. Its equipment is fairly austere: headlights, a computerized crew station (p. \pageref{itm:terminal_com_crew}), a HUD (p. \pageref{itm:terminal_hud}), an inertial compass (p. \pageref{itm:inertial_compass}), a medium radio (p. \pageref{itm:radio_comm_medium}), and a personal computer (p. \pageref{itm:person_com}).

\multicolinterrupt{
    \begin{table}[H]
        \hrule height 1pt\medskip
        \subsection{Hovercraft Table}
        \centering
        \small
        \rowcolors{1}{}{\colorvehicles}
        \begin{tabularx}{\linewidth}{lccccccrcrrc}
            \textbf{Vehicle} & \textbf{ST/HP} & \textbf{Hnd/SR} & \textbf{HT} & \textbf{Move} & \textbf{LWt.} & \textbf{Load} & \textbf{SM} & \textbf{Occ.} & \textbf{DR} & \textbf{Cost} & \textbf{Locations} \\
            % name && sthp & hndsr & ht & move & lwt & load & sm & occ & dr & \$ & locations \\
            Armored Hovercraft & 130 & -2/4 & 11f & 5/50 & 20 & 2 & +5 & 2+8S & 150/70 & \$500K & tX \\
            Hover Jeep & 50 & -1/3 & 12 & 3/40 & 2 & 1 & +4 & 2+3 & 10 & \$40K & OX \\
        \end{tabularx}
        % \caption{Caption}
        \label{tab:hovercraft}
        \hrule height 1pt\medskip
    \end{table}
}

\section{Minisubs}
These are vehicles designed for underwater research, salvage, and special operations. All three subs described below share certain characteristics.

Each has a bridge with a pair of holographic crew stations (p. \pageref{itm:terminal_holo_crew}) plus a variable number of passenger seats, and a one-man airlock.

Other equipment includes a medium hydrophone (p. \pageref{itm:medium_hydrophone}), a medium sonar (p. \pageref{itm:medium_sonar}), microframe computer (p. \pageref{itm:microframe}), a periscope (15') with a medium radio (p. \pageref{itm:radio_comm_medium}) and thermal imaging sensors (p. \pageref{subsec:infrared_sensors}), and a medium sonar comm (p. \pageref{itm:medium_sonar}). A sound baffling system gives a -3 penalty on rolls to detect the submarines with hydrophones, but only when they are moving at speeds below 50 mph.

A stabilized turret is standard for all the minisubs described below. Civilian versions equip the turret with a searchlight (p. \pageref{itm:searchlight}) and heavy laser torch (p. \pageref{subsec:laser_plasma_torches}); military or paramilitary models often carry a blue-green strike laser (p. \pageref{subsubsec:option_blue-green}).

\subsection{Deep~-Sea Minisub}\label{subsec:deep-sea_minisub}
This is a 30~-foot~-long submarine that can dive into the deepest parts of the ocean or explore the seas of alien worlds. It is saucer~-shaped, with a spherical pressure hull surrounded by an unpressurized engineering section that houses twin hydrojet propellers. It uses a nuclear power plant which gives it unlimited range, and it can safely operate at a depth of up to 10 miles. Another feature is a pair of ST 30 robot manipulator arms that an operator can control with his own DX and skills, using either virtual reality gloves or a neural interface.

\subsection{Supercav Minisub}\label{subsec:supercav_minisub}
This is a short-range sub designed for underwater courier, attack, or patrol duties. The vessel has a streamlined wedge-shaped body, and is propelled by vortex~-combustor ramjet engines that combine aluminum dust with water (this serves as both oxidizer and reaction mass). It uses a gas generator to create a supercavitating bubble around the vehicle, reducing its drag and permitting very high underwater speeds. The supercav minisub can dive to a depth of 900 feet and has a range of 200 miles.

\subsection{Nuclear Minisub}\label{subsec:nuclear_minisub}
This is a fusion-powered multi-purpose minisub. Its features are identical to the deep-sea minisub except that it has ST 45 arms, full life support (limited only by the food that is carried aboard) and unlimited cruising range. The reactor is good for 200 years.

\section{Diver Propulsion Systems}\label{sec:diver_propulsion_systems}
These gadgets let single divers travel long distances underwater. Naval black ops teams find them particularly useful, and in underwater colonies, everyone might use them.

\subsection{Aquasled}\label{subsec:aquasled}
This one-man underwater propulsion system resembles a small sled equipped with a hydrojet propulsion system. The diver grips the control handles on the sled and is pulled forward. It has a headlight and depth gauge; it may carry up to 10 pounds of other gadgets, such as weapons or sonar. It runs for eight hours on a D cell and weighs 60 lbs. LC4.

\subsection{Underwater Jet Pack}\label{subsec:underwater_jet_pack}
This is a backpack underwater propulsion system using a vortex-combustor ramjet. It has a range of 10 miles and weighs 40 lbs. It takes three seconds to strap on or remove. An extra fuel cylinder is \$40 and 20 lbs. LC4.

\multicolinterrupt{
    \begin{table}[H]
        \hrule height 1pt\medskip
        \subsection{Minisub and Diver Propulsion System Table}
        \centering
        \small
        \rowcolors{1}{}{\colorvehicles}
        \begin{tabularx}{\linewidth}{lXcccccrcrrcX}
            \textbf{Vehicle} & \textbf{ST/HP} & \textbf{Hnd/SR} & \textbf{HT} & \textbf{Move} & \textbf{LWt.} & \textbf{Load} & \textbf{SM} & \textbf{Occ.} & \textbf{DR} & \textbf{Cost} & \textbf{Locations} & \textbf{Draft} \\
            % name && sthp & hndsr & ht & move & lwt & load & sm & occ & dr & \$ & locations \\
            Aquasled & 15 & 0/2 & 12 & 2/15 & 0.13 & 0.1 & -1 & 1 & 5 & \$2K & E & 2 \\
            Deep~Sea Minisub & 150 & 0/3 & 12 & 1/6 & 28 & 1 & +6 & 6PS & 100 & \$25M & 2Argst & 10 \\
            Supercav Minisub & 135 & +1/3 & 11 & 8/150 & 20 & 0.5 & +5 & 2PS & 30 & \$30M & gst & 6 \\
            Nuclear Minisub & 150 & 0/3 & 12 & 2/18 & 28 & 1 & +6 & 6PS & 150 & \$50M & 2Argst & 10 \\
            Underwater Jet Pack & 11 & 0/1 & 12 & 6/12 & 0.11 & 0.1 & -3 & 1 & 5 & \$600 & E & 2 \\
        \end{tabularx}
        % \caption{Caption}
        \label{tab:minisub_diver_propulsion_system}
        \hrule height 1pt\medskip
    \end{table}
}

\section{Tilt~-Rotor Transport}\label{sec:tilt-rotor_transport}
Tilt-rotor airplanes have two oversized propellers that can swivel between a vertical position (to fly like a helicopter) and horizontal position (for efficient, high~-speed airplane flight). They are especially useful for military special ops, but may also be popular commuter and cargo aircraft.

The tilt rotor has two computerized crew stations (p. \pageref{itm:terminal_com_crew}) for the pilot and co-pilot, with a cabin and cargo area to the rear. Access is provided by two side doors and a rear cargo door under the tail. The aircraft is sealed with limited life support (60 man-hours). Other onboard systems include an inertial navigation system (p. \pageref{subsubsec:inertial_nav_system}), a personal computer (p. \pageref{itm:person_com}), a medium multi~-mode radar (p. \pageref{subsec:multi-mode_radar}), and a medium radio (p. \pageref{itm:radio_comm_medium}).

A tilt-rotor pilot uses Piloting (Heavy Airplane) when in fixed-wing flight (required for speeds over 150 mph) and Piloting (Helicopter) when in a helicopter mode. Electronics Operation (Comm, Sensors) and Navigation (Air) skills are useful. A co-pilot is not required, but can share the workload.

\subsection{Tactical Tilt~-Rotor}\label{subsec:tactical_tilt-rotor}
This is an armored special ops version of the tilt rotor. It has the same capabilities as the tilt-rotor transport plus infrared cloaking (p. \pageref{subsec:infrared_cloaking}), radar stealth (p. \pageref{subsec:radar_stealth}), and a large radar (p. \pageref{itm:large_radar}). A small independent turret is under the nose. The pilot or co~-pilot will usually have Gunner (Beams or Machine Gun) skill.

\section{Utility Vertol}\label{sec:utility_vertol}
These are wingless direct-lift transport vehicles, similar to the air car (p. \pageref{subsubsec:air_car}) but larger. They perform the same roles as helicopters, but their lack of wings or rotors lets them maneuver in built~-up areas. Typical missions include aerial assault, flying ambulance, logistics support, and VIP transport.

\subsection{Utility Vertol}\label{subsec:utility_vertol}
This is a streamlined vehicle like a wingless cargo jet, with a tail assembly, four pods containing vectored-thrust ducted fan engines, and a retractable skid undercarriage. It is lightly armored, but its redundant systems enable it to fly despite systems failures or combat damage.

It has a front cockpit with two crew seats for a pilot and a co~-pilot; behind that is a small cabin with eight passenger seats and a cargo bay. There are doors on either side of the fuselage, and two under the tail. All seats are provided with crashwebs (p. \pageref{subsec:vehicle_systems}). Electronics include a pair of computerized crew stations (p. \pageref{itm:terminal_com_crew}), an inertial navigation system (p. \pageref{subsubsec:inertial_nav_system}), a medium multi~-mode radar (p. \pageref{subsec:multi-mode_radar}), a medium radio (p. \pageref{itm:radio_comm_medium}), two personal computers (p. \pageref{itm:person_com}), and a large radar (p. \pageref{itm:large_radar}). Military models add additional stealth systems -- see the \hyperref[part:defenses]{Defenses} for various options. It has a sealed hull with an NBC kit (p. \pageref{subsec:vehicle_systems}).

The pilot uses Piloting (Vertol) skill. Other useful skills are Electronics Operation (Communications and Sensors) and Navigation (Air). A co~-pilot is common, and will have the same skills.

\multicolinterrupt{
    \begin{table}[H]
        \hrule height 1pt\medskip
        \subsection{Tilt~-Rotor and Vertol Table}
        \small
        \centering
        \rowcolors{1}{}{\colorvehicles}
        \begin{tabularx}{\linewidth}{lXcccccrcrrcX}
            \textbf{Vehicle} & \textbf{ST/HP} & \textbf{Hnd/SR} & \textbf{HT} & \textbf{Move} & \textbf{LWt.} & \textbf{Load} & \textbf{SM} & \textbf{Occ.} & \textbf{DR} & \textbf{Cost} & \textbf{Locations} & \textbf{Stall} \\
            % name && sthp & hndsr & ht & move & lwt & load & sm & occ & dr & \$ & locations & stall \\
            Tacitcal Tilt~-Rotor & 130 & -1/4 & 12f & 4/200 & 20 & 3 & +6 & 2+18 & 30 & \$40M & gt3WrWi & 0 \\
            Tilt~-Rotor Transport & 130 & -1/4 & 11f & 4/200 & 20 & 4 & +6 & 2+28 & 6 & \$20M & G3WrWi & 0 \\
            Utility Vertol & 90 & +3/3 & 11fx & 4/200 & 10 & 2 & +5 & 2+8S & 30 & \$12M & g3Rr & 0 \\
        \end{tabularx}
        % \caption{Caption}
        \label{tab:tilt-rotor_vertol}
        \hrule height 1pt\medskip
    \end{table}
}

\section{Grav Bikes and Platforms}\label{sec:grav_bikes_platforms}
These are small vehicles that use superscience contragravity (CG) generators for lift and propulsion. Grav vehicles are quiet, unless they are hybrid machines that use contragravity only to cancel lift and some other propulsion system for thrust.

\subsection{Grav Bike}\label{subsec:grav_bike}
Another of Grav~-Tech's latest line of consumer gwhel~-drive vehicles, this flying bike is small and agile. It has two saddle seats and an aerodynamic windshield. Its electronics include a built-in small computer (p. \pageref{itm:small_com}), a portable terminal (p. \pageref{itm:terminal_portable}), and a windscreen HUD (p. \pageref{itm:terminal_hud}). Use Piloting (Contragravity) skill. Its three E cells give it a range of 1,000 miles.

\section{Microplanes}\label{sec:microplanes}
These are portable aircraft that can be stored in kit form and assembled with a few tools.

\subsection{Dragonfly Microlight}\label{subsec:dragonfly_microlight}
This small propeller airplane is often used as a recreational aircraft or carried by explorers, but it is also useful for covert insertions. The wings and body are constructed of transparent, high~-strength polymers over foamed metal structural membranes. The Dragonfly can be broken down for transport into two backpack modules, each weighing a mere 35 lbs. Assembly or disassembly takes a single person only nine minutes; a Mechanic+2 roll and a tool kit are required.

It carries one person in an open saddle. It lands and takes off on skids; it has a range of 100 miles, or more if it can glide with a good tail wind. Its construction provides it with radar stealth (p. \pageref{subsec:radar_stealth}).

\subsection{Backpack Dragonfly}\label{subsec:backpack_dragonfly}
This advanced version of the Dragonfly folds into a single 35-lb. backpack. No assembly is required. It takes three seconds for the aircraft to unfold or contract.

\multicolinterrupt{
    \begin{table}[H]
        \hrule height 1pt\medskip
        \subsection{Grav Bike and Microplane Table}
        \centering
        \small
        \rowcolors{1}{}{\colorvehicles}
        \begin{tabularx}{\linewidth}{lXcccccrcrrcX}
            \textbf{Vehicle} & \textbf{ST/HP} & \textbf{Hnd/SR} & \textbf{HT} & \textbf{Move} & \textbf{LWt.} & \textbf{Load} & \textbf{SM} & \textbf{Occ.} & \textbf{DR} & \textbf{Cost} & \textbf{Locations} & \textbf{Stall} \\
            % name & sthp & hndsr & ht & move & lwt & load & sm & occ & dr & \$ & locations & stall \\
            Dragonfly & 16 & +2/5 & 12 & 5/35 & 0.14 & 0.11 & +2 & 1 & 2 & \$4K & E2R2Wi & 15 \\
            Backpack Dragonfly & 16 & +2/5 & 12 & 5/35 & 0.13 & 0.11 & +2 & 1 & 2 & \$6K & E2R2Wi & 15 \\
            Grav Bike & +4/2 & 11 & 11 & 20/80 & 0.4 & 0.2 & 0 & 1+1 & 3 & \$250K & E & 0 \\
        \end{tabularx}
        % \caption{Caption}
        \label{tab:grav_bike_microplane}
        \hrule height 1pt\medskip
    \end{table}
}

\section{Flight Packs}\label{sec:flight_packs}
These are strap-on aerial propulsion systems. Most flight packs are controlled by a panel built into an arm curving in front of the user; computer autopiloting is standard, so only one hand is required to operate them. Instrument readouts are usually projected into a helmet HUD, but the pack can connect to a neural interface for hands-free operation.

It takes four seconds to strap on a flight pack, two seconds to remove it. All of these “vehicles” require Piloting (Flight Pack) skill to operate.

\subsection{Helipack}\label{subsec:helipack}
This is a pair of three~-foot wide ducted fans attached to a backpack harness and control unit. It's useful for emergency rescue work, and thanks to the relatively quiet power plant, has some military applications. It won't operate in a trace or vacuum atmosphere. It requires two yards of clearance to either side of the wearer -- he can't fly through narrow passages or doors. The helipack weighs 200 lbs. and uses an E cell for power. It has a range of 200 miles. LC3.

\section{Zero~-G Thruster}\label{sec:zero-g_thrusters}
These are microgravity flight rigs used for short-range travel outside of spacecraft or space stations. They use cold~-gas thrusters to provide maneuverability, and can be easily donned, doffed, and serviced by a single individual. Use Free Fall skill to operate them.

\subsection{Hand Thrusters}\label{subsec:hand_thrusters}
A hand thruster propels the user with bursts of compressed gas. Each burst accelerates or decelerates a normal~-mass human by one yard per second in the direction opposite to that in which the thruster is pointed. A successful roll against Free Fall or Vacc Suit skill is necessary to point the thruster in the desired direction. The unit's cylinder is good for 30 one-second bursts. A hand thruster weighs four pounds, including the cylinder; extra cylinders cost \$10, weigh one pound and take three seconds to replace.

\subsection{Thruster Pack}\label{subsec:thruster_pack}
A strap-on unit for short jaunts in free fall. It consists of a thruster pack, a pair of arms with reverse thrusters, and a control arm that curves in front of the user. Maneuver jets are located at strategic points along the entire pack; a built-in autopilot assists the wearer. It takes 10 seconds and a Vacc Suit roll (which can be tried again every five seconds if missed) to strap into the thruster pack. The large cylinder allows 100 seconds of full acceleration. Successful Free Fall rolls allow the user to control his speed and direction. It weighs 40 pounds, including one cylinder. Extra cylinders cost \$30, weigh 10 lbs. and take five seconds to replace.

\multicolinterrupt{
    \begin{table}[H]
        \hrule height 1pt\medskip
        \subsection{Flight Pack and Thruster Pack Table}
        \centering
        \small
        \rowcolors{1}{}{\colorvehicles}
        \begin{tabularx}{\linewidth}{lXcccccrcrrcX}
            \textbf{Vehicle} & \textbf{ST/HP} & \textbf{Hnd/SR} & \textbf{HT} & \textbf{Move (G)} & \textbf{LWt.} & \textbf{Load} & \textbf{SM} & \textbf{Occ.} & \textbf{DR} & \textbf{Cost} & \textbf{Locations} & \textbf{Stall} \\
            % name & sthp & hndsr & ht & move & lwt & load & sm & occ & dr & \$ & locations & stall \\
            Hand Thruster & 6 & +1/1 & 12 & 1/30 (0.1G) & 0.1 & 0.1 & -5 & 1 & 5 & \$50 & -- & -- \\
            Helipack & 17 & +2/2 & 12 & 36/30 & 0.25 & 0.15 & -2 & 1 & 10 & \$20K & E & 0 \\
            Thruster Pack & 14 & +3/1 & 12 & 3/300 (0.3G) & 0.12 & 0.1 & -2 & 1 & 10 & \$2K & E & -- \\
        \end{tabularx}
        % \caption{Caption}
        \label{tab:flight_thruster_pack}
        \hrule height 1pt\medskip
    \end{table}
}

\section{Drop Capsules}\label{sec:drop_capsules}
These are designed to let their occupants enter atmosphere safely. They have small rocket engine clusters that provide limited maneuverability, but careful landing is a manner of good navigation. De-orbiting takes two or three rotations around a planet with an Earthlike atmosphere (more for a planet with a thinner atmosphere, such as Mars). During this time, radio, radar, and all passive sensors will be blinded due to plasma effects.

Life pods and drop capsules incorporate a computer with Navigation (Space)-12, or the user can override this and program his own landing. Critical success means the user lands within a mile of where he intended. Success means he's within 5d ¥ 100 miles, less 200 miles times his margin of success (e.g., success by 5 reduces the radius by 1,000 miles), minimum one mile. Failure means he could be anywhere on the planet. Critical failure means a disaster of some sort: landing in rough terrain, getting stuck in orbit without fuel to deorbit, or optionally, a too-steep reentry that results in the capsule burning up (a fate best left to NPCs).

\subsection{Life Pod}\label{subsec:life_pod}
This is a four-person escape capsule designed to let people evacuate a spacecraft or space station in the event of disaster.

If launched from a vessel in deep space, a life pod is designed to maneuver a safe distance away from a damaged vessel and broadcast a distress signal.

If launched from a vessel in planetary orbit, the pod provides its occupants the option to land; if they are not responsive, it will do so if its library data indicates it is safe to do so (e.g., it won't try to land on a gas giant!). Reentry is handled by an autopilot. After a series of braking parachutes have reduced speed, the capsule uses a parachute to descend to a soft landing. If it lands in water, air bags are automatically inflated, and the capsule will float.

A life pod is equipped with padded acceleration seats for four people and a pair of lockers holding 200 lbs. of cargo. These are usually stuffed with medical and survival kits, but in an emergency the lockers can be emptied, allowing an extra person to cram into each locker.

The capsule is equipped with a medium radio beacon (p. \pageref{itm:radio_comm_medium}), an inertial navigation system (p. \pageref{subsubsec:inertial_nav_system}), and 90 man-days of limited life support (p. \pageref{subsec:vehicle_systems}). Its internal energy bank will power its beacon and life support system for up to a month.

The pod's surface is equipped with a programmable camouflage (p. \pageref{subsec:program_camouflage}) intended to give it a radar-reflective surface or adjust the exterior for high visibility. The pod is controlled by a personal computer (p. \pageref{itm:person_com}) and a simple crew console. Its maneuvering rockets require Piloting (High-Performance Spacecraft), but most pods are equipped with an AI so they can be used by unskilled escapees. LC4.

\subsection{Drop Capsules}\label{subsec:drop_capsules}
A drop capsule is a re-entry capsule protected by an ablative shield, allowing an occupant or cargo canister to be safely dropped from a spaceship in low orbit. It takes two minutes to load a drop capsule. The capsule must be launched from a vehicle bay or missile launcher on a trajectory that will de-orbit it.

Re-entry is handled by an autopilot. After a series of braking parachutes have reduced descent speed, the capsule breaks up a mile or so above the surface. A conventional parachute, parawing, or grav belt can then be used. The drop capsule is not reusable. A drop capsule's split DR is DR 100 ablative armor (all of which is usually gone after the re-entry) on its underside, plus DR 20 from its composite body. LC3.

\subsection{Stealth Capsules}\label{subsec:stealth_capsules}
These are similar to standard drop capsules, but are made of material with a low sensor signature and packed with ECM equipment and decoys. They have radar stealth (p. 100) and sensor jammers (p. 99). A stealth capsules automatically launches radar and infrared decoys and activates its own jamming systems, giving itself an extra -5 to be struck by homing missiles. It may also deploy a spare parachute to “jink” itself off a sensor screen. This generally triggers a second roll (at -5) by the sensor operator to avoid losing contact. A stealth capsule is somewhat more cramped than a drop capsule. LC2.

\multicolinterrupt{
    \begin{table}[H]
        \hrule height 1pt\medskip
        \subsection{Drop Capsules Table}
        \centering
        \small
        \rowcolors{1}{}{\colorvehicles}
        \begin{tabularx}{\linewidth}{lXXcccccrcrrc}
            \textbf{Vehicle} && \textbf{ST/HP} & \textbf{Hnd/SR} & \textbf{HT} & \textbf{Move (G)} & \textbf{LWt.} & \textbf{Load} & \textbf{SM} & \textbf{Occ.} & \textbf{DR} & \textbf{Cost} & \textbf{Locations} \\
            % name && sthp & hndsr & ht & move & lwt & load & sm & occ & dr & \$ & locations \\
            Life Pod && 50 & -5/1 & 13 & 1/1,000 (0.1G) & 1 & 0.5 & +2 & 4SV & 100/20 & \$100K & -- \\
            Drop Capsule && 50 & -- & 13 & -- & 1 & 0.5 & +2 & 2SV & 100/20 & \$10K & -- \\
            Stealth Capsule && 50 & -- & 13 & -- & 1 & 0.3 & +2 & 1SV & 100/200 & \$50K & -- \\
        \end{tabularx}
        % \caption{Caption}
        \label{tab:drop_capsules}
        \hrule height 1pt\medskip
    \end{table}
}

\section{Other Transport}\label{sec:matter_transmission}

\subsection{Matter Transmission Booths}\label{subsec:matter_transmission_booths}
Also known as Psi Transmission Booths. These booths use psi~-amplifying technology and a compatible espers with teleportation abilities to move matter or people from a transmitter booth to a receiver booth. These booths require two espers, one on each end, to function and they must coordinate, either by using network~-connected neural interfaces or by way of a third, telepathic esper forming a bridge. Technically, each connection can be temporary, only for the length of the transmission, after which the booth could connect to a different booth to do a separate transmission.

These are expensive. The booths themselves are \$100,000 each, weight 1,000 lb., and run on external power. They require an esper~-mind capable of teleporting others, which paying an esper is generally around \$2,500 a month, plus \$100 per transmission. Some planets have established networks which such booths can b connected to, these usually cost around \$1,000 per month to be connected.

\subsection{Witch Gate}\label{subsec:witch_gate}
During the reign of the witch covens, many of these structures were created universe wide, arcane portals capable of connecting vastly distant points by traveling acrosss ley lines. While witchcraft is now outlawed near universally, these structures remain. The most obvious of them have been destroyed by witchhunters as a possible root through which the covens could rise and sweep across the galaxy. Still, though, the subtle gates remain. 

Constructing a new gate, permanent or temporary, is an act of advanced witchcraft, and even reopening closed gates is a difficult, albeit easier, ritual as well. For gates that are still open, though, even those without arcane knowledge may step though, granted they can uncover how they work. Some are simple, walk through the portal, doorway, archway, etc. and appear on the other side. Others, though, are shrouded in secrecy, requiring specific steps to ``properly'' step through. Many such gates are discovered by layfolk by accident and it can be quite unfortunate to find yourself transported miles or even light years away.

\subsection{Hydra Gates}\label{subsec:hydra_gates}
The Hydra Gate Network is an expansive network of large gates which lead to the inside of the leviathan Hydra, a space known colloquially as ``meatspace.'' The gates themselves are these large, mechanical structures which are made to orbit planets, moons, and occasionally stars (and one black hole!). The network connects nearly 600 star systems and three galaxies. Currently, there are nearly fifty in~-construction gates, twelve of which are being built in the galaxy ND~-685 in the frontier. 

Inside meatspace are tunnels and caves with walls of the Hydra's Flesh. Navigating is difficult, requiring Hydra Network Navigation software which is Complexity 9 software, \$15,000, and an inertial navigation system (p. \pageref{subsubsec:inertial_nav_system}). This software is considered the basic tools to be able to use Navigation (Meatspace) skill to travel through the network. Furthermore, given the shifting nature of the network (the layout changes at 14:00 Universal Time, every second Tuesday of each month) the software needs an updated database of the tunnels, generally for sale at entrance gates for between \$300 and \$1,000. Specific ``a~to~b'' routes are generally available for sale, running from the \$400 to \$800 range and offering a +2 (quality) bonus to Navigation (Meatspace) but only for navigating that specific route (and only for the tunnel layout when that route was mapped).

Travel times vary with the specific layout of the tunnels that month. Generally, going between any two gates will take no more than a week's travel, though a particularly bad route can see this extended up to a month. Traveling without mapping software is the equivalent of no equipment for Navigation (Meatspace) so any rolls are at -10; -5 if only lacking an updated layout database. When navigating through meatspace, make a Navigation (Meatspace) roll. Success means that there is no notable mishaps; critical success means a shortcut was found, cutting travel time by 30\% (which the data for which is potentially worth selling at a gate or checkpoint). Failure means you get lost and are delayed (perhaps encountering hostile Hydrites you must fend off); critical failure means that something has gone very wrong perhaps you are completely lost or that your software is glitching (or a scam!) and led you astray.

Gaining entry to meatspace requires passing through one of the gates, paying the toll and associated fees. The toll is standard \$500, though occasionally there are additional surcharges such as high traffic periods. This toll is waved if that month's gate is still unmapped, as Hydra Gate Company can in no way guarantee safety (they will even pay for mapping data if you make it out alive). Throughout the network are checkpoint stations, which serve as hubs for refueling, resupplying, and updating navigation software.

In addition to the official Hydra Gates, there are rumored to be black market gates operated by some syndicates and militaries in order to use the network for themselves. The Hydra Gate Company's internal enforcement agency works to stamp down on any such illicit use of its network.