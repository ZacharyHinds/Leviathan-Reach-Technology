\section{Passive Visual Sensors}\label{sec:passive_visual_sensors}
These systems work like normal vision, but extend the limits of human sight. They include light-intensifying, infrared, ultraviolet, and hyperspectral sensors.

Passive sensors often incorporate levels of telescopic magnicication. Each doubling in magnification lets the user ignore -1 in range penalties ($\log_2(x)$) on Vision rolls when using the sensor. The user can also "zoom in" on a particular target by taking an Aim maneuver. This doubles the benefit against that target but eliminates the bonus to spot other targets.

All passive sensors incorporate a digital camera.

All these sensors provide anti-glare protection and DR 2 for the eyes.



\begin{table}[H]
    \caption*{\textbf{Magnification Range Penalty Negation}}
    \centering
    \rowcolors{1}{}{\colorcomms}
    \begin{tabularx}{\columnwidth}{XccX}
        & 1× & 0 & \\
        & 2× & -1 & \\
        & 4× & -2 & \\
        & 8× & -3 & \\
        & 16× & -4 & \\
        & 32× & -5 & \\
        & 64× & -6 & \\
        & 128× & -7 & \\
    \end{tabularx}
    \label{tab:magnif_penalty}
\end{table}

\subsection{Passive Visual Sensor Configurations}
Standard models:

\textit{Binoculars:}\label{itm:conf_binoculars} A manual hand-held viewer. It limits the user's vision to a 120\degree{} forward arc (see No Peripheral Vision, B151) and requires one free hand an Aim maneuvers to use. Binoculars incorporate a built-in HUD, a laser rangefinder, and a digital camera. They can be used as basic equipment for Photography skill.

\textit{Goggles or Visor:}\label{itm:conf_goggles_visor} These are wearable hands-free optics with a wide field of view, but lower magnification than equivalent binosculars. They also incorporate a built-in HUD and digital camera, but the simple controls for the latter give a -5 (quality) modifier to Photography skill.

\textit{Imaging Sensor Array} or \textit{Surveillance Camera:}\label{itm:conf_sensor_array} A security system or vehicle-mounted sensor. It does not come with come a display; it requires a separate terminal as its interface. It limits the user's vision to 120\degree{} forward arc (No Peripheral Vision, B151), but is often mounted on a rotating turret or tripod. It can be used as a digital camera with +1 (quality) bonus to Photography skill.

\textit{Video Glasses:}\label{itm:conf_video_glass} These resemble ordinary sunglasses. They have the same capabilities as goggles, but less magnification. It takes a Ready maneuver to don or remove them.

\textit{Video Contacts:}\label{itm:conf_video_contacts} These rigid gas-permeable contact lenses contain intricate microcircuitry and auto-focusing system. They have the capabilities of goggles, but much less magnification. It takes a day to adjust to wearing contacts; until then, vision rolls are -1. It takes six seconds to insert or remove both lenses. They must be taken out every week and cleaned or else risk eye infection (HT rolls to resist infection are at a penalty equal to the number of weeks without removal/cleaning). They're powered by body heat or piezoelectricty.

\subsection{Night Vision Optics}
\label{subsec:night_vision}
These devices use near-infrared and computer-enhanced light intensification to amplify ambient light levels. They are rated for their level of Night Vision. Each level (to a maximum of nine) lets the user ignore -1 in combat or vision penalties due to darkness. However, they have no effect on the -10 penalty for total darkness.

They come in the classes detailed under Passive Visual Sensor Configurations (above), with various levels of telescopic magnification and night vision.

\textit{Electro-Optical Binoculars ("Televiewers"):}\label{itm:nv_binoculars} Night Vision 9 and 128× magnification. \$500, 0.6 lb., 2B/100 hr. LC4.

\textit{Electro-Optical Surveillance Camera:}\label{itm:nv_surveillance_cam} Night Vision 9 and 8× magnification. \$250, 0.6 lb., 2B/100 hr. Often uses external power. LC4.

\textit{Night Vision Contacts:}\label{itm:nv_contacts} Night Vision 7 and 2× magnification. \$200, neg. LC4.

\textit{Night Vision Glasses ("Night Shades"):}\label{itm:nv_glasses} Night Vision 8 and 4× magnification. \$250, 0.1 lb., A/10 hr. LC4.

\textit{Night Vision Goggles} or \textit{Visor:}\label{itm:nv_goggles} Night Vision 9 and 8× magnification. \$1,000, 0.3 lb., 2B/100 hr. LC4.

\subsection{Infrared Imaging Sensors}
\label{subsec:infrared_sensors}
This is technically known as thermal imaging, and is equivalent to the Infravision advantage. These sensors detect the infrared (heat) spectra emitted by objects at different temperatures, then build up a false-color television image of the environment.

Infrared sensors lets the user observe or fight at no penalty even in absolute darkness, if the target emits heat (this includes all living beings and most machines). The sensors give a +2 on all vision rolls to spot such targets, since their heat stands out against the background. It can also distinguish targets that are colder than their surroundings (there is no bonus). Infrared sensors can be used to follow a heat trail when tracking: add +3 to Tracking rolls if the trail is no more than an hour old.

Infrared sensors do not distinguish real colors (which may limit the ability to use some controls), and only allow the user to judge the general size and shape of heat-emitting objects. Roll at -4 to distinguish objects of similar size and shape. Attempting to read by reflected heat requires a Vision-4 roll. Flare, fiery explosions, infrared lasers and other sudden flashes of heat can blind the imaging system, just as a flash of light can blind ordinary vision.

Infrared sensors usually come with one or more levels of telescopic magnification. The user can switch freely between normal vision and infravision.

The infrared sensors described below also have a daylight TV optical channel as well. This gives telescopic magnification at the same level without providing infravision. It takes a Ready maneuver to switch settings.

They come in styles and features described under Passive Visual Sensor Configurations, with various levels of telescopic magnification.

\textit{Infrared Imaging Sensor Array:}\label{itm:infrared_array} 128× magnification. \$40,000, 50 lb., 2D/12 hr. LC3.

\textit{Infrared Binoculars:}\label{itm:infrafred_binoculars} 32× magnification. \$2,500, 3 lb., C/10 hr. LC4.

\textit{Infrared Surveillance Camera:}\label{itm:infrared_surveillance_cam} 8× magnification. \$250, 0.6 lb., 2B/100 hr. Often uses external power. LC4.

\textit{Infrared Goggles} or \textit{Visor:}\label{itm:infrared_goggles_visor} 4× magnification. They're an integral feature of many suit helmets, but if purchased separately are \$500, 0.6 lb., B/10 hr. LC4.

\textit{Infrared Video Glasses:}\label{itm:infrared_glasses} 2× magnification. \$500, 0.1 lb., A/10 hr. LC4.

\textit{Infrared Contacts:}\label{itm:infrared_contacts} 1× magnification. \$300, neg. LC4.

\subsection{Hyperspectral Imaging Sensors}\label{subsec:hyperspec_imaging_sensor}
These optical sensors electronically fuse passive radar, infrared, visual, and ultraviolet imagery into a single false-color television image. The integrated picture often reveals details that are invisible to those who see in only one of these frequencies.

If there is any light at all, hyperspectral imaging grants near-perfect night vision with no vision or combat penalties. In total darkness, it functions exactly like infrared sensors (above). It also gives +3 to all Vision rolls, all Tracking rolls, and all rolls to spot hidden clues or objects with Forensics, Observation, or Search skill. These capabilities are not cumulative with other passive visual sensors or similar advantages.

Hyperspectral imaging sensors all incorporate the above capabilities plus one or more levels of telescopic optics. If the hyperspectral imaging is turned off, the sensors function as daylight television systems.

\textit{Hyperspectral Imaging Sensor Array:}\label{itm:hyperspec_array} 64× magnification. \$160,000, 50 lb., 2D/12 hr. LC3.

\textit{Hyperspectral Binoculars:}\label{itm:hyperspec_binoculars} 32× magnification. \$10,000, 3 lb., C/10 hr. LC4.

\textit{Hyperspectral Surveillance Camera:}\label{itm:hyperspec_surveillance_cam} 8× magnification. \$2,000, 1 lb., C/100 hr. Often uses external power. LC4.

\textit{Hyperspectral Goggles} or \textit{Visor:}\label{itm:hyperspec_goggles} 2× magnification. An integral feature of many suit helmets, or available for \$2,000, 0.6 lb., B/10 hr. LC4.

\subsection{Passive Electromagnetic Sensor Arrays (PESA)}
These are similar to hyperspectral imaging sensors, but they see even farther into the electromagnetic spectrum. They provide Hyperspectral Vision (Extended Low Band), allowing the user to "see" microwave emissions.

\textit{PESA Sensor Array:}\label{itm:pesa_array} 32× magnification. \$160,000, 50 lb., 2D/12 hr. LC3.

\textit{PESA Binoculars:}\label{itm:pesa_binoculars} 16× magnification. \$10,000, 3 lb., C/10 hr. LC4.

\textit{PESA Surveillance Camera:}\label{itm:pesa_surveillance_cam} 4× magnification. \$2,000, 1 lb., C/100 hr. Often uses external power. LC4.

\textit{PESA Goggles} or \textit{Visor:}\label{itm:pesa_goggles} 1× magnification. An integral feature of many suit helmets, or available for \$2,000, 0.6 lb., B/10 hr. LC4.

\section{Indirect Passive Sensors}
\label{sec:indirect_passive_sensors}
These sensors are omnidirectional, and do not require a line of sight.

\subsection{Chemsniffer}\label{subsec:chemsniffer}
An artificial nose that registers the presence of almost any odor by comparing it to a database. The user must set the chemsniffer for a particular odor or scent. When so programmed, it allows the use of Electronics Operation (Sensors) skill for tasks that would require Smell rolls. It can recognize people, places, and things by scent (provided they have been scanned before, or are common items). It can't detect anything in a sealed environment, underwater, or in vacuum.

The sensor has a computerized database of olfactory "signatures" that can quickly be compared to new sensory impressions. The sensor can record a new signature by analyzing a scent. Its bonus is not cumulative with the Discriminatory Smell or Acute Taste and Smell advantages. A chemsniffer gives +5 on any Electronics Operation (Sensors) roll to detect targets, +5 to Tracking skill, and +9 to analyze or recognize targets by scent.

\textit{Personal Chemsniffer:}\label{itm:personal_chemsniff} This takes 10 seconds to analyze a new smell. Incorporates a built-in tiny computer. \$2,000, 2 lb., A/1 wk. LC4.

\textit{Dedicated Chemsniffer:}\label{itm:dedicated_chemsniff} Optimized to detect a single particular category of scents, e.g., explosives, human beings, drugs, etc. \$100, 0.2 lb., A/1 wk. LC4.

\textit{Tactical Chemsniffer:}\label{itm:tactical_chemsniff} Takes only three seconds for the system to scan a new scent. Can track 10 different scents at the same time. \$100,000, 40 lb., B/1 wk. LC3.

\subsection{Electronic Support Measure (ESM)}
This system detects and classifies electromagnetic emissions. On a successful Electronics Operation (EW) roll, this sensor detects radar or radio signals and reveals the distance to each source. Signals are usually detected at twice their range; low-probability intercept signals are detected at 1.5 times their range.

The system will also function as a lensor sensor, detecting ladar, targeting laser, and laser comm signals that are beamed directly at it.

The brief warning the ESM system provides give a +1 bonus to Dodge any attack aimed with an active targeting sensor that the ESM can detect.

The operator may take more time and make an Electronics Operation (EW) roll to analyze the signal. Each attempt requires a Concentrate maneuver; success distinguishes a random emission from a target lock, and can determine known types of emitters. An ESM can also be set to detect and analyze signals autonomously, using its own Electronics Operation (EW) skill for this purpose.

\textit{ESM Detector:}\label{itm:esm_detector} A hand-held or belt-mounted system, often used as a counter-surveillance device. It has Electronics Operation (EW)-10. \$250, 0.25 lb., A/1 wk. LC3.

\textit{Tactical ESM Detector:}\label{itm:tactical_esm_detector} A heavier and more expensive model. Adds a +1 (quality) bonus or uses Electronics Operation (EW)-12. \$1,000. 2 lb., B/1 wk. LC3.

These systems are also commonly built into suits, vehicles, etc.

\subsection{Sound Detector}
\label{subsec:sound_detector}
This is a sensitive array of microphones and sound-profiling software that provides the superhuman ability to distinguish between sounds.

The user can always identify people by voice, and can recognize individual machines by their “sound signature.” In tactical situations, sound detectors are often programmed to respond to particular sounds made by specific weapons, engine noises, breaking armor, etc.

The system can memorize a sound by monitoring it for at least one minute, then adding it to the signature library. It gives +4 on any Hearing roll, +4 to Shadowing skill when following a noisy target, and +8 to Electronics Operation (Sensors) rolls made to analyze and identify a particular sound. Sound detectors can also magnify sounds from a distant point for eavesdropping purposes; this requires an Aim maneuver.

Sound detectors only work in air (hydrophones are used under water). They are useless in vacuum. They can detect an air sonar at double its range.

\textit{Personal Sound Detector:}\label{itm:personal_sound_detector} This device can zoom in and amplify a particular sound by 16×. Must be connected to a Complexity 4+ computer. \$1,000, 1 lb., A/1 wk. LC3.

\textit{Tactical Sound Detector:}\label{itm:tactical_sound_detector} A sensitive “phased array” of microphones, often built into a vehicle hull. It can amplify a particular sound by 64×. It must be connected to a Complexity 4+ computer. \$30,000, 30 lb., B/1 wk. LC3.

\subsection{Hydrophone}
\label{subsec:hydrophone}
These are underwater microphones connected to discriminatory sound signature-profiling software. This can detect and track moving or noisy objects in the water, provided the hydrophone is submerged. To do so, make an Electronics Operation (Sonar) roll at the detection bonus shown below. Consult the Size and Speed/Range Table (B550); apply separate bonuses for the target's size and speed, and a penalty for the range to the target. Swift currents will generate "noise" that interferes with the sense. Find the speed of the current on the table and assess the relevant speed penalty.

A successful roll reveals the size, location, speed, and direction of movement of the target. It reveals the target's general class based on sounds (e.g., "whale" or "nuclear sub"), location, and vector, giving +8 to identify it, +4 to shadow it, and +3 to hit it with an aimed attack. It does not provide any information about the object's shape or color. Once the object is detected, it can be attacked. The modifiers that applied to the skill roll also apply to the attack roll, but can never give a bonus to hit over the +3.

Hydrophones automatically detect anyone using sonar or sonar communicators at twice that system's range (or 1.5 times range if it is low-probability intercept sonar).

\textit{Small Hydrophone:}\label{itm:small_hydrophone} +10 to the detection roll. \$5,000, 5 lb., B/1 wk. LC3.

\textit{Medium Hydrophone:}\label{itm:medium_hydrophone} +12 to the detection roll. \$25,000, 25 lb., C/1 wk. LC3.

\textit{Large Hydrophone:}\label{itm:large_hydrophone} +14 to the detection roll. \$100,000, 100 lb., D/1 wk. LC3.

\textit{Search Hydrophones:}\label{itm:search_hydrophone} This system is used for underwater research, fishing, or perimeter surveillance. It does not provide a targeting bonus, but costs 1/10 as much. LC4

\subsection{Gravscanner}
\label{subsec:gravscanner}
These devices detect the strong gravity waves produced by operating gravitic devices; see Gravity Control. They can also detect massive objects (a million tons or more) such as giant spacecraft, asteroids, planets, stars, or black holes. They provide an estimate of the bearing and strength of the gravity emanation. They can receive messages from gwhel-ripple comms but they cannot send them.

Electronics Operation (Sensors) skill is used to operate them.

\textit{Very Large Gravscanner:} +12 to detection. \$500,000, 1,000 lb., external power. LC4.

\subsection{Radscanner}
\label{subsec:radscanner}
This detects electrical or magnetic fields and radiation sources of all kinds (including radar and radio signals, not just radioactivity). The user must set the sensor to detect a particular type of radiation, such as radio waves or gamma radiation.

The detector can provide range, strength, and bearing. It does not emit a scanning signal. Detection requires a roll against Electronics Operation (Sensors) skill.

Range depends on the strength of the source -- for sensor or communicator signal detection, range is usually twice the radiating system's range. If detecting other sources of radiation, add modifiers from the Size and Speed/Range Table. If attempting to detect operating power cells, the skill roll is at -12 for an AA cell, -9 for an A cell, -6 for a B cell, -3 for a C cell, 0 for a D cell, +3 for an E cell, +6 for an F cell.

Radscanners also analyze radiation. Make a Physics or Electronics Operation (EW) roll.

\textit{Large Radscanner:}\label{itm:large_radscanner} +18 bonus to detection (or ×1,000 range when detecting signals). \$100,000, 150 lb., external power. LC4.

\textit{Medium Radscanner:}\label{itm:medium_radscanner} +12 bonus to detect radiation sources (or ×100 range for signals). \$10,000, 5 lb., B/24 hr. LC4.

\textit{Small Radscanner:}\label{itm:small_radscanner} +6 bonus to detection (or ×10 range for signals). \$1,000, 0.5 lb., AA/24 hr. LC4.

\subsection{Voidscanner}
\label{subsec:voidscanner}
Voidscanners detect the presence of "Void Energy" (VE) which is output by Ravens, Leviathans, and Witchcraft.

They provide the bearing, range, and strength of the VE. Using the scanner requires a roll against Electronics Operation (Sensors) skill. The scanner does not, by default, differentiate between the sources, but analysis of the readings can estimate the source, this requires a Biology (Void) or Thaumatology skill roll.

Range depends on the strength of the VE source. For Ravens and Leviathans this adds modifiers from the Size and Speed/Range Table plus modifiers based on the abilities of the creature. For Witchcraft, Witches and low-power charms also just use the Size and Speed/Range Table, but when casting rituals or when detecting high-power charms, the strength of the magic also adds a bonus (Witches can also mask their rituals making them harder to detect).

\textit{Large Voidscanner:}\label{itm:large_voidscanner} +9 bonus to detection. \$100,000, 150 lb., external power. LC3.

\textit{Medium Voidscanner:}\label{itm:medium_voidscanner} +6 bonus to detection. \$25,000, 5 lb., B/24 hr. LC3.

\textit{Small Voidscanner:}\label{itm:small_voidscanner} +3 bonus to detectionm. \$5,000, 0.5 lb., AA/24 hr. LC4

\subsection{Psi Scanner}
\label{subsec:psiscanner}
Psi-scanners can detect the presence of psionic energy, output by espers.

The scanners provide the bearing, range, and strength of the psionic energy. Using the scanner requires a roll against Electronics Operation (Sensors) skill. The scanner cannot differentiate the type of psionic powers being used.

Range depends on the strength or the esper or the ability they are using. For espers themselves, use firstly the Size and Speed/Range table penaltes/bonuses and then add a bonus based on the strength of the esper. For espers actively using their abilities, there is also a hidden bonus (i.e. the GM will not declare) based on the strength of the specific ability being used. Espers can use the Hidden Signature technique for their abilities to mask their powers, which applies a penalty to the detection roll.

\textit{Large Psiscanner:}\label{itm:large_psiscanner} +9 to the detection roll. \$100,000, 150 lb., external power. LC3.

\textit{Medium Psiscanner:}\label{itm:medium_psiscanner} +6 to the detection roll. \$10,000, 5 lb., B/24 hr. LC3.

% \textit{Small Psiscanner:}\label{itm:small_psiscanner} +3 to the detection roll. \$1,000, 0.5 lb., B/1 wk. LC3.


\section{Active Sensors}
%\addcontentsline{toc}{section}{Active Sensors}
\label{sec:active_sensors}
Active sensors detect objects by bouncing energy off them and analyzing the returned signal.

Active sensors are rated for the type of sensor and a range in miles or yards. An Electronics Operation skill roll is required to use an active sensor to detect hidden targets or fine detail. Active sensors can sense objects out to their rated maximum range at no range penalty; each doubling of range beyond that gives -2 to skill.

The scanning wave of an active sensor can be detected by specialized detectors. Normally, this is at twice its range. It is \textit{possible} to detect them at longer range given that most scanners radiate energy, but as these sensors operate on multiple frequencies, detection is difficult. The detector required depends on the sensor.

Unless otherwise noted, assume an active sensor scans a 120\degree arc in front of it (see \textit{No Peripheral Vision}, p. B151).

\subsubsection{Special Modes}
\label{subsubsec:special_modes}
\textit{Targeting:} Active sensors are available in tactical versions that incorporate a rangefinder mode. This works the same way for all active sensors: it generates a narrow targeting beam. It requires an Aim maneuver to ``lock onto'' a particular target that has already been detected. This determines its precise range and speed, and gives +3 to hit with an aimed attack used in conjunction with targeting software (p. TODO).

\textit{Low~-Probability Intercept (LPI):} The sensor uses a rapid frequency-agile burst of radar energy. This halves range, but results in a radar signal that can only be detected at 1.5 times the halved range rather than twice the normal range.

\textit{Disruption or Blinding:} Some sensors have the ability to emit high-power narrow beams that can be used as weapons -- see the individual sensor descriptions.

\subsubsection{Vehicular Arrays}
\label{subsubsec:vehicular_arrays}
Aircraft, submarines, or spacecraft often have very large active sensor arrays that cover a sizable fraction of their surface area on one or more facings. Active arrays operate indefinitely off vehicle power; the cost and weight are included as part of the vehicle, as the capabilities depend on the vehicle’s surface area.

\subsection{Ladar}
\label{subsec:ladar}
%\addcontentsline{toc}{subsection}{Ladar}
This high~-resolution sensor emits laser energy, then analyzes the returned signal. A ladar can discern a target's size and shape, and pick out othe rphysical details, such as the shape of a face. It can't determine flat details such as writing. Anyone who can sense the signal you emit can detect the ladar, out to twice its own range.

Ladars are of limited use in detecting unknown targets due to the narrowness of the beam -- make an Electronics Operation (Sensors) roll at -4 to spot a previously unknown target. However, they are excellent for identifying targets that have already been spotted by other sensors, roll at +4 (even to detect fine details).

Ladar can be used to ``lock onto'' a target that has already been detected. This determines its precise range and speed, and gives +3 to hit that target with an aimed ranged attack. This bonus is not cumulative with that from other active sensors that have locked onto the target.

Ordinary radar detectors do not detect ladar; specialized laser sensors are required. Ladar cannot penetrate solid objects. It has 10~-50\% range in falling rain or snow, and can be tuned to use blue~-green frequencies. Underwater, ladar functions at 1\% of its normal range with a maximum range of 200 yards.

\textit{Large Ladar:}\label{itm:large_ladar} A powerful ladar, usually vehicle~-mounted. It has a 200~-mile range. \$200,000, 100 lb., D/8 hr. LC4.

\textit{Medium Ladar:}\label{itm:medium_ladar} A portable ladar set. Can be worn as a pack, or mounted on a tripod, vehicle, or robot. It has a 60~-mile range. \$20,000, 10 lb., C/8 hr. LC4.

\textit{Small Ladar:}\label{itm:small_ladar} A mini ladar with a 20~-mile range. It comes in a hand-held version, or attaches to a shoulder mount, and plugs into a HUD (p. \pageref{itm:terminal_hud}). \$2,000, 1 lb., B/8 hr. LC4.

\textit{Small, Medium,} or \textit{Large Tactical Ladar:}\label{itm:tactical_ladar} A military~-style target~-acquisition ladar. It can track up to 10 targets at once out to the listed range, and gives +3 to hit any of them with an aimed attack. +4 CF. LC2.

\subsubsection{Ladar Smartskin}
\label{subsubsec:ladar_smartskin}
This is a phased array ladar integrated into the vehicle's surface area. It functions as a tactical ladar with a range specified in the vehicle's description, and as a laser commincator (p. \pageref{subsec:laser_comms})

\textit{Tactical Ladar Arrays:} These have an ``optical countermeasures'' mode -- see Blinding Lasers (p. TODO). Weight and cost are included in the vehicle statistcs; the array can't be added later.

\subsubsection{Laser Chemscanner}\label{subsubsec:laser_chemscanner}
Chemicals absorb laser energy at known wavelengths. This system uses a laser to detect airborne chemical compounds, as well as surface contaminants such as a slick of chemicals coating an object of the ground. It is most often used to identify chemical weapons or pollution levels in the atmosphere. It can also analyze the light scattered from swarms of microbots or nanomachines and identify them by matching the patterns with known models.

A dedicated laser chemscanner is half as expensive as a ladar, but has twice the range. A chemscanner mode for a ladar adds +0.2 CF.

\subsection{Multi~-Mode Radar}\label{subsec:multi-mode_radar}
This provides a search mode for locating potential targets, and an imaging mode for identifying them as they get closer. Generally, most moving targets that fit the radar's criteria are detected automatically. If a target is using radar countermeasures or being stealthy then roll a quick contest between the radar operator's Electronic Operation (Sensors) skill against the target's Stealth.

\textit{Search Radar:} This searches a fan~-shaped, 120\degree area in front of the user, hunting for rat-sized (SM -6) or larger moving targets and displaying them as blips on a screen. Darkness, smoke, and bad weather do not impair it, but it cannot see over the horizon or through solid obstacles. It provides a digital readout of target speed, altitude, position, and approximate size. It cannot, generally, distinguish between different objects of the same size (such as a human and a similarly sized robot). Background items make spotting stationary human-sized (SM 0) or smaller objects on the ground virtually impossible in anything but open terrain. Non-moving targets are impossible to distinguish from ground clutter unless the user has seen that particular ``blip'' moving.

\textit{Imaging Radar:}\label{itm:imaging_radar} This uses millimeter~-wave radar. This means it has a shorter range than search radar, but can spot small objects and determine their shape. An Electronics Operation (Sensor) roll is needed to distinguish fine relief. Imaging Radar can see through thin fabric or vegetation. It gives a +3 to Search rolls to locate objects like concealed weapons, and may ignore penalties for spotting objects hidden behind light brush. Ordinary radar detectors detect Imaging Radar at -4. Imaging Radar does not work underwater. The effects are similar to the Imaging Radar advantage. It has 1/10th the range of the radar in search mode.

Switching settings takes a Ready maneuver. If desired, a longer cable can connect the radar and its control panel -- this sometimes proves tactically desirable, since radar emissions can be detected over quire a distance.

\textit{Large Radar:}\label{itm:large_radar} A powerful multi~-mode radar suite, usually vehicle~-mounted. It has a 200~-mile range in search mode, 20~-mile range in imaging mode. \$100,000, 100 lb., D/8 hr. LC4.

\textit{Medium Radar:}\label{itm:medium_radar} A portable radar set. It can be worn as a pack, or mounted on a tripod, vehicle, or robot. It has a 60~-mile range in search mode, 6~-mile in imaging mode. It has no display screen of it own, but can be plugged into a computer monitor, HUD, or interface. \$10,000, 10 lb., C/8 hr. LC4.

\textit{Small Radar:}\label{itm:small_radar} A mini radar set with a 20~-mile range in search mode, 2~-mile in imaging mode. It’s available in a
hand-held version, or one that mounts on the shoulder and plugs into a HUD. \$1,000, 1 lb., B/8 hr. LC4.

\textit{Small, Medium,} or \textit{Large Tactical Radar:}\label{itm:tactical_radar} Military~-style multi-mode radar. It can track up to 10 targets at once out to the listed range, identify them at 1/10 that range, and give +3 to hit any of them with an aimed attack. +4 CF. LC2.

\subsubsection{Tactical Active Electromagnetic Sensor Array (AESA)}\label{subsubsec:aesa}
Some vehicles have large multi~-mode tactical radar antenna arrays buried in their hulls, often covering a good fraciton of their surface. These arrays are rated for their range in miles; see the vehicle descriptions.

AESA arrays are powerful enough to be used in \textit{disruption mode}. This uses a narrow microwave beam to jam or burn out enemy electroinc systems. See Microwave Disruptors (p. TODO) for combat statistics of AESA arrays.

A vehicular AESA operates indefinitely off vehicle power. Cost and weight are included in the vehicle's statistics, as the capabilities depend on the vehicle's surface area.

\subsection{Sonar}\label{subsec:sonar_active}
This is an active sonar using ultrasonic sound waves. Sonar can spot small objects and determine their shape, but an Electronics Operation (Sonar) skill roll is required to distinguish fine relief (e.g., to identify a face). Sonar can be ``jammed'' or fooled by explosions and other loud noises. Individuals or devices with Ultrahearing can detect sonar.

Sonar gadgets must be designed for air or water. Standard range is for underwater sonars. Air sonars have shorter ranges: 1/10th normal, multiplied by air pressure in atmospheres (one atmosphere on Earth). All sonars are ineffective in vacuum.

Electronics Operation (Sonar) rolls are used to detect objects. Ambient noise from sea life and other ships will interfere with detection; apply a -1 penalty for being near noisy sea life, or -6 for detecting an object in a busy, cramped harbor.

A sonar can sense objects out to its rated maximum range at no penalty; each doubling of range beyond that gives -2 to skill. Detection is limited to a 120\degree arc. Under ideal conditions, sonars can be detected at twice their own range, but ambient noise can interfere.

\textit{Large Sonar:}\label{itm:large_sonar} A powerful multi-mode sonar suite, usually vehicle-mounted. It has a 20,000~-yard range. \$20,000, 100 lb., D/8 hr. LC4.

\textit{Medium Sonar:}\label{itm:medium_sonar} A portable sonar, often used by small boats or underwater robots. It has a 2,000~-yard range. \$2,000, 10 lb., C/8 hr. LC4.

\textit{Small Sonar:}\label{itm:small_sonar} A small sonar used by divers, underwater battlesuits, and robots. It has a 200~-yard range. It comes in a hand-held version or one that mounts on the shoulder and plugs into a HUD. \$200, 1 lb., B/8 hr. LC4.

\textit{Tactical Sonar:}\label{itm:tactical_sonar} Military-style multi~-mode targeting sonar. It can track and identify up to 10 targets at once out to the listed range, and gives +3 to hit any of them with an aimed attack. +9 CF. LC2.

\subsection{Terahertz Radar}\label{subsec:terahertz_radar}
This uses the ``t~-ray'' wavelengths that lie between infrared radiation and the millimeter~-waves used by imaging radar. A terahertz radar can penetrate clothing, brush, or thin walls (up to a few inches thick) to see inside objects. It can also be used to spot small objects and determine their shape, and eliminates penalties to spot objects behind light cover. It gets +4 to locate concealed weapons, and while it still requires an Electronics Operation (Sensors) roll to distinguish fine relief, this roll is also at +4. Only special~-purpose sensors can detect its radar emissions. It doesn't work
underwater.

\textit{Large Terahertz Radar:}\label{itm:terhertz_large} 4,000~-yard range. \$200,000, 100 lb., D/10 hr. LC4.

\textit{Medium Terahertz Radar:}\label{itm:terahertz_medium} A portable radar set. It has a 1,200~-yard range. \$20,000, 10 lb., C/8 hr. LC4.

\textit{Small Terahertz Radar:}\label{itm:terahertz_small} 400~-yard range. \$2,000, 1 lb., B/10 hr. LC4.

\textit{Tactical Terahertz Radar:}\label{itm:terahertz_tactical} It can track up to 10 targets at once out to the listed range, and gives +3 to hit any of them with an aimed attack. +4 CF. LC2.

\section{Combination Sensors}\label{sec:combo_sensors}
%\addcontentsline{toc}{section}{Combination Sensors}
\subsection{Extensible Sensor Pod (ESP)}\label{subsec:extensible_sensor_pod}
This is expensive sensor suite is used by special ops teams or battlesuit troopers for urban warfare. It is a backpack unit with a short periscope, tipped with a multi~-sensor head. It swivels, and can be extended up to a yard vertically or horizontally. This lets the user see around corners or over cover. (If shot at, it is SM -8, HP 4, and DR 20)

The pod has a sound detector (p. \pageref{subsec:sound_detector}) and a hyperspectral imaging sensor (p. \pageref{subsec:hyperspec_imaging_sensor}), both with 8× magnification, plus a tactical terahertz radar (p. \pageref{subsec:terahertz_radar}) with a range of 200 yards. The user will need a HUD and either a terminal or neural interface to use it. \$50,000, 10 lb., 2C/100 hr. LC3.

\subsection{Tactical Sensor Turret}\label{subsec:tactical_sensor_turret}
This battlefield sensor system uses a rotating mini~-turret to look in any direction. It includes a set of hyperspectral imaging sensors (p. \pageref{subsec:hyperspec_imaging_sensor}) with 20× magnification and a tactical ladar (p. \pageref{itm:tactical_ladar}) with a range of 20 miles.

It must be controlled from either a terminal (p. \pageref{subsec:terminals}) or neural interface (p. \pageref{sec:neural_interfaces}).

Two versions are available:

\textit{Sensor Turret:}\label{itm:sensor_turret} A ball turret installed on the roof or in the nose of vehicles. It can also be placed on top of a building. \$300,000, 70 lb., external power. LC3.

\textit{Sensor Periscope:}\label{itm:sensor_periscope} The same system on a telescoping mast that can be extended up to seven yards (21') above teh roof of the vehicle or base camp it is installed in. Often used by submarines, and by specialized armored fighting vehicles that need to look over hills. \$350,000, 150 lb. LC3.

\section{Scientific Equipment}\label{sec:scientific_equip}
%\addcontentsline{toc}{section}{Scientific Equipment}
\subsection{Portable Laboratories}\label{subsec:portable_laboratories}
These provide the scientific equipment necessary to conduct research ``in the field.'' They include an array of scientific instruments, a dataport for linking them to a computer, and sealed sub~-compartments for storing solid, liquid, and gaseous samples. They fulfill the basic equipment requirements for gathering and analyzing samples.

Portable labs are specialized to a specific scientific skill: Archaeology, Biology, Chemistry, Farming, Forensics, Geology, Metallurgy, Paleontology, and Pharmacy. 

These labs incorporate high-precision systems that purify, separate, pump, stir, filter, and transfer minuscule samples with single~-molecule accuracy.

\textit{Suitcase Lab:}\label{itm:suitcase_lab} Fulfills basic equipment requirements for using the skill. Takes 10 seconds to set up or pack up. \$3,000, 10 lb., 4C/10 hr. LC4.

\textit{Pocket Analyzer:}\label{itm:pocket_analyzer} These are basic equipment for analysis of small samples. They are -5 for other tasks. \$500, 0.6 lb., 2B/5 hr. LC4.

\textit{Semi~-Portable Lab:}\label{itm:semi-portable_lab} Contains good-quality scientific equipment; takes 1 minute to set up or pack. +1 (quality) bonus to the skill. \$15,000, 40 lb., 2D/10 hr. LC4.

\textit{Mobile Lab:}\label{itm:mobile_lab} Enough lab equipment to fill a room; takes 15 minutes to set up or pack. +2 (quality) bonus to the skill. \$75,000, 200 lb., external power. LC4.

\subsection{Sensor Gloves}\label{subsec:sensor_gloves}
These gloves are equipped with sensitive tactile, pressure, chemical, and biometric sensors. They can weigh items by lifting them, measure the hardness and smoothness of materials, detect chemicals, read ink printing, and scan any of this information into computer by touch. The user's fingertips can sense residual heat in a chair, or feel faint vibrations in the floor as someone approaches. Add +4 to any task using the sense of touch.

The bonuses from Sensor Gloves are not cumulative with bonuses from the Acute Touch or Sensitive Touch advantages. Each glove: \$1,000, 0.2 lb., A/2 wk. LC4.

\subsection{Wristwatch Rad Counter}\label{subsec:wristwatch_rad}
This measures and siplays the amount of radiation that the user is exposed to, and can be programmed to set off an alarm if dosage exceeds a designated level. Can be connected to a HUD or be built into a helmet visor. \$100, neg., A/6 mo. LC4.