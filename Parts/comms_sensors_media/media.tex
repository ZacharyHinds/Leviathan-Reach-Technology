\subsection{Word Processing Software}\label{subsec:word_processing_software}
\textit{Voice Processor:} A smart voice-interactive word processing suite, with dedicated AI editing capabilities. It converts ordinary speech to text, giving +1 (quality) to Writing skill for composition, and +2 for editing. Voice processors are very common programs. Complexity 5, \$10.\label{itm:voice_processor}

\section{Recording and Playback}\label{sec:recording_and_playback}
%\addcontentsline{toc}{section}{Recording and Playback}
\subsection{Digital Data Storage}
%\addcontentsline{toc}{subsection}{Digital Data Storage}
Modular digital media are similar to older computer disk and digital video disks, but use three-dimensional data storage with greater capacity.

\textit{Data Bank:} A large unit used as a modular backup or expansion for computers. It has a cable jack. It is \$100, 1 lb. per 100 PB. \label{itm:data_bank}

\textit{Datachip:} About 1/4-inch square. Chip readers may be built into many other electronic devices. It holds 1 PB. \$1, 0.01 lb. A datachip drive is built into many devices; purchased on its own, it's \$5, 0.02 lb., AA/100 hr.\label{itm:datachip}

\textit{Data Dot:} Used for covert information storage, this tiny unit holds 1 TB. \$0.1, neg. A data dot drive accessory is \$5, 0.01 lb., AA/200 hr.\label{itm:data_dot}

\subsection{Digital Cameras and Camcorders}
\label{subsec:digital_cam}
Passive visual sensors can be used as digital cameras, but camera systems produce higher-quality images. They have a removable datachip, plus internal data storage with the same capacity. Each TB of storage holds about 12 hours of uncompressed, studio-quality imagery, or two weeks of compressed imagery (which is good enough for most purposes). Use Photography skill to take good pictures. All systems include a datachip drive, a microphone jack, and a display that can be used for simple editing tasks; treat as improvised equipment for any complicated audio-video production.

\textit{Flatcam:} A palm-sized digital audio-visual recorder. It has Night Vision 7 image-intensification lenses and 8× optical magnification. This is basic equipment for Photography skill. \$50, 0.1 lb., A/10 hr.\label{itm:flatcam}

\textit{Pocketcam:} A high-quality digital audio-video camcorder with 32× optical magnification and Night Vision 8 image intensification. It gives a +1 (quality) bonus to Photography skill. \$200, 0.25 lb., B/10 hr.\label{itm:pocketcam}

\textit{Portacam:} A professional-quality movie camera for news gathering, intelligence work, or video production. It provides 32× parabolic audio magnification, 128× optical magnification, and Night Vision 9. It gives a +2 (quality) bonus to Photography skill, and can be mounted on a tripod for extra stabilization. \$2,000, 4 lb., C/40 hr. LC4.\label{itm:portacam}

\textit{3D Cameras:} Camera built with specialized lenses to capture the depth needed for 3D imagery. They can also be used to record holotech projections. The above cameras are available as 3D cameras for the same cost. 3D images use 100 times as much storage space as flat images.\label{itm:3d_cam}

\subsubsection{Vid Glasses}\label{subsubsec:vid_glasses}
Tough sunglasses incorporate a HUD (p. \pageref{itm:terminal_hud}), earbuds (p. \pageref{subsec:earbud}), and the same camera as a flatcam (above). A cheaper alternative to ``night shades.'' Provides DR 2 to eyes. \$60, 0.1 lb., A/10 hr. LC4.

\subsection{Media Players}\label{subsec:media_players}
\textit{Book Reader:} The size of a slim paperback, this dedicated device is built as a digital text display, with a screen optimized for maximum readability. It can also read texts aloud. It stores a petabyte of text internally and has a data chip drive, a cable jack, and a radio microcommunicator. \$20, 0.1 lb. 2A/100 hr.\label{itm:book_reader}

\textit{Data Player:} An inexpensive palm-sized viewing screen and speaker for audio, video, text, or other data. It has a datachip drive, a cable jack, and a radio microcommunicator for connecting a HUD, computer, or ear phones. \$5, 0.05 lb., A/100 hr.\label{itm:data_player}

\textit{Entertainment Console:} This Complexity 6 computer is +1 Complexity when running computer games, virtual reality, and sensie programs. It includes a datachip drive, a portable terminal, and a cable jack. \$500, 1 lb., 4B/5 hr. or external power. LC4.\label{itm:entertainment_console}

\textit{Video Wall:} A flat, flexible, low-wattage video display pasted or painted on a wall. \$10 and 0.05 lb. per square foot. Uses external power.\label{itm:video_wall}

\textit{Multi~-Media Wall:} As above, but also ripples to generate sound, allowing a direction speaker effect. Many residences have these; they are also used for ad wall and other displays. \$20 and 0.05 lb. per square foot. External power.\label{itm:multi-media_wall}

\textit{3D Media Wall:} Higher resolution, providing realistic depth. \$50, 0.05 lb. per square foot. External power. LC4.\label{itm:3d_media_wall}

\subsection{Scent Synthesizers}\label{subsec_scent_synth}
A programmable odor generator used for air conditioning, parties, and art, with a specialized molecular assembler that produces realistic scents. Programming a known scent from the library takes one minute; creating a new one takes at least an hour and a Chemistry roll. Original or artistic scents may require days to perfect. It cannot generate biochemical agents such as pheromones or sleep gas, but it can create odors that mask other odors (-5 on rolls to detect things by smell), or produce a nauseating odor (treat as a mild form of riot gas; roll vs HT at no penalty to resist).

\textit{Odor Synthesizer:} Fills a medium-sized room or vehicle; affects a five-yard radius outdoors. \$500, 1 lb., B/100 hr. The cartridge is good for 100 mixes (lingers for a minute). LC4.\label{itm:odor_synth}

\textit{Programmable Perfume:} A wearable unit. Some scnets may be complementary, but it is a good idea to wash off one before trying another. Affects a two-yard radius, including the wearer. \$200, 0.1 lb., A/100 hr. LC4.\label{itm:program_perfume}

\subsection{Sonic Projector}\label{subsec:sonic_projector}
This uses acoustic heterodyning technology to transform a spoken or recorded message into a directional sound beam. The sound appears to emanate from the location the beam is directed at, rather than the projector. It has a microphone for voice transmission, and a datachip drive for playing recorded sounds.

It can be used for communication, so that the recipient hears a voice that seem to be right beside him, even if the sender is hundreds of yards away. Personal, theater, or concert sound systems often integrate sound projection technology to create 3D audio that emanates from multiple locations around the listener.

Stores, billboards, or vending machines can use a sonic projector to address targeted ads to individuals passing by (cameras and AI programs identify the most likely customers). It’s also useful for covert operations – for example, a softly-spoken message can be beamed to a distant target without anyone noticing. This can be used for psychological manipulation.

Complex effects (e.g., beaming a recording of someone walking behind a subject, so he thinks he’s being followed by invisible footsteps) require an Electronics Operation (Media) skill roll. Focusing on a moving target requires an attack roll: it is Acc 6, Bulk -2; use Beam Weapons (Projector) skill.

A sonic projector requires an atmosphere to conduct sound, and is not designed for underwater use (for that purpose, see Sonar Communicator, pp. 44-45). The signal travels at about 0.2 miles per second (at sea level).

Various projector sizes are available:

\textit{Large:} 300-yard range; can project up to four signals simultaneously. \$500, 2 lb., 2C/10 hr. LC4.

\textit{Medium:} 150-yard range. \$200, 0.5 lb., 2B/10 hr. LC4.

\textit{Small:} 15-yard range. \$50, 0.05 lb., 2A/10 hr. LC4.

\subsection{Holoprojectors}\label{subsec:holoprojectors}
These projectors are capable of projecting three-dimensional images at a distance into empty space and/or around objects. They can project images, movies, or still shots. These devices are often coupled with a sonic projector to create sounds that appear to emanate from the holographic image. All holoprojectors incorporate cable jacks and radio micro communicators, allowing them to be remotely controlled for use as displays, entertainment systems, or decoys.

\textit{Holoprojector:} The projection range is 12 yards. The visual projection can fill up to 216 cubic feet (2 × 2 × 2 yards). The projection area can also be moved at up to 12 yards/second, although doing so without disrupting the illusion requires an Electronics Operation (Media) roll. \$8,000, 4 lb., C/1 day. LC4\label{itm:holoprojector}

\textit{Holotech Player:} A simple holoprojector the size of a sugar cube, and can project a single image or short sequence (up to 30 seconds) with a range of one yard; the sequence or image is permanently stored in the device. It is often built into lockets and other keepsakes. \$10, 0.1 lb., A/1 day. LC4.\label{itm:holotech_player}

\textit{Mini Holoprojector:} This pocket-sized holoprojector has a range of seven yards, filling an area up to 54 cubic feet (1 × 1 × 2 yards). The projection area can be moved at seven yards/second, but doing so without disrupting the illusion requires an Electronics Operation (Media) roll. Mini holoprojectors are often built into other devices, such as a computer, a helmet, a ``magic wand," or an implant. \$2,000, 1 lb., B/1 day. LC4.\label{itm:mini_holoproj}

\textit{Super Holoprojector:} This powerful projector has a range of 33 yards, and can fill up to 5,000 cubic feet (e.g., 25' × 20' × 10'). Super holoprojectors may be used for entertainment, but are also government propaganda, and delivering villainous ultimatums. The zone can move at up to 33 yards/second, but doing so without disrupting the illusion requires an Electronics Operation (Media) roll. \$200,000, 100 lb., D/6 day. LC4.\label{itm:super_holoproj}

\subsubsection{Holotech Editing Program}\label{subsubsec:holotech_editing_program}
Software for creating or editing holotech and 3D camera images. It can be used to produce computerized holographic animation, special effects, etc. Use Electronics Operation (Media). Complexity 6 software. \$300. LC4 \label{itm:holotech_editing}

\subsection{Interactive Holoprojection}\label{interactive_holoproj}
This artificial intelligence software lets a holoprojector-user control projections ``on the fly," usually via a neural input device or direct neural interface. The operator combines objects from an image library with various pre-programmed and artificially-intelligent behavior sets. All imagery must remain in the projector area.

The operator takes a Concentrate maneuver to project animated, three-dimensional images of anything he can visualize. The images and sounds can occupy any frequency range, including spectra that are beyond human perception. They persist for as long as the device is operating.

In combat, a holoprojection can deceive and distract. Roll a Quick Contest of Electronic Operation (Media) against the Perception of anyone in a position to notice it. Success means the projection seems real to that individual (although if he knows it's a holoprojection, he'll just be impressed!)

To make a holoprojection disturbing enough to cause a Fright Check, win a Quick Contest of Artist (Holoprojection) against the higher of IQ or Perception for each victim. To trick someone into believing in a projection of someone she knows, roll the lower of Acting, Electronics Operation (Media), or Artist (Holoprojection) skill against the higher of a target's IQ or Perception.

Roll a new Quick Contest when someone fooled suddenly changes how they're interacting with the projection; e.g., they attack a holographic monster, or falls through a chair that isn't there. If they win or tie, the operator can't simulate a believable response to their action (such as the monster dodging, or the chair slipping) and the victim catches on.
\\ \\
{\color{Cerulean}
    \indent \textit{Modifiers:} A victim gets +4 if someone who knows about the projection warns him, or if you critically fail in a Quick Contest against someone else. He gets +10 if you create the holoprojection unsubtly and in plain sight, or if he examines it with a sense you can't deceive -- most often touch. Inappropriate projections give a further +1 to +10, while believable ones (e.g., you pull out a holographic gun) give from -1 to -5.
}
\\

It's hard to animate a convincing semblance of a holographic person for direct, personal interaction, such as dueling or conversation. Multiple fake people are progressively more robot and unresponsive; anyone rolling a Quick Contest to spot the projection is at +4 per construction after the first. Holotech projections obstruct vision but are otherwise intangible. They glow in the dark; apply a -1 penalty on rolls to fool or otherwise distract someone per -1 darkness penalty (unless the object would ordinarily be glowing in the dark).
Interactive holoprojection requires a computer running Complexity 6 software plus an interface for controlling the holoprojector. Apply a -6 to skill if attempting to control interactive holoprojection through anything other than a neural interface! LC4.

\section{Virtual Reality}\label{sec:virtual_reality}
%\addcontentsline{toc}{section}{Virtual Reality}
The simplest form of virtual reality is a visual display. The user dons goggles or a helmet that blocks out the real world and replaces it with a wrap-around view of computer-generated imagery. VR displays are popular means of receiving sensor input from computer games or simulations, from scientific and other sensors, and from sensors and instruments on vehicles or robots. Most remote-control drones use some form of VR display as part of their control system.

\subsection{Multi~-User VR}\label{subsec:multi-user_vr}
In the modern era, Multi-User VR has managed to take off as a popular means of socializing digitally. On planets with highly developed planetary networks, VR spaces and meetings have become more common than phone or radio calls. In these sorts of civilizations, service providers will generally offer a mix of private VR spaces and open public forums, such as virtual parks, bars, shopping malls, or streets. Virtual malls sell both virtual and physical goods and services, often incorporating simulations to allow users to try out goods.

Travel speed in a virtual reality may be limited to walking, but some users may be granted the ability to teleport, fly, etc., or board virtual public or private transportation.

Many service providers allow subscribers to design and rent their own personalized locations, either in public forums or in private-access areas. Corporations may have VR offices. Individuals should take care before using VR for confidential meetings. A system operator can design software to monitor or record events in “private” spaces.

Access to large “public” VR environments may be free (perhaps sponsored by corporations, or treated as the equivalent of public parks). Other VR sites may have dues ranging from \$1 per month to \$1 per minute, although the latter charge is likely only for sophisticated game sites or private clubs. Price may depend on how congested communications bandwidth is.

How a user interacts with a virtual reality depends on his VR rig. All VR rigs must be linked to a computer that is running a virtual-reality program.

\subsection{VR Gloves}\label{subsec:vr_gloves}
A simple set of gloves, used in conjunction with a HUD. It allows a user to manipulate virtual objects using the gloves. It requires a computer of at least Complexity 2 to use. \$20, 0.3 lb. (plus a HUD). LC4. VR gloves can be incorporated into any set of body armor or other suit gloves. \label{itm:vr_gloves}

\subsection{Basic VR Suit}\label{subsec:basic_vr_suit}
The user has VR gloves, plus small movement ``tracers" attached to various points on the body. He can move around a virtual reality and have a ``body" there, but only experiences full tactile stimulation in his hands. The suit takes 10 seconds to put on or remove, and requires a Complexity 3+ computer. It can be worn with any armor or clothing. \$200, 1 lb. LC4. \label{itm:basic_vr_suit}

\textit{Basic Neural VR:} Someone with a direct neural interface can omit the suit and run the equivalent of basic VR through a Complexity 4 program. \$30. LC4.\label{itm:basic_neural_vr}

\subsection{Full VR Suit}\label{subsec:full_vr_suit}
This consists of sealed helmet, gloves, and a sensor-equipped body stocking. The helmet blocks out the real world, creating 3D images, sound, and scents. The body stocking and gloves house feedback sensors and pressure devices. The suit allows the user to move about a virtual reality and manipulate objects as if they were real (subject to the constraints of the program). The suit will sense the user's movements and provide tactile force-feedback (including sexual stimuli, if this feature is enabled), although not strongly enough to suffer any injury. It takes a minute to put on, 30 seconds to remove. It requires a Complexity 5+ computer. \$2,000, 5 lb. LC4.\label{itm:full_vr_suit}

\textit{Full Neural VR:} Someone with a direct neural interface can omit the suit and run full VR througha Complexity 5 program. \$100. LC4.\label{itm:full_neural_vr}

\subsection{Total VR}\label{subsec:total_vr}
This is only available as a computer program accessed through direct neural interface. It provides the same effects as a full VR suit, with the difference that all the user's senses are engaged. The only limit is whatever safety factors are programmed into the system.

If safety interlocks are engaged, the user may feel discomfort or dislocation, but not pain. If they are not engaged, a person in a total VR simulation can feel real pain. She won't suffer injury, but psychological damage can result if she is hurt, killed, or tortured in VR. This results in Fright Checks.

A standard feature in total VR systems are ``consent-level'' protocols limiting how much ``reality'' (in terms of discomfort or pain) the user is willing to take. Additionally, they generally include a ``safeword'' function. If the user speaks a specific code word, they are immediately pulled out. Sabotage or system operator connivance might neutralize such features. Total VR is a Complexity 6 program. \$300. LC4.

\subsection{VR Manager}\label{sec:vr_manager}
This is the back-end program which manages the interactions of users within shared VR and must be run on whatever computer is maintaining the virtual environment. Each program can handle up to 10 users. For more people, additional instances are required. The manager can grant varying degrees of access to individual users to design characters or places within the environment. Its Complexity and cost depend on the most complex VR interface it can support:
\\\\
\indent \textit{Complexity 4:} Supports VR gloves or basic VR. \$10.

\textit{Complexity 5:} Supports up to full VR. \$30.

\textit{Complexity 6:} Supports up to total VR. \$100.
\\\\
The level of ``reality'' experienced is the lower of the VR interface or the VR manager.

\subsection{VR Environmental Database}
This stores a virtual environment to be accessed by a VR manager. Users of interactive networks might also store their own environmental databases (e.g., personal character avatars) on their own systems, to be uploaded to the VR manager.

Memory requirements vary widely depending on the number of different objects stored in it and their level of detail. A forest of identical trees is much smaller than a small room with a hundred different knickknacks. Some typical database sizes are:

\begin{table}[H]
    \centering
    % \caption{Caption}
    \rowcolors{1}{}{\colorcomms}
    \begin{tabularx}{\columnwidth}{lXr}
        \textbf{Imagery Database} &&  \\
        Virtual character && 0.001 TB \\
        Virtual room && 0.001 TB \\
        Virtual house or park && 0.01 TB \\
        Virtual mansion or wilderness && 0.1 TB \\
        Virtual street or mall && 1 TB \\
        Virtual neighborhood && 10 TB \\
        Virtual town && 100 TB \\
        Virtual city && 1K TB \\
        Virtual small nation && 10K TB \\
        Virtual large nation && 100K TB \\
        Virtual planet && 1M TB \\
        Virtual interplanetary state && 10M TB \\
        Virtual interstellar state && 100M TB \\
        Virtual galactic empire && 1B TB
    \end{tabularx}
    \label{tab:vr_env_database}
\end{table}

Virtual wilds, streets, malls, cities, and worlds include simulations of animals or people as well as live users, but they are not really ``alive" until someone else encounters them. Large areas may also use ``generic scenery" to fill in backgrounds. A virtual city may only have a few thousand specific building interiors, assembling other rooms from ``cut and paste" programs whenever individuals visit them.

Divide the required database by 10 for a ``cartoon" level of imagery; multiply by 10 for ``lifelike" imagery with fewer generic details. ``Lifelike" imagery experienced with full or total VR is nearly indistinguishable from reality.

Packaged Characters and Settings: Buying off-the-shelf realities or standard character avatars cost \$1 per TB. For customized settings and characters, may cost significantly more. Many system managers prefer to program their own characters and environments.

\subsection{Private Realities}
Some commercial computer networks, such as those offered by Virtual Authentic, allow users to construct and/or rent private VRs on the network that only they are allowed to access. Pricing is based on what the network provider charges for storage (generally \$1 per PB).

\subsection{VR-Enabled Software}
Many software programs support a VR interface, including repair programs and games. See \nameref{sec:ar}.

\subsection{Interactive Total VR: Dreamgames}
\label{sec:dreamgames}
These are interactive total VR games and simulations.. The user connects and is plunged into the setting and fiction genre of his choice.

Playing dreamgames can be addictive. This uses the rules for non-chemical addictions (B122): addiction to dreamgames is generally cheap, legal, and incapacitating [-10]. (May also be the grounds for a Delusion based on their games/characters.)

Dreamgame addiction is a growing social concern, considering the ease of access of direct neural interfaces and cheap computers powerful enough to run the dreamgames. One in five Eurydice parents say their child has struggled with dreamgame addiction.

Dreamgames are usually Complexity 6+, but due to massive distribution, commercial games are usually one-tenth standard cost. Specialized corporate, government, or military training sims will be full cost. Some high-end programs may be Complexity 7+, and correspondingly more expensive. There is also a booming market for independently-developed dreamgames. Most dreamgames are LC4.

\section{Augmented Reality}
%\addcontentsline{toc}{section}{Augmented Reality}
\label{sec:ar}
\subsubsection{Hardware}
Augmented reality is usually presented using video glasses (p. \pageref{itm:conf_video_glass}; or \textit{Vid Glasses} p. \pageref{subsubsec:vid_glasses}) or with a computer implant (p. TODO). A HUD (p. \pageref{itm:terminal_hud}) and a camera (p. \pageref{subsec:digital_cam}) is also sufficient equipment; either or both could be part of a helmet. A cyborg with bionic eyes and a computer implant running optical-recognition and database programs could keep everything in her skull. Digital minds can use augmented reality without any special interface.

\subsection{Memory Augmentation}
\label{sec:memory_aug}
This ``mug shot'' database is a common AR program. It uses stored and/or net-accessible databases ranging from the commonplace to the job-specific. Most people accumulate personal databases of people they meet or expect to meet, co-workers, and so on.

If user's wearable camera (or eyes, if they use a brain implant) spots someone whose face is in the database, the program will automatically display that person's name and a brief identifier. This functionality can be customized based on the database and can be told to ignore relatives and other constant companions. Similar programs exist for recognizing artwork, wildlife, and vehicles. Memory Augmentation programs are generally Complexity 5 (though additional functionality can increase this). \$100. LC4 (Some more advanced are LC3).

\subsection{Video and Sensory Processing}
AR can digitally process what the user sees, improving his vision.

\subsubsection{Visual Enhancement}
\label{sec:visual_enhance}
This gives +1 to Vision rolls. Complexity 4, \$1,000. LC4.

\subsubsection{Cosmetic Filter}
\label{sec:cosmetic_filter}
A common AR program, this controls the audio-video display on a communication system. When activated, the video uplnk picks up the user's image as usual, but filters it through a preprogammed ``ideal" of beauty before transmitting it to the receiver. The user still looks like herself, but the program tightens sagging jowls, erases crow's feet and wrinkles, and removes or minimizes blemishes. The user's video Appearance rises by one level, but cannot exceed Very Handsome. Any enhancement above Attractive has the Off-the-Shelf Looks (B21) modifier applied. Complexity 4, \$400. LC4.

\subsubsection{Video Masking}
\label{sec:video_masking}
Works as the cosmetic filter (above), except that it can change the user's features and voice. The user may resemble another person, or adopt a persona created by the program. Complexity 5, \$800. LC4.

\subsection{Smart Diagnostic}
\label{sec:smart_diagnostic}
Most modern technology incorporates built-in sensors to monitor their own status. The specifics vary from object to object; a precision machine measuring microstresses in its components or even a milk carton checking to see if the milk is spoiled. The data from these sensors is continuously uploaded to local (or planetary) networks, and accessed by looking at the object.

\subsection{Virtual Tutors}
\label{sec:virtual_tutors}
These systems simplify tasks such as repairing a car engine or building a prefabricated house. A mechanism may have dozens (or thousands) of different parts tagged with microcommunicators and positional sensors. Integral databases know where each part goes and virtual tutoring software can track both the parts and the user's own hand movements, aiding in assembly, disassembly, preparation, or maintenance.

\subsubsection{Virtual Tutor}
\label{sec:virtual_tutor}
This augmented reality program coaches the user in a specific task, such as assembling electronics or fixing a car engine. The user has an effective skill of 12. Complexity 3 if the task normally uses an Easy skill, Complexity 4 if it uses a harder skill or if it uses several skills in concert. Any necessary parts must be purchased as Instructor kits. Normal cost. LC4.

\section{Sensies}
%\addcontentsline{toc}{section}{Sensies}
\label{sec:sensies}
Sensies are a relatively new invention. They are recordings or transmissions of another person's sensory experiences. They are sensory telepathy transmitted through total virtual reality media. Users require direct neural interfaces and experience full sensory input as if they were really there.

Transmitting or recording a sensie requires a specialized device that picks up the subject's sensory experiences. If it's recorded, a sensie can be replayed by anyone with a direct neural interface; they'll see and feel everything the original subject did.

Sensies don't have to be made from humans. Recording nonhuman allows a user to ``become'' a cat, a bird, or even a Raven. (Commercial sensies of very simple creatures like butterflies or worms usually have more understandable virtual reality experiences dubbed over the simple-minded experience of the actual creature).

Some edited sensie programs come with multiple view points, so that you can try out the show or story line from the perspective of more than one character in it.

\subsection{Sensie Uses}
\label{sec:sensie_uses}
In recent decades, sensies have exploded in popularity as a growing entertainment medium. They are also used for surveillance and control. Emerging industries exist for using sensies to monitor prisoners, children, and even employees.

\subsection{Sensie Mass Media}
\label{subsec:sensie_mass_media}
Sensies have seen growth both as commercial products and as a form of reality entertainment in the form of sensie-blogs, sensie livestreams, and more.

Commercial sensies come in a variety of genres and formats. Pornography, drama, and travel and sports shows are very popular. The most popular programs are ``Sensie Experiences," the types of programs which bring the user on unique experiences: eating exotic food, scuba diving, sky-diving, zero-g free fall, and so on.

Many sensies are edited to remove any unpleasant sensations the viewpoint character may experience, such as sunburn, pain, hunger, or cold. However, black-market sensies may feature injuries, painful deaths, or torture. These find a market with jaded masochists, or as torture devices. Normally these are illegal, since the person making the sensie was harmed or killed. A sensie of sort will impose one or more Fright Checks on the user, at a penalty based on severity.

\subsubsection{Sensie Stars}
\label{subsubsec:sensie_stars}
Anyone using a sensory uplink can make a sensie transmission, but some people have a gift for recording a satisfying sensory experience. These individuals make good ``sensie stars." High HT and Acute Senses are valuable traits to have.

\subsection{Experiencing a Sensie}
A sensie is experienced from a live or recorded transmission of another individual's sensory experiences. Someone accessing a sensie experiences all the sensory data of the original subject: seeing through their eyes, hearing what they hear, sharing tactile sensations, etc.

There are two ways to experience a sensie:

In \textit{immersion mode}, the user is unable to use his own senses and is submerged in the transmission. If the transmission includes pain or physical afflictions, the user also feels pain and suffers shock effects, but takes no damage. If it includes terrifying events, severe injury, torture, or death, it requires one or more Fright Checks. Since the user's own senses are immersed, he might miss almost anything that didn't wreck the headset or media player, all mass-produced sensie equipment is programmed to turn off the sensie in the event of an emergency (such as a fire or burglar alarm).

In \textit{surface mode}, the receiver experiences the transmitted sensory perceptions, but they are muted. The receiver can still function, but she will be distracted. This imposes a -3 on other activities, unless the task is one that would benefit from intimate knowledge of what the subject is feeling; e.g., attempting to interrogate or seduce them. The user suffers only half the transmitter's shock penalties, and makes any required HT, Will rolls or Fright Checks at a +4 bonus. Most commercial sensory interface experiences are transmitted in surface mode.

Experiencing a real-time sensie in immersion mode requires a transmission speed of at least 1 GB per second; surface mode requires at least 0.1 GB per second. This generally means one has to ``jack in" to experience a sensie.

\subsection{Sensie Equipment}
Creating or experiencing sensie requires a neural interface and appropriate software.

\textit{Sensie Player:}\label{itm:sensie_player} This software lets someone experience sensie media. They must use a direct neural interface to connect their mind to a computer running this program. This lets them access recorded or live sensie feeds stored on their computer, or transmitted over networks or via communicator. Complexity 6. \$300. LC4.

\textit{Sensie Uplink:}\label{itm:sensie_uplink} This software lets someone transmit or record his sensory experiences as sensie media. The link requires a direct neural interface that is in communication with a computer running this program. The data is then sent to a recorder, or broadcast using a communicator or net connection. Complexity 7, \$1,000. LC4.

\textit{Braintaps:} These specialized cybernetic implants only record and transmit sensies. See TODO.

Anything with a digital mind -- AIs and mind emulations -- can record its experiences without the need for any kind of sensie uplink, since it experiences everything in digital form already.

A typical sensie program occupies about 100 GB/hour, recorded in standard digital media. Cost is about \$10 per hour for mass-market entertainment sensies, but may be considerably more for specialized ones such as tutorials. Sensie-rental fees are usually about 20\% of the purchase price.

\subsection{Sensie Editor}
\label{subsec:sensie_editor}
This is a software suite that someone who can play sensies can use to edit raw sensory recordings. The user can wipe portions of a recorded sensie, compress time with smooth jumps, fadeouts, or transitions, tone down sensory experiences, or splice several recordings together. It also can be used to analyze a sensie recording to tell whether it is ``raw" or edited, what kind of equipment was used, etc.

Sensie editors are necessary to make commercial-quality sensies from raw recordings. For instance, if sensie superstar Selena Usagi records her latest travel sim “Beautiful in Bali,” and takes an hour-long walk down a moonlit beach before skinny-dipping in the warm tropical ocean with her co-star, the editor might condense it to the most stimulating 10 minutes. The quality of the sensie-editing job matters as much as the actual experience that generated the sensie; experiencing a poorly edited sensie can be disorienting and unpleasant! Electronics Operation (Media) skill is used to operate a sensie editor. Complexity 6 program; \$5,000. LC4.

\section{Mass Media}
%\addcontentsline{toc}{section}{Mass Media}
\textit{Augmented Reality:} Traditional text, video, and other media may be delivered at all times as an overlay on daily life.

\textit{Total VR and Sensies:} Fully-interactive sensory experiences offer high levels of realism and excitement, and create new frontiers for artistic effects. Their main limitations are high bandwidth requirements, and the need for expensive and invasive neural interfaces.

\textit{Media Walls:} Cheap audio-video walls have lead to a renaissance in billboard technology as well as the emergence of video graffiti using paint-on screens.

\textit{Holotech Projections:} Super holoprojectors allow for giant-sized images that tower over entire communities.

\section{Teaching and Learning Aids}
%\addcontentsline{toc}{section}{Teaching and Learning Aids}
\subsection{AI Tutors}
\label{subsec:ai_tutors}
The growth of AI software has led to the emergence of AI tutors for education purposes. AI tutors can train the user in a variety of mental skills, languages, and learnable mental advantages. By using full or total VR, they can train any skill.

Using a Non-Volitional AI teaches at half the speed of a human teacher (the equivalent of self-study). Volitional AI is the equivalent of a human teacher.

AI tutors need Teaching skill and the trait or skill the user will study. See Purchasing Machines for the cost of skilled AI software.

\subsection{Training Robots}
\label{subsec:training_robots}
For a long time, robots were a common training aid for sports to combat training to medicine. They have fallen out of favor with the explosive growth of total VR.

\subsection{Virtual Education}
\label{subsec:virtual_ed}
VR allows the user to study IQ-based skills or languages with a distant teacher as if he were present. If the user has access to a Basic Vr rig or better, DX-based skills can also be learned. HT-based skills require total VR.

\subsection{Dream Teacher}
\label{subsec:dream_teacher}
This is an advanced form of total virtual reality. It transforms the user's dream-state into a teaching environment via direct neural interface. The user goes to sleep (or is sedated) while connected via direct neural interface to a computer that is running the program. As she sleeps, the program interfaces with her dreams to create lifelike simulations that reinforce rote aspects of a skill and teach new situations.

Dream teacher programs allow the user to perform Intensive Training (B293), while sleeping, in any IQ-based skill or lanuage. DX- and Ht-based skills are not quite as effective and are only learned as the same as Education (B293).

Dream teacher programs are individual for each skill or trait. Programs are Complexity 6 for Easy skills, 7 for Average skills, 8 for Hard skills and languages, or 9 for Very Hard skills. Behavior modification programs are Complexity 7 for -1 point disadvantages, Complexity 8 for -2 to -10, Complexity 9 otherwise. Standard software costs. Generally are LC4, but behavior modification programs and those that teach military or espionage skills will be LC3 or lower.