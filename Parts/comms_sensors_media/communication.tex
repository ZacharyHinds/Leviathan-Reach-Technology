\subsection{Cable Jack}\label{subsec:cable_jack}
This basic communications system is simply a plug for a fiber-optic optical cable. These are the backbone of many planetary communication networks. Optical cable provides a high-bandwidth data link for computers and other electronic gadgets, transferring 1 TB per second.

\textit{Cable Jack:}\label{itm:cable_jack} A socket and cable for plugging into other cable jack-equipped gadgets or into a building's network. It can be added to any gadget with greater than negligible weight. \$5, negligible weight.

\textit{Optical Cable:}\label{itm:optical_cable} Fiber-optic cable costs \$0.10 and weighs 0.01 pounds per yard. It comes in various lengths. Use Electronics Repair (Communications) skill to lay or install datacable networks.

See also Networks (p. \pageref{sec:networks}).

\section{Communicators}
\label{sec:comms}
Communicators send and receive voice transmissions. If connected to a terminal or a computer, they can exchange text, video, or data.

Most communicators only send and receive to others of the same type (e.g. radio to radio) or to individuals with an appropriate Telecommunication advantage. An exception is that laser retinal imaging can beam signals to anyone.

All communicators use Electronics Operation (Comm) (B189) to operate and Electronics Repair (Comm) (B190) skill for servicing and repairs. No roll is required for operation under normal circumstances (unless the user is unskilled).

Communicators are either broadcast or directional. Broadcast (omnidirectional) signals can be picked up by every communicator tuned to the same frequency within range. Directional signals are beamed toward a particular target, and unless noted, are limited by line of sight; terrain and the curve of the horizon block the beam. To overcome these line-of-sight restrictions, relay stations are generally used.

Communicator ranges are given in yards or miles. Interplanetary comm ranges are measured in astronomical units (AU). Interstellar ranges are in parsecs.

Communicators generally travel at the speed of light. This is effectively instantaneous for planetary communications, but across space, the time lag between sending a message and receiving a replay may be significant. A light-speed message crosses one AU in approximately 500 seconds.

When transmitting large files, the data transfer rate of a communicator is important. Data transfer rates are specified for different communication systems. Repeating the same data several times takes longer, lowering effective data transfer rate ("bandwidth"), but gives a significant boost to range: 1/4 speed doubles the range; 1/100 speed multiplies the range by 10, and 1/10,000 speed multiplies the range by 100.

All communicators can be, and often are, equipped with encryption systems.

\subsubsection{Standard Comm Sizes}
\label{subsubsec:standard_comm_sizes}

Communicators are available in these sizes:

\textit{Micro:} "Comm dots" too small for humans to use directly, but are built into many electronic devices that share data with each other. The short range makes detection unlikely. Not all comms have a micro-sized version.

\textit{Tiny:} A button-sized communicator that may be wrist-mounted (with a video display), worn as a voice-activated badge or ear piece, or built into many other devices such as helmets.

\textit{Small:} A palm-sized handset or built into powered armor helmets or vehicles. It has a small video display.

\textit{Medium:} A hefty communicator usually worn on a shoulder strap or backpack, or built into vehicles. It has a video display.

\textit{Large:} A vehicle-mounted unit, often with a sizeable antenna.

\textit{Very Large:} A room-sized installation, often with a large antenna, used for dedicated communications relay stations or spacecraft.

\subsubsection{Communicators with Different Ranges}
\label{subsubsec:comm_different_ranges}
The relative size of a comm determines its range. Not all comms come in all sizes. The listed range for a given size assumes that both transmitter and receiver are that size. If they differ use the range given for the smaller comm modified for the size of the larger ones as follows:

\begin{table}[H]
    \centering
    \rowcolors{1}{}{\colorcomms}
    \begin{tabularx}{\columnwidth}{XllX}
        & \textbf{Size Difference} & \textbf{Modified Range} & \\
        & One size greater & 3 × shorter range & \\
        & Two sizes greater & 10 × shorter range & \\
        & Three sizes greater & 30 × shorter range & \\
        & Four sizes greater & 100 × shorter range & \\
        & etc. & etc. &
    \end{tabularx}
    % \caption{Caption}
    \label{tab:comm_size_diff}
\end{table}

\subsection{IR Communicators}
\label{subsec:ir_comms}
IR comms use infrared directional signals. Its beam scatters somewhat and can bounce off solid objects. Make an Electronics Operation (Comms) roll to take advantage of this. Roll vs. Electronics Operation (EW) to eavesdrop on another IR comm's beam if you are within a few degrees of the beam path. The data transfer rate is 10 GB/minutes.

The beam is invisible, but infrared or hyperspectral vision can see it at up to double its range if it is aimed directly at the observer, or in dust or fog.

\textit{Large:} 50-mile range. \$2,000, 50 lb., 2D/10 hr. LC3. \label{itm:ir_comm_large}

\textit{Medium:} 5-mile range. \$500, 5 lb., 2C/10 hr. LC4. \label{itm:ir_comm_medium}

\textit{Small:} 1,000-yard range. \$100, 0.5 lb., 2B/10 hr. LC4. \label{itm:ir_comm_small}

\textit{Tiny:} 100-yard range. \$20, 0.05 lb., 2A/10 hr. LC4. \label{itm:ir_comm_tiny}

\textit{Micro:} 10-yard range. \$5, neg., AA/100 hr. LC4. \label{itm:ir_comm_micro}

\subsection{Laser Communicators}
\label{subsec:laser_comms}
All laser comms use a modulated multi~-frequency laser beam to transmit~-directional signal. The narrow beam and line-of-sight requirement makes it hard to eavesdrop on a laser comm signal; someone must be in the direct path of the beam to intercept it. The beam is invisible and eye-safe, and tunes itself automatically to penetrate snow, fog, etc. Laser comms may also be tuned to use blue-green frequencies to reach underwater. The signal range is 1\% of normal underwater, with a maximum range of 200 yards. The data transfer rate is 1 TB per minute.

\textit{Very Large:} 100,000-mile range. \$40,000, 400 lb., external power. LC3. \label{itm:laser_comm_very_large}

\textit{Large:} 10,000-mile range. \$10,000, 50 lb., 2D/10 hr. LC3. \label{itm:laser_comm_large}

\textit{Medium:} 1,000-mile range. \$2,000, 5 lb., 2C/10 hr. LC4. \label{itm:laser_comm_medium}

\textit{Small:} 100-mile range. \$400, 0.5 lb., 2B/10 hr. LC4. \label{itm:laser_comm_small}

\textit{Tiny:} 10 mi. \$100, 0.05 lb., 2A/10 hr. LC4.\label{itm:laser_comm_tiny}

\textit{Micro:} 2,000-yard range, but usually broadcasts at lower output with range of 10 to 20 yards. \$20, neg., AA/100 hr. LC4. \label{itm:laser_comm_micro}

\subsubsection{Laser~-Retinal Imaging}
\label{subsubsec:laser-retinal_imaging}
This upgrade for laser comms allows them to beam graphics or text files directly into the retina of a single eye. It is difficult to aim; this is treated as an attack aimed at the eye (-9 to hit) but with an Acc 12 (or 18 if mounted on a tripod or vehicle). Roll Electronics Operation (Comms) to hit. If the subject is standing still or walking slowly, the laser can continue to track once a hit is achieved.

The subject doesn't need a communicator to receive a signal, making this a good way to send covert messages over a few miles. They can, however, interrupt a retina message by closing their eyes or turning their head. Glare~-resistant optics will also filter out a message.

Can only send images (or text as an image). It can flicker several hundred images per second, but most subjects would only see a blur at that speed. The subject's comprehension limits the data-transfer rate. Sending text limits the transmission to the subject's reading speed (which the sender must estimate). Since the transmission is one-way, the sender likely has no idea whether the subject read the information.

Fitting a laser comm with the computer chips for laser-retinal imaging costs \$1,000, but adds no weight. LC3.

\subsection{Radio Communicators}
\label{subsec:radio_comms}
Radio comms are broadcast communicators which use radio waves. All radio comms incorporate spread-spectrum technology in which the clarity and reliability of communications is improved by spreading the signal over a range of frequencies. The frequency hopping also keeps the transmitter from being "bright" in any given frequency, making it very hard to detect.

Radio range may drop by a factor of 10 in urban environments or underground. The data transfer rate is 0.1 GB per minute, but range drops significantly (divide by 10) when transmitting real-time-audio-visual signals.

\textit{Very Large:} 20,000-mile range. \$20,000, 400 lb., external power. LC3.\label{itm:radio_comm_very_large}

\textit{Large:} 2,000-mile range. \$4,000, 50 lb., 2D/10 hr. LC3.\label{itm:radio_comm_large}

\textit{Medium:} 200-mile range. \$1,000, 5 lb., 2C/10 hr. LC3.\label{itm:radio_comm_medium}

\textit{Small:} 20-mile range. \$200, 0.5 lb., 2B/10 hr. LC4.\label{itm:radio_comm_small}

\textit{Tiny:} 2-mile range. \$50, 0.05 lb., 2A/10 hr. LC4.\label{itm:radio_comm_tiny}

\textit{Micro:} 500-yard range, but usually broadcasts at lower output with a range of two to four yards. \$10, neg., AA/100 hr. LC4.\label{itm:radio_comm_micro}

\subsection{Sonar Communicators}
\label{subsec:sonar_comms}
Sonar comms use modulated sound beams to broadcast communication. It travels at the speed of sound: almost a mile per second underwater or 0.2 miles per second in air (at sea level). Generally are designed for underwater operation, but there are some that are tunable to operate in air (in which case, the range is equal to 1\% of the listed range multiplied by the air pressure of the atmosphere). Do not work in a vacuum. The data transfer rate is very slow: 0.1 MB/minute.

Signals can be detected (but not understood) at twice the comm range by passive sonars, or by anyone with Ultrahearing or Vibration Sense advantages. The only way to jam the signal is with powerful, specialized sonar jammers -- but underwater explosions cause transient interference.

\textit{Large:} 450-mile range. \$5,000, 50 lbs., 2D/10 hr. LC3.\label{itm:sonar_comm_large}

\textit{Medium:} 45-mile range. \$1,000, 5 lbs., 2C/10 hr. LC3.\label{itm:sonar_comm_medium}

\textit{Small:} 4.5-mile range. \$200, 0.5 lbs., 2B/10 hr. LC4.\label{itm:sonar_comm_small}

\textit{Tiny:} 900-yard range. \$40, 0.05 lbs., 2A/10 hr. LC4.\label{itm:sonar_comm_tiny}

\textit{Micro:} 90-yard range. \$10, neg., AA/100 hr. LC4.\label{itm:sonar_comm_micro}

\subsection{Sonic Communicators}
\label{subsec:sonic_comms}
A sonic projector can be used to beam voice or audio signals.

\subsection{Ghwel~-Ripple Communicators}
\label{subsec:gwhel_comms}
These communicators use gwhel~-generators (p. \pageref{subsubsec:gwhel_generator}) to manipulate gravity to create gravity waves to transmit data. The signal is omnidirectional; eavesdroppers must roll against Electronics Operation (EQ) to listen in. The data transfer rate is 1 GB/minute.

Gravity waves reach underwater and penetrate solid objects at no penalty. Intense gravity sources such as neutron stars, pulsars, and black holes can disrupt the signal.

\textit{Very Large:} 100,000-mile range. \$400,000, 400 lb., external power. LC3.\label{itm:gwhel_comm_very_large}

\subsection{Neutrino Communicators}
\label{subsec:neutrino_comms}
Neutrino comms are directional communicators which use a modulated beam of neutrinos. It is nearly impossible to jam or intercept, and functions in any environment -- it can reach underwater or penetrate solid objects at no penalty, and isn't blocked by the horizon.

Neutrino transmissions uses specialized particle accelerators.

\textit{Very Large:} 100,000-mile range. \$500,000, 400 lb., external power. LC3.\label{itm:neutrino_comm_very_large}

\begin{ravenbox}
    \subsection{Gwhels}\label{subsec:gwhels_ravenbox}
    Gwhels, also known as ``Gravity Whales'' or ``G~-Whales'' are massive creatures which roam space in search of the rare materials which make up their diet. 
    
    Physically, they resemble Earth's whales, though covered in ceramic-like scales, razer~-sharp teeth, and they can grow to the size of small moons. While they only possess animal intelligence, they are quite smart and vicious, capable of learning quickly. 
    
    They move through space with their propulsion bladder. Scientists reverse~-engineered the bladders to create gwhel~-generators (p. \pageref{subsubsec:gwhel_generator}). Their blubber is also a valuable resource. Gwhel bladders easily sell for \$1,000 per ton, but the real value is living, captured gwhels which are easily \$200,000 per ton. 
\end{ravenbox}

\section{Encryption}
%\addcontentsline{toc}{section}{Encryption}
\label{sec:encryption}

\subsection{Encryption Systems}
\label{subsec:encryption_systems}
Basic Encryption: Encryption standard complex enough to be reasonably secure, but not so complex that it slows down operations by taking up excessive bandwidth or computer processing time. A Complexity 10 computer may attempt to break this encryption once per hour. This standard is adequate for business transactions and personal privacy. It comes standard in all communicators and computers at no extra cost. LC4.

\textit{Secure Encryption:}\label{itm:secure_encryption} More complicated encryption, often used to secure classified government or military information. There may be a delay of one or two seconds as messages are sent or data is processed. Breaking it in an hour requires a Complexity 12 computer. A secure encryption chip for a computer or comm is \$500; neg. weight. The chip also lets the system generate or encrypt one-time pads. LC2.

Cryptography skill is used to crack encryption systems. Rather than the modifiers on B186, apply modifiers for the quality of the decryption program and for the time spent relative to the base time (see above).

The encryption standards specifies the Complexity of computer required to make an hourly attempt at decrypting it. A higher-Complexity computer reduces the time by a factor of 10 per +1 level over it (six minutes for +1, 36 seconds for +2, or in real time as the message arrives for +4 or more). Using a computer of lower Complexity multiplies the time by 10 for each -1 Complexity (10 hours, 100 hours, 1,000 hours, etc.).

\textit{Decryption Program:} Contains a database of hacks and shortcuts. Gives a +1 (quality) bonus to Cryptography. Complexity 2, \$500. LC3.\label{itm:decryption_program}

\textit{Quantum Computers:}\label{itm:quantum_decryption} A quantum computer adds +5 to its Complexity for the purpose of decryption. Also, if the quantum computer is of lower Complexity than the encryption, each -1 under triples the time required (3 hours, 10 hours, 30 hours, etc.) rather than causing a 10-fold increase.

\subsection{One-Time Pads}
\label{subsec:one-time_pads}
There is one way to ensure that an encrypted message is not broken: the “one-time pad” system. The message is encrypted using a completely random key that is only used once. Unlike public-key encryption, the encryption and decryption keys are the same. Thus, both the sender and recipient must already have the key.

To use one-time pads, one or more of them are generated and passed to the parties who wish to use them to communicate (e.g., before sending a spy on a mission). That way, the only signal that need be sent is something like “use pad \#231.”

One-time pads are only for data transmission. The key must be at least as long as the message it encodes (i.e., it takes up as much bandwidth). Secure encryption systems have hardware-based random number generators that use electrical or atmospheric noise or nuclear particle decay to generate the true random numbers suitable for one-time pads.

The other disadvantage of one-time pads is that safe delivery often requires a physical courier or advance arrangement – transmitting them as public key-encrypted messages risks someone decrypting them, which defeats the entire point. Delivery and retrieval of disks containing a one-time pad are an opportunity for adventure. However, a faster alternative is to use quantum communications to transmit a one-time pad key, since any eavesdropper on a quantum channel would be detected.

\subsection{Quantum Communications}
\label{subsec:quantum_comms}
In quantum theory, certain pairs of physical properties are complementary, in that measuring one property necessarily disturbs the other. By using quantum phenomena to carry information, a communication system can be designed which always detects eavesdropping.

A laser communicator, neutrino comm, or optical cable can have a quantum channel option. Laser or neutrino comm range is 10\% of normal when using it. If both sender and receiver use quantum channels, the result is highly secure: If anything intercepts the signal, the users are alerted instantly. Multiply the cost of a laser or neutrino comm with a quantum channel by 10; multiply the cost of optical fiber systems by 100. LC3.

\section{Translators}
%\addcontentsline{toc}{section}{Translators}
\label{sec:translators}

\subsection{Translator Program}
\label{subsec:translator_program}
This program translates conversations from one language into another in real time. It can be used with any computer with an appropriate interface. Spoken languages require a microphone or speaker, whether built-in or provided by a linked communicator. Some users speak into their communicators (or use a neural interface) and let the computer speak for them.

Each translation (e.g. English-Portuguese) is a separate program. The program's level of comprehension can never exceed the input.

\textit{Broken:} This translates speech at a Broken comprehension level. Each language requires at least a 10GB database. Complexity 3. \label{itm:trans_program_broken}

\textit{Accented:} Translates speech at the Accented comprehension level. Each language requires at least a 30GB database. Complexity 4. \label{itm:trans_program_accent}

\textit{Native:} Translates speech at the Native comprehension level. Each language requires at least a 100GB database. Complexity 5.\label{itm:trans_program_native}

Reduce program Complexity by 1 if either language is an artificial construct designed for ease of learning and/or translation. If this is the case for both languages, the modifier is cumulative.

Increase Complexity by 1 if translating between different species (such as Raven and Human). Complexity also increases by 1 if the system translates from one sense to another, such as sign language to spoken language, or between different frequencies (ultrasonic signals to human voice). Appropriate input and output sensors will be needed.

\subsection{Field Translator}\label{subsec:field_translator}
A high-capacity, small computer with with a datapad terminal and a full range of microcomms. The Complexity 5 computer runs a non-volitional AI with an IQ 10 and two Native-level spoken or visual language translation programs. It can store 100,000 Native-level spoken language databases, which must be purchased separately. \$3,000 plus cost of translator program (above). 

\subsection{Translator Disk}\label{subsec:translator_disk}
A much smaller version of the field translator (above), this is a high-capacity tiny computer, a small sonic projector, a mini-camera, and the culture's full range of microcomms. It can store 10,000 Native-level spoken language databases. The sonic projector can be set to allow only the target to hear the translation (for crowded spaceport bars and embassy cocktail parties); otherwise, the computer's infra- and ultrasonic-capable speaker is used. It has IQ 6. It comes as a stick-on medallion or earpiece. It's similar to the field translator in all other respects. \$150 plus the cost of software, 0.125 lbs., B/36 hrs. LC4.

\section{Neural Interfaces}
%\addcontentsline{toc}{section}{Neural Interfaces}
\label{sec:neural_interfaces}
Neural interfaces capture and amplify nerve impulses and/or muscle movements, translating them into digital commands for an electronic device or a computer interface. Permits commands to be entered with "the speed of thought," often much faster than speech or typing.

\subsection{Neural Input Receiver}
%\addcontentsline{toc}{subsection}{Neural Input Receiver}
\label{subsec:neutral_input_receiver}
These systems pick up neural signals indirectly from the user's muscle movement, eye/facial movement, or brain waves. They pick up basic commands, but cannot transmit sensory feedback to the user. They're built into wearable devices such as goggles or contact lenses for hands-free operation, usually in concert with a physical HUD display.

\textit{Neural Input Headset:} Picks up brain waves. It can replace a computer mouse or equivalent device. \$50, 0.1 lb. A/100 hr. LC4. \label{itm:neural_input_headset}

\subsection{Direct Neural Interface}
%\addcontentsline{toc}{subsection}{Direct Neural Interface}
\label{subsec:direct_neural_interface}
Often referred to as a "neural interface," this sophisticated device allows the user's brain to communicate with computers and control complex equipment. It can do anything that a neural input device can do, and much more.

The interaction is two-way: data displays, physical feedback, and other sensory information can be transmitted directly into the user's brain. There is no need for a user to touch controls or see physical data displays. They can have the equivalent of a HUD overlaid on their visual field, so they can "live" in augmented reality. A direct neural interface is required for certain technologies, such as dream teachers, sensies, and total virtual reality.

When using a neural interface, the user is opening up their nervous system and brain to intrusion -- or even being hacked. Like any networked computer, the user's safety depends on their encryption systems, the products they use, and those associates or superiors to whom they grant access.

\textit{Neural Interface Implant:} This involves implanting sensitive electrodes in the brain along with an implanted communicator. See Direct Neural Interface Implant in Cybernetics.\label{itm:neural_interface_implant}

\textit{Neural Interface Helmet:} Largely defunct, but still available some places, this "crown of thorns" hlmet invades the skull with tiny nanowires. They inflict no damage, but users may find the idea disturbing! The helmet takes four seconds to don or remove; yanking it off before disconnecting causes 1d injury. It includes a cable jack and radio micro communicator. \$10,000, 2 lb., C/100 hr. LC3. \label{itm:neural_interface_helmet}

\textit{Neural Induction Helmet:} Uses non-invasive neural induction process to "write" data to the brain. \$5,000, 2 lb., C/100 hr. LC3. \label{itm:neural_induction_helmet}

\subsection{Brainlocks}
\label{subsec:brainlocks}
Any neural input device or neural interface may include a brainlock. This is an interface programmed to only accept a user who has a specific brainwave pattern. The user list can be hard-wired into the system; otherwise, any interfaced user can use a password to alter the lock's parameters. If attached gadgets have multiple functions, only some might be brainlocked. A brainlock can also grant partial access to computerized records or other data, based on Security Clearance or other criteria. A brainlock has no extra cost. LC4

\section{Networks}
%\addcontentsline{toc}{section}{Networks}
\label{sec:networks}
\subsection{Planetary Network}
%\addcontentsline{toc}{subsection}{Planetary Network}
\label{subsec:planetary_network}
Most civilized worlds have networks that cover the entire planet. The exact structure of this may vary from civilization to civilization. Some may be multiple decentralized networks, like the Internet. Others may be business or state monopolies.

A planetary network consists of high-bandwidth communications backbone (often using optical cables), an infrastructure of repeater stations, communication satellites and other relays, supporting databanks and software, and the people or machines that maintain it. Generally, this will be a subscription paid to the service, though some may be provided by the government. This is generally included in cost of living as part of the utility bill, but if paired for separately ranges from \$10 to \$60 per month.

All subscribers with compatible communication gear may call or send messages on the network at no extra cost. Accounts include voice and the equivalent of email addresses.

Storage of data on the provider's system is usually included. Storing lots of information costs extra. \$1 per petabyte per month.

\subsubsection{Cable Connection}
\label{subsubsec:cable_connect}
Most users connect to a planetary network through a cable box. Generally included with subscription; if purchased, a box is \$100, 0.2 lb., external power. LC4. \label{itm:network_cable_box}

\subsubsection{Mobile Access and Cellular Communicators}
\label{subsubsec:cell_comms}
Subscribers using compatible communicators can route calls through a planetary network provided they're in range of a local repeater station.

In places without a working repeater station, network access is generally available via satellite connection. The user's comm needs at least a 10-mile range.

Most cellular networks are based on radio or laser systems.

\textit{Cellular Communicator:} A comm that can only access a planetary data network is available at half the normal cost. Generally this is a tiny or small radio comm. \label{itm:cell_comm}

\subsubsection{Comphone}\label{subsubsec:comphone}
The next evolution of the personal communicator, this device consists of a tiny computer with the compact and slow options(Complexity 4), a data player, a GPS receiver, an inertial compass, a network-only radio microcomm, a laser microcomm, and a tiny radio receiver. They are small enough that they come as a medallion, wristband, or badge with ``stick pad'' backing. Comphones have a tiny screen and some buttons, but their main interface is voice, or they can be hooked up to an external input. \$35, 0.08 lb., 2A/16 hr. LC4.

A more expensive version with a real datapad, full tiny radio, and a regular computer (Complexity 5): \$150, 0.2 lb., B/24 hr. LC4.

\subsection{Earbud}\label{subsec:earbud}
This earplug contains a radio microcomm with a deliberately shorter range (10 yards), a speaker, and a filtered external pickup that gives +1 to resist loud noises. The filtered pickup can also be set to a ``sensitive'' mode, boosting volume of ambient noise, giving +1 to Hearing rolls (but -2 to resist loud noises). Used as a headset for a comphone or data player. Double cost for two connected by a short length of optical cable and quadruple for a wireless pair (connected using tiny, dedicated radio microcomm). \$2, neg., non-rechargeable AA/2,000 hrs. LC4.

\section{Mail and Freight}
%\addcontentsline{toc}{section}{Mail and Freight}
\label{sec:mail_freight}
\subsection{Suborbital Express Mail}
\label{subsec:suborbital_mail}
While not standard on every planet, most have some form of high-speed, high-priority courier service. Generally use hypersonic aircraft and spaceplanes to ship mail at 10 times the speed of sound. Typical price: \$100 per pound.

\subsection{The Couriers' Guild}
\label{subsec:courier_mail}
While the Couriers have become known to offer a wide variety of services, their original and primary function was the delivery of mail and freight across the universe. The specific rates vary guild to guild but as a general rule they charge \$15 per AU per ton (or \$150 per passenger per AU) though trips that require FTL travel (such as through the Hydra Gates or other method) have a significant up-charge.