Even as technology gives thieves and spies the ability to bypass old security systems, it creates new ones to replace them. As the average criminal becomes more sophisticated, so does the security that tracks him down. This chapter covers security systems that protect against both physical and electronic intrusion, as well as advanced law~-enforcement tools that allow police and security forces to track, identify, and detain criminals more effectively -- or simply suppress a riot.

It might be possible to build an impregnable security system -- but more layers of security add more complexity and difficulty. Generally, security comes at the expense of convenience and efficiency.

\section{Barriers, Mines, and Traps}\label{sec:barriers_mines_traps}
Many dangerous traps have low LC. 

In addition to the systems described here, construction foam (p. \pageref{subsec:construction_foam}) can make useful barriers.

\subsection{Armored Doors}\label{subsec:armored_doors}
A heavy door made of an inch of composite armor. It has HP 50 and DR 150. It is \$1,000, 200 lb., per 10 square feet. Typically, these are made from metal~-matrix composites, though older (and weaker: DR 100) doors might use ceramic composites. 

The lock is usually in the adjacent wall rather than the door.

\subsection{Laser Fences}\label{subsec:laser_fences}
These project a continuous beam between two emitters, which may be built into fence posts, doorways, or corridors. Each emitter weighs 10 pounds and may be no more than 10 yards apart.

\textit{Open:}\label{itm:open_laser_fence} This style of laser fence produces either a fixed or moving pattern of beams that can be avoided with an Acrobatics-3 or Escape-3 roll.

\textit{Tight:}\label{itm:tight_laser_fence} A thick, continuous energy field. It can't be avoided; anyone passing through takes damage. This requires significantly more power to generate, so many facilities take advantage of a computer~-controlled system which starts in the open pattern, then switches to a tight pattern if an intruder is detected.

\textit{Laser Fence:}\label{itm:laser_fence} This inflicts up to 6d(2) tight~-beam burn damage. \$5,000 per post for an open fence, double cost for a tight fence. LC3.

\textit{Electrolaser Fence:}\label{itm:electrolaser_fence} An electrical fence using energy beams instead of wires. It delivers a HT-6 (2) affliction attack plus linked 1d-3 burn damage; use the rules for military electrolasers (p. TODO). The fence can be set to ``stun'' or ``kill.'' \$5,000 per post for an open fence, double cost for a tight fence. LC3.

\textit{Rainbow Laser Fence:}\label{itm:rainbow_laser_fence} This inflicts up to 6d(3) tight~-beam burn damage. \$3,000 per post for an open fence, double cost for a tight fence. LC2.

All of these fences use external power.

\subsection{Electronic Locks}\label{subsec:electronic_locks}
An electronic lock may be mounted on doors, consoles, briefcases, and anything else that needs to keep people out. It uses a numeric keypad, or a small electronic key card. Picking it requires Electronic Repair tools (p. \pageref{subsec:tool_kits}) or an electronic lockpick (p. \pageref{subsec:elec_lockpick}).

\textit{Simple Lock:}\label{itm:simple_lock} Typical of homes, hotel or shipboard staterooms, etc. No modifier to Lockpicking rolls. Uses building power. \$25. LC4.

\textit{Complex Lock:}\label{itm:complex_lock} Typical of secure installations. -4 to skill rolls to pick. 

Electronic locks may also incorporate a scanlock (p. TODO) for additional security. 

\subsection{Remote~-Controlled Weapons}\label{subsec:remote-con_weapons}
These are usually connected to sensors with a cable and controlled by a computer, either manually or via an AI. Roll vs. Traps-9 to spot them first.

\textit{Defense Globe:}\label{itm:defense_globe} A remote~-control weapon mounted in a small turret, usually disguised as an ordinary light fixture or smoke detector. Install any ranged weapon up to 4 lb. It is SM-6, HP 8, and can't use active defenses. Beam weapons use building power. \$100/lb. of weapon, plus the cost of a weapon and a smart sight (p. TODO).

\textit{Spray Canisters:}\label{itm:spray_canisters} Thes do not require sophisticated mounting systems; they're normally disguised as building fire extinguishers, or placed in air ducts. They can be built by adding optical cable or a communicator (p. \pageref{sec:comms}) to a spray tank (p. TODO). Numerous types of gas and nano can be deployed -- see \textit{Gases and Clouds} (p. TODO). \$100 for installation.

\subsection{Safes and Vaults}\label{subsec:safes_vaults}
These delay or deter thieves. Safes use electronic locks (above) and biometric scanners (p. TODO) to limit access to valuables; add normal cost and weight to the safe.

\textit{Wall Safe:}\label{itm:wall_safe} A typical home or business safe; one cubic foot, DR 150, HP 25. \$100, 50 lb. LC4.

\textit{Small Safe:}\label{itm:small_safe} A high~-security safe; five cubic feet, DR 450, 80 HP. \$500, 0.5 tons. LC4.

\textit{Armored Vault:}\label{itm:armored_vault} A small walk~-in vault. 50 cubic feet, DR 1,200, 100 HP. \$30,000, 2 tons. LC4.

\textit{MT Vault:}\label{itm:mt_vault} An MT booth (p. TODO) may be placed inside a room in a hidden location. LC3.

\subsection{Sonic Barrier}\label{subsec:sonic_barrier}
This generates a curtain of high~-intensity sound, inaudible until someone tries to cross it. It can be turned on or off remotely. They generate a faint ripple in the air (make a VIsion roll to notice) from the sonic field. It inflicts a HT-6 affliction attack on anyone trying to cross, with the effect of a sonic nauseator (p. TODO) beam weapon. \$3,000, 10 lb. per 10 square yards of field, external power. LC4.

\subsection{Wire Fences}\label{subsec:wire_fences}
The fencing materials here are designed to be easily stored and quickly deployed. A typical ``unit'' of fencing stretches up to 15 yards when uncoiled or unfolded, and stands four feet tall. All fencing is free~-standing, and flexible enough to form a curbed enclosure or surround an odd~-shaped area. Stakes and other fixtures can make the fencing more permanent. Multiple layers of fencing can be ``stacked'' for extra protection.

I takes one man~-minute per yard to deploy fencing. If protective gloves, wirecutters, and fasteners are not available, the time required is tripled.

\subsubsection{Cutting Wire}\label{subsubsec:cutting_wire}
Cutting wire comes coiled into tight rolls. The wire is wound with triangular segments of memory metal that extend when the wire is subjected to an electric pulse, forming thousands of small jagged cutting edges. Once ``popped,'' the wire cannot be returned to its original form. The inner core of the wire is flexible and shear~-resistant, making it difficult to cut. 

Passing through an area of cutting wire requires a roll at DX-5 each yard. Failure deals 1d-1 cutting damage, and requires a Will roll to avoid yelping or cursing as barbs tear clothing and skin (unless you have High Pain Threhold; at a -3 penalty with Low Pain Threshold). A 15~-yard coil of cutting wire is \$100, 15 lb. LC4.

\subsubsection{Fragwire}\label{subsubsec:fragwire}
This looks like ordinary wire, but the core is tightly coiled memory~-metal. When cut, the wire explodes outward with a loud \textit{ping}! The burst of sharp fragments does 1d-1 cutting damage over a two~-yard radius. Fragwire is often wound around cutting or sensor wire to dissuade infiltrators. A 15~-yard coil of fragwire is \$200, 30 lb. LC2.

\subsubsection{Sensor Wire}\label{subsubsec:sensor_wire}
This wire includes an optical~-fiber core. Each end of a strand terminates in a short wire plug that can be connected to another strand of wire or a hidden transmitter. If the wire is cut or snapped, the signal running between the two emitters is interrupted and the communicator sends an alert. Each coil has a unique identification code, allowing security monitors to determine exactly where the wire was breached. A 15~-yard coil of sensor wire costs \$150 and weighs 15 lb. LC4.

\subsubsection{Monowire}\label{subsubsec:monowire_fence}
Deadly and nearly invisible, monowire (p. \pageref{subsec:monowire_spool}) can cut an intruder to pieces without warning. 

Any roll to see monowire requires a Vision or Traps roll at -4 (-1 if a searcher is specifically looking for it). Anyone walking through a monowire ``spiderweb'' will take 2d(10) cutting damage per strand. This damage is reduced to 1d(10) if moving very slowly, and is increased to 3d(10) if running. Care must be taken to avoid injury when stringing monowire, since it is hard to see and cuts almost anything. On a critical failure when using it, the user takes 1d(10) cutting damage to one hand. LC3.

\subsubsection{Neural Disruptor Field}\label{subsec:neural_disruptor_field}
This device is built into furniture or flooring. It produces an area effect identical to a neural disruptor (p. TODO). The grid can be left on, activated by sensors, or activated by remove control. Anyone moving through or ending his turn in an activated field must roll against HT-1 or suffer the effects of the neural disruptor. Add +1 to HT to resist for every DR 2 of sealed armor worn.

The effects lasts for as long as the power remains on and the victim remains on the grid, and for minutes equal to the margin of failure afterward. Neural disruptor fields producing a specific effect such as agony or paralysis are \$10,000 to install, plus \$1,000 per square yard. (Tunable fields cost double.) Neural disruptor fields run on building or ship power. LC3.

\subsection{Gravity Web}\label{subsec:gravity_web}
This is a possible use of a gwhel generator (p. \pageref{subsubsec:gwhel_generator}) as a form of defense by increasing the gravity to slow, disorient, or immobilize intruders.

\section{Security Scanners}\label{sec:security_scanners}
%\addcontentsline{toc}{section}{Security Scanners}
Security scanners are fixed sensor installations designed to identify intruders.

\subsection{Biometric Scanner}\label{subsec:biometric_scanner}
A multipurpose identity scanner. It can identify fingerprints, retina prints, voiceprints, or DNA prints, if this data is in a database it can access. Fingerprints and retina prints must be taken from a one~-yard range, while DNA prints require a hair or blood sample. All types of scans take one second.

\textit{Handheld Biometric Scanner:}\label{itm:handheld_biometric_scanner} A handheld device used by security personnel to check identities. \$1,000, 1 lb., A/1 day.

\textit{Biometric Scanlock:}\label{itm:briometric_scanlock} Integrated into a lock on a door, case, or other device. Cost, weight, and power are the same as the handheld scanner.

\subsection{Surveillance Sensors}\label{subsec:surveillance_sensors}
Security sensors are designed to detect an intruder and then take action, whether sounding an alarm, activating gas, or closing doors. They run indefinitely using vehicle or building power; most have backup power cells. Make a Traps-2 roll to spot them. Electro~-optical cameras (p. \pageref{itm:nv_surveillance_cam}), infrared cameras (p. \pageref{itm:infrared_surveillance_cam}), hyperspectral cameras (p. \pageref{itm:hyperspec_surveillance_cam}), short~-range terahertz radars (p. \pageref{subsec:terahertz_radar}), and imaging radars (p. \pageref{itm:imaging_radar}) are among the most common types. Add \$100/lb. to cover the mount and installation costs.

\subsection{Portal Scanners}\label{subsec:portal_scanners}
These are short~-range, ultra~-high resolution sensors that scan whatever passes between them. The device usually consists of transmitter and receiver with one~- to three~-yard range. They can be concealed in doorways or luggage conveyor belts, and may be set to trigger automatic doors, weapons, etc. if they detect anything. They are remotely controlled, with information displayed on a video screen or other interface. They work automatically, but their results must be interpreted by the Search skill roll of a person or AI.

\textit{X~-ray Scanner:}\label{itm:x-ray_scanner} This device uses X~-rays to see inside objects. It comes with a microcommunicator, a data port, and X~-ray analysis software. Add +4 to Explosives (EOD) skill when using it to defuse a bomb, and to Search skill when examining the contents of a package. \$2,000, 5 lb. (2.5 lb. per module), C/4 hr. LC3.

\textit{T~-Ray Portal Scanner}\label{itm:t-ray_scanner} This illuminates the target with tunable terahertz radiation. The absorption spectra of the resulting image ins analyzed and cross~-referenced with a database to check for chemicals of interest. This is good for locating drugs or other chemicals, explosives, and weapons. Gives +4 to Search skill for identifying non~-living objects. \$10,000, 10 lb., C/4 hr. LC3.

\textit{Explosives Scanner:}\label{itm:explosives_scanner} A nuclear quadrupole resonance (NQR) scanner excites a specific material (typically nitrogen) into a higher quantum mechanical energy state using a radiofrequency beam. When the material ``relaxes'' it gives off a distinct signal. The scan provides an unambiguous yes/no answer to the presence of all chemical explosives, but does not detect energy~-based explosives. It provides +4 to Search skill to detect chemical epxlosives. \$4,000, 10 lb., C/4 hr. LC3.

\textit{Ultrascan Portal:}\label{itm:ultrascan_portal} This uses para~-radar to perform a fast atomic~-level scan of the target's body, including a detailed bio~-scan. It gives a +5 (quality) bonus to Search rolls to find \textit{anything}, and can match a person by their genetic code against a database. It can be fooled by distortion fields (p. \pageref{itm:no-eye_distort_field_belt}. \$12,500, 10 lb., D/100 hr. LC3.

\subsection{Security Swarm}\label{subsec:security_swarm}
A swarm of data~-gathering microbots with short~-range infrared, tactile, visual, and chemical sensors. They search anyone they contact with Search-12 and Diagnosis-8, crawling over the subject's body and noting what is carried where. They are limited to performing either a skin search (+3 to Search skill) or a body cavity search (+5 to Search skill); see \textit{Search} (p. B219). They also record the subject's physical dimensions (height and build), species, gender, and can note if they're running a fever.

Since a swarm can't store data from more than one sweep, it should be uploaded to a computer where it can be reviewed.. Alternatively, the data can be accessed in real time as the swarm collects it. In addition to performing their own search, security swarms can be teleoperated to remotely perform skin or body cavity searches; as such, they provide a +1 (quality) bonus to Search skill. \$1,000/square yard. LC3.

\section{Surveillance and Tracking Devices}\label{sec:surveillance_tracking_devices}
%\addcontentsline{toc}{section}{Surveillance and Tracking Devices}
Passive sensors (p. \pageref{sec:indirect_passive_sensors}) and audio~-visual recorders (p. \pageref{sec:recording_and_playback}) are the basic tools of surveillance. Active sensors (p. \pageref{sec:active_sensors}) are useful if the subject lacks appropriate detection gear, of if letting her know she's being watched isn't a problem.

The following devices are useful for covertly obtaining information, or for following people or objects.

\subsection{Com Tap}\label{subsec:com_tap}
This device can tap into an optical or electrical cable line. It is a 100~-yard, hair~-thin optical cable ending in a clip head, connected to a pocket~-sized unit which includes both a monitor and a recorder that uses standard computer storage media. An Electronics Operation (Surveillance) roll is needed to succeed without damaging the line being tapped into; tapping an optical cable is at -3 to skill. \$500, 0.1 lb. A/100 hr. LC3.

\subsection{Homing Beacon}\label{subsec:homing_beacon}
A tiny tracer (SM -11) which can be set to activate when it receives a coded signal (sleeper mode), or to broadcast continuously. Its signal can be picked up by radio emissions scanners up to 50 miles away. \$40, negs., AA/100 hrs (1 year in ``sleeper'' mode). LC4.

\subsection{Nanobug}\label{subsec:nanobug}
A pinhead~-sized sensor/recorder unit (SM -18) with an adhesive backing, which is usually placed somewhere it can scan an entire room. Its camera and microphone can record constantly, listen for a specific voiceprint before recording, scan at specific times of day, or scan when its sensors detect light or motion in the room. It includes a microcommunicator (p. \pageref{subsubsec:standard_comm_sizes}) that can transmit recorded data in a short ``burst'' upon receiving a coded radio command. It can also be set to transmit after a specific time has passed. Once it transmits, it may be programmed to erase everything it has stored and begin recording again, or to self~-destruct. It will also self~-destruct if tampered with (roll vs. Explosives (EOD)-3 ot Traps-3 to defuse). \$100, AA/1 yr. LC3

\textit{Emissions Nanobug:}\label{itm:emissions_nanobug} As above, but instead of audiovisual sensors, it has field~-emission sensors that can read data sent to or from an electronic device that the bug is in direct contact with. It cannot read stored data that is not being accessed. Same cost, weight, power cost, and LC.

\textit{Microbot Nanobug:}\label{itm:mircobot_nanobug} A single, tiny microbot spy, tiny microbot spy (SM -16). As a regular nanobug, but add any microbot swarm chassis (p. TODO) at 1\% the usual cost. Mobility as per a cyberswarm. One hit destroys it.

\subsection{Laser Microphone}\label{subsec:laser_microphone}
This device turns any window or faceplate into a bug by reflecting an invisible laser beam off the glass and picking up vibrations caused by speech within the room. The user may roll Electronics Operation (Surveillance) to hear whatever is behind the window as if he were present on the other side of it. Extraneous noise such as loud music or running facuets is easily filtered out. Very heavy curtains, anti~laser coating, or triple~-glazing may defeat this method; bug stompers (p. \pageref{subsec:bug_stomper}) and privacy field white~-noise generators (p. TODO) never do. Laser sensors (p. TODO) can sense a laser mic.

\textit{Laser Microphone:}\label{itm:laser_mic} Range 6,000 yards. \$200, 2.5 lb., C/10 hr. LC3.

\textit{Pocket Laser Mic:}\label{itm:pocket_laser_mic} Range 600 yards. \$40, 0.1 lb., A/1 hr. LC3.

\subsection{Surveillance Worm}\label{subsec:surveillance_worm}
A flexible robotic snake only 0.1" wide and 2" long. A high~-tech endoscope. It has a light which illuminates a two~-yard cone and it has Infravision. 

It comes with a small screen and remote connected via the built~-in radio microcommunicator, but it can also be connected using a fiber~-optic cable. A series of surveillance worms can also relay signals to each other, either by physically connecting or by transmitting. The user can see whatever the worm is looking at, but has No Depth Perception.

Surveillance worms provide +3 on Search attempts. A Vision-5 roll is required to spot the worm. \$100, 0.1 lb., A/1 hr. LC4.

\subsection{Computer Pill}\label{subsec:computer_pill}
Designed to be disguised as a piece of candy, a raisin, a pill, or a seed, this is a disposable organic computer that activates once swalled. It attaches itself to the user's stomach, remaining in the body. It includes a radio microcommunicator, which only has a range of a few feet inside a body. This is enough range to contact any radio communicator implant in or acrried by the person who consumed the pill. 

The computer is Complexity 4, stores 1 PB, and costs \$50. Its integral power suplly operates it for one week.

\textit{Messenger Pill:}\label{itm:messenger_pill} A computer pill that incorporates a genetic scanner that can determine if it's been swallowed by the right person (or family, or species, or whatever). If the scan comes up with a match, the pill will try to call the communicator of its host. If not, it self~-destructs, and is dissolved by the body's own digestive system. Double the cost of a normal computer pill.

\subsection{Surveillance Swarm ("Surveillance Dust")}\label{subsec:surveillance_swarm}
This microbot swarm (p. TODO) mounts tiny video cameras and audio sensors, collectively the equivalent to a nanobug (p. \pageref{subsec:nanobug}). The swarm is programmed to remin in a particular place, observe for a period of time, and then return; it can also transmit information or be ordered to go to a different location. Base cost is \$50/square yard, modified by the \textit{Swarm Chassis} (p. TODO). LC3.

\section{Counter~-Surveillance and ECM}\label{sec:counter-surveillance}
%\addcontentsline{toc}{section}{Counter~-Surveillance and ECM}
These devices are used to warn of or defeat surveillance attempts.

\subsection{Bug Detectors}\label{subsec:bug_detectors}
\textit{RF Bug Detector:}\label{itm:rf_bug_detector} Detects and locates radio transmitters and microphones. This requires a Quick Contest between the operator's Electronics Operation (Surveillance) skill and the Electronics Operation (Surveillance) skill of whoever hid the bug. Locating a bug involves sweeping hte room with the device. The detector has a range of one yard, and takes 10 seconds to scan 100 square feet. \$200, 0.1 lb., A/10 hr. LC4.

\textit{Multispectral Bug Sweeper:}\label{itm:multispectral_bug_sweep} A counter~-surveillance radio frequency detector integrated with IR sensors and sensitive microphones. Adds +2 (quality) to Electronics Operation (Surveillance) skill when used to sweep for transmitters (infrared, sonic, and radio), comm taps, microphones, recorders, sound detectors, and other bugs. Alternatively, it can automatically scan with its own skill of 15. It can check or monitor comm lines, or everything within a 10~-yard radius. It folds up into a small briefcase. \$10,000, 5 lb. C/10 hr. LC3.

\subsection{Bug Stomper}\label{subsec:bug_stomper}
This is a pocket~-sized ``pink noise'' generator, which prevents audio surveillance devices from picking up anything but static within three yards of the device. It will jam the listening ability of a programmable bug, but not its visual sensors. It has not effect on laser mics, and advanced bugs can be built to filter out the noise. \$600, 0.25 lb., B/24 hr. LC3.

\subsection{Bughunter Swarm}\label{subsec:bughunter_swarm}
This microbot swarm (p. TODO) is equipped with emissions sensors. Each square yard of bughunters can sweep one square yard per minute to locate nanobugs, comm taps, surevillance swarms, or other surevillance devices. They have Electronics Operation (Surveillance)-12 for this function only, or give +2 (quality) to an operator's skill. \$4,000/square yard. LC3.

\subsection{Privacy Field}\label{subsec:privacy_field}
This sonic generator creates a spherical interference pattern that blocks all normal sound waves. No one inside the field's boundaries can hear any sound originating outside the field, and no sounds within the field are audible to anyone outside it. The generator can only be used within an atmosphere, with the diameter of the field increasing with atmospheric density. In a standard atmosphere, the bubble has a four~-yard radius. A privacy bubble will block audio bugs, but not laser listening devices. It also gives DR 15 against sonic weapons fired across its boundaries. \$5,000, 4 lb. C/8 hr.