These modifications can be added to just about any gadget that has both a specified cost and weight (i.e., not softtware, drugs, etc.)

\subsection{Disguised}
A gadget or weapon may be disguised as something else of similar shape, such as a laser rifle built into an umbrella. Double the cost for a mass~-produced disguised item; multiply cost by 5 for a custom ~-built one.

\subsection{Styling}\label{subsec:styling}
Styling alters the device's appearance to appeal to the potential customer's aesthetic sensibilities. Styling grants a bonus to reaction rolls from collectors and potential buyers, and to Merchant skill rolls made as Influence rolls (p. B359) on such people: +1 to rolls for ~x2 cost, +2 for ~x5 cost, or +3 for ~x10 cost.

\subsection{Rugged}\label{subsec:rugged}
Rugged gadgets are built to withstand abuse, harsh weather, and physical damage. Rugged systems incorporate modifications such as shock~-mounted brackets, heavy~-duty heat sinks, and redundant power supplies. A rugged gadget gets a +2 HT bonus and has twice its normal DR. Add 20\% to weight and double the cost.

\subsection{Cheap and Expensive Gadgets}
Cheap gadgets use inexpensive materials, older electronics, etc. They are generally 1.5 times normal weight (excluding the weight of any power cells) but half normal cost. 

Expensive gadgets use lightweight materials or have been deliberately designed to save weight. They are generally 2/3 normal weight (excluding the weight of any power cells) but cost twice as much.

\section{Plug~-in Gadgets}\label{sec:plug-in_gadgets}
Electronic gadgets can plug into other gadgets, either directly or using data cables. This allows them to link their functions, or to turn multiple functions on or off with a single Ready maneuver. Most often, this permits a computer to talk to (and control) multiple devices as peripherals, but other combinations can exist. 

Linking devices usually takes between 10 seconds and a minute, assuming the gadgets are compatible. If they aren't, or if a particular combination is very complex, connecting them requires a toolkit and Electronics Operation roll. Useful devices for linking gadgets include optical cable (p. \pageref{itm:optical_cable}), cable jacks (p. \pageref{itm:cable_jack}), and microcommunicators (p. \pageref{sec:comms}). A neural interface (p. \pageref{sec:neural_interfaces}) is a device for mentally linking a person to one or more gadgets.

Gadgets which use similar communicators, such as two gadgets both with radio or ir comms, can also be connected wirelessly. If the devices are compatible, this takes less time than physically connected them, between 5 to 30 seconds. If they are not, make a roll as above, but at a -2 penalty.

Most electronics can be preprogrammed for a few simple remote functions. Almost all electronics have a simple ``clock'' function, so they can be set to turn features on or off or activate various functions at a specific time, or upon receiving particular input.

For example, a recorder could be plugged into a communicator to play a message at a certain time, or upon receiving a specific signal, or to act as an answering machine. A detonator plugged into an inertial compass could go off when the subject reached a specific destination. Wireless connectivity is also possible: plug in a communicator set to a specific frequency, and you can talk to the device using a computer and communicator.

Devices that must be aimed are difficult to operate remotely. A gun with a communicator plugged into it could fire, but unless it also had a plugged-in sensor, the firer wouldn't know whether there was a target. And unless a gun with a sensor was attached to something like a powered tripod (p. \pageref{subsubsec:powered_tripod_mount}), it could only be fired at someone who crossed its sights. Some pieces of gadget programming may not be possible due to limitations.

\section{Combination Gadgets}
Want to invent a device featuring an inertial locator, multi-mode ladar, and neutrino communicator in one handy unit? Here's how.

If the gadgets can be used all at once, the weight is that of the heaviest gadget plus 80\% of the weight of the others, the weight savings being due to shared housing and components.

If only one of the combined gadgets can work at once, the weight is based on the highest weight among all gadgets plus 50\% of the other gadget weights, due to shared electronics and mechanical parts. (Make this calculation using the empty weight of the gadget, after subtracting the weight of any power cells and ammunition.)

The same applies to cost, based on the costliest of the gadgets. LC is always based on the lowest LC among all component gadgets.

Combined gadgets may end up using several different power cells. To make them all run off the same size of power cell, adjust endurance based on relative cell size. Since a D cell is 10 times the power of a C cell, a gadget that switched to using C cells will operate for one-tenth as long. Don't forget that changing the types of power cells will modify the gadget's actual weight – subtract the weight of the old power cell(s), and add the weight of the new one(s).

\section{Gear for Nonhumans}
All equipment listed in this book are assumed to be designed for humans or humanoids. If equipment is designed for non~-humans (Ravens, uplifted animals, non~-humanoids androids\ldots aliens?) it may have different controls or displays to accommodate the target species' hands or senses. The latter could be quite odd, such as olfactory readouts, colors or sounds in frequencies outside the human range of perception.

Such gadgets that are awkward to use will impose a penalty to skill equivalent to the Bad Grip disadvantage (-2 to -6). Gadget that requires missing senses or limbs may be unusable without technologies or advantages to emulate them. Adapting incompatible ``alien'' hardware is +10\%~-100\% of the original cost (and possibly weight).

Hardware for nonhumans and robots generally have identical stats, but some stats may differ. The exception to this are devices where statistics vary based on size and surface area. See \textit{Adjusting for SM} (below).

\begin{calloutbox}
    \section{Adjusting for SM}
    Some gadgets note to ``adjust for SM'' after their weight, cost, and power requirement. This means the weight, cost, and number of power cells are multiplied by a factor depending on the user's Size Modifier. For ordinary~-sized humans (SM 0) there is no change. However, if used by larger or smaller individuals, or if added to vehicles or robots with a higher or lower SM, multiply as follows:

    \begin{table}[H]
        \centering
        \rowcolors{1}{}{white!25}
        \begin{tabularx}{\columnwidth}{llXll}
             \textbf{SM} & \textbf{Modifier} && \textbf{SM} & \textbf{Modifier} \\
             SM -4 & ×1/20 && SM +4 & ×20 \\
             SM -3 & ×1/10 && SM +5 & ×50 \\
             SM -2 & ×1/5 && SM +6 & ×100 \\
             SM -1 & ×1/2 && SM +7 & ×200 \\
             SM +1 & ×2 && SM +8 & ×500 \\
             SM +2 & ×5 && SM +9 & ×1,000 \\
             SM +3 & ×10 && SM +10 & ×2,000 \\
        \end{tabularx}
        % \caption{Caption}
        \label{tab:adjust_for_sm}
    \end{table}
\end{calloutbox}
