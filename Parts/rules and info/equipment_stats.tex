Most gadgets detailed in subsequent chapters use standard format for statistics

\subsection{Cost, Weight, Power, LC}
Many gadgets list these four statistics at or towards the end of their description.

\subsubsection{Cost}
This is the price in generic \textbf{\textit{GURPS}} dollars. The price \textit{does not} include power cells, fuel, or ammunition. It also does not account for currency conversions.

\subsubsection{Weight}
This is the gadget's mass, as well as its weight, under a normal Earth gravity (1 G). It is given in pounds (lb.), or in some cases in tons (of 2,000 lb.). Weight \textit{does} include any power cells, fuel, or ammunition.

\subsubsection{Power}
If a non~-weapon gadget requires power, the letter designation for the type of power cell it uses is listed, along with the number of cells, if it requires more than one. See \textit{Power Cells} (p. \pageref{sec:power_cells}). This is followed by operating time, usually in hours (hr.), days, or weeks (wk.). Thus, ``D/12 hr.'' means the device requires one D cell that operates it for 12 hours of continuous use; ``2A/3 days'' means two A cells that collectively power it for 3 days. In some cases, a gadget's endurance is listed in ``uses'' or ``shots'' rather than time.

Some items rarely use power cells -- they're usually plugged into a building's electrical system or built into a vehicle. These have the notation ``external power.'' Some items such as computers have both notations: they're used with an external power supply, but also have power cells as a backup.

\subsubsection{Legality Class (LC)}
LC measures how likely an item is to be legally or socially controlled. If a LC is omitted, it means the item is not likely to be controlled even by the most repressive regime. For details of LC, see p. B267 and p. B507.

\subsection{Specialized Equipment}
Certain types of equipment are described in a different format.

\textit{Robots:} Senitent machines big enough to see are described using racial templates or as animals. See \textit{Machines as Characters} (p. TODO). 

\textit{Weapopns:} These use the format on p. B268~-271, with the exception that beam weapons list a power cell type instead of ammunition weight. ``7/2C'' means the weapon is powered by a pair of C cells, which are included in its 7 lb. weight.

\textit{Armor, Suits, and Protective Gear:} These use the format described on p. B282.

\textit{Software:} Computer programs have a Complexity raiting, which is the minimum Complexity of computer that can run it (see \textit{Software}, p. B472).

\textit{Vehicles:} These are described using the format on p. B464.

\subsection{Equipment Bonuses}
Several gadgets are examples of basic, good, fine, or even best-quality equipment in terms of \textit{Equipment Modifiers} (p. B345). Better quality equipment is \textit{usually} heavier and more expensive. A gadget's quality grade is always followed by ``(quality)'' in item descriptions, e.g., ``provides a +2 (quality) bonus to Electronics Repair (Armoury) skill.'' Quality is basic if there is no bonus, good if the bonus is +1, fine if at least +2 but less than +5, and best if +5.

Gadgets may also add an intrinsic bonus to skill because the underlying technology is easy to use or doesn't fail very often -- an example is the bonus that high~-tech surgical instruments provide. This is comparable to a ranged weapon's Accuracy. Any bonus that isn't marked ``(quality)'' is an intrinsic bonus. It has nothing to do with quality, and applies whenever you use that variety of gadget. An intrinsic bonus ``stacks'' with the quality modifier, if any.

\subsection{HP, HT, DR}
\textit{HP:} A gadget’s hit points are calculated from its weight. Use the chart on p. B558. Almost all gadgets will use the Unliving/Machine column.

\textit{HT:} A gadget is assumed to have HT 10 unless otherwise noted. Rugged gadgets (p. \pageref{subsec:rugged}) are HT 12.

\textit{DR:} Use the guidelines on p. B483. Most gadgets are made of plastic with DR 2. Weapons are normally DR 4, or DR 6 for solid metal melee weapons. Armor, suits, vehicles, etc. have their specified DR. Rugged gadgets have twice their normal DR.

\subsection{Bulk}
\textit{Bulk:} A general measure of size and handiness. The larger the penalty, the more bulky the item. Bulk modifies weapon skill when you take a Move and Attack maneuver with a ranged weapon, and serves as a penalty to Holdout skill when you attempt to conceal the gadget.