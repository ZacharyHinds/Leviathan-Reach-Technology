Legal equipment can usually be purchased from local shops or in the expansive Siren Corp catalogs (offering more than just Siren Corp equipment!). Very expensive items, those that require a license, or are otherwise specialized will require the buyer to find a dealer, work directly with a manufacturer, or other specialist to create it for them. Expensive but commonplace civilian gear is generally easy to purchase, but others may require more time and effort to arrange.

\section{Black Markets}\label{sec:black_markets}
If an item is illegal to own, expensive to acquire, or rare, it may be available from one or more outlets in the underground economy -- the black market.

\subsection{Availability}\label{subsec:availability_black_market}
Acquiring goods usually requires a Contact (pp. B44-B45) who knows how to reach the local black market merchants. Alternately, Streetwise (p. B223) can be used to locate new connections. These rolls are made secretly by the GM and a failure may result in unwanted attention from local law enforcement or criminal syndicates, while a critical failure may mean worse (perhaps a police ``sting,'' a criminal ambush, or defective goods).\\

{\color{Cerulean}
    \indent \textit{Modifiers:} Subtract the local Control Rating (p. B506) and Cultural Familiarity modifiers (p. B23); +1 if the area includes a major shipping port, is bordering a low~-CR country, or has ineffectual (i.e., corrupt or undermanned) law enforcement; -3 in an unfamiliar area.
}

\subsection{Markets}\label{subsec:markets_black_markets}
The term ``black market'' describes any business that operates illegally. Some sell proscribed services, while others circumvent tax or safety laws to undercut legitimate competitors. Within a black market, there may be various niches that cater to specific customers and require specialized Contacts (or skill penalties) to deal with. Some examples include:

\textit{Electronics:} This can include prototype or custom~-made computers, pirated copies of name~brand software, and banned software. Failure means the product you bought doesn't work as advertised or was fake. A critical failure may mean a knock on the door by the legal owner or patent holder -- who's to say how they found you and why.

\textit{Medical:} This can include cut~-rate surgical, unlicensed implant clinics, stolen bionics (perhaps with the previous owner still attached), and cheap drugs or nanosymbionts. Failure means you don't find what you're looking for, or the seller can't provide the amount requested. A critical failure may leave you with scars, placebos, or worse.

\textit{Entities:} Androids, robots, slaves, organs, and illegal clones (your own, or someone else's). Failure means you receive a defective product, or that having it around will be a risk to life and limb. A critical failure means you're the target of a raid or your new acquisition has a hidden homicidal streak.

\textit{Weapons:} Stolen or smuggled firearms, banned ammunition, and purloined military vehicle and combat robots. Failure means you can only get an inferior version of what you were looking for. Critical failure means it will malfunction on first use, or that someone is tracking it. Caveat emptor

\subsection{Prices}\label{subsec:prices_black_market}
The black market operates in competition with the normal market for many goods. To sell goods readily available from legal channels, it can only compete by making things easier to acquire (which is rare), or by selling for a lower price (by not charging for taxes, selling cheap copies, or fencing stolen goods). Easily copied media and textiles can sell for as little as 5\% of normal price, but most other items sell for 60\% of normal price.

The black market is opportunistic: if an item is hard to acquire legally, it has an edge over the legitimate market and will exploit it ruthlessly. Successful haggling with the Merchant skill can bring the price down, but black market dealers rarely have any incentive to offer big discounts!

Local availability and demand is a major factor in the final price. The inhabitants of a war-torn country may sell military weapons at a huge discount to anyone with hard currency, while electronics and food are sold at outrageous markups. A rich, peaceful country with thriving black markets in cheap alcohol and pirated movies may not have LC3 or lower weapons available for any price.